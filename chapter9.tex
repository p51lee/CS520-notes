\chapter{An Eager Functional Language}

\section{Motivation}

\begin{enumcirc}
	%
	\item
	%
	In \cref{ch:lambda-calculus}, we learned the lambda calculus and the eager
	evaluation for it.
	%
	They form the basis of most (or nearly all) call-by-value functional
	programming languages such as OCaml, Clojure, Scala, Scheme etc.
	%
	But developing such a real eager functional programming language involves much
	more than including the lambda calculus and adopting the eager evaluation.
	%
	The goal of this chapter and the subsequent few chapters is to understand these
	additional things.
	%
	\item
	%
	In the chapter, we will mainly study two topics related to the following
	questions:
	%
	\begin{enumrm}
		%
		\item
		%
		In order to help solve most real-world computational problems naturally, a
		functional programming language should include constants and operations for
		primitive types, such as \texttt{int}, and mechanisms for building data
		structures.
		%
		How can we do this?
		%
		How should we change the abstract grammar, the notion of canonical form,
		evaluation relation and denotational semantics?
		%
		\item
		%
		Also, we need to support recursive definitions.
		%
		What should we do?
		%
	\end{enumrm}
	%
	\item
	%
	We will focus on answering these questions.
	%
	In Chapter 10 of the textbook, Reynolds explains a lot more than what we will
	cover.
	%
	He even gives well-known list-manipulation examples of functional programming.
	%
	If you are interested in them, have a look at chapter 10.
	%
\end{enumcirc}

\section{Constants and primitive operations, basic (dynamic) types}

\begin{enumcirc}
	%
	\item
	%
	Recall the syntax of the lambda calculus and the notion of canonical form.
	%
	\begin{center}
		\begin{minipage}{0.5\textwidth}
			\begin{grammar}
				<exp> ::= <var>
				| <exp> <exp>
				| $\lambda$ <var> . <exp>

				<cfm> ::= <funcfm>

				<funcfm> ::= $\lambda$ <var> . <exp>
			\end{grammar}
		\end{minipage}
	\end{center}
	%
	Here we expressed the notion of canonical forms using the abstract grammar.

	We also recall the rules for evaluation; where we omit $\textrm{E}$ in
	$\RightarrowE$ since we are interested in eager evaluation only here.

	Canonical Forms
	%
	\[
		\inferrule{\;}{\lambda v. e \RightarrowE \lambda v. e}
	\]

	$\beta_\textrm{E}$-evaluation
	%
	\[
		\inferrule
		{
			e \RightarrowE \lambda v . e'' \\
			e' \RightarrowE z' \\
			\prths{\subst{e''}{v}{z'}} \RightarrowE z}
		{e \; e' \Rightarrow z}
	\]

	($z$ and $z'$ are canonical forms.
	%
	In general, we use $z$ with primes, subscripts or superscripts to denote
	canonical forms.)
	%
	\item
	%
	Adding constants, primitive operations, and those for basic type constructors
	(tuple, alternative / sum) amounts to extending $\gram{exp}$, $\gram{cfm}$, and
	$\Rightarrow$.
	%
	Extending $\gram{cfm}$ with more cases intuitively means that we make the
	language support values other than lambda expressions.
	%
	The extension of $\Rightarrow$ encodes the meanings of newly introduced
	constants and operations.
	%
	\item
	%
	We add four new kinds of values.
	%
	That is, we change the grammar for $\gram{cfm}$ as follows:
	%
	\begin{center}
		\begin{minipage}{0.4\textwidth}
			\grammarindent5em
			\begin{grammar}
				<cfm> ::= <funcfm>
				\alt <intcfm> \;|\; <boolcfm>
				\alt <tuplecfm> \;|\; <altcfm>

				<intcfm> ::= \dots \;|\; -2 \;|\; -1 \;|\; 0 \;|\; 1 \;|\; 2 \;|\; \dots

				<boolcfm> ::= \true \;|\; \false

				<tuplecfm> ::= $\langle$ <cfm> , \dots , <cfm> $\rangle$

				<altcfm> ::= @ <tag> <cfm>

				<tag> ::= 0 \;|\; 1 \;|\; 2 \;|\; \dots
			\end{grammar}
		\end{minipage}
	\end{center}
	%
	This extension means that expressions in this language denote five different
	kinds of values.
	%
	Another important point is that we only include operations for constructing
	tuples and alternatives, not those for destructing them;
	%
	such as projection and case operation.
	%
	In a sense, this comes from the fact that destructors represent \ul{unfinished}
	computation, and we regard something as value or canonical from when it
	represents \ul{completed} computation.
	%
	\item
	%
	Next we extend $\gram{exp}$ with appropriate constants and operations so that
	we can write expressions denoting computations with those newly introduced
	canonical forms, i.e., integers, booleans, tuples, and alternatives.
	%
	\begin{center}
		\begin{minipage}{0.8\textwidth}
			\grammarindent5em
			\begin{grammar}
				<exp> ::= <var> \;|\; <exp> <exp> \;|\; $\lambda$ <var> . <exp>
				\alt 0 \;|\; 1 \;|\; 2 \;|\; \dots \;|\; - <exp> \;|\; <exp> $\substack{+\\-\\\times\\\div\\\textrm{rem}}$ <exp>
				\alt \true \;|\; \false \;|\; $\neg$ <exp> \;|\; <exp> $\substack{\wedge\\\vee\\\Rightarrow\\\Leftrightarrow}$ <exp>
				\alt if <exp> then <exp> else <exp> \;|\; <exp> $\substack{=\\\ne\\\textrm{\textless}\\\le\\\textrm{\textgreater}\\\ge}$ <exp>
				\alt $\langle$ <exp> , \dots , <exp> $\rangle$ \footnotemark \;|\; <exp> . <tag> \footnotemark
				\alt @ <tag> <exp> \footnotemark \;|\; sumcase <exp> of ( <exp>, \dots , <exp> ) \footnotemark
				\alt error \;|\; typeerror
			\end{grammar}
		\end{minipage}
	\end{center}
	\footnoteeqn[-3]{constructing a tuple}
	\footnoteeqn{projection destructing a tuple}
	\footnoteeqn{constructing an alternative}
	\footnoteeqn{case analysis for destructing an alternative}
	%
	We already said that introducing four new kinds of canonical forms amounts to
	introducing four new runtime (or dynamic) types to the language, namely, int,
	bool, tuple and alternative.
	%
	For tuple and alternative, we have operations for constructing tuple and
	alternative expressions, and also operations for destructing them.
	%
	This is a typical pattern that appears repeatedly when one designs a new
	programming language.
	%
	When he or she adds a static or dynamic/runtime type to a language, he or she
	also introduces appropriate constructors and destructors for the type to the
	language.
	%
	\item
	%
	Evaluation relation $\Rightarrow$ should also be extended to incorporate
	intended semantics of newly added operations.
	%
	For integers and booleans, we change $\Rightarrow$ according to the standard
	semantics of primitive operations.
	%
	Hence, we give rules for only some operations.
	%
	Other cases are similar.
	%
	\[
		\begin{array}{c}
			\inferrule
			{e \Rightarrow i}
			{-e \Rightarrow \hat{-}i}
			\qquad \qquad
			\inferrule
			{e \Rightarrow b}
			{\neg e \Rightarrow \hat{\neg} b}
			\\[2em]
			\inferrule
			{e \Rightarrow i                                                       \\ e' \Rightarrow i'}
			{e \textrm{ op } e' \Rightarrow i \hat{\textrm{ op }} i'}
			\qquad
			\substack{
			\textrm{ where } \textrm{op} \in \braces{+,-,\times,=,\ne,<,\le,>,\ge} \\
				\textrm{ or } \prths{\textrm{op} \in \braces{\div, \textrm{rem}} \textrm{ and } i' \neq 0}
			}
			\\[2em]
			\inferrule
			{e \Rightarrow b                                                       \\ e' \Rightarrow b'}
			{e \textrm{ op } e' \Rightarrow b \hat{\textrm{ op }} b'}
			\qquad
			\substack{
				\textrm{ when } \textrm{op} \in \braces{\wedge,\vee,\Rightarrow,\Leftrightarrow}
			}
			\\[2em]
			\inferrule
			{e \Rightarrow \true                                                   \\ e' \Rightarrow z}
			{\textrm{if } e \textrm{ then } e' \textrm{ else } e'' \Rightarrow z}
			\qquad \qquad
			\inferrule
			{e \Rightarrow \false                                                  \\ e'' \Rightarrow z}
			{\textrm{if } e \textrm{ then } e' \textrm{ else } e'' \Rightarrow z}
		\end{array}
	\]
	%
	Note that $i$ and $b$ are integer and boolean constants, respectively.
	%
	Also, I used $\hat{\;}$ to emphasize mathematical operations.

	For operations for tuples and alternatives, we include rules for constructors
	that just evaluate their arguments, and those for destructors that convert
	constructed tuples and alternatives back to some of their components.
	%
	\[
		\begin{array}{c}
			\inferrule
			{e_0 \Rightarrow z_0 \quad \dots \quad e_{n-1} \Rightarrow z_{n-1}}
			{\chevrons{e_0, \dots, e_{n-1}} \Rightarrow \chevrons{z_0, \dots, z_{n-1}}}
			\qquad \qquad
			\inferrule
			{e \Rightarrow \chevrons{z_0, \dots, z_{n-1}}}
			{e . k \Rightarrow z_k}
			\substack{\textrm{ when } k < n}
			\\[2em]
			\inferrule
			{e \Rightarrow z}
			{@ \; k \; e \Rightarrow @ \; k \; z}
			\qquad \qquad
			\inferrule
			{e \Rightarrow @ \; k \; z \\ e_k \; z \Rightarrow z'}
			{\textrm{sumcase} \; e \; \textrm{of} \; \prths{e_0, \dots, e_{n-1}} \Rightarrow z'}
			\substack{\textrm{ when } k < n}
		\end{array}
	\]
	%
\end{enumcirc}

\section{Recursion}

\begin{enumcirc}
	%
	\item
	%
	We include letrec:
	%
	\begin{center}
		\begin{minipage}{0.7\textwidth}
			\begin{grammar}
				<exp> ::= \dots \;|\; letrec <var> $\equiv$ $\lambda$ <var> . <exp> in <exp>
			\end{grammar}
		\end{minipage}
	\end{center}
	%
	% ($\textrm{letrec } v \equiv \lambda w . e \textrm{ in } e'$)
	($\lletrec{v}{w}{e}{e'}$)
	%
	defines a recursive function $v$ and performs $e'$ with $v$ bound to this
	recursive function.
	%
	Thus,
	%
	\[
		\fv{\lletrec{v}{w}{e}{e'}} = \prths{\prths{\fv{e} \setminus \braces{w}} \cup \fv{e'}} \setminus \braces{v}
	\]
	%
	This means that the occurrence of $v$ in $e$ denotes $\lambda w . e$, the
	recursive function being defined here, not the value of a free variable $v$.
	%
	\item
	%
	This construct imposes two important constraints.
	%
	First, recursively defined entities should be functions like $\lambda w . e$.
	%
	For instance, we can't do
	%
	\[
		\textrm{letrec } v \equiv \chevrons{1, v} \textrm{ in } e
	\]
	%
	which defines an infinite tuple
	%
	$v = \chevrons{1, \chevrons{1, \chevrons{1, \dots}}}$.

	Second, the RHS of $\equiv$ should be a canonical form.
	%
	For instance, the following is not allowed.
	%
	\[
		\textrm{letrec } v \equiv \prths{\lambda x . \prths{\lambda y . 3}} \prths{\lambda z . z} \textrm{ in } e
	\]
	%
	Both restrictions are included because we use the eager evaluation.
	%
	In a programming language based on the normal-order evaluation such as Haskell,
	we don't need to impose those restrictions.
	%
	When we discuss denotational semantics, you will understand where these
	restrictions come from.
	%
	\item
	%
	Since adding letrec doesn't add a new kind of denotable values by expressions,
	we don't change $\gram{cfm}$.
	%
	(Although we don't show, letrec can be expressed using lambda expressions and applications.)
	%
	But we need to add a rule for evaluating letrec expressions.
	%
	Here is the rule.
	%
	\[
		\inferrule
		{\subst{e'}{v}{\prths{\subst{\lambda w . e}{v}{\lletrec{v}{w}{e}{v}}}} \footnotemark \Rightarrow z}
		{\lletrec{v}{w}{e}{e'} \Rightarrow z}
		\footnotemark
	\]
	\footnoteeqn[-1]{unrolling of recursive call $v$ in $\lambda w . e$}
	\footnoteeqn{execution of $e'$ with $v$ bound to its definition}
	%
	\item
	%
	Programming exercise:

	Suppose that we represent binary trees with integer leaves using alternative
	and tuple as follows.
	%
	\begin{itemize}
		%
		\item
		      %
		      $@ \; 0 \; n$ \dots for a terminal note (or leaf) labelled by the integer $n$.
		      %
		\item
		      %
		      $@ \; 1 \; \chevrons{l, r}$ \dots for a non-terminal node with left subtree $l$ and right subtree $r$.
		      %
	\end{itemize}
	%
	Write a program that sums all the integers in a given tree.

	Answer:
	\[
		\begin{array}{l}
			\lletrec
			{\textrm{add}}
			{t}
			{
				\textrm{ sumcase } t \textrm{ of } \prths{
					\lambda n . n,\;
					\lambda t' . \prths{\textrm{add} \prths{t' . 0} + \textrm{add} \prths{t' . 1}}
				}
			}
			{\textrm{add}}
		\end{array}
	\]
	%
\end{enumcirc}

\section{Denotational Semantics}

\begin{enumcirc}
	%
	\item
	%
	Recall the denotational semantics for the lambda calculus, and the eager
	evaluation, in particular domains used and the form of the semantics function
	$\bbrackets{-}$.
	%
	\begin{align*}
		D             & \defeq V_\bot                                                                                \\
		V             & \simeq V \toc D \quad , \quad \textrm{more precisely, } V \lrsupsubarrow{\phi}{\psi} V \to D \\
		\bbrackets{-} & \in \brackets{\gram{exp} \to V^{\gram{var}} \to D}                                           \\
	\end{align*}
	%
	Since wee added four new kinds of values, we should change $\simeq$ so that the
	isomorphism says $V$ consists of not just continuous functions but also those
	new values.
	%
	Also, we have to change $V_\bot$ to account for errors and failures of the
	runtime typechecker.
	%
	\item
	%
	We use the following $V$ and $V_*$ (which corresponds to $D$ above) and change
	the form of $\bbrackets{-}$ accordingly.
	%
	\begin{align*}
		V_*                  & \defeq \prths{V + \braces{\textrm{error} ,\; \textrm{typeerror}}}_\bot                                                               \\
		V                    & \lrsupsubarrow{\phi}{\psi} \subsctext{V}{int} + \subsctext{V}{bool} + \subsctext{V}{fun} + \subsctext{V}{tuple} + \subsctext{V}{alt} \\
		\subsctext{V}{int}   & \defeq \Z                                                                                                                            \\
		\subsctext{V}{bool}  & \defeq \B = \braces{\ttt, \fff}                                                                                                      \\
		\subsctext{V}{fun}   & \defeq V \toc V_*                                                                                                                    \\
		\subsctext{V}{tuple} & \defeq \bigcup_{n = 0}^\infty V^n = \bigcup_{n = 0}^\infty V \times \dots \times V                                                   \\
		\subsctext{V}{alt}   & \defeq \N \times V                                                                                                                   \\
	\end{align*}
	%
	\begin{enumrm}
		%
		\item
		%
		Don't be confused between $V$ here and $V$ in \circled{1}.
		%
		They are different.
		%
		Also, note the parallel between the definition of $\gram{cfm}$ and the
		isomorphism for $V$ here.
		%
		This isomorphism is in a sense a denotational way of saying there are five
		kinds of canonical forms (or values).
		%
		\item
		%
		$V_*$ extends $V$ not just with $\bot$ but also with error and typeerror, so that the semantics can express such errors.
		%
		\item
		%
		In general, the denotational semantics of an eager functional language has the
		form:
		%
		\[
			\bbrackets{-} \in \brackets{\gram{exp} \to V^{\gram{var}} \to T \footnotemark \prths{V}}
		\]
		\footnoteeqn[0]{$T$: domain constructor like $\prths{-}_\bot$ and $\prths{-}_*$}
		%
		and interprets functions using
		%
		\[
			V \toc T\prths{V}
		\]
		%
		This indicates that variables always get bound to values/canonical forms, not
		to arbitrary computations.
		%
	\end{enumrm}
	%
	\item
	%
	The actual definition of $\bbrackets{-}$ is involved, but in a sense
	straightforward.
	%
	The only things to be noteworthy are the uses of $f_*$ and $g_{\theta *}$ for
	$\theta \in \braces{\textrm{int}, \textrm{bool}, \textrm{fun}, \textrm{tuple},
			\textrm{alt}}$.
	%
	\[
		\begin{array}{c}
			f \in \brackets{V \toc V_*} \qquad f_* \in \brackets{V_* \toc V_*}
			\\[2em]
			f_* \prths{a} =
			\begin{cases}
				f\prths{b} & \textrm{if } a = \chevrons{0, b} \footnotemark \\
				a          & \textrm{otherwise}
			\end{cases}
			\\[2em]
			g \in \brackets{V_\theta \toc V_*} \qquad g_{\theta *} \in \brackets{V_* \toc V_*}
			\\[1em]
			\prths{
				\theta \in \braces{\textrm{int}, \textrm{bool}, \textrm{fun}, \textrm{tuple}, \textrm{alt}}
			}
			\\[1em]
			g_{\theta *} \prths{a} =
			\begin{cases}
				g\prths{b}                       & \textrm{if } a = \chevrons{0, b} \textrm{ and }
				\exists i, b' \textrm{ s.t. } \substack{b = \psi \chevrons{i, b'}                  \\ \textrm{ and } b' \in V_\theta} \\
				\chevrons{1, \textrm{typeerror}} & \textrm{if } a = \chevrons{0, b} \textrm{ and }
				\neg \prths{\substack{b = \psi \chevrons{i, b'}                                    \\ \textrm{ and } b' \in V_\theta}} \\
				a                                & \textrm{otherwise}
			\end{cases}
		\end{array}
	\]
	\footnoteeqn[0]{the first component of $V+\braces{\textrm{error},\textrm{typeerror}}$}
	%
	Intuitively, when $g_{\theta *}$ is applied to an element in $V$
	%
	(a value, not an error or $\bot$),
	%
	i.e. $\chevrons{0, b}$, it first does runtime typechecking and finds out
	whether $b$ is a value of type $\theta$.
	%
	If not, $g_{\theta *}$ returns a typeerror $\chevrons{1, \textrm{typeerror}}$.
	%
	If yes, on the other hand, it casts $b$ to a value $b'$ of type $\theta$, and
	runs $g$ on $b'$.
	%
	Thus, $g_{\theta *}$ does type-testing and type-casting.
	%
	\item
	%
	Here is the definition of $\bbrackets{-}$.
	%
	We present only some cases.
	%
	Look at the textbook for the complete definition.
	%
	\begin{align*}
		\bbrackets{-}               & \in \brackets{\gram{exp} \to V^{\gram{var}}\footnotemark \to V_*} \\
		\bbrackets{v} \eta          & = \chevrons{0, \eta\prths{v}}                                     \\
		\bbrackets{e_0 \; e_1} \eta & =
		\begin{cases}
			\bbrackets{e_0} \eta             & \textrm{if } \bbrackets{e_0} \eta \in \braces{\bot, \chevrons{1, \textrm{error}}, \chevrons{1, \textrm{typeerror}}}      \\
			\chevrons{1, \textrm{typeerror}} & \textrm{else if } \bbrackets{e_0} \eta = \chevrons{0, b} \textrm{ but }
			\neg \prths{\substack{
			\exists b' \textrm{ s.t. } b = \psi \chevrons{2, b'}                                                                                                        \\
			\wedge \; b' \in \subsctext{V}{fun}}}                                                                                                                       \\
			\bbrackets{e_1} \eta             & \textrm{else if } \bbrackets{e_1} \eta \in \braces{\bot, \chevrons{1, \textrm{error}}, \chevrons{1, \textrm{typeerror}}} \\
			b'\prths{a}                      & \textrm{else if }
			\prths{
				\begin{array}{l}
					\bbrackets{e_0} \eta = \chevrons{0, b} \textrm{ and }
					\prths{\substack{
					\exists b' \textrm{ s.t. } b = \psi \chevrons{2, b'} \\
					\wedge \; b' \in \subsctext{V}{fun}}}                \\
					\textrm{ and } \bbrackets{e_1} \eta = \chevrons{0, a}
				\end{array}
			}
		\end{cases}
	\end{align*}
	\footnoteeqn[0]{often written as $E$ to emphasize that it is the set of \ul{environments}}
	%
	More succinctly, we can write:
	%
	\begin{align*}
		\bbrackets{e_0 \; e_1} \eta                      & =
		\prths{
			\lambda f . \prths{ \lambda z . f \prths{z}}_* \prths{\bbrackets{e_1} \eta}
		}_{\textrm{fun}*} \prths{\bbrackets{e_0} \eta}\footnotemark                       \\
		\bbrackets{\lambda v . e} \eta                   & =
		\chevrons{0, \psi\prths{\chevrons{2, \lambda z . \bbrackets{e}\aug{\eta}{v}{z}}}} \\
		\bbrackets{\chevrons{e_0, \cdots, e_{n-1}}} \eta & =
		\prths{
			\lambda z_0 . \cdots \prths{\lambda z_{n-1} . \chevrons{0, \chevrons{z_0, \cdots \!, z_{n-1}}}}_*
		\! \prths{\bbrackets{e_{n-1}} \eta} \cdots }_* \! \prths{\bbrackets{e_0} \eta}    \\
		\bbrackets{e.k} \eta                             & =
		\prths{
			\lambda t . \textrm{ if } k \in \textrm{dom}\prths{t}\footnotemark
			\begin{cases}
				\textrm{ then } \chevrons{0, t_k} \\
				\textrm{ else } \chevrons{1, \textrm{typeerror}}
			\end{cases}
		}_{\textrm{tuple}*} \prths{\bbrackets{e} \eta}                                    \\
		\bbrackets{@ \; k \; e} \eta                     & =
		\prths{
			\lambda z . \chevrons{0, \psi \prths{\chevrons{4, \chevrons{k, z}}}}
		}_* \prths{\bbrackets{e} \eta}                                                    \\
	\end{align*}
	\footnoteeqn[-1]{
	%
	$\lambda f$, $\lambda z$ and $f\prths{z}$: function definitions and application in math,
	%
	$\prths{\;}_{\textrm{fun}*}$: typechecking and typecasting
	}
	\footnoteeqn{$t$ consists of at least $k+1$ components}
	%
	\vspace{-4em}
	%
	\begin{multline*}
		\bbrackets{\textrm{sumcase} \; e \; \textrm{of} \; \prths{e_0, \dots, e_{n-1}}} \eta \\
		= \prths{
			\lambda \chevrons{k, z} \textrm{ if } k < n
			\begin{cases}
				\textrm{ then } \prths{\lambda f . f \prths{z}}_{\textrm{fun}*} \prths{\bbrackets{e_k} \eta} \\
				\textrm{ else } \chevrons{1, \textrm{typeerror}}
			\end{cases}
		}_{\textrm{alt}*} \prths{\bbrackets{e} \eta}
	\end{multline*}
	%
	\begin{align*}
		\bbrackets{k} \eta                  & = \chevrons{0, \psi \prths{\chevrons{0, k}}}    \\
		\bbrackets{e_0 + e_1} \eta          & = \prths{
			\lambda i . \prths{
				\lambda i' . \chevrons{0, \psi{\chevrons{0, i + i'}}}
			}_{\textrm{int}*} \prths{\bbrackets{e_1} \eta}
		}_{\textrm{int}*} \prths{\bbrackets{e_0} \eta}                                        \\
		\bbrackets{\textrm{typeerror}} \eta & = \chevrons{1, \textrm{typeerror}}              \\
		\bbrackets{\true} \eta              & = \chevrons{0, \psi \prths{\chevrons{1, \ttt}}} \\
	\end{align*}

	\begin{exercisetab}
		%
		Suppose that $\bbrackets{e} \eta = \bot$.
		%
		What are $\bbrackets{e + \true} \eta$ and $\bbrackets{\true + e} \eta$?
		%
	\end{exercisetab}
	%
	\item
	%
	Although complex, the definition of $\bbrackets{e}$ is so far is in a sense
	straightforward because it is almost automatically derived from our informal
	understanding of $e$'s meaning,
	%
	the form of $\bbrackets{-}$ and the definition of $V$ and $V_*$ (i.e., semantic
	types), and the idea of inserting type-testing and type-casting using
	$\prths{-}_{\theta *}$.
	%
	But the case for recursive function
	%
	\[
		\bbrackets{\lletrec{v}{u}{e}{e'}} \eta
	\]
	%
	is not that straightforward.
	%
	To see why, consider the following naive (and incorrect) attempt:
	%
	\[
		\begin{array}{c}
			\bbrackets{\lletrec{v}{u}{e}{e'}} \eta =
			\bbrackets{\,e'\,} \aug{\eta}{v: Y_{V_*}\prths{F}} \\[1em]
			F: V_* \to V_* \qquad F\prths{b} = \bbrackets{\lambda u . e} \aug{\eta}{v: b}
		\end{array}
	\]
	%
	What's wrong with this?
	%
	$\aug{\eta}{v: Y_{V_*}\prths{F}}$ and $\aug{\eta}{v: b}$ are not environments
	because $Y_{V_*}\prths{F}$ and $b$ might not be values; i.e., elements of the
	form $\chevrons{0, c}$ for some $c \in V$.
	%
	This mathematical problem comes from the mismatch between the semantics here
	and the operation semantics
	%
	(i.e., evaluation relation)

	What should we do?
	%
	We use the fact that $\bbrackets{\lambda u . e} \eta$ is always of the form
	%
	\[
		\chevrons{0, \psi \prths{\chevrons{2, \lambda z . \bbrackets{e} \aug{\eta}{u: z}}}}
		\in V \toc V_* = \subsctext{V}{fun}.
	\]
	%
	Also, note that although $V$ is not a domain, $\subsctext{V}{fun} = V \toc V_*$
	is a domain.
	%
	Thus, every continuous function on $V \toc V_*$ has the least fixed point.
	%
	This means that we can define
	%
	\begin{align*}
		F                     & \in \brackets{\brackets{V \toc V_*} \to \brackets{V \toc V_*}} \\
		F \prths{f} \prths{z} & = \bbrackets{e}\augtwo{\eta}{v: f}{u: z}
	\end{align*}
	%
	Then, we can interpret the recursive definition as follows:
	%
	\[
		\bbrackets{\lletrec{v}{u}{e}{e'}} \eta =
		\bbrackets{e'} \aug{\eta}{v: \chevrons{0, \psi\prths{\chevrons{2, \subscmath{Y}{V \toc V_*}\prths{F}}}}}
	\]
	%
	If $v$ is not bound to a function or $v$ is not bound to a canonical form, we
	cannot use this interpretation.
	%
	This is one of the reasons that we consider only this restricted form of
	recursive definitions in eager functional programming languages.
	%
	OCaml adopted the same restriction.
	%
\end{enumcirc}

