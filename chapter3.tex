\chapter{Program Specifications and Their Proofs}

\section{Motivation}

\begin{enumcirc}
	%
	\item
	%
	Are there any methods that let us specify desired properties or intended
	behaviors of a program and prove that the specified properties indeed hold?
	%
	For instance, consider the program:
	%
	\[
		\subsctext{c}{div3} =
		\prths{
			\begin{array}{l}
				a := 0;                                        \\
				b := x;                                        \\
				\textrm{while } \prths{b \geq 3} \textrm{ do } \\
				\qquad b := b - 3;                             \\
				\qquad a := a + 1                              \\
			\end{array}
		}
	\]
	%
	We want to express formally that the program divides $x$ by $3$ and stores the
	quotient in $a$ and the remainder in $b$.
	%
	We also want to prove this formal specification.
	%
	\item
	%
	We will study Hoare logic and its variant for total correctness.
	%
	They provide the kind of methods that we are looking for.
	%
	\item
	%
	Hoare logic and its total-correctness variant are the basis of modern automatic
	software verifiers, such as Facebook's infer.
	%
	\item
	%
	Another reason for studying Hoare logic is that it shows why we need or where
	we use denotational semantics that we studied.
	%
\end{enumcirc}

\section{Syntax and semantics of specifications}

\begin{enumcirc}
	%
	\item
	%
	Specifications are a new type of phrases that formally express properties of
	programs.
	%
	\item
	%
	Syntax in terms of abstract grammar:
	%
	\begin{center}
		\begin{minipage}{0.5\textwidth}
			\grammarindent5em
			\begin{grammar}
				<spec> ::= \{<assert>\} <comm> \{<assert>\}
				\alt [<assert>] <comm> [<assert>]
			\end{grammar}
		\end{minipage}
	\end{center}
	%
	\begin{exampletab}
		%
		\todo
		%
	\end{exampletab}
	%
	\item
	%
	Intuitive reading:
	%
	\begin{enumrm}
		%
		\item
		%
		$\braces{p} c \braces{q}$ holds iff when $c$ is run in a state satisfying $p$,
		\ul{and it terminates normally}, \footnote{condition} then the final state satisfies $q$.
		%
		\item
		%
		$\brackets{p} c \brackets{q}$ holds iff when $c$ is run in a state satisfying $p$,
		then \ul{it terminates normally}, \footnote{conclusion} and the final state satisfies $q$.
		%
	\end{enumrm}
	%
	Note that $\brackets{p} c \brackets{q}$ expresses a stronger property than
	$\braces{p} c \braces{q}$.
	%
	The former is called \ul{total correctness specification}, \footnote{also total
		correctness triple} and the latter \ul{partial correctness specification}.
	\footnote{also partial correctness triple, Hoare triple or triple}

	$p$ \dots precondition or precedent.

	$q$ \dots postcondition or consequent.

	\begin{exercise}
		%
		Among all the partial correctness and total correctness specification from
		above, pick those that hold.
		%
	\end{exercise}
	%
	\item
	%
	Formal semantics:
	%
	\begin{align*}
		\bbrackets{-}                           & \in \brackets{\gram{spec} \to \B} \\
		\bbrackets{\braces{p} c \braces{q}}     & = \ttt \quad \textrm{ iff } \quad
		\bbrackets{p} \sigma = \ttt \wedge \bbrackets{c} \sigma \neq \bot \implies
		\bbrackets{q} \prths{\bbrackets{c} \sigma} = \ttt                           \\
		\bbrackets{\brackets{p} c \brackets{q}} & = \ttt \quad \textrm{ iff } \quad
		\bbrackets{p} \sigma = \ttt \implies \bbrackets{c} \sigma \neq \bot \wedge
		\bbrackets{q} \prths{\bbrackets{c} \sigma} = \ttt
	\end{align*}

\end{enumcirc}

