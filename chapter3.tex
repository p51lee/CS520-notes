\chapter{Program Specifications and Their Proofs}

\section{Motivation}

\begin{enumcirc}
	%
	\item
	%
	Are there any methods that let us specify desired properties or intended
	behaviors of a program and prove that the specified properties indeed hold?
	%
	For instance, consider the program:
	%
	\[
		\subsctext{c}{div3} =
		\prths{
			\begin{array}{l}
				a := 0;                                        \\
				b := x;                                        \\
				\textrm{while } \prths{b \geq 3} \textrm{ do } \\
				\qquad b := b - 3;                             \\
				\qquad a := a + 1                              \\
			\end{array}
		}
	\]
	%
	We want to express formally that the program divides $x$ by $3$ and stores the
	quotient in $a$ and the remainder in $b$.
	%
	We also want to prove this formal specification.
	%
	\item
	%
	We will study Hoare logic and its variant for total correctness.
	%
	They provide the kind of methods that we are looking for.
	%
	\item
	%
	Hoare logic and its total-correctness variant are the basis of modern automatic
	software verifiers, such as Facebook's infer.
	%
	\item
	%
	Another reason for studying Hoare logic is that it shows why we need or where
	we use denotational semantics that we studied.
	%
\end{enumcirc}

\section{Syntax and semantics of specifications}

\begin{enumcirc}
	%
	\item
	%
	Specifications are a new type of phrases that formally express properties of
	programs.
	%
	\item
	%
	Syntax in terms of abstract grammar:
	%
	\begin{center}
		\begin{minipage}{0.5\textwidth}
			\grammarindent5em
			\begin{grammar}
				<spec> ::= \{<assert>\} <comm> \{<assert>\}
				\alt [<assert>] <comm> [<assert>]
			\end{grammar}
		\end{minipage}
	\end{center}
	%
	\begin{exampletab}
		%
		\todo
		%
	\end{exampletab}
	%
	\item
	%
	Intuitive reading:
	%
	\begin{enumrm}
		%
		\item
		%
		$\braces{p} c \braces{q}$ holds iff when $c$ is run in a state satisfying $p$,
		\ul{and it terminates normally}, \footnote{condition} then the final state satisfies $q$.
		%
		\item
		%
		$\brackets{p} c \brackets{q}$ holds iff when $c$ is run in a state satisfying $p$,
		then \ul{it terminates normally}, \footnote{conclusion} and the final state satisfies $q$.
		%
	\end{enumrm}
	%
	Note that $\brackets{p} c \brackets{q}$ expresses a stronger property than
	$\braces{p} c \braces{q}$.
	%
	The former is called \ul{total correctness specification}, \footnote{also total
		correctness triple} and the latter \ul{partial correctness specification}.
	\footnote{also partial correctness triple, Hoare triple or triple}

	$p$ \dots precondition or precedent.

	$q$ \dots postcondition or consequent.

	\begin{exercise}
		%
		Among all the partial correctness and total correctness specification from
		above, pick those that hold.
		%
	\end{exercise}
	%
	\item
	%
	Formal semantics:
	%
	\begin{align*}
		\bbrackets{-}                           & \in \brackets{\gram{spec} \to \B} \\
		\bbrackets{\braces{p} c \braces{q}}     & = \ttt \quad \textrm{ iff } \quad
		\bbrackets{p} \sigma = \ttt \wedge \bbrackets{c} \sigma \neq \bot \implies
		\bbrackets{q} \prths{\bbrackets{c} \sigma} = \ttt                           \\
		\bbrackets{\brackets{p} c \brackets{q}} & = \ttt \quad \textrm{ iff } \quad
		\bbrackets{p} \sigma = \ttt \implies \bbrackets{c} \sigma \neq \bot \wedge
		\bbrackets{q} \prths{\bbrackets{c} \sigma} = \ttt
	\end{align*}
	%
\end{enumcirc}

\section{Inference rules}

\begin{enumcirc}
	%
	\item
	%
	Methods or rules for proving or deriving partial or total correctness triples.
	%
	\item
	%
	\;\vspace{-1.5em}
	%
	\[
		\inferrule{\textrm{premises}}{\textrm{conclusion}}
		\qquad \qquad
		\inferrule{\varphi_1 \quad \varphi_2 \quad \dots \quad \varphi_n}{\psi}
	\]
	%
	(if $\varphi_1, \varphi_2, \dots, \varphi_n$ are true, then $\psi$ is true)
	%
	\item
	%
	Rules associated with program constructs:
	%
	\[
		\begin{array}{c}
			\inferrule
			{ }
			{\braces{p} \cskip \braces{p}}
			\qquad \qquad
			\inferrule
			{ }
			{\brackets{p} \cskip \brackets{p}}
			\\[2em]
			\inferrule
			{\braces{p} c_1 \braces{r}                    \\ \braces{r} c_2 \braces{q}}
			{\braces{p} c_1 ; c_2 \braces{q}}
			\qquad \qquad
			\inferrule
			{\brackets{p} c_1 \brackets{r}                \\ \brackets{r} c_2 \brackets{q}}
			{\brackets{p} c_1 ; c_2 \brackets{q}}
			\\[2em]
			\inferrule
			{\braces{p \wedge b} c_1 \braces{q}           \\ \braces{p \wedge \neg b} c_2 \braces{q}}
			{\braces{p} \cif{b}{c_1}{c_2} \braces{q}}
			\qquad \qquad
			\inferrule
			{\brackets{p \wedge b} c_1 \brackets{q}       \\ \brackets{p \wedge \neg b} c_2 \brackets{q}}
			{\brackets{p} \cif{b}{c_1}{c_2} \brackets{q}}
			\\[2em]
			\inferrule
			{ }
			{\braces{\subst{q}{x}{e}} x := e \braces{q}}
			\qquad \qquad
			\inferrule
			{ }
			{\brackets{\subst{q}{x}{e}} x := e \brackets{q}}
			\\[2em]
			\inferrule
			{\braces{i \wedge b} c \braces{i}}
			{\braces{i} \cwhile{b}{c} \braces{i \wedge \neg b}}
			\qquad \quad
			\inferrule
			{i \wedge b \Rightarrow e \ge 0 \footnotemark \\ \brackets{i \wedge b \wedge e = v_0 \footnotemark} c \brackets{i \wedge e < v_0}}
			{\brackets{i} \cwhile{b}{c} \brackets{i \wedge \neg b}}
		\end{array}
	\]
	\footnoteeqn[-1]{$i \wedge b \Rightarrow e \ge 0$ should be valid. That is, $\bbrackets{i \wedge b \Rightarrow e \ge 0} \sigma = \ttt$ for all $\sigma$.}
	\footnoteeqn{when $v_0$ does not occur free in $i$, $b$, $c$ or $e$}
	%
	\begin{enumrm}
		%
		\item
		%
		Note that rules for total correctness and the corresponding ones for partial
		correctness are identical except for the case of loop.
		%
		This is expected because these two notions differ only in their treatment of
		non-termination.
		%
		\item
		%
		The rules for while require that $i$ should be preserved by the body of the
		loop.
		%
		The one for total correctness additionally requires that the value of $e$
		should decrease whenever we run the loop body $c$ once, but it cannot be
		negative.
		%
		All these requirements together give the conclusions of the rules.
		%
		\item
		%
		The rules for assignment also deserve some thoughts.
		%
		They in a sense say that running an assignment backward symbolically is the
		same as doing substitution.
		%
		It holds because of the substitution theorem (Prop 1.3 and Prop 1.4 in the
		textbook).

		Reminder of the theorem specialized to our case here:
		%
		\[
			\begin{tikzcd}[cramped, sep=7em]
				\Sigma
				\arrow[r, "\lambda \sigma . \aug{\sigma}{x:\bbrackets{e}\sigma}"]
				\arrow[d, swap, "\bbrackets{\subst{q}{x}{e}}"] &
				\Sigma
				\arrow[d, "\bbrackets{q}"]\\
				\B \arrow[r,-,double, "="] &
				\B
			\end{tikzcd}
		\]
		%
	\end{enumrm}
	%
	Rules not associated with any specific program constructs
	%
	(sometimes called structural rules or adaptation rules):
	%
	\[
		\inferrule
		{p \Rightarrow p' \\ \braces{p'} c \braces{q'} \\ q' \Rightarrow q}
		{\braces{p} c \braces{q}}
		\qquad \qquad
		\inferrule
		{p \Rightarrow p' \\ \brackets{p'} c \brackets{q'} \\ q' \Rightarrow q}
		{\brackets{p} c \brackets{q}}
	\]
	%
	They are called \ul{the rule of consequence}.
	%
	They often enable us to use the other rules, in particular, those for loop and
	if.
	%
	\item
	%
	I omit many structural rules and the rule for newvar.
	%
	Look at the textbook if you are interested.
	%
\end{enumcirc}

\section{Example proof}

\todo

\begin{enumcirc}
	%
	\item
	%
	In practice, people use the rule of consequence, without mentioning it
	explicitly.
	%
	Also, they use many derived rules.
	%
	\item
	%
	This proof has the flavor of running a program backward symbolically because of
	its heavy use of the assignment rule and the fact that the rule of consequence
	is used only when it is necessary.

	\begin{exercisetab}
		%
		\todo
		%
	\end{exercisetab}

	\begin{exercisetab}
		%
		\todo
		%
	\end{exercisetab}
	%
\end{enumcirc}

\section{Soundness}

\begin{theorem}[Soundness theorem]
	%
	If $\brackets{p} c \brackets{q}$ is derivable using
	%
	\ul{the rules that we studied},
	%
	\footnote{called rules in Hoare logic}
	%
	then $\bbrackets{\braces{p} c \braces{q}} = \ttt$, i.e.,
	%
	the triple $\braces{p} c \braces{q}$ holds.
	%
	If $\brackets{p} c \brackets{q}$ is derivable, then
	%
	$\bbrackets{\brackets{p} c \brackets{q}} = \ttt$.
	%
\end{theorem}
%
\begin{proof}
	%
	Intuitively, the theorem says that all rules are correct.
	%
	In fact, typical proofs of the theorem show the correctness of the rules in the
	following sense:

	If $\inferrule{\varphi_1 \quad \varphi_2 \quad \dots \quad \varphi_n}{\psi}$
	%
	then
	%
	\[
		\bbrackets{\varphi_1} = \ttt \wedge
		\bbrackets{\varphi_2} = \ttt \wedge \dots \wedge
		\bbrackets{\varphi_n} = \ttt \implies
		\bbrackets{\psi} = \ttt.
	\]
	%
	The rules for loop (or while) are the most important cases.
	%
	We will consider only the one for partial correctness.
	%
	\[
		\inferrule
		{\braces{i \wedge b} c \braces{i}}
		{\braces{i} \cwhile{b}{c} \braces{i \wedge \neg b}}
	\]
	%
	We first do a bit of rewriting for the semantics of specifications.
	%
	\begin{multline*}
		\bbrackets{\brackets{p} c \brackets{q}}
		\quad \textrm{ iff } \\
		\forall \sigma \in \Sigma ,\;
		\bbrackets{p} \sigma = \ttt \implies
		\bbrackets{q}_\bot \prths{\bbrackets{c} \sigma} \sqsubseteq \ttt \\
		\textrm{ where }
		\bbrackets{q}_\bot \in \brackets{\Sigma_\bot \to \B_\bot} \\
		\textrm{ s.t. }
		\bbrackets{q}_\bot \sigma = \bbrackets{q} \sigma \textrm{ for all } \sigma \in \Sigma
		\textrm{ and }
		\bbrackets{q}_\bot \prths{\bot} = \bot
	\end{multline*}
	%
	We need to prove that, if
	%
	\begin{equation}{} \label{eq:while-partial-correctness} \tag{$\star$}
		\forall \sigma .\;
		\bbrackets{i \wedge b} \sigma = \ttt \implies
		\bbrackets{i}_\bot \prths{\bbrackets{c} \sigma} \sqsubseteq \ttt
	\end{equation}
	%
	then
	%
	\[
		\forall \sigma .\;
		\bbrackets{i} \sigma = \ttt \implies
		\bbrackets{i \wedge \neg b}_\bot \prths{\bbrackets{\cwhile{b}{c}} \sigma} \sqsubseteq \ttt
	\]

	Assume that \cref{eq:while-partial-correctness} holds.
	%
	Let
	%
	\begin{align*}
		F                          & \in \brackets{\prths{\Sigma \to \Sigma_\bot} \toc \prths{\Sigma \to \Sigma_\bot}}                      \\
		F \prths{f} \prths{\sigma} & = \cif{\bbrackets{b} \sigma = \ttt}{\prths{ f_\doublebot \circ \bbrackets{c}} \prths{\sigma}}{\sigma}.
	\end{align*}
	%
	Define $f_n := F^n \prths{\bot}$ for all $n \ge 0$.
	%
	Then, $\displaystyle \bbrackets{\cwhile{b}{c}} = \bigcup_{n = 0}^\infty f_n$.
	%
	We will show that for all $n \ge 0$,
	%
	\begin{equation} \label{eq:while-partial-correctness-2} \tag{$\star\star$}
		\forall \sigma .\;
		\bbrackets{i} \sigma = \ttt \implies
		\bbrackets{i \wedge \neg b}_\bot \prths{f_n \prths{\sigma}} \sqsubseteq \ttt.
	\end{equation}
	%
	This is sufficient because for all $\sigma \in \Sigma$ s.t. $\bbrackets{i}
		\sigma = \ttt$,
	%
	\begin{align*}
		\bbrackets{i \wedge \neg b}_\bot \prths{\bbrackets{\cwhile{b}{c}} \sigma}
		 & = \bbrackets{i \wedge \neg b}_\bot \prths{\prths{\bigcup_{n = 0}^\infty f_n} \prths{\sigma}}       \\
		 & = \bbrackets{i \wedge \neg b}_\bot \prths{\bigcup_{n = 0}^\infty f_n \prths{\sigma}}               \\
		 & = \footnotemark \bigcup_{n = 0}^\infty \bbrackets{i \wedge \neg b}_\bot \prths{f_n \prths{\sigma}} \\
		 & \sqsubseteq \footnotemark \bigcup_{n = 0}^\infty \ttt = \ttt.
	\end{align*}
	\footnoteeqn[-1]{because $\bbrackets{i \wedge \neg b}_\bot$ is continuous}
	\footnoteeqn{because of \cref{eq:while-partial-correctness-2}}
	%
	Our proof of \cref{eq:while-partial-correctness-2} uses induction on $n$.

	\begin{itemize}
		%
		\item
		      %
		      Base case $n = 0$: $f_0 = \bot$.
		      %
		      \[
			      \therefore
			      \bbrackets{i \wedge \neg b}_\bot \prths{f_0 \prths{\sigma}} = \bbrackets{i \wedge \neg b}_\bot \prths{\bot} = \bot \sqsubseteq \ttt.
		      \]
		      %
		\item
		      %
		      Inductive case $n = m + 1$:
		      %
		      Pick $\sigma$ s.t. $\bbrackets{i} \sigma = \ttt$.
		      %
		      \begin{align*}
			       & \quad\, \bbrackets{i \wedge \neg b}_\bot \prths{f_{m + 1} \prths{\sigma}}                                    \\
			       & = \bbrackets{i \wedge \neg b}_\bot \prths{F\prths{f_m} \prths{\sigma}}                                       \\
			       & = \bbrackets{i \wedge \neg b}_\bot
			      \prths{\cif{\bbrackets{b} \sigma = \ttt}{\prths{ {f_m}_\doublebot \circ \bbrackets{c}} \prths{\sigma}}{\sigma}} \\
			       & = \cif
			      {\bbrackets{b} \sigma = \ttt}
			      {\prths{\bbrackets{i \wedge \neg b}_\bot \circ {f_m}_\doublebot} \prths{\bbrackets{c} \prths{\sigma}}}
			      {\bbrackets{i \wedge \neg b}_\bot \prths{\sigma}}                                                               \\
		      \end{align*}
		      %
		      \vspace{-3em}\\
		      %
		      Since $\bbrackets{i} \sigma = \ttt$, if $\bbrackets{b} \sigma \ne \ttt$, then
		      %
		      \[
			      \bbrackets{i \wedge \neg b}_\bot \prths{\sigma} = \bbrackets{i}_\bot \prths{\sigma} = \ttt \sqsubseteq \ttt.
		      \]
		      %
		      If $\bbrackets{b} \sigma = \ttt$ and $\bbrackets{c} \sigma = \bot$, then
		      %
		      \[
			      \prths{\bbrackets{i \wedge \neg b}_\bot \circ {f_m}_\doublebot} \prths{\bbrackets{c} \prths{\sigma}} = \bot \sqsubseteq \ttt.
		      \]
		      %
		      If $\bbrackets{b} \sigma = \ttt$ and $\bbrackets{c} \sigma \ne \bot$, then
		      %
		      $\bbrackets{i}_\bot \prths{\bbrackets{c}\sigma} = \ttt$
		      %
		      by \cref{eq:while-partial-correctness}.
		      %
		      Thus,
		      %
		      \[
			      \prths{\bbrackets{i \wedge \neg b}_\bot \circ {f_m}_\doublebot} \prths{\bbrackets{c} \prths{\sigma}} =
			      \bbrackets{i \wedge \neg b}_\bot \prths{ f_m \prths{\bbrackets{c} \prths{\sigma}} } \sqsubseteq \ttt
		      \]
		      %
		      by induction hypothesis.
		      %
	\end{itemize}

\end{proof}

