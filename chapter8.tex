\chapter{The Lambda Calculus}

\section{Motivation}

\begin{enumcirc}
	%
	\item
	%
	Most programming languages support a mechanism for declaring functions and
	applying them to arguments.
	%
	In fact, in functional languages, such as Ocaml, Haskell, Clojure and Scala (to
	some extent),
	%
	function declaration and application (not state access and update) is the main
	device of computation.
	%
	\item
	%
	The lambda calculus is a simple formal language that lets us study principles
	behind function declaration and application without being distracted by the
	complexities of usual real-world programming languages.
	%
	If forms the basis of many functional languages.
	%
	Also, it can be used to define a notion of computability.
	%
	\item
	%
	One interesting construct of the lambda calculus is so-called lambda
	abstraction.
	%
	\[
		\lambda x. e
	\]
	%
	which denotes a function with formal argument $x$ and body $e$.
	%
	Nowadays most mainstream languages (c++, Java, Python, etc.) support this
	construct.
	%
	The lambda abstraction is particularly useful when we use higher-order
	functions.
	%
	For instance, to express
	%
	\[
		\int_{0}^{1} dx \int_{0}^{x} dy \prths{x+y}^2
	\]
	%
	in a programming language with the ``\ul{integrate}'' primitive, we can write

	\todo

\end{enumcirc}

\section{Syntax}

\begin{center}
	\begin{minipage}{0.5\textwidth}
		\begin{grammar}
			<exp> ::= <var>
			| <exp> <exp> \footnotemark
			| $\lambda$ <var> . <exp> \footnotemark
		\end{grammar}
	\end{minipage}
\end{center}
\footnoteeqn[-1]{called application}
\footnoteeqn{called abstraction or lambda expression}

\begin{enumcirc}
	%
	\item
	%
	examples
	%
	\todo
	%
	\item
	%
	The set of free variables (in an expression of the lambda calculus), the
	substitution operator, and the $\alpha$-equivalence (or renaming equivalence)
	for the lambda-calculus expressions are defined as you expect.
	%
	Once you get the idea that the lambda operator in $\lambda x. e$ binds the
	variable $x$ in the expression $e$, just as the quantifier $\forall$ in
	$\forall x. p$ binds the variable $x$ in an assertion $p$.
	%
	\begin{enumrm}
		%
		\item
		\begin{align*}
			\fv{v}            & = \braces{v}                  \\
			\fv{e_1 \; e_2}   & = \fv{e_1} \cup \fv{e_2}      \\
			\fv{\lambda v. e} & = \fv{e} \setminus \braces{v} \\
		\end{align*}
		%
		\item
		%
		$\delta$ is a substitution, i.e., a map from $\chevrons{var}$ to $\chevrons{exp}$:
		%
		\begin{align*}
			v / \delta                    & = \delta\prths{v}                                                             \\
			\prths{e_1 \; e_2} / \delta   & = \prths{e_1 / \delta} \prths{e_2 / \delta}                                   \\
			\prths{\lambda v. e} / \delta & = \lambda \subsctext{v}{new}. \prths{e / \aug{\delta}{v: \subsctext{v}{new}}} \\
			where \subsctext{v}{new}      & \notin \bigcup_{w \in \fv{e} \setminus \braces{v}} \fv{\delta\prths{w}}
		\end{align*}
		%
		\item
		%
		The renaming or change of bound variable means the operation of replacing an
		occurrence of $\lambda v. e$ by
		%
		$\lambda \subsctext{v}{new}. \prths{\subst{e}{v}{\subsctext{v}{new}}}$,
		%
		\footnote{
			This means the substitution that maps all variables to themselves except $v$, which
			it maps to $\subsctext{v}{new}$.
		}
		%
		Two expressions $e_1$ and $e_2$ are \ul{$\alpha$-equivalent} or
		\ul{renaming-equivalent} if we can obtain $e_2$ from $e_1$ by applying this
		renaming operation to some subexpressions of $e_1$ zero or multiple times.
		%
		We write $e_1 \equiv e_2$ to denote their $\alpha$-equivalence.
		%
		\begin{example}
			\todo
		\end{example}
		%
	\end{enumrm}
\end{enumcirc}

\section{Reduction}

\begin{enumcirc}
	%
	\item
	%
	When we studied operational semantics, we used transition relation
	$\rightarrow$ to model one-step computation.
	%
	We do something similar for the lambda calculus.
	%
	We define a binary relation
	%
	$\rightarrow \in \chevrons{exp} \times \chevrons{exp}$,
	%
	called \ul{contraction relation}, which models or formalizes single-step
	computation.
	%
	Then, we will call the reflexive transitive closure of $\rightarrow$, i.e.,
	$\rightarrow^*$, the \ul{reduction relation}.
	%
	\item
	%
	Here is the definition of the contraction relation using inference rules.
	%
	\begin{enumrm}
		%
		\item
		%
		$\beta$-reduction
		%
		\[
			\inferrule
			{\;}
			{\prths{\lambda v. e}{e^\prime} \rightarrow \subst{e}{v}{e^\prime}}
		\]
		%
		\item
		%
		Renaming
		%
		\[
			\inferrule
			{e_0 \rightarrow e_1 \\ e_1 \equiv e_1^\prime}
			{e_0 \rightarrow e_1^\prime}
		\]
		%
		\item
		%
		Contextual closure
		%
		\[
			\inferrule
			{e_0 \rightarrow e_1}
			{e_0^\prime \rightarrow e_1^\prime}
		\]
		%
		$e_1^\prime$ is obtained from $e_0^\prime$ by replacing one occurrence of
		$e_0$ in $e_0^\prime$ by $e_1$.
	\end{enumrm}
	%
	\item
	%
	The real change happens by the first rule, $\beta$-reduction.
	%
	The second rule means that the contraction relation is defined on
	$\alpha$-equivalence classes.
	%
	The third rule says that any subexpression of a given $e$ can be contracted by
	the $\beta$-reduction rule.
	%
	Here are a few examples that may help you to see what goes on here.
	%
	\todo
	%
	\item
	%
	Because of the third rule, the contraction relation is not deterministic.
	%
	That is, $e_0 \rightarrow e_1$ and $e_0 \rightarrow e_2$ do not imply that $e_1
		\equiv e_2$.
	%
	For a counterexample, look at the example above.
	%
	However, this nondeterminism comes from any nondeterministic constructs of the
	lambda calculus (which do not exist).
	%
	The following theorem shows one consequence of this, and expresses that the
	contraction relation is essentially deterministic.
	%
	\begin{property}[Church-Rosser Theorem]
		%
		\label{church-rosser}
		\ \\
		%
		If $e \rightarrow^* e_0$\footnote{ reflexive transitive closure of
			$\rightarrow$ } and $e \rightarrow^* e_1$, then there exists $e_2$ s.t. $e_0
			\rightarrow^* e_2$ and $e_1 \rightarrow^* e_2$.
		%
	\end{property}
	%
	\item
	%
	An expression $e$ \ul{is a $\beta$-normal form} if it cannot be contracted.
	%
	Intuitively, such an expression denotes the final result of some computation,
	and the reduction relation $\rightarrow^*$ performs computation by transforming
	a given expression to normal form.
	%
	\cref{church-rosser} implies that every expression can be reduced to at most
	one normal-form expression modulo $\alpha$-equivalence.
	%
	\begin{property}
		%
		\ \\
		If $e_0 \rightarrow^* e_1$, $e_0 \rightarrow e_2$ and $e_1$ and $e_2$ are
		$\beta$-normal forms, then $e_1 \equiv e_2$. (i.e. $e_1$ and $e_2$ are
		$\alpha$-equivalent)
		%
	\end{property}
	%
	\begin{proof}
		%
		By \cref{church-rosser}, $\exists e_3$ s.t.
		%
		$e_1 \rightarrow^* e_3$ and $e_2 \rightarrow^* e_3$.
		%
		Since $e_1$ and $e_2$ are $\beta$-normal forms,
		%
		$e_1 \equiv e_3$ and $e_2 \equiv e_3$.
		%
		$\therefore e_1 \equiv e_2$.
		%
	\end{proof}
	%
	\item
	%
	One natural question is whether we can find a good strategy to use the
	nondeterminism in the third ``Contextual closure'' rule, so that if an
	expression $e$ can be reduced to a normal form, this strategy indeed transforms
	$e$ to such a normal-form expression.

	To see this issue, consider the following two reduction sequences:
	%
	\begin{align*}
		\prths{\lambda u. \lambda v. v}
		\prths{
			\prths{\lambda x. x \; x}
			\prths{\lambda x. x \; x}
		}
		 & \rightarrow \lambda v. v \\
		\prths{\lambda u. \lambda v. v}
		\prths{
			\prths{\lambda x. x \; x}
			\prths{\lambda x. x \; x}
		}
		 & \rightarrow
		\prths{\lambda u. \lambda v. v}
		\prths{
			\prths{\lambda x. x \; x}
			\prths{\lambda x. x \; x}
		}                           \\
		 & \rightarrow \cdots
	\end{align*}
	%
	Only the first gives an expression in the normal form.
	%
	\item
	%
	\ul{The normal-order reduction} is a particular way of using the ``Contextual closue'' rule.
	%
	It picks a redex\footnote{reducible expression} ($\beta$-redex) in an
	expression $e$ that is not included in any other redex.
	%
	Also, if there are multiple such \ul{outermost} redexes, it picks the
	\ul{leftmost} one.
	%
	In our example, the normal-order reduction doesn't pick the redex
	%
	$\prths{\lambda x. x \; x} \prths{\lambda x. x \; x}$
	%
	because it is included in the redex
	%
	$\prths{\lambda u. \lambda v. v} \prths{\prths{\lambda x. x \; x} \prths{\lambda x. x \; x}}$.
	%
	The normal-order reduction is also called \ul{outermost} \ul{leftmost}
	reduction.
	%
	\begin{property}
		%
		If $e \rightarrow^* e^\prime$ for some normal-form $e^\prime$,
		%
		then $e \rightarrow^*_\text{normal} e^\prime$
		%
		where $\rightarrow^*_\text{normal}$ means the contraction relation of the
		normal-order reduction.
		%
	\end{property}
	%
\end{enumcirc}

\section{Normal-order evaluation and eager evaluation}

\begin{enumcirc}
	%
	\item
	%
	In functional languages, we use restricted versions of the reduction relation
	to specify how function calls should be handled.
	%
	We will look at two well-known restrictions used in Haskell and Ocaml, and call
	them normal-order evaluation and eager evaluation.
	%
	Note that we use the world ``evaluation'' instead of ``reduction'' or
	``contraction''.
	%
	\item
	%
	Both normal-order evaluation and eager evaluations are defined for closed
	expressions only, i.e., expressions that do not have any free variables.
	%
	Also, they are formalized as big-step semantics where the evaluation relation
	$\Rightarrow$ transforms an expression to a final result in one go, instead of
	in multiple steps in the reduction relation.
	%
	Finally, these evaluations do not contract any subexpressions inside lambda.
	%
	Thus, their results might not be normal forms.
	%
	They will instead be a \ul{canonical form}.
	%
	\item
	%
	Normal-order evaluation $\Rightarrow$:
	%
	\[
		e\footnotemark \Rightarrow z\footnotemark
	\]
	\footnoteeqn[-1]{closed expression}
	\footnoteeqn{expression in the canonical form (i.e., lambda expression)}
	%
	\ul{A canonical form $z$} is a lambda expression.

	Canonical Forms
	%
	\[
		\inferrule{\;}{\lambda v. e \Rightarrow \lambda v. e}
	\]

	Application ($\beta$-evaluation)
	%
	\[
		\inferrule
		{e \Rightarrow \lambda v . e'' \\ \prths{\subst{e''}{v}{e'}} \Rightarrow z}
		{e \; e' \Rightarrow z}
	\]
	%
	\begin{property}
		%
		For any closed expression $e$ and canonical form $z$,
		%
		$e \rightarrow^* z$ iff $e \Rightarrow z$.
		%
	\end{property}
	%
	\begin{exercise}
		\todo
	\end{exercise}
	%
	\item
	%
	Intuitively, the normal-order evaluation works by postponing the evaluation of
	the arguments of a function.
	%
	The arguments get evaluated when they are needed.
	%
	However, this evaluation strategy may be inefficient because it may repeat the
	evaluation of one argument.
	%
	For instance, look at the following example.
	%
	\[
		\prths{\lambda x. x \; x}
		\prths{
			\prths{\lambda y. y}
			\prths{\lambda z. z}
		}
	\]
	%
	In the normal-order evaluation, the redex
	%
	$\prths{\lambda y. y} \prths{\lambda z. z}$
	%
	gets contracted twice; because of the two occurrences of $x$ in the body of
	%
	$\lambda x. x \; x$.
	%
	\item
	%
	The eager evaluation takes a different approach.
	%
	It evaluates the arguments of a function before applying the function.
	%
	Most programming languages implement this eager evaluation strategy.
	%
	\item
	%
	Eager evaluation (formalized by) $\RightarrowE$:
	%
	\[
		e\footnotemark \RightarrowE z\footnotemark
	\]
	\footnoteeqn[-1]{closed expression}
	\footnoteeqn{canonical form}

	Canonical Forms
	%
	\[
		\inferrule{\;}{\lambda v. e \RightarrowE \lambda v. e}
	\]

	Application ($\beta_\textrm{E}$-evaluation)
	%
	\[
		\inferrule
		{
			e \RightarrowE \lambda v . e'' \\
			e' \RightarrowE z' \\
			\prths{\subst{e''}{v}{z'}} \RightarrowE z}
		{e \; e' \Rightarrow z}
	\]
	%
	\begin{exercise}
		\todo
	\end{exercise}
	%
\end{enumcirc}

\section{Deotational semantics}

\begin{enumcirc}
	%
	\item
	%
	Interpreting the lambda calculus denotationally is not easy.
	%
	It was one of the longstanding open problems in the 1960's, until Scott solved
	it using the techniques from the domain theory (which Scott himself developed
	partly for this purpose).
	%
	\item
	%
	To understand why it was an open problem, let's try to interpret expressions in
	the lambda calculus (denotationally) using sets and functions.
	%
	This trial will fail as we explain shortly and will show the challenge of
	handling the fact that in the lambda calculus, functions can cat on themselves.

	The first thing that we should do (in this trial) is to find an appropriate
	space $S$ (which is this case is a set) for the meanings of the expressions.
	%
	Let's suppose that we somehow managed to find such $S$.
	%
	Then, $S$ should include all functions on $S$ that may be denoted by
	expressions in the lambda calculus.
	%
	Let's make a slightly stronger but nicer assumption that
	%
	\[
		\brackets{S \rightarrow S}\footnotemark \subseteq S
	\]
	\footnoteeqn[0]{set of all functions on $S$}
	%
	we will now show that $S$ is the singleton set.
	%
	This is because the assumption implies that every function $f$ on $S$ has a
	fixed point, which can happen only if $S$ is a singleton set.
	%
	What fixed point does a function $f \in \brackets{S \rightarrow S}$ have?

	Here is an answer for the question.
	%
	Let $p$ a function on $S$ s.t. $p\prths{x} = f\prths{x\prths{x}}$ for all
	%
	$x \in \brackets{S \rightarrow S} \subseteq S$.
	%
	Such $p$ exists.
	%
	Then $p\prths{p} = f\prths{p\prths{p}}$.
	%
	So, $p\prths{p}$ is a fixed point of $f$.

	We have just shown that $\brackets{S \rightarrow S} \notin S$ for any set $S$
	that contains more than one element.
	%
	\item
	%
	What should we do?
	%
	We need to use the domain theory and the categorical tools, in particular
	general categorical fixed-point theorem and the category
	$\textrm{DOM}^\textrm{EP}$, which consists of domains and a particular kind of
	strict conditions functions called embeddings.
	%
	If you are curious about these, look at the notes on
	%
	``5. Famous Example of the Fixed point Theorem'' (6 Nov 2018).
	%
	Using these tools, we can find domains $D_1, D_2, D_3$ and $V_2, V_3$ s.t.
	%
	\begin{enumrm}
		%
		\item
		%
		\[
			D_1\footnotemark \simeq\footnotemark \brackets{D_1 \toc\footnotemark D_1}
		\]
		\footnoteeqn[-2]{also denoted by $D_\infty$ in other textbooks}
		\footnoteeqn{isomorphism between domains}
		\footnoteeqn{
			\begin{minipage}{0.9\textwidth}
				continuous functions. We often omit $c$ but here we wrote it explicitly to
				emphasize the fact that we are considering continuous functions only here.
			\end{minipage}
		}
		%
		\item
		%
		\begin{align*}
			D_2 & \simeq \brackets{D_2 \toc D_2} \times \brackets{D_2 \toc D_2} \\
			D_3 & \defeq \prths{V_3}_\bot                                       \\
		\end{align*}
		%
		\item
		%
		\begin{align*}
			V_2 & \defeq \brackets{D_2 \toc D_2}              \\
			V_3 & \simeq \brackets{V_3 \toc \prths{V_3}_\bot} \\
		\end{align*}
		%
	\end{enumrm}
	%
	\item
	%
	Note that $D_1$, $D_2$ and $V_3$ are solutions of slightly different recursive
	domain equations.
	%
	Why do we consider three such equations, instead of one?
	%
	It is because the contraction relation, the normal-order evaluation and the
	eager evaluation provide three different meanings to the expressions in the
	lambda calculus,
	%
	and these equations capture these differences.
	%
	$D_1$ is for the contraction relation, $D_2$ and $V_2$ are for the normal-order
	evaluation, and $D_3$ and $V_3$ are for the eager evaluation.
	%
	\item
	%
	As before, understanding these domains is the key to understand the
	denotational semantics of the lambda calculus under three different notions of
	computations:
	%
	\begin{enumrm}
		%
		\item
		%
		the contraction relation
		%
		\item
		%
		the normal-order evaluation
		%
		\item
		%
		the eager evaluation
		%
	\end{enumrm}
	%
	The equations for $\prths{D_2, V_2} and \prths{D_3, V_3}$ indicate that under
	(ii) and (iii), we need to differentiate expressions that don't terminate and
	those that do and become lambda expressions.
	%
	$D_2$ and $D_3$ are domains for all expressions, and $V_2$ and $V_3$ are domains
	for the latter kind of expressions.
	%
	Sometimes $V_2$ and $V_3$ are called \ul{domains for values}, and $D_2$ and
	$D_3$ are called \ul{domains for computations}.
	%
	Note that we don't make this kind of distinction for $D_1$.
	%
	For instance, in $D_2$ and $D_3$, $\bot$ is different from the constant
	function that always return $\bot$. (i.e. $\lambda x. \bot$)
	%
	On the other hand, in $D_1$, they are regarded the same.
	%
	(In $D_1$, $\lambda x . \bot$ is the least element up to the isomorphism $D_1
		\simeq \brackets{D_1 \toc D_1}$.)

	Why is $D_1$'s way of defining $\bot$ different from $D_2$ and $D_3$'s?
	%
	Because the normal-order evaluation and the eager evaluation are do not reduce
	subexpressions under the lambda abstraction ($\lambda x. e$), whereas the
	contraction relation do reduce such subexpressions.
	%
	\item
	%
	Now let's try to understand the difference between $\prths{D_2, V_2}$ and
	$\prths{D_3, V_3}$.
	%
	Rewriting isomorphisms and definitions slightly can help us to see this
	difference more easily.
	%
	\begin{align*}
		D_2 & \defeq \prths{V_2}_\bot \qquad V_2 \simeq \prths{\brackets{D_2 \toc D_2}}_\bot \\
		D_3 & \defeq \prths{V_3}_\bot \qquad V_3 \simeq \brackets{V_3 \toc D_3}              \\
	\end{align*}
	%
	The key difference lies in the fact that $V_2$ has a function space
	%
	$\brackets{\underline{D_2} \toc D_2}$,
	%
	whereas $V_3$ has a function space
	%
	$\brackets{\underline{V_3} \toc D_3}$.
	%
	The argument domain for the normal-order evaluation is that for computations,
	while the argument domain for the eager evaluation is that for values.
	%
	This difference comes from the fact that in the eager evaluation, we pass only
	canonical forms (which are lambda expressions) to functions as arguments, while
	in the normal-order evaluation, we pass any expressions as function arguments.
	%
	So, when we use the normal-order evaluation, variables may be bound to
	expressions that may denote any computations.
	%
	But if we use the eager evaluation, variables should be bound to lambda
	expressions or more generally canonical forms that denote values.
	%
	\item
	%
	Here is the denotational semantics for the contraction relation.
	%
	\[
		D_1 \lrsupsubarrow{\phi}{\psi} \brackets{D_1 \toc D_1}
	\]
	%
	\[
		\begin{array}{c}
			\bbrackets{-} \in
			\brackets{ \chevrons{exp} \rightarrow \brackets{ D_1^{\chevrons{var}}\footnotemark \toc D_1 } }
			\qquad \eta \in D_1^{\chevrons{var}} \dots \textrm{called \ul{environment}}                 \\[1em]
			\bbrackets{v}\eta = \eta\prths{v} \qquad
			\bbrackets{e_1 \; e_2}\eta = \phi\prths{\bbrackets{e_1}\eta} \prths{\bbrackets{e_2}\eta}    \\
			\bbrackets{\lambda x. e}\eta = \psi\prths{\lambda a \in D_1 . \bbrackets{e}\aug{\eta}{x:a}} \\
		\end{array}
	\]
	\footnoteeqn[0]{$D_1^{\chevrons{var}}=\brackets{\chevrons{var} \rightarrow D_1}$ ordered pointwise}
	%
	\begin{property}[Textbook 10.8]
		%
		\;\\
		%
		Well-defined. That is,
		%
		$\lambda a \in D_1 . \bbrackets{e}\aug{\eta}{x}{a}$
		%
		is continuous.
		%
	\end{property}
	%
	\begin{property}[Textbook 10.12 and 10.13]
		%
		\[
			\textrm{If } e_0 \rightarrow e_1, \textrm{ then }
			\bbrackets{e_0}\eta = \bbrackets{e_1}\eta \textrm{ for all } \eta \in D_1^{\chevrons{var}}.
		\]
	\end{property}
	%
	The contraction relation preserves the semantics.
	%
	Note that this implies that the reduction relation $\rightarrow^*$ also
	preserves the semantics.
	%
	\item
	%
	Here are the denotational semantics for the normal-order evaluation and that
	for the eager evaluation.

	\ul{normal-order evaluation:}
	%
	\[
		\begin{array}{c}
			\displaystyle
			D_2 = \prths{V_2}_\bot \qquad
			V_2 \lrsupsubarrow{\phi}{\psi} \brackets{D_2 \toc D_2} \times \brackets{V_2 \toc V_2}                                                                      \\[1em]
			\normalsem{-} \in \brackets{\chevrons{exp} \rightarrow \brackets{ D_2^{\chevrons{var}} \toc D_2 }}                                                         \\[1em]
			\normalsem{v}\eta = \eta\prths{v}                                                                                                                          \\
			\normalsem{e_0 \; e_1}\eta = \begin{cases}
				                             \bot                                                       & \quad \textrm{if} \quad \normalsem{e_0}\eta = \bot \\
				                             \phi\prths{\normalsem{e_0}\eta}\prths{\normalsem{e_1}\eta} & \quad \textrm{otherwise}                           \\
			                             \end{cases} \\[2em]
			\normalsem{\lambda x. e}\eta = \psi\prths{\lambda a \in D_2 . \normalsem{e}\aug{\eta}{x:a}}                                                                \\
		\end{array}
	\]

	\ul{eager evaluation:}
	%
	\[
		\begin{array}{c}
			\displaystyle
			D_3 = \prths{V_3}_\bot \qquad
			V_3 \lrsupsubarrow{\phi}{\psi} \brackets{V_3 \toc D_3}                                                                                \\[1em]
			\eagersem{-} \in \brackets{\chevrons{exp} \rightarrow \brackets{ V_3^{\chevrons{var}} \toc D_3 }}                                     \\[1em]
			\eagersem{v}\eta = \eta\prths{v}                                                                                                      \\
			\eagersem{e_0 \; e_1}\eta = \begin{cases}
				                            \bot                                                     &
				                            \quad \textrm{if} \quad \eagersem{e_0}\eta = \bot \textrm{ or } \eagersem{e_1}\eta = \bot \\
				                            \phi\prths{\eagersem{e_0}\eta}\prths{\eagersem{e_1}\eta} &
				                            \quad \textrm{otherwise}                                                                  \\
			                            \end{cases} \\[2em]
			\eagersem{\lambda x. e}\eta = \psi\prths{\lambda a \in V_3 . \eagersem{e}\aug{\eta}{x:a}}                                             \\
		\end{array}
	\]

	Both semantics are well-defined.
	%
	They validate the normal-order evaluation relation and the eager evaluation
	relation.
	%
	That is,
	%
	\begin{align*}
		\prths{e \Rightarrow e'} \quad  & \Rightarrow\quad \normalsem{e}\eta = \normalsem{e'}\eta \textrm{ for all } \eta \in D_2^{\chevrons{var}} \\
		\prths{e \RightarrowE e'} \quad & \Rightarrow\quad \eagersem{e}\eta = \eagersem{e'}\eta \textrm{ for all } \eta \in V_3^{\chevrons{var}}   \\
	\end{align*}
\end{enumcirc}
