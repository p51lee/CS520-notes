% \documentclass[12pt]{book}
\documentclass[ebook,9pt,twoside,openany]{memoir}

\usepackage[paperwidth=6in, paperheight=9in, margin=0.5in]{geometry}
% \usepackage[paperwidth=148.5mm, paperheight=210mm, margin=10mm]{geometry}
\usepackage[english]{babel}
% \usepackage{graphicx}
% \usepackage{framed}
% \usepackage[normalem]{ulem}
\usepackage{amsmath}
\usepackage{amsthm}
\usepackage{amssymb}
\usepackage{amsfonts}
\usepackage{enumerate}
% \usepackage{kotex}
% \usepackage{setspace}
\usepackage{listings}
% \usepackage{xcolor}
% \usepackage{bbm}
% \usepackage{lipsum}
\usepackage{tikz}
\usepackage{enumitem}
\usepackage{syntax}
\usepackage{changepage}
% \usepackage{mathrsfs}
\usepackage{stmaryrd}
\usepackage[perpage]{footmisc}
\usepackage{mathpartir}
\usepackage{soul}
\usepackage{pifont}
\usepackage{tikz-cd}
\usepackage{hyperref}
\usepackage[capitalise,nameinlink]{cleveref}
\usepackage{lmodern}
\delimitershortfall-1sp
% \usepackage{mleftright}
% \mleftright

\crefformat{section}{#2\S{}#1#3}

\newtheoremstyle{tabstyle}
  {\topsep} % Space above
  {\topsep} % Space below
  {\normalfont} % Body font
  {} % Indent amount
  {\bfseries} % Theorem head font
  {.} % Punctuation after theorem head
  {.5em} % Space after theorem head
  {} % Theorem head spec (can be left empty, meaning ‘normal’)

\theoremstyle{definition}
\newtheorem{theorem}{Theorem}[chapter]
\newtheorem*{corollary}{Corollary}
\newtheorem{property}{Property}[chapter]
\newtheorem{lemma}{Lemma}[chapter]
\newtheorem*{definition}{Definition}
\newtheorem{note}{Note}

\theoremstyle{tabstyle}
\newtheorem{exercise}{Exercise}[chapter]
\newtheorem*{example}{Example}
\newenvironment{exampletab}
  {\begin{example}\;\begin{adjustwidth}{1em}{0em}}
  {\end{adjustwidth}\end{example}}
\newenvironment{exercisetab}
  {\begin{exercise}\;\begin{adjustwidth}{1em}{0em}}
  {\end{adjustwidth}\end{exercise}}

\newenvironment{enumarab}
  {\begin{enumerate}[label=(\arabic*)]}
  {\end{enumerate}}
\newenvironment{enumcirc}
  {\begin{enumerate}[label=\protect\circled{\arabic*}]}
  {\end{enumerate}}
\newenvironment{enumrm}
  {\begin{enumerate}[label=(\roman*)]}
  {\end{enumerate}}
\newenvironment{enumalpha}
  {\begin{enumerate}[label=(\alph*)]}
  {\end{enumerate}}


\newcommand*\circled[1]{\tikz[baseline=(char.base)]{
            \node[shape=circle,draw,inner sep=2pt] (char) {#1};}}

\newcommand{\inred}{\color{red}}
\newcommand{\todo}{{\inred TODO}}
\newcommand{\<}{\langle}
\renewcommand{\>}{\rangle}
\newcommand{\abs}[1]{\left\vert #1 \right\vert}
\newcommand{\prths}[1]{\left( #1 \right)}
\newcommand{\braces}[1]{\left\{ #1 \right\}}
\newcommand{\brackets}[1]{\left[ \, #1 \, \right]}
\newcommand{\bbrackets}[1]{\left\llbracket #1 \right\rrbracket}
\newcommand{\chevrons}[1]{\left\< #1 \right\>}
\newcommand{\set}[2]{\left\{ #1 \; \middle\vert \; #2 \right\}}
\newcommand{\aug}[2]{\left[ #1 \; \middle\vert \; #2 \right]}
\newcommand{\augtwo}[3]{\left[ #1 \; \middle\vert \; #2 \; \middle\vert \; #3 \right]}
\newcommand{\subsctext}[2]{#1_{\textrm{\tiny #2}}}
\newcommand{\subscmath}[2]{#1_{\textrm{\tiny \ensuremath{{}_{#2}}}}}
\newcommand{\supsctext}[2]{#1^{\textrm{\tiny #2}}}
\newcommand{\supscmath}[2]{#1^{\textrm{\tiny \ensuremath{{}_{#2}}}}}

\newcommand{\intexpsem}[1]{\bbrackets{#1}_{\textrm{\tiny intexp}}}
\newcommand{\boolexpsem}[1]{\bbrackets{#1}_{\textrm{\tiny boolexp}}}
\newcommand{\commexpsem}[1]{\bbrackets{#1}_{\textrm{\tiny comm}}}
\newcommand{\assertsem}[1]{\bbrackets{#1}_{\textrm{\tiny assert}}}
\newcommand{\normalsem}[1]{\bbrackets{#1}_{\textrm{\tiny n}}}
\newcommand{\eagersem}[1]{\bbrackets{#1}_{\textrm{\tiny e}}}

\newcommand{\gram}[1]{\chevrons{\textit{#1}}}
\newcommand{\doublebot}{%
  {\bot \mkern-11mu \bot}%
}

\newcommand{\Z}{\mathbb{Z}}
\newcommand{\B}{\mathbb{B}}
\newcommand{\R}{\mathbb{R}}
\newcommand{\N}{\mathbb{N}}
\newcommand{\Cc}{\mathcal{C}}
\newcommand{\Oc}{\mathcal{O}}
\newcommand{\Dc}{\mathcal{D}}
\newcommand{\true}{\textrm{true}}
\newcommand{\false}{\textrm{false}}
\newcommand{\ttt}{\textbf{tt}}
\newcommand{\fff}{\textbf{ff}}
\newcommand{\fv}[1]{\textrm{FV}\prths{ #1 }}
\newcommand{\fa}[1]{\textrm{FA}\prths{ #1 }}
\newcommand{\fvintexp}[1]{\textrm{FV}_{\textrm{\tiny intexp}}\prths{ #1 }}
\newcommand{\fvassert}[1]{\textrm{FV}_{\textrm{\tiny assert}}\prths{ #1 }}
\newcommand{\where}{\quad \textrm{ where } \;}
\newcommand{\elemarith}{\substack{+\\-\\\times\\\div}}
\newcommand{\plustimes}{\substack{+\\ \times}}
\newcommand{\twostack}[2]{\substack{#1\\#2}}
\newcommand{\toc}{\xrightarrow[c]{}}
\newcommand{\abort}{\textrm{abort}}

\newcommand{\cskip}{\textrm{skip}}
\newcommand{\cseq}[2]{#1 ;\; #2}
\newcommand{\cif}[3]{\textrm{if } #1 \textrm{ then } #2 \textrm{ else } #3}
\newcommand{\cwhile}[2]{\textrm{while } #1 \textrm{ do } #2}
\newcommand{\cassign}[2]{#1 := #2}
\newcommand{\cfail}{\textrm{fail}}
\newcommand{\cout}[1]{\textrm{!} #1}
\newcommand{\cin}[1]{\textrm{?} #1}
\newcommand{\cnew}[3]{\textrm{newvar } #1 := #2 \textrm{ in } #3}
\newcommand{\cfor}[4]{\textrm{for } #1 := #2 \textrm{ to } #3 \textrm{ do } #4}

\newcommand{\lletrec}[4]{\textrm{letrec } #1 \equiv \lambda #2 . #3 \textrm{ in } #4}

\newcommand{\Obj}[1][]{\subscmath{\textrm{Obj}}{#1}}
\newcommand{\Hom}{\textrm{Hom}}
\newcommand{\id}{\textrm{id}}
\newcommand{\Id}{\textrm{Id}}
\newcommand{\Homset}[3][]{\subscmath{\Hom}{#1}\brackets{#2, #3}}
\newcommand{\inj}{\textrm{inj}}
\newcommand{\Dom}[1][]{{\textrm{Dom}}^{\textrm{#1}}}

\newcommand{\ibot}{\iota_\bot}
\newcommand{\iterm}{\iota_{\textrm{term}}}
\newcommand{\iabort}{\iota_{\textrm{abort}}}
\newcommand{\iout}{\iota_{\textrm{out}}}
\newcommand{\iin}{\iota_{\textrm{in}}}
\newcommand{\botO}{\bot_\Omega}

\newcommand{\defeq}{\stackrel{\text{def}}{=}}
\newcommand{\subst}[3]{#1/_{#2 \rightarrow #3}}
\newcommand{\RightarrowE}{\Rightarrow_{\textrm{E}}}
\newcommand{\lrsupsubarrow}[2]{ \mathrel{\mathop{\rightleftarrows}^{\mathrm{#1}}_{\mathrm{#2}}} }

\newcommand{\footnoteeqn}[2][1]{\addtocounter{footnote}{#1}\footnotetext{#2}}

\renewcommand{\thefootnote}{\fnsymbol{footnote}}

\title{\Huge Theory of Programming Languages}
\author{Hongseok Yang}
\date{\today}

\OnehalfSpacing
% \counterwithin*{footnote}{page}

\begin{document}

\maketitle

\tableofcontents

\chapter*{Preface}
\addcontentsline{toc}{chapter}{Preface} % If you want it in the table of contents

\noindent\textbf{Editor's Note:}

\noindent
This document is transcribed from Professor Yang's handwritten graduate
programming language lecture notes for improved readability and accessibility.

\medskip
\noindent Should you find any errors within these documents, contributions and
error reports are welcomed at the following GitHub repository:
\href{https://github.com/p51lee/CS520-notes}{github.com/p51lee/CS520-notes}.

\medskip
\noindent In this document, I will be using symbols \ding{168}, \ding{169},
\ding{170}, \ding{171}, \ding{72}, \ding{96}, and \ding{44} to express the
additional arrowed notes of the original lecture notes, as a footnote.

\noindent So don't be surprised or confused if you see a footnote symbol in the
middle of a sentence or an equation; just look down at the bottom of the page to
see what it means.
%
If you are using a PDF viewer, you can click on the footnote symbol to jump to
the footnote.

\DefineFNsymbols{footsym}{
	{\tiny\ding{168}}
		{\tiny\ding{169}}
		{\tiny\ding{170}}
		{\tiny\ding{171}}
		{\tiny\ding{72}}
		{\tiny\ding{96}}
		{\tiny\ding{44}}
}
\setfnsymbol{footsym}

% !TEX root = main.tex

\chapter{Predicate Logic}

\section{Motivation or objective}

\begin{enumerate}[label=\protect\circled{\arabic*}]
  \item
    Learn four key tools in PL that will be used throughout this course.
    \begin{enumerate}[label=(\roman*)]
      \item Abstract Syntax
      \item Denotational Semantics
      \item Inference Rule
      \item Binding
    \end{enumerate}
  \item Learn the basics of predicate logic (or first-order logic)
  \item We plan to go ghrough some of (i) - (iv) twice, first using integer
    expressions and then using predicate logic.
\end{enumerate}

\section{Integer expressions}

\begin{enumerate}[label=\protect\circled{\arabic*}]
  \item
    How to analyze integer expresion found in logic and programming languages
    mathematically? We will first have to define the syntax and teh semantics
    for them.
  \item
    Examples: $x + 3 \times y$ , $x \div 2 + x \times x$ ...
  \item
    We also want to develop mathematical tools to reason about or manipulate
    integer expressions.
\end{enumerate}

\section{Abstract syntax and initial algebra}

\begin{enumerate}[label=\protect\circled{\arabic*}]
  \item Abstract Syntax: \\
    Specification of \emph{abstract phrases}
    \footnote{
      vague words, but will be made rigorous when we define initial algebra.
    }
    in a formal language, such as the language of integer expressions and
    predicate logic.
  \item
    Typically, we use \emph{abstract grammar}
    \footnote{
      \begin{minipage}[t]{\textwidth}
      Here is the explanation of the word with an enumerated list:
      \begin{enumerate}[label=(\roman*)]
        \item grammar without any concern on parsing ofr surface syntax.
        \item In this case, parse trees in the grammar are abstract phrases.
      \end{enumerate}
      \end{minipage}
    }
    to describe abstract syntax.
  \item Abstract grammar for integer expressions:
    \setlength{\grammarindent}{6em} % increase separation between LHS/RHS
    \begin{center}
    \begin{minipage}{0.4\textwidth}
    \begin{grammar}
    <intexp> ::= 0 | 1 | 2 ...
    \alt <var>
    \alt - <intexp>
    \alt <intexp> $+$ <intexp>
    \alt <intexp> $-$ <intexp>
    \alt <intexp> $\times$ <intexp>
    \alt <intexp> $\div$ <intexp>
    \end{grammar}
    \end{minipage}
    \end{center}
    (abstract) integer expressions are finite deriviation trees in this grammar.
    For instance,

    \begin{center}
    \begin{tikzpicture}[
      level 1/.style={sibling distance=10mm, level distance=10mm},
      level 2/.style={sibling distance=10mm},
      level 3/.style={sibling distance=5mm}]

      \node {+}
        child {node {$\times$} }
        child {node {$\times$}
          child {node {3}}
          child {node {y}}
        };
    \end{tikzpicture}
    $\qquad$
    \begin{tikzpicture}[
      level 1/.style={sibling distance=20mm, level distance=10mm},
      level 2/.style={sibling distance=10mm},
      level 3/.style={sibling distance=5mm}]

      \node {+}
        child {node {$\div$}
          child {node {x}}
          child {node {2}}
        }
        child {node {$-$}
          child {node {x}}
          child {node {x}}
        };
    \end{tikzpicture}
    \end{center}

    Note that infinite trees are not included.

  \item
    A more accurate view is to view abstract syntax as an inital algebra. This
    view will help us to see shy we can define various operations on abstract
    phrases or integer expressions using syntax-directed definition.

  \item
    \emph{Algebra} $A$ $\cdots$ Set with operations and constraints. \\
    \emph{Signature} $S$ $\cdots$ Type of an algebra.

    \begin{exampletab}
      \begin{enumerate}[label=(\roman*)]
        \item TODO
        \item TODO
      \end{enumerate}
    \end{exampletab}

  \item
    \emph{Algebra homomorphism} $\cdots$ map between algebras that
    preserves constants and operations.
    \begin{align*}
      S &= \prths{
      t,\; c_1 : t,\; \cdots ,\; c_n : t ,\;
      \textrm{op}_1 : t \times \cdots \times t \rightarrow t ,\; \cdots
      \textrm{op}_m : t \times \cdots \times t \rightarrow t
      }
      \\
      A_0 &= \prths{
      \mathcal{U}_0\footnotemark,\;
      c_1^0 \in \mathcal{U} ,\;
      \cdots ,\;
      c_n^0 \in \mathcal{U} ,\;
      \textrm{op}_1^0 :
      \mathcal{U} \times \cdots \times \mathcal{U} \rightarrow \mathcal{U},\; \cdots
      \textrm{op}_m^0 :
      \mathcal{U} \times \cdots \times \mathcal{U} \rightarrow \mathcal{U}
      }
      \\
      A_1 &= \prths{
      \mathcal{U}_1,\;
      c_1^1 \in \mathcal{U} ,\;
      \cdots ,\;
      c_n^1 \in \mathcal{U} ,\;
      \textrm{op}_1^1 :
      \mathcal{U} \times \cdots \times \mathcal{U} \rightarrow \mathcal{U},\; \cdots
      \textrm{op}_m^1 :
      \mathcal{U} \times \cdots \times \mathcal{U} \rightarrow \mathcal{U}
      }
    \end{align*}
    \footnotetext{notation: $\abs{A_0}$}

    $f \in \mathcal{U}_0 \rightarrow \mathcal{U}_1$ is a \emph{homomorphism} if
    \begin{enumerate}[label=(\roman*)]
      \item $f \prths{c_i^0} = c_i^1$ for all $i$.
      \item $f \prths{\textrm{op}_i^0 \prths{x_1,\; \cdots ,\; x_k}} =
        \textrm{op}_i^1 \prths{x_1,\; \cdots ,\; x_k}$ for all $i$.
    \end{enumerate}

  \item
    \emph{Initial algebra of a signature} $S$
    \begin{enumerate}[label=(\roman*)]
      \item
        An algebra $A$ of the signature $S$ s.t. for all algebras $A^\prime$ of
        the same signature, there is a \emph{unique} homomorphism $f$ from $A$
        to $A^\prime$.
      \item
        $A_{\textrm{grammar}}$ is initial.
      \item
        Formally, an abstract syntax fixes a signature and it denotes an initial
        algebra of the signature. An abstract phrase is an element of that
        algebra.
    \end{enumerate}
    \begin{exercise}
      Prove that $A_{\textrm{grammar}}$ is indeed an initial algebra.
    \end{exercise}
    \begin{exercise}
      Let $A_0$ and $A_0$ be initial algebras of the same signature $S$. Then,
      there are homomorphisms $f \in \abs{A_0} \rightarrow \abs{A_1}$ and
      $g \in \abs{A_1} \rightarrow \abs{A_0}$ s.t. $f \circ g = \textrm{id}$ and
      $g \circ f = \textrm{id}$.\\
      This means that all initial algebrras of $S$ are essentially the same,
      i.e. isomorphic. Prove this fact.
    \end{exercise}

\end{enumerate}

\section{Syntax-directed definition and denotational semantics}
\begin{enumerate}[label=\protect\circled{\arabic*}]
  \item
    Definition of a map on integer expressions using a form of induction and
    case analysis.
  \item
    $
    \textrm{FV (Free Variables)}: \chevrons{intexp} \rightarrow 2^{\chevrons{Var}}
    $
    \begin{align*}
      \fv{e\footnotemark} &= V \footnotemark \\
      \fv{c\footnotemark} &= \phi \\
      \fv{x\footnotemark} &= \braces{x} \\
      \fv{-e} &= \fv{e} \\
      \fv{e_1 \; \substack{+\\-\\\times\\\div} \; e_2} &= \fv{e_1} \cup \fv{e_2}
    \end{align*}
    \addtocounter{footnote}{-3} \footnotetext{integer expression}
    \stepcounter{footnote} \footnotetext{Set of free variables in $e$}
    \stepcounter{footnote} \footnotetext{constant}
    \stepcounter{footnote} \footnotetext{variable} 

  \item
    Two features: case analysis, recursive calls on subphrases.

  \item
    $
      \bbrackets{-} \in \chevrons{intexp} \; \rightarrow \;
      \Sigma \rightarrow \z
      \where \Sigma = \chevrons{var} \rightarrow \z , \textrm{ a set of states }
      \sigma.
    $
    \begin{align*}
      \bbrackets{c}\sigma &= c \\
      \bbrackets{x}\sigma &= \sigma\prths{x} \\
      \bbrackets{-e}\sigma &= -\prths{\bbrackets{e}\sigma} \\
      \bbrackets{
        e_1 \;\substack{+\\-\\\times\\\div}\; e_2
      }\sigma &= \prths{\bbrackets{e_1}\sigma}
      \;\substack{+\\-\\\times\\\div}\;
      \prths{\bbrackets{e_1}\sigma} \footnotemark
    \end{align*}
    \footnotetext{some special treatment of the divide-by-zero case}
    Intuitively, $\bbrackets{-}$ maps tress to mathematical functions in a
    synatax-directed (also called compositional) way. Such a compositional
    mapping from syntactic entities to mathematical entities is called
    \emph{denotational semantics}.

\end{enumerate}


\chapter{The Simple Imperative Language}

\section{Motivation or goal}

\begin{enumcirc}
	%
	\item
	%
	Most real-worlds PLs support computation by state update and that by function
	application.
	%
	The former is often referred to as imperative computation, and the latter as
	functional or applicative computation
	%
	Our goal is to study core PL concepts and ideas for imperative computation.
	%
	\item
	%
	Actually, it is more appropriate to say that our aim is a formal and
	mathematical analysis of the core PL concepts for imperative computation.
	%
	We will study or learn mathematical tools for this.
	%
	Also, we will show how to express and analyze big design decisions of such an
	imperative PLs.
	%
	\item
	%
	We will look at (i) some basic concepts and results of domain theory, (ii)
	variable declaration and binding, (iii) syntactic sugar error handling and (iv)
	the notions of soundness and full abstraction.%
	%
	(Well, I just listed all the key items in the chapter 2 of the book.)
	%
	\item
	%
	A good way to learn the material of this chapter is to ask yourself:
	%
	What should you do in order to design an imperative programming language and
	build a foundation of the designed language?
	%
	Think about this a little, and compare your answer with what I'll explain.
	%
\end{enumcirc}

\section{Syntax}

\begin{enumcirc}
	%
	\item
	%
	Variables, and read and update of them \dots key concepts or operations for
	imperative computation.
	%
	\item
	%
	Syntax that supports these concepts and operations:
	%
	\begin{center}
		\begin{minipage}{0.9\textwidth}
			\begin{grammar}
				<intexp> ::= 0 | 1 | 2 | \dots | <var> | ... \footnotemark

				<boolexp> ::= true | false | ... \footnotemark

				<comm> ::= <var> := <intexp> \footnotemark | skip | <comm> ; <comm> \footnotemark
				\alt if <boolexp> then <comm> else <comm>
				\alt while <boolexp> do <comm>
			\end{grammar}
		\end{minipage}
	\end{center}
	\footnoteeqn[-3]{same as before}
	\footnoteeqn{
		\begin{minipage}{0.9\textwidth}
			%
			almost the same as that of $\chevrons{\textit{assert}}$.
			%
			The exception is that $\chevrons{\textit{boolexp}}$ doesn't include
			quantifiers. (Why?)
			%
		\end{minipage}
	}
	\footnoteeqn{update of a variable}
	\footnoteeqn{order matters}

	As in the case of predicate logic, you can understand $\gram{comm}$ as the set
	of all finite derivation trees, or as a multi-sorted initial algebra for the
	signature determined by the grammar.
	%
	\item
	%
	It is a language for expressing a sequence of variable reads and variable
	updates.
	%
\end{enumcirc}

\section{Baby domain theory}

\begin{enumcirc}
	%
	\item
	%
	Giving a denotational semantics to our simple imperative language is not as
	straightforward as doing so with predicate logic, because of loop.
	%
	\item
	%
	$\bbrackets{-}$ : $\gram{comm} \to \dots$\;\footnote{don't worry about this target set now.}

	We want the following equation for loop unrolling to hold:
	%
	\begin{align*}
		\bbrackets{\cwhile{b}{c}} & = \bbrackets{\cif{b}{c ; \cwhile{b}{c}}{\cskip}}             \\
		                          & = \dots \bbrackets{\cwhile{b}{c}} \dots                      \\
		                          & = F \prths{\bbrackets{\cwhile{b}{c}}} \textrm{ for some } F.
	\end{align*}
	%
	But a function $F$ on a set may or may not have such a fixed point.
	%
	\item
	%
	Then, why should $F$ in the above have a fixed point?
	%
	Because it is something that can be implemented by a program.
	%
	\[
		F \; ``=" \footnotemark \bbrackets{\cif{b}{c ; \Box}{\cskip}}.
	\]
	\footnoteeqn[0]{informally}
	%
	One objective of domain theory is to formalize good properties enjoyed by such
	program implementable functions without going into the low-level details of
	computability theory.
	%
	\item
	%
	High-level meta heuristic behind domain theory:
	%
	\begin{enumrm}
		%
		\item
		%
		Consider a set together with some structure.
		%
		\item
		%
		Use functions between such sets that respect the structures.
		%
		\item
		%
		Why on (i) and (ii)?
		%
		Because if done well, functions in (ii) will always have fixed points.
		%
	\end{enumrm}
	%
	\item
	%
	Key definitions:
	%
	\begin{definition}[partial order]
		%
		A binary relation $\sqsubseteq$ on a set $S$ is a \ul{partial order} if
		%
		\begin{enumrm}
			%
			\item
			%
			$x \sqsubseteq x$ for all $x \in S$ (reflexivity);
			%
			\item
			%
			for all $x, y, z \in S$, if $x \sqsubseteq y$ and $y \sqsubseteq z$,
			%
			then $x \sqsubseteq z$ (transitivity); and
			%
			\item
			%
			for all $x, y \in S$, if $x \sqsubseteq y$ and $y \sqsubseteq x$, then $x = y$
			(anti-symmetry).
			%
		\end{enumrm}
		%
		A set $S$ with a partial order $\sqsubseteq$ is called a
		%
		\ul{partially ordered set} or \ul{poset}.
		%
	\end{definition}
	%
	\begin{definition}
		%
		A \ul{chain} in a poset $\prths{S, \sqsubseteq}$ is a (countably) infinite
		sequence
		%
		\[
			x_0, x_1, x_2, \dots , x_n, \dots
		\]
		%
		of elements in $S$ s.t. $x_n \sqsubseteq x_{n+1}$ for all $n \geq 0$.
	\end{definition}
	%
	\begin{definition}
		A \ul{pre-domain} is a poset $\prths{S, \sqsubseteq}$ such that \ul{all
			chains have least upper bounds}.
		%
		That is, for every chain $\braces{x_n}_{n \geq 0}$ in $S$, there exists $y$
		%
		\footnote{notation for $y$: $\displaystyle\bigsqcup_{n \geq 0} x_n$}
		%
		in $S$ s.t.
		%
		\begin{enumrm}
			%
			\item
			%
			\footnote{$y$ is an upper bound}
			%
			$x_n \sqsubseteq y$ for all $n \geq 0$
			%
			\item
			%
			\footnote{$y$ is the least such}
			%
			for any $z$ in $S$, if $x_n \sqsubseteq z$ for every $n \geq 0$, then $y
				\sqsubseteq z$.
			%
		\end{enumrm}
		%
	\end{definition}
	%
	\begin{definition}
		%
		A \ul{domain} is a pre-domain $\prths{S, \sqsubseteq}$ that has
		%
		\ul{the least element}, often denoted $\bot$.
		%
		(meaning: for all $x \in S$, $\bot \sqsubseteq x$)
		%
	\end{definition}
	%
	\begin{definition}
		%
		Let $\prths{S_1, \sqsubseteq_1}$ and $\prths{S_2, \sqsubseteq_2}$ be
		pre-domains.
		%
		A function $f : S_1 \to S_2$ is \ul{continuous} if for every chain
		%
		$\braces{x_n}_{n \geq 0}$ in $S_1$,
		%
		$f \prths{\bigsqcup_{n \geq 0} x_n}$ is the least upper bound of
		%
		$\braces{f \prths{x_n}}_{n \geq 0}$ in $S_2$.
		%
	\end{definition}
	%
	\begin{definition}
		%
		A function $f : S_1 \to S_2$ is \ul{monotone} if for all $x, y \in S_1$,
		%
		\[
			x \sqsubseteq_1 y \implies f \prths{x} \sqsubseteq_2 f \prths{y}.
		\]
		%
		When $S_1$ and $S_2$ are domains with least elements $\bot_1$ and $\bot_2$, we
		say that a function
		%
		$f : S_1 \to S_2$ is \ul{strict} if
		%
		$f \prths{\bot_1} = \bot_2$.
		%
	\end{definition}
	%
	\begin{exercise} \label{ex:cont-mon}
		%
		Show that if $f$ is continuous, it is monotone.
		%
	\end{exercise}
	\item
	%
	What's going on here?
	%
	What are the intuitions behind these definitions?
	%
	\begin{enumrm}
		%
		\item
		%
		$x \sqsubseteq y$ \dots intuitively means that $y$ has more information than $x$ or
		$x$ and $y$ have the same amount of information.
		%
		\newpage
		%
		\begin{exampletab}
			%
			\begin{enumalpha}
				%
				\item
				%
				$\Z^* \cup \Z^\omega$ \dots finite or infinite sequence \footnote{
					results produced so far by sequence-producing computation.
				} of integers.
				%
				(without the infinite part, not a pre-domain)
				%
				$x \sqsubseteq y$ iff $x$ is an initial subsequence (or prefix) of $y$.
				%
				\begin{align*}
					\chevrons{3, 1, 4} & \sqsubseteq \chevrons{3, 1, 4, 1, 5, 9} \\
					\chevrons{3, 1, 4} & \not\sqsubseteq \chevrons{3, 1, 5, 9}   \\
				\end{align*}
				%
				\item
				%
				$\Z_{\bot} \defeq \Z \cup \braces{\bot}$ \dots lifted $\Z$.
				%
				\footnote{integer output of an integer-returning computation}
				%
				\[
					\forall x, y \in \Z_{\bot}, \quad x \sqsubseteq y \iff x = \bot \textrm{ or } x = y.
				\]
				%
				\item
				%
				$\prths{2^\Z, \subseteq}$ \dots power set of $\Z$.
				%
				\item
				%
				vertical domain of natural numbers \dots
				%
				$
					\left.
					\begin{array}{c}
						\infty =: \top \\
						\vdots         \\
						2              \\
						1              \\
						0 =: \bot
					\end{array}
					\right)
					=: \N^\top
				$
				%
			\end{enumalpha}
		\end{exampletab}
		%
		\item
		%
		The monotonicity means the preservation of the approximation
		relation.\footnote{ editor: if $a$ is an approximation of $b$, then $f(a)$ is
			an approximation of $f(b)$ }
		%
		One intuition behind continuity of $f$ is that in order to produce finite
		amount of information in its output, $f$ consumes only finite amount
		information in its input.
		%
		\begin{exampletab}
			%
			\begin{enumalpha}
				%
				\item
				%
				\begin{align*}
					\textrm{set}                                    & \in \brackets{Z^{*,\omega} \to 2^\Z} \\
					\textrm{set} \prths{\chevrons{x_1, \dots, x_n}} & = \braces{x_1, \dots, x_n}           \\
					\textrm{set} \prths{\chevrons{x_1, x_2, \dots}} & = \braces{x_1, x_2, \dots}           \\
				\end{align*}
				%
				If $A \subseteq \textrm{set}\prths{s}$ and $A$ is finite, then there is a
				finite prefix $s_0$ of $s$ (i.e., $s_0 \sqsubseteq s$) s.t. $A \subseteq
					\textrm{set}\prths{s_0}$.
				%
				\begin{exercisetab}
					%
					Show that if $f \in \brackets{Z^{*,\omega} \to 2^\Z}$ is continuous, it
					satisfies the above property.
					%
					Also, show that if $f \in \brackets{Z^{*,\omega} \to 2^\Z}$ is monotone and
					satisfies the property, then $f$ is continuous.
					%
				\end{exercisetab}
				%
				\item
				%
				$f \in \brackets{2^\Z \to \N^\top}$
				%
				\[
					f \prths{A} = \begin{cases}
						\abs{A} \textrm{ if } $A$ \textrm{ is finite } \\
						\top \textrm{ if } $A$ \textrm{ is infinite }
					\end{cases}
				\]
				%
				\begin{exercisetab}
					%
					A function from $2^\Z$ to a predomain $P$ is \ul{finitely generated} if for all
					$A \in 2^\Z$, $f \prths{A}$ is the least upper bound of
					%
					$\set{f\prths{A_0}}{A_0 \subseteq A \textrm{ and } A_0 \textrm{ is finite}}$.
					%
					Show that $f$ is continuous iff it is finitely generated.
					%
				\end{exercisetab}
				%
			\end{enumalpha}
			%
		\end{exampletab}
		%
		\item
		%
		John Reynolds' phrase in page 108: \\
		%
		\emph{
			``\dots Instead, it is based on the
			physical \footnote{ Domain theory attempts to capture this aspect of computation
			} limitations of communication: one cannot predict the future of input, nor
			receive an infinite amount of information in a finite amount of time, nor
			produce output except at finite times \dots''
		}
		%
	\end{enumrm}
	%
	\item
	%
	One important reason of doing domain theory is to have the following ``least
	fixed-point theorem'':
	%
	\begin{property}[Least Fixed-Point Theorem]
		%
		If $D$ is a domain and $f$ is a continuous function from $D$ to $D$,
		%
		\[
			x = \bigsqcup_{n = 0}^{\infty} f^n \prths{\bot \footnotemark}
		\]
		\footnoteeqn[0]{least element of $D$}
		%
		is the least fixed-point of $f$.
		%
		(That is, $f \prths{x} = x$ and for all $y \in D$ s.t. $f \prths{y} = y$, $x \sqsubseteq y$.)
		%
	\end{property}
	%
	\begin{proof}
		%
		By \cref{ex:cont-mon}, $f$ is monotone.
		%
		Using induction and $\bot$'s being the least element, we can show that
		%
		$\braces{f^n \prths{\bot}}_{n \geq 0}$ is a chain in $D$.
		%
		Since $D$ is a domain, the least upper bound
		%
		$\bigsqcup_{n = 0}^{\infty} f^n \prths{\bot}$ exists.
		%
		Furthermore, by the continuity of $f$,
		%
		\[
			f \prths{x} =
			f \prths{\bigsqcup_{n = 0}^{\infty} f^n \prths{\bot}} =
			\bigsqcup_{n = 0}^{\infty} f^{n + 1} \prths{\bot} =
			\bigsqcup_{n = 0}^{\infty} f^n \prths{\bot} =
			x.
		\]
		%
		$\therefore x$ is a fixed point of $f$.

		To show that x is a least such, consider a fixed point $y$ of $f$.
		%
		Then, by induction, we can show that $y$ is an upper bound of the chain
		%
		$\braces{f^n \prths{\bot}}_{n \geq 0}$.
		%
		$\therefore x \sqsubseteq y$.
		%
	\end{proof}
	%
	\item
	%
	When $P, P'$ are predomains, we write $\brackets{P \toc P'}$ for the set of
	continuous functions.
	%
	When $\brackets{P \toc P'}$\footnote{
		\begin{minipage}{0.9\textwidth}
			%
			Although we rush here, this function space construction is very important.
			%
			It lets domain theory be applicable to functional languages.
			%
		\end{minipage}
	} is given pointwise order
	$\sqsubseteq$,
	%
	\[
		f \sqsubseteq g \iff f \prths{x} \sqsubseteq_{P'} g \prths{x}
		\textrm{ for all } x \in P, \quad f, g \in \brackets{P \toc P'},
	\]
	it becomes a predomain where the limit of a chain $\braces{f_n}_{n \geq 0}$ is
	defined pointwise $x \mapsto \bigsqcup_{n = 0}^\infty f_n \prths{x}$.
	%
	Furthermore, if $P'$ is a domain with the least element $\bot'$, then
	$\brackets{P \toc P'}$ is also a domain with $x \mapsto \bot'$ as its least
	element.
	%
	\item
	%
	$D$ is a domain with the least element $\bot$.
	%
	Define $Y_D$ to be the following function from $\brackets{D \toc D}$ to $D$:
	%
	\[
		Y_D \prths{f} = \bigsqcup_{n = 0}^{\infty} f^n \prths{\bot}.
	\]
	%
	\begin{lemma}
		%
		$Y_D$ is continuous\footnote{%
			This means that the very act of computing a fixed point of a given function is continuous.%
		}.
		%
	\end{lemma}
	\begin{proof}
		%
		Exercise.
		%
	\end{proof}
	%
	\item
	%
	There are a lot of interesting results in domain theory, some of which we will
	cover later in the course.
	%
	Before finishing this mini review of domain theory, I want to explain the
	lifting construction.
	%
	\begin{enumrm}
		%
		\item
		%
		\begin{itemize}
			%
			\item
			      %
			      $P_\bot := P \cup \braces{\bot}$ for a predomain $P$.
			      %
			\item
			      %
			      $x \sqsubseteq_{P_\bot} y$ iff $x = \bot$ or $x, y \in P \textrm{ and } x \sqsubseteq_P y$
			      for $x, y \in P_\bot$.
			      %
			\item
			      %
			      Intuitively, we are adding the least element to $P$ and converting $P$ to a
			      domain.
			      %
		\end{itemize}
		%
		\item
		%
		$i_\uparrow \in \brackets{P \toc P_\bot}$, sometimes called \ul{unit}.
		%
		\[
			i_\uparrow \prths{x} = x \textrm{ for all } x \in P.
		\]
		%
		\item
		%
		For each $f \in \brackets{P \toc P'_\bot}$,
		%
		\begin{align*}
			f_\doublebot              & \in \brackets{P_\doublebot \toc P'_\doublebot} \\
			f_\doublebot \prths{\bot} & = \bot                                         \\
			f_\doublebot \prths{x}    & = f \prths{x} \textrm{ for all } x \in P.
		\end{align*}
		%
		sometimes called \ul{Kleisli extension}.
		%
		\item
		%
		Why should we care about (ii) and (iii)?
		%
		Because they allow us to compose continuous functions from $P$ to $P'_\bot$
		%
		\[
			f \in \brackets{P \toc P'_\bot} \; \textrm{ and } \;
			g \in \brackets{P' \toc P''_\bot} \implies
			g_\doublebot \circ f \in \brackets{P \toc P''_\bot}.
		\]
		%
		We can view $(-)_\doublebot \circ (-)$ as a new composition operator $\circ'$.
		%
		Then, $\circ'$ is associative and $i_\uparrow \circ' f = f = f \circ'
			i_\uparrow$.
		%
		This means that $\prths{-_\bot, i_\uparrow, -_\doublebot}$ gives rise to a
		monad on predomains.
		%
	\end{enumrm}
	%
\end{enumcirc}

\section{Denotational semantics of the simple imperative language}

\begin{enumcirc}
	%
	\item
	%
	Recall that $\Sigma = \gram{var} \to \Z$.
	%
	$\Sigma$ is a predomain when given the discrete order $\sqsubseteq$.
	%
	(That is, $x \sqsubseteq y$ iff $x = y$ for all $x, y \in \Sigma$.)
	%
	\begin{align*}
		\intexpsem{-}  & \in \gram{intexp} \to \brackets{\Sigma \to \Z} (same as before)  \\
		\boolexpsem{-} & \in \gram{boolexp} \to \brackets{\Sigma \to \B} (same as before) \\
		\commexpsem{-} & \in \gram{comm} \to \brackets{\Sigma \toc \Sigma_\bot}
	\end{align*}
	%
	\begin{align*}
		\bbrackets{\cassign{x}{e}} \sigma     & = \aug{\sigma}{x: \bbrackets{e} \sigma}                                                        \\
		\bbrackets{\cskip} \sigma             & = \sigma                                                                                       \\
		\bbrackets{\cseq{c_1}{c_2}} \sigma    & = \bbrackets{c_2}_\doublebot \prths{\bbrackets{c_1} \sigma}                                    \\
		\bbrackets{\cif{b}{c_1}{c_2}} \sigma  & = \cif{\bbrackets{b} \sigma}{\bbrackets{c_1} \sigma}{\bbrackets{c_2} \sigma}                   \\
		\bbrackets{\cwhile{b}{c}} \sigma      & = Y_{\Sigma_\bot} \prths{F}                                                                    \\
		\textrm{ where } F                    & \in \brackets{\Sigma \toc \Sigma_\bot} \toc \brackets{\Sigma \toc \Sigma_\bot}                 \\
		\textrm{and }F\prths{f}\prths{\sigma} & := \cif{\bbrackets{b} \sigma}{\prths{f_\doublebot \circ \bbrackets{c}} \prths{\sigma}}{\sigma}
	\end{align*}
	%
	Note: $Y_{\Sigma_\bot}$ and $\toc$ are where we get something by using domain
	theory.
	%
	\item
	%
	Why \ul{least} fixed point?
	%
	Because the least fixed point maps an input state to $\bot$
	%
	(denoting non-termination, hence, the absence of any information)
	%
	whenever the corresponding output state is not uniquely determined by the
	equation $F(f) = f$.
	%
	For example,
	%
	\begin{enumrm}
		%
		\item
		%
		least fixed point \dots
		%
		$\bbrackets{\cwhile{\textrm{true}}{\cskip}} \sigma = \bot$.
		%
		\item
		%
		non-least fixed point
		%
		\dots $\bbrackets{\cwhile{\textrm{true}}{\cskip}} \sigma = \sigma$.
		%
		\begin{align*}
			F\prths{f}\prths{\sigma} & = \cif{\bbrackets{\textrm{true}} \sigma}{\prths{f_\doublebot \circ \bbrackets{\cskip}} \prths{\sigma}}{\sigma} \\
			                         & = \prths{f_\doublebot \circ \bbrackets{\cskip}} \prths{\sigma}                                                 \\
			                         & = f_\doublebot \prths{\bbrackets{\cskip} \prths{\sigma}}                                                       \\
			                         & = f_\doublebot \prths{\sigma}                                                                                  \\
			                         & = f\prths{\sigma}                                                                                              \\
		\end{align*}
		%
		That is, $F(f) = f$.
		%
	\end{enumrm}
	%
	Later when we consider the correspondence between denotational semantics and
	operational semantics, we will answer this question more rigorously.
	%
	\item
	%
	Design decision of our language seen in the semantics:
	%
	\[
		\intexpsem{-} : \gram{intexp} \to \Sigma \to \Z
	\]
	%
	all integer expressions terminate and do not raise exceptions.
	%
	A similar remark applies to boolean expressions as well.
	%
	\item
	%
	Choosing the type of the semantics function such as
	%
	$\gram{intexp} \to \Sigma \to \Z$
	%
	is the most important step in defining the semantics.
	%
	It also clarifies certain major design decisions of the target programming
	language.
	%
\end{enumcirc}

\section{Variable declaration and substitution}

\begin{enumcirc}
	%
	\item
	%
	\begin{grammar}
		<comm> ::= ... | newvar <var> := <intexp> in <comm>
	\end{grammar}
	%
	This is a construct that doesn't increase the expressivity of the language but
	enables the programmers to combat the complexity of software by introducing the
	ideas of scope and local variables.
	%
	\item
	%
	\[
		\bbrackets{\cnew{v}{e}{c}} \sigma =
		\prths{
			\prths{
				\lambda \sigma' \in \Sigma \;.\; \aug{\sigma'}{v : \sigma v}
			}_\doublebot \circ \bbrackets{c}
		} \prths{
			\aug{\sigma}{v : \bbrackets{e} \sigma}
		}
	\]
	%
	\dots
	%
	$\lambda \sigma' \in \Sigma \;.\; \aug{\sigma'}{v : \sigma v}$
	%
	means the function $\sigma' \mapsto \aug{\sigma'}{v: \sigma v}$, restoring the
	old value.
	%
	We didn't have to do something like this when we interpreted quantifications in
	predicate logic.
	%
	This is because there we didn't return a state, but a boolean value.
	%
	\item
	%
	How do we know that this is a sensible definition?
	%
	By checking expected properties like \cref{prop:coincidence} and
	\cref{prop:renaming}.

	$\fv{c}$ \dots free variables appearing in $c$. (textbook page 40)

	$\fa{c}$ \dots free assigned variables appearing in $c$. (textbook page 41)

	\begin{property}[Coincidence]\label{prop:coincidence}
		%
		\;\\
		%
		\vspace{-1.5em}
		%
		\begin{enumalpha}
			%
			\item
			%
			$\sigma w = \sigma' w \textrm{ for all } w \in \fv{c}$
			%
			\begin{align*}
				\implies & \prths{\bbrackets{c}\sigma = \bbrackets{c}\sigma' = \bot} \textrm{ or } \\
				         & \prths{
					\bbrackets{c}\sigma, \bbrackets{c}\sigma' \in \Sigma \textrm{ and }
					\prths{\bbrackets{c}\sigma} w = \prths{\bbrackets{c}\sigma'} w \textrm{ for all } w \in \fv{c}
				}
			\end{align*}
			%
			\item
			%
			$
				\bbrackets{c}\sigma \neq \bot \implies
				\prths{\bbrackets{c}\sigma} w = \sigma w \textrm{ for all } w \notin \fa{c}.
			$
		\end{enumalpha}
		%
	\end{property}
	%
	\begin{property}[Renaming]\label{prop:renaming}
		%
		\begin{multline*}
			\subsctext{v}{new} \notin \fv{c'} - \braces{v} \\
			\implies \bbrackets{\cnew{v}{e}{c'}} \sigma =
			\bbrackets{\cnew{\subsctext{v}{new}}{e}{\subst{c'}{v}{\subsctext{v}{new}}}} \sigma
		\end{multline*}
		%
	\end{property}
	%
\end{enumcirc}

\section{Syntactic Sugar}

\begin{enumcirc}
	%
	\item
	%
	Introduction of a construct by defining its meaning in terms of existing
	constructs in the language.
	%
	\item
	%
	Three definitions of for loop:
	%
	\begin{enumrm}
		%
		\item
		%
		$
			\prths{
				\cfor{v}{e_0}{e_1}{c}} :=
			\prths{
				\cseq{
					\cassign{v}{e_0}
				}{
					\cwhile{v \le e_1}{
						\prths{
							\cseq{c}{\cassign{v}{v + 1}}
						}
					}
				}
			}
		$
		%
		\item
		%
		$
			\prths{
				\cfor{v}{e_0}{e_1}{c}} :=
			\prths{
				\cnew{v}{e_0}{
					\cwhile{v \le e_1}{
						\prths{
							\cseq{c}{\cassign{v}{v + 1}}
						}
					}
				}
			}
		$
		%
		\item
		%
		$
			\prths{
				\cfor{v}{e_0}{e_1}{c}} := \\
			\textrm{\quad}
			\prths{
				\cnew{w}{e_1}{
					\cnew{v}{e_0}{
						\cwhile{v \le w}{
							\prths{
								\cseq{c}{\cassign{v}{v + 1}}
							}
						}
					}
				}
			}
		$,

		where $w \ne v$ and $w \notin \fv{e_0} \cup \fv{c}$.
		%
		\item
		      %
		      (iii) with the condition $v \notin \fv{c}$.
		%
	\end{enumrm}
	%
	\item
	%
	The for loop should be something easier to understand than while.
	%
	In this regard, (i) $<$ (ii) $<$ (iii) $<$ (iv).
	%
\end{enumcirc}

\section{Arithmetic errors}

\begin{enumcirc}
	%
	\item
	%
	How should we deal with $x \div 0$, underflow and overflow?
	%
	\item
	%
	Two approaches:
	%
	\begin{enumrm}
		%
		\item
		%
		early stop with error
		%
		\item
		%
		some default choice and computation continued:
		%
		ad hoc but it can become less ad hoc if we ensure that the default choices
		satisfy \ul{certain properties} such as
		%
		\begin{align*}
			\bbrackets{\prths{x + y} \times 0} \sigma                       & = 0                     \\
			\bbrackets{x \div 0 = x \div 0} \sigma                          & = \ttt                  \\
			\bbrackets{\cseq{\cassign{y}{x \div 0}}{\cassign{y}{e}}} \sigma & =
			\bbrackets{\cassign{y}{e}} \sigma \textrm{ when } y \notin \fv{e}                         \\
			\bbrackets{\cif{x + y = z}{c}{c}} \sigma                        & = \bbrackets{c} \sigma. \\
		\end{align*}
		%
	\end{enumrm}
	%
\end{enumcirc}

\section{Soundness and full abstraction}

\begin{enumcirc}
	%
	\item
	%
	The semantics defined so far looks ok, but is there any formal way to confirm
	this?
	%
	\item
	%
	One approach is to show that the semantics assigns the same meaning to two
	commands $c_1$ and $c_2$ only when $c_1$ and $c_2$ should be equal intuitively.
	%
	That is,
	%
	\[
		\bbrackets{c_1} = \bbrackets{c_2} \implies c_1 ``=" \footnotemark c_2.
	\]
	\footnoteeqn[0]{our intuitive notion of equality defined separately}
	%
	This property is called \ul{soundness}.
	%
	Its converse
	%
	\[
		\bbrackets{c_1} = \bbrackets{c_2} \impliedby c_1 ``=" c_2
	\]
	%
	is called \ul{full abstraction}.
	%
	\item
	%
	Now how to define $``="$?
	%
	We use a set of observable phrases with a hole or \ul{contexts}, $\Cc$, and a
	set of \ul{observations}, $\Oc$, which are functions from observable phrases to
	outcomes.
	%
	\[
		c_1 ``=" c_2 \iff \forall c \in \Cc \footnotemark ,\; \forall o \in \Oc \footnotemark,\;
		o \prths{c \brackets{c_1}\footnotemark} = o \prths{c \brackets{c_2}}
	\]
	\footnoteeqn[-2]{intuitively means all use cases}
	\footnoteeqn{intuitively means user's observations}
	\footnoteeqn{filling the hole of $c$ with $c_1$}
	%
	\begin{enumrm}
		%
		\item
		%
		Intuitively, this condition says that under all use cases, the user cannot
		observe the difference between $c_1$ and $c_2$.
		%
		\item
		%
		This is sometimes called \ul{observataional equivalence}.
		%
		\item
		%
		Note that this is not a syntax-directed (or compositional) definition.
		%
	\end{enumrm}
	%
	\begin{example}
		%
		Assume that $v_0, \dots , v_{n-1}$ are all the free variables in $c_1$ and
		$c_2$.
		%
		\[
			\Cc = \set{
				\begin{array}{c}
					\begin{aligned}
						\cnew{v_0     & }{k_0}{        \\
						\cnew{v_1     & }{k_1}{        \\
						              & \vdots         \\
						\cnew{v_{n-1} & }{k_{n-1}}{}}}
					\end{aligned}
					\\
					\prths{
						\begin{aligned}
							\brackets{-}; \textrm{ if } v_i = k & \textrm{ then } \cskip                         \\
							                                    & \textrm{ else } \cwhile{\textrm{true}}{\cskip}
						\end{aligned}
					}
				\end{array}
			}{
				\begin{array}{c}
					k, k_0, \dots , k_{n-1} \in \Z \\
					i \in \braces{0, \dots , n-1}
				\end{array}
			}
		\]
		%
		\vspace{1em}
		%
		\[
			\Oc = \braces{\lambda c \;.\; \cif{\bbrackets{c} \sigma_0 = \bot}{0}{1}}
		\]
		%
		where $\sigma_0 \prths{x} = 0$ for all $x$.
	\end{example}

\end{enumcirc}
\chapter{Program Specifications and Their Proofs}

\section{Motivation}

\begin{enumcirc}
	%
	\item
	%
	Are there any methods that let us specify desired properties or intended
	behaviors of a program and prove that the specified properties indeed hold?
	%
	For instance, consider the program:
	%
	\[
		\subsctext{c}{div3} =
		\prths{
			\begin{array}{l}
				a := 0;                                        \\
				b := x;                                        \\
				\textrm{while } \prths{b \geq 3} \textrm{ do } \\
				\qquad b := b - 3;                             \\
				\qquad a := a + 1                              \\
			\end{array}
		}
	\]
	%
	We want to express formally that the program divides $x$ by $3$ and stores the
	quotient in $a$ and the remainder in $b$.
	%
	We also want to prove this formal specification.
	%
	\item
	%
	We will study Hoare logic and its variant for total correctness.
	%
	They provide the kind of methods that we are looking for.
	%
	\item
	%
	Hoare logic and its total-correctness variant are the basis of modern automatic
	software verifiers, such as Facebook's infer.
	%
	\item
	%
	Another reason for studying Hoare logic is that it shows why we need or where
	we use denotational semantics that we studied.
	%
\end{enumcirc}

\section{Syntax and semantics of specifications}

\begin{enumcirc}
	%
	\item
	%
	Specifications are a new type of phrases that formally express properties of
	programs.
	%
	\item
	%
	Syntax in terms of abstract grammar:
	%
	\begin{center}
		\begin{minipage}{0.5\textwidth}
			\grammarindent5em
			\begin{grammar}
				<spec> ::= \{<assert>\} <comm> \{<assert>\}
				\alt [<assert>] <comm> [<assert>]
			\end{grammar}
		\end{minipage}
	\end{center}
	%
	\begin{exampletab}
		%
		\[
			\subsctext{c}{fib} = \prths{
				\begin{array}{c}
					k := 1;\; y := 0;\; x:=1;\; \\
					\cwhile{k \ne n}{           \\ \prths{t := y;\; y := x;\; x := x + t;\; k := k + 1}}
				\end{array}
			}
		\]
		%
		\[
			\begin{array}{c}
				\braces{n \ge 0}
				\subsctext{c}{fib}
				\braces{x = \textrm{fib}\prths{n}}
				\qquad \qquad
				\braces{\true}
				\subsctext{c}{fib}
				\braces{x = \textrm{fib}\prths{n}}
				\\[0.3em]
				\brackets{n \ge 0}
				\subsctext{c}{fib}
				\brackets{x = \textrm{fib}\prths{n}}
				\qquad \qquad
				\brackets{\true}
				\subsctext{c}{fib}
				\brackets{x = \textrm{fib}\prths{n}}
				\\[0.3em]
				\braces{x \ge 0}
				\subsctext{c}{div3}
				\braces{x = 3a + b \wedge 0 \le b < 3 \wedge a \ge 0}
				\\[0.3em]
				\brackets{x \ge 0}
				\subsctext{c}{div3}
				\brackets{x = 3a + b \wedge 0 \le b < 3 \wedge a \ge 0}
				\\[0.3em]
				\brackets{\true}
				\subsctext{c}{div3}
				\brackets{x = 3a + b \wedge 0 \le b < 3 \wedge a \ge 0}
				\\[0.3em]
				\brackets{\true}
				\subsctext{c}{div3}
				\brackets{x = 3a + b \wedge 0 \le b < 3 \wedge a \ge 0}
			\end{array}
		\]
		%
	\end{exampletab}
	%
	\item
	%
	Intuitive reading:
	%
	\begin{enumrm}
		%
		\item
		%
		$\braces{p} c \braces{q}$ holds iff when $c$ is run in a state satisfying $p$,
		\ul{and it terminates normally}, \footnote{condition} then the final state satisfies $q$.
		%
		\item
		%
		$\brackets{p} c \brackets{q}$ holds iff when $c$ is run in a state satisfying $p$,
		then \ul{it terminates normally}, \footnote{conclusion} and the final state satisfies $q$.
		%
	\end{enumrm}
	%
	Note that $\brackets{p} c \brackets{q}$ expresses a stronger property than
	$\braces{p} c \braces{q}$.
	%
	The former is called \ul{total correctness specification}, \footnote{also total
		correctness triple} and the latter \ul{partial correctness specification}.
	\footnote{also partial correctness triple, Hoare triple or triple}

	$p$ \dots precondition or precedent.

	$q$ \dots postcondition or consequent.

	\begin{exercise}
		%
		Among all the partial correctness and total correctness specification from
		above, pick those that hold.
		%
	\end{exercise}
	%
	\item
	%
	Formal semantics:
	%
	\begin{align*}
		\bbrackets{-}                           & \in \brackets{\gram{spec} \to \B} \\
		\bbrackets{\braces{p} c \braces{q}}     & = \ttt \quad \textrm{ iff } \quad
		\bbrackets{p} \sigma = \ttt \wedge \bbrackets{c} \sigma \neq \bot \implies
		\bbrackets{q} \prths{\bbrackets{c} \sigma} = \ttt                           \\
		\bbrackets{\brackets{p} c \brackets{q}} & = \ttt \quad \textrm{ iff } \quad
		\bbrackets{p} \sigma = \ttt \implies \bbrackets{c} \sigma \neq \bot \wedge
		\bbrackets{q} \prths{\bbrackets{c} \sigma} = \ttt
	\end{align*}
	%
\end{enumcirc}

\section{Inference rules}

\begin{enumcirc}
	%
	\item
	%
	Methods or rules for proving or deriving partial or total correctness triples.
	%
	\item
	%
	\;\vspace{-1.5em}
	%
	\[
		\inferrule{\textrm{premises}}{\textrm{conclusion}}
		\qquad \qquad
		\inferrule{\varphi_1 \quad \varphi_2 \quad \dots \quad \varphi_n}{\psi}
	\]
	%
	(if $\varphi_1, \varphi_2, \dots, \varphi_n$ are true, then $\psi$ is true)
	%
	\item
	%
	Rules associated with program constructs:
	%
	\[
		\begin{array}{c}
			\inferrule
			{ }
			{\braces{p} \cskip \braces{p}}
			\qquad \qquad
			\inferrule
			{ }
			{\brackets{p} \cskip \brackets{p}}
			\\[2em]
			\inferrule
			{\braces{p} c_1 \braces{r}                    \\ \braces{r} c_2 \braces{q}}
			{\braces{p} c_1 ; c_2 \braces{q}}
			\qquad \qquad
			\inferrule
			{\brackets{p} c_1 \brackets{r}                \\ \brackets{r} c_2 \brackets{q}}
			{\brackets{p} c_1 ; c_2 \brackets{q}}
			\\[2em]
			\inferrule
			{\braces{p \wedge b} c_1 \braces{q}           \\ \braces{p \wedge \neg b} c_2 \braces{q}}
			{\braces{p} \cif{b}{c_1}{c_2} \braces{q}}
			\qquad \qquad
			\inferrule
			{\brackets{p \wedge b} c_1 \brackets{q}       \\ \brackets{p \wedge \neg b} c_2 \brackets{q}}
			{\brackets{p} \cif{b}{c_1}{c_2} \brackets{q}}
			\\[2em]
			\inferrule
			{ }
			{\braces{\subst{q}{x}{e}} x := e \braces{q}}
			\qquad \qquad
			\inferrule
			{ }
			{\brackets{\subst{q}{x}{e}} x := e \brackets{q}}
			\\[2em]
			\inferrule
			{\braces{i \wedge b} c \braces{i}}
			{\braces{i} \cwhile{b}{c} \braces{i \wedge \neg b}}
			\qquad \quad
			\inferrule
			{i \wedge b \Rightarrow e \ge 0 \footnotemark \\ \brackets{i \wedge b \wedge e = v_0 \footnotemark} c \brackets{i \wedge e < v_0}}
			{\brackets{i} \cwhile{b}{c} \brackets{i \wedge \neg b}}
		\end{array}
	\]
	\footnoteeqn[-1]{$i \wedge b \Rightarrow e \ge 0$ should be valid. That is, $\bbrackets{i \wedge b \Rightarrow e \ge 0} \sigma = \ttt$ for all $\sigma$.}
	\footnoteeqn{when $v_0$ does not occur free in $i$, $b$, $c$ or $e$}
	%
	\begin{enumrm}
		%
		\item
		%
		Note that rules for total correctness and the corresponding ones for partial
		correctness are identical except for the case of loop.
		%
		This is expected because these two notions differ only in their treatment of
		non-termination.
		%
		\item
		%
		The rules for while require that $i$ should be preserved by the body of the
		loop.
		%
		The one for total correctness additionally requires that the value of $e$
		should decrease whenever we run the loop body $c$ once, but it cannot be
		negative.
		%
		All these requirements together give the conclusions of the rules.
		%
		\item
		%
		The rules for assignment also deserve some thoughts.
		%
		They in a sense say that running an assignment backward symbolically is the
		same as doing substitution.
		%
		It holds because of the substitution theorem (Prop 1.3 and Prop 1.4 in the
		textbook).

		Reminder of the theorem specialized to our case here:
		%
		\[
			\begin{tikzcd}[cramped, sep=7em]
				\Sigma
				\arrow[r, "\lambda \sigma . \aug{\sigma}{x:\bbrackets{e}\sigma}"]
				\arrow[d, swap, "\bbrackets{\subst{q}{x}{e}}"] &
				\Sigma
				\arrow[d, "\bbrackets{q}"]\\
				\B \arrow[r,-,double, "="] &
				\B
			\end{tikzcd}
		\]
		%
	\end{enumrm}
	%
	Rules not associated with any specific program constructs
	%
	(sometimes called structural rules or adaptation rules):
	%
	\[
		\inferrule
		{p \Rightarrow p' \\ \braces{p'} c \braces{q'} \\ q' \Rightarrow q}
		{\braces{p} c \braces{q}}
		\qquad \qquad
		\inferrule
		{p \Rightarrow p' \\ \brackets{p'} c \brackets{q'} \\ q' \Rightarrow q}
		{\brackets{p} c \brackets{q}}
	\]
	%
	They are called \ul{the rule of consequence}.
	%
	They often enable us to use the other rules, in particular, those for loop and
	if.
	%
	\item
	%
	I omit many structural rules and the rule for newvar.
	%
	Look at the textbook if you are interested.
	%
\end{enumcirc}

\section{Example proof}

\[
	\subsctext{c}{div3} =
	\prths{
		\begin{array}{l}
			a := 0;                                        \\
			b := x;                                        \\
			\textrm{while } \prths{b \geq 3} \textrm{ do } \\
			\qquad b := b - 3;                             \\
			\qquad a := a + 1                              \\
		\end{array}
	}
\]

Goal: prove that
%
\[
	\braces{x \ge 0}
	\subsctext{c}{div3}
	\braces{x = 3a + b \wedge 0 \le b < 3}
\]

Proof:

{
\fontsize{3pt}{3pt}\selectfont
\[
	\scriptscriptstyle
	\inferrule{
		x \ge 0 \Rightarrow x = 3 \times 0 + x \wedge x \ge 0
		\\
		\inferrule{
			\inferrule{ }{
				\braces{x = 3 \times 0 + x \wedge x \ge 0}
				a := 0
				\braces{x = 3a + x \wedge x \ge 0}
			}
		}{
			\braces{x \ge 0}
			a := 0
			\braces{x = 3a + x \wedge x \ge 0}
		}
		\\
		\inferrule{ }{
			\braces{x = 3a + x \wedge x \ge 0}
			b := x
			\braces{x = 3a + b \wedge b \ge 0}
		}
	}{
		\braces{x \ge 0}
		a := 0;\; b := x
		\braces{x = 3a + b \wedge b \ge 0}
	}
\]

\[
	\inferrule{
		\inferrule{
			\inferrule{
				\substack{
					x = 3a + b \wedge b \ge 0 \wedge b \ge 3\\
					\Downarrow\\
					x = 3(a + 1) + (b - 3) \wedge b - 3 \ge 0\\
				}
				\\
				\inferrule{ }{
					\braces{x = 3(a + 1) + (b - 3) \wedge b - 3 \ge 0}
					b := b - 3
					\braces{x = 3(a + 1) + b \wedge b \ge 0}
				}
			}{
				\braces{x = 3a + b \wedge b \ge 0 \wedge b \ge 3}
				b := b - 3
				\braces{x = 3(a + 1) + b \wedge b \ge 0}
			}
			\\
			\inferrule{ }{
				\braces{x = 3(a + 1) + b \wedge b \ge 0}
				a := a + 1
				\braces{x = 3a + b \wedge b \ge 0}
			}
		}{
			\braces{x = 3a + b \wedge b \ge 0 \wedge b \ge 3}
			b := b - 3;\; a := a + 1
			\braces{x = 3a + b \wedge b \ge 0}
		}
	}{
		\braces{x = 3a + b \wedge b \ge 0}
		\cwhile{b \ge 3}{b := b - 3;\; a := a + 1}
		\braces{x = 3a + b \wedge 0 \le b < 3}
	}
\]

\[
	\inferrule
	{
		\braces{x \ge 0}
		a := 0;\; b := x
		\braces{x = 3a + b \wedge b \ge 0}
		\\
		\braces{x = 3a + b \wedge b \ge 0}
		\cwhile{b \ge 3}{b := b - 3;\; a := a + 1}
		\braces{x = 3a + b \wedge 0 \le b < 3}
	}{
		\braces{x \ge 0}
		\textrm{c}_{div3}
		\braces{x = 3a + b \wedge 0 \le b < 3}
	}
\]
}

\begin{enumcirc}
	%
	\item
	%
	In practice, people use the rule of consequence, without mentioning it
	explicitly.
	%
	Also, they use many derived rules.
	%
	\item
	%
	This proof has the flavor of running a program backward symbolically because of
	its heavy use of the assignment rule and the fact that the rule of consequence
	is used only when it is necessary.

	\begin{exercisetab}
		%
		Prove:
		%
		\begin{enumalpha}
			%
			\item
			%
			\[
				\braces{n \ge 1}
				\subsctext{c}{fib}
				\braces{x = \textrm{fib}\prths{n}}
			\]
			%
			\item
			%
			\[
				\subsctext{c}{Euclid} =
				\prths{
					\begin{array}{l}
						while \prths{a \ne b} \textrm{ do }                  \\
						\qquad \textrm{if } a > b \textrm{ then } a := a - b \\
						\qquad \textrm{else } b := b - a
					\end{array}
				}
			\]
			%
			\[
				\braces{a \ge 1 \wedge b \ge 1 \wedge a = a_0 \wedge b = b_0}
				\subsctext{c}{Euclid}
				\braces{a = \textrm{gcd}\prths{a_0, b_0}}
			\]
			%
		\end{enumalpha}
		%
	\end{exercisetab}

	\begin{exercisetab}
		%
		Find a forward rule for assignment.
		%
		That is, for all $p$ and $x, e$, find $q$ s.t.
		%
		\[
			\inferrule{ }{\brackets{p} x := e \brackets{q}}
		\]
		%
	\end{exercisetab}
	%
\end{enumcirc}

\section{Soundness}

\begin{theorem}[Soundness theorem]
	%
	If $\brackets{p} c \brackets{q}$ is derivable using
	%
	\ul{the rules that we studied},
	%
	\footnote{called rules in Hoare logic}
	%
	then $\bbrackets{\braces{p} c \braces{q}} = \ttt$, i.e.,
	%
	the triple $\braces{p} c \braces{q}$ holds.
	%
	If $\brackets{p} c \brackets{q}$ is derivable, then
	%
	$\bbrackets{\brackets{p} c \brackets{q}} = \ttt$.
	%
\end{theorem}
%
\begin{proof}
	%
	Intuitively, the theorem says that all rules are correct.
	%
	In fact, typical proofs of the theorem show the correctness of the rules in the
	following sense:

	If $\inferrule{\varphi_1 \quad \varphi_2 \quad \dots \quad \varphi_n}{\psi}$
	%
	then
	%
	\[
		\bbrackets{\varphi_1} = \ttt \wedge
		\bbrackets{\varphi_2} = \ttt \wedge \dots \wedge
		\bbrackets{\varphi_n} = \ttt \implies
		\bbrackets{\psi} = \ttt.
	\]
	%
	The rules for loop (or while) are the most important cases.
	%
	We will consider only the one for partial correctness.
	%
	\[
		\inferrule
		{\braces{i \wedge b} c \braces{i}}
		{\braces{i} \cwhile{b}{c} \braces{i \wedge \neg b}}
	\]
	%
	We first do a bit of rewriting for the semantics of specifications.
	%
	\begin{multline*}
		\bbrackets{\brackets{p} c \brackets{q}}
		\quad \textrm{ iff } \\
		\forall \sigma \in \Sigma ,\;
		\bbrackets{p} \sigma = \ttt \implies
		\bbrackets{q}_\bot \prths{\bbrackets{c} \sigma} \sqsubseteq \ttt \\
		\textrm{ where }
		\bbrackets{q}_\bot \in \brackets{\Sigma_\bot \to \B_\bot} \\
		\textrm{ s.t. }
		\bbrackets{q}_\bot \sigma = \bbrackets{q} \sigma \textrm{ for all } \sigma \in \Sigma
		\textrm{ and }
		\bbrackets{q}_\bot \prths{\bot} = \bot
	\end{multline*}
	%
	We need to prove that, if
	%
	\begin{equation}{} \label{eq:while-partial-correctness} \tag{$\star$}
		\forall \sigma .\;
		\bbrackets{i \wedge b} \sigma = \ttt \implies
		\bbrackets{i}_\bot \prths{\bbrackets{c} \sigma} \sqsubseteq \ttt
	\end{equation}
	%
	then
	%
	\[
		\forall \sigma .\;
		\bbrackets{i} \sigma = \ttt \implies
		\bbrackets{i \wedge \neg b}_\bot \prths{\bbrackets{\cwhile{b}{c}} \sigma} \sqsubseteq \ttt
	\]

	Assume that \cref{eq:while-partial-correctness} holds.
	%
	Let
	%
	\begin{align*}
		F                          & \in \brackets{\prths{\Sigma \to \Sigma_\bot} \toc \prths{\Sigma \to \Sigma_\bot}}                      \\
		F \prths{f} \prths{\sigma} & = \cif{\bbrackets{b} \sigma = \ttt}{\prths{ f_\doublebot \circ \bbrackets{c}} \prths{\sigma}}{\sigma}.
	\end{align*}
	%
	Define $f_n := F^n \prths{\bot}$ for all $n \ge 0$.
	%
	Then, $\displaystyle \bbrackets{\cwhile{b}{c}} = \bigcup_{n = 0}^\infty f_n$.
	%
	We will show that for all $n \ge 0$,
	%
	\begin{equation} \label{eq:while-partial-correctness-2} \tag{$\star\star$}
		\forall \sigma .\;
		\bbrackets{i} \sigma = \ttt \implies
		\bbrackets{i \wedge \neg b}_\bot \prths{f_n \prths{\sigma}} \sqsubseteq \ttt.
	\end{equation}
	%
	This is sufficient because for all $\sigma \in \Sigma$ s.t. $\bbrackets{i}
		\sigma = \ttt$,
	%
	\begin{align*}
		\bbrackets{i \wedge \neg b}_\bot \prths{\bbrackets{\cwhile{b}{c}} \sigma}
		 & = \bbrackets{i \wedge \neg b}_\bot \prths{\prths{\bigcup_{n = 0}^\infty f_n} \prths{\sigma}}       \\
		 & = \bbrackets{i \wedge \neg b}_\bot \prths{\bigcup_{n = 0}^\infty f_n \prths{\sigma}}               \\
		 & = \footnotemark \bigcup_{n = 0}^\infty \bbrackets{i \wedge \neg b}_\bot \prths{f_n \prths{\sigma}} \\
		 & \sqsubseteq \footnotemark \bigcup_{n = 0}^\infty \ttt = \ttt.
	\end{align*}
	\footnoteeqn[-1]{because $\bbrackets{i \wedge \neg b}_\bot$ is continuous}
	\footnoteeqn{because of \cref{eq:while-partial-correctness-2}}
	%
	Our proof of \cref{eq:while-partial-correctness-2} uses induction on $n$.

	\begin{itemize}
		%
		\item
		      %
		      Base case $n = 0$: $f_0 = \bot$.
		      %
		      \[
			      \therefore
			      \bbrackets{i \wedge \neg b}_\bot \prths{f_0 \prths{\sigma}} = \bbrackets{i \wedge \neg b}_\bot \prths{\bot} = \bot \sqsubseteq \ttt.
		      \]
		      %
		\item
		      %
		      Inductive case $n = m + 1$:
		      %
		      Pick $\sigma$ s.t. $\bbrackets{i} \sigma = \ttt$.
		      %
		      \begin{align*}
			       & \quad\, \bbrackets{i \wedge \neg b}_\bot \prths{f_{m + 1} \prths{\sigma}}                                    \\
			       & = \bbrackets{i \wedge \neg b}_\bot \prths{F\prths{f_m} \prths{\sigma}}                                       \\
			       & = \bbrackets{i \wedge \neg b}_\bot
			      \prths{\cif{\bbrackets{b} \sigma = \ttt}{\prths{ {f_m}_\doublebot \circ \bbrackets{c}} \prths{\sigma}}{\sigma}} \\
			       & = \cif
			      {\bbrackets{b} \sigma = \ttt}
			      {\prths{\bbrackets{i \wedge \neg b}_\bot \circ {f_m}_\doublebot} \prths{\bbrackets{c} \prths{\sigma}}}
			      {\bbrackets{i \wedge \neg b}_\bot \prths{\sigma}}                                                               \\
		      \end{align*}
		      %
		      \vspace{-3em}\\
		      %
		      Since $\bbrackets{i} \sigma = \ttt$, if $\bbrackets{b} \sigma \ne \ttt$, then
		      %
		      \[
			      \bbrackets{i \wedge \neg b}_\bot \prths{\sigma} = \bbrackets{i}_\bot \prths{\sigma} = \ttt \sqsubseteq \ttt.
		      \]
		      %
		      If $\bbrackets{b} \sigma = \ttt$ and $\bbrackets{c} \sigma = \bot$, then
		      %
		      \[
			      \prths{\bbrackets{i \wedge \neg b}_\bot \circ {f_m}_\doublebot} \prths{\bbrackets{c} \prths{\sigma}} = \bot \sqsubseteq \ttt.
		      \]
		      %
		      If $\bbrackets{b} \sigma = \ttt$ and $\bbrackets{c} \sigma \ne \bot$, then
		      %
		      $\bbrackets{i}_\bot \prths{\bbrackets{c}\sigma} = \ttt$
		      %
		      by \cref{eq:while-partial-correctness}.
		      %
		      Thus,
		      %
		      \[
			      \prths{\bbrackets{i \wedge \neg b}_\bot \circ {f_m}_\doublebot} \prths{\bbrackets{c} \prths{\sigma}} =
			      \bbrackets{i \wedge \neg b}_\bot \prths{ f_m \prths{\bbrackets{c} \prths{\sigma}} } \sqsubseteq \ttt
		      \]
		      %
		      by induction hypothesis.
		      %
	\end{itemize}

\end{proof}


\chapter{Failure, Input-Output, and Continuations}

\section{Motivation}

\begin{enumcirc}
	%
	\item
	%
	Realistic programming languages have a wide range of language constructs and
	features.
	%
	In particular, they have constructs for input and output and those for altering
	the flow of exectuion.
	%
	\item
	%
	We will study mathematical tools that allow us to study or analyze constructs
	for input-output and the fail operation, the simplest operator for changing the
	flow of execution.
	%
	\item
	%
	Specifically, we will study recursively defined domains, the disjoint union of
	domains and continuations, and use them to define denotational semantics of an
	imperative language with input-output and failure.
	%
	\item
	%
	Here are a few transferable high-level messages that I want you to learn in the
	chapter.
	%
	\begin{enumrm}
		%
		\item
		%
		Recursively defined domains can be used to model computations with intermediate
		results\footnote{such as computations with outputs}
		%
		and computations that may be stopped in the middle and be resumed
		later\footnote{such as computations with inputs}.
		%
		\item
		%
		Continuations may simplify semantic definitions by providing a canonical
		treatment of sequential composition operator.
		%
	\end{enumrm}
	%
\end{enumcirc}

\section{Syntax of a programming language with failure and input-output}

\begin{center}
	\begin{minipage}{0.9\textwidth}
		\begin{grammar}
			<comm> ::= \dots | newvar <var> := <intexp> in <comm> | fail\footnotemark | ?<var>\footnotemark | !<intexp>\footnotemark
		\end{grammar}
	\end{minipage}
\end{center}
\footnoteeqn[-2]{failure}
\footnoteeqn{input}
\footnoteeqn{output}
%
Fail terminates the current execution and makes the current state the final
state of execution modulo the restoration of values of global variables.

\begin{exercise}
	%
	Write two interesting programs in this language.
	%
\end{exercise}

\begin{exercise}
	%
	What should be the final state of the following program?
	%
	\begin{center}
		\begin{minipage}{0.6\textwidth}
			\begin{verbatim}
        x := 124; y := 0;
        (newvar x := 3 in fail);
        y := 2
      \end{verbatim}
		\end{minipage}
	\end{center}
	%
\end{exercise}

\section{Semantics}

\begin{enumcirc}
	%
	\item
	%
	The most important step in defining the semantics is to decide the form of the
	interpretation function for commands:
	%
	\[
		\bbrackets{-} \in \brackets{\chevrons{\textit{comm}} \to \brackets{\underline{A} \toc^{\footnotemark} \underline{B}}}
	\]
	\footnotetext{continuous function}
	%
	What predomains or domains should we use for $\underline{A}$ and
	$\underline{B}$?
	%
	\item
	%
	$\underline{A}$ should be $\Sigma = \brackets{\textit{Var} \to \Z}$,
	%
	the set of (or predomain) of states.\footnote{
		\begin{minipage}{0.8\textwidth}
			\vspace{0.2em}
			Recall that a set can be viewed
			as a predomain when it is given $=$ as its order $\sqsubseteq$, which is called
			discrete order.
			\vspace{0.2em}
		\end{minipage}
	}
	%
	\item
	%
	$\underline{B}$ is complex.
	%
	Its element describe computations that can be stopped and resumed, and may
	output some integers in the middle.
	%
	Formally,
	%
	\begin{align*}
		\underline{B} & = \Omega \simeq\footnotemark \prths{\hat{\Sigma} + \prths{\Z \times \Omega} + \prths{\Z \toc \Omega}}_\bot
		\footnotemark \footnotemark                                                                                                \\
		\hat{\Sigma}  & = \Sigma \cup \braces{\abort} \times \Sigma \simeq \Sigma + \Sigma
	\end{align*}
	\footnoteeqn[-2]{is isomorphic to}
	\footnoteeqn{$\Omega$ satisfies a form of equation}
	\footnoteeqn{
		\begin{minipage}{0.8\textwidth}
			\vspace{0.2em}
			The RHS of the isomorphism says that there are five kinds of outcomes of
			running a command at a given state. We list these cases below.
			\vspace{0.2em}
		\end{minipage}
	}
	%
	\item
	%
	Many things here are not defined nor explained.
	%
	We will look at them one by one.
	%
	But before doing so, let's try to understand intuitions behind this definition.
	%
	\begin{enumrm}
		%
		\item
		%
		non-termination \dots $\bot$
		%
		\item
		%
		normal termination with a state $\sigma$ \dots
		%
		$\sigma \in \Sigma$
		%
		\item
		%
		abnormal termination with a state $\sigma$ \dots
		%
		$\chevrons{\abort, \sigma} \in \braces{\abort} \times \Sigma$
		%
		\item
		%
		output $n$ and the rest of computation $\omega$ \dots
		%
		$\chevrons{n, \omega} \in \Z \times \Omega$
		%
		\item
		%
		suspended computation $g$ that waits for an input \dots
		%
		$g \in \prths{\Z \toc \Omega}$
		%
	\end{enumrm}
	%
	\item
	%
	The description of $\Omega$ uses two pre-domain constructors, sum $+$ and
	product $\times$.
	%
	Let $P_0, \cdots, P_{n-1}$ be pre-domains and let
	%
	$\sqsubseteq_0, \cdots, \sqsubseteq_{n-1}$ be the partial orders of these pre-domains.
	%
	From these pre-domains, we can construct the following two pre-domains, their
	sum and product:
	%
	\[
		\begin{array}{c}
			P_0 + \cdots + P_{n-1} = \set{\chevrons{i, x}}{i \in \braces{0, \cdots, n-1}, x \in P_i}          \\[0.5em]
			\chevrons{i, x} \sqsubseteq \chevrons{j, y} \textrm{ iff } i = j \textrm{ and } x \sqsubseteq_i y \\[1em]
			P_0 \times \cdots \times P_{n-1} = \set{\chevrons{x_0, \cdots, x_{n-1}}}{x_i \in P_i}             \\[0.5em]
			\chevrons{x_0, \cdots, x_{n-1}} \sqsubseteq \chevrons{y_0, \cdots, y_{n-1}} \textrm{ iff }
			x_i \sqsubseteq_i y_i \textrm{ for all } i \in \braces{0, \cdots, n-1}
		\end{array}
	\]
	%
	\begin{enumrm}
		%
		\item
		%
		\begin{exercise}
			%
			Show that $P_0 + \cdots + P_{n-1}$ and $P_0 \times \cdots \times P_{n-1}$ are
			pre-domains.
			%
			Also, prove that $P_0 \times \cdots \times P_{n-1}$ is a domain if all $P_i$'s
			are domains.
			%
		\end{exercise}
		%
		\underline{Hint/Information}:
		%
		\begin{enumalpha}
			%
			\item
			%
			The least upper bound of a chain
			%
			$\braces{\chevrons{x_0^{(k)}, \cdots, x_{n-1}^{(k)}}}_k$ in
			%
			$P_0 \times \cdots \times P_{n-1}$ can be computed component-wise.
			%
			\[
				\bigsqcup_k^\infty \chevrons{x_0^{(k)}, \cdots, x_{n-1}^{(k)}} =
				\chevrons{\bigsqcup_k^\infty x_0^{(k)}, \cdots, \bigsqcup_k^\infty x_{n-1}^{(k)}}
			\]
			%
			\item
			%
			For every chain $\braces{z_k}_k$ in $P_0 + \cdots + P_{n-1}$, there are some
			$i$ and a chain $\braces{x_k}_k$ in $P_i$ such that $z_k = \chevrons{i, x_k}$
			for all $k$.
			%
		\end{enumalpha}
		%
		\item
		%
		These predomain constructors correspond to the sum and product type
		constructors in programming languages.
		%
		\item
		%
		For each case, we have a way to construct an element and a way to destruct an
		element.
		%
		Constructors:
		%
		\begin{enumalpha}
			%
			\item
			%
			injection function $\iota_k \in \brackets{P_k \toc P_0 + \cdots + P_{n-1}}$:
			%
			$\iota_k \prths{x} = \chevrons{k, x}$
			%
			\item
			%
			For
			%
			$f_0 \in \brackets{P \toc P_0}, \cdots, f_{n-1} \in \brackets{P \toc P_{n-1}}$,\\
			%
			the ``target-tupling'' function
			%
			$f_0 \otimes \cdots \otimes f_{n-1} \in \brackets{P \toc P_0 \times \cdots \times P_{n-1}}$:\\
			%
			\[
				\prths{f_0 \otimes \cdots \otimes f_{n-1}}\prths{x} = \chevrons{f_0\prths{x}, \cdots, f_{n-1}\prths{x}}
			\]
			%
		\end{enumalpha}
		%
		Destructors:
		%
		\begin{enumalpha}
			%
			\item
			%
			For
			%
			$f_0 \in \brackets{P_0 \toc P}, \cdots, f_{n-1} \in \brackets{P_{n-1} \toc P}$,\\
			%
			the ``source-tupling'' function
			%
			$f_0 \oplus \cdots \oplus f_{n-1} \in \brackets{P_0 + \cdots + P_{n-1} \toc P}$:\\
			%
			\[
				\prths{f_0 \oplus \cdots \oplus f_{n-1}}\prths{\chevrons{i, x}} = f_i\prths{x}
			\]
			%
			\item
			%
			projection function
			%
			$\pi_k \in \brackets{P_0 \times \cdots \times P_{n-1} \toc P_k}$:
			%
			\[
				\pi_k\prths{\chevrons{x_0, \cdots, x_{n-1}}} = x_k
			\]
			%
		\end{enumalpha}
		%
		These constructors and destructors are mutually inverse in a sense.
		%
		Prop 5.1 and Prop 5.2 in the textbook express such inverse relationships.
		%
		\item
		%
		The sum and product operators can be applied to continuous functions so as to
		build a new continuous function.
		%
		For instance, if
		%
		\[
			f_0 \in \brackets{P_0 \toc P_0^\prime}, \cdots, f_{n-1} \in \brackets{P_{n-1} \toc P_{n-1}^\prime}
		\]
		%
		Then we have the following two \ul{continuous} functions:
		%
		\begin{align*}
			\prths{f_0 + \cdots + f_{n-1}}           & \in \brackets{P_0 + \cdots + P_{n-1} \toc P_0^\prime + \cdots + P_{n-1}^\prime}                     \\
			\prths{f_0 + \cdots + f_{n-1}}           & \chevrons{i, x}  = \chevrons{i, f_i\prths{x}}                                                       \\
			\prths{f_0 \times \cdots \times f_{n-1}} & \in \brackets{P_0 \times \cdots \times P_{n-1} \toc P_0^\prime \times \cdots \times P_{n-1}^\prime} \\
			\prths{f_0 \times \cdots \times f_{n-1}} & \chevrons{x_0, \cdots, x_{n-1}} = \chevrons{f_0\prths{x_0}, \cdots, f_{n-1}\prths{x_{n-1}}}
		\end{align*}
		%
	\end{enumrm}
	%
	Many predomain constructors similarly induce constructors for continuous
	functions.
	%
	This is because they are functors on the category of predomains and continuous
	functions.
	%
	We will look at such category-theoretic formulation in some later lectures.
	%
	\item
	%
	One nontrivial important concept is a recursively-defined domain.
	%
	Recall the description of $\Omega$ in the beginning of this lecture:
	%
	\begin{align*}
		\Omega       & \simeq \prths{\hat{\Sigma} + \prths{\Z \times \Omega} + \prths{\Z \to \Omega}}_\bot \\
		\hat{\Sigma} & = \Sigma\footnotemark + \Sigma\footnotemark
	\end{align*}
	\footnoteeqn[-1]{normal termination}
	\footnoteeqn{failed termination}
	%
	$\simeq$ means the presence of two continuous functions $\phi$ and $\psi$:
	%
	\[
		\Omega
		\mathrel{\mathop{\rightleftarrows}^{\mathrm{\phi}}_{\mathrm{\psi}}}
		\prths{\hat{\Sigma} + \prths{\Z \times \Omega} + \prths{\Z \to \Omega}}_\bot
	\]
	%
	such that $\phi \circ \psi = \mathrm{id}$ and $\psi \circ \phi = \mathrm{id}$.

	Note that the RHS of $\simeq$ contains $\Omega$ itself, something that is being
	defined.
	%
	It is a bit like $\Omega$ is defined in terms of itself, i.e. recursively.
	%
	Or we can say that $\Omega$ is a fixed point of some equations over domains.
	%
	\begin{enumrm}
		%
		\item
		%
		$\Omega$ is the domain of possible outcomes.
		%
		The isomorphism $\phi$ confirms that it consists of five kinds of elements
		corresponding to five different outcomes described earlier.
		%
		\item
		%
		$\Omega$ is not just a solution of the recursive domain equation.
		%
		It satisfies the following \ul{minimality condition}\footnote{ also called
			initiality }.

		For any domain $D$ and any continuous function $\alpha$
		%
		\[
			\alpha \in \brackets{\prths{\hat{\Sigma} + \prths{\Z \times D} + \prths{\Z \to D}}_\bot \toc D},
		\]
		%
		there exists a unique continuous function $\beta \in \brackets{\Omega \toc D}$
		such that the following diagram commutes.
		%
		\[
			\begin{tikzcd}[cramped, sep=9em]
				\prths{\hat{\Sigma} + \prths{\Z \times \Omega} + \prths{\Z \to \Omega}}_\bot
				\arrow[r, "\prths{\textrm{id}_{\hat{\Sigma}}+\prths{\textrm{id}_\Z\times\beta}+\prths{\Z\to\beta}\footnotemark}"]
				\arrow[d, "\phi"] &
				\prths{\hat{\Sigma} + \prths{\Z \times D} + \prths{\Z \to D}}_\bot \arrow[d, "\alpha"]\\
				\Omega \arrow[r, "\beta"']                                                                                         & D
			\end{tikzcd}
		\]
		\footnoteeqn[0]{
			\begin{minipage}{0.8\textwidth}
				\vspace{0.2em}
				\begin{enumalpha}
					%
					\item
					%
					$\prths{\Z \to \beta} \in \brackets{\prths{\Z \to \Omega} \toc \prths{\Z \to D}}$\\
					%
					$\prths{\Z \to \beta}\prths{g}\prths{n} = \beta\prths{g\prths{n}}$
					%
					\item
					%
					% asdf
					$\forall f \in \brackets{D^\prime \toc D^{\prime\prime}}$, we have
					%
					$f_\bot \in \brackets{D^\prime_\bot \toc D^{\prime\prime}_\bot}$ \\ s.t.
					%
					$f_\bot\prths{\bot} = \bot$ and $f_\bot\prths{x} = f\prths{x}$ for all $x \in D^\prime$.
				\end{enumalpha}
				\vspace{0.2em}
			\end{minipage}
		}

		One important consequence is that we can define a continuous function from
		$\Omega$ by case analysis, just as we can define a function on programs in a
		syntax-directed manner.
		%
		\item
		%
		Why does such $\Omega$ exist?
		%
		Because the predomain constructors used in the RHS of $\simeq$ are all very
		good, that is, they satisfy so-called local continuity.
		%
		The situation is very similar to the one for the fixed point theorem.
		%
		There we require a function to be continuous.
		%
		There, we use an analogous property on an operator that constructs a domain.
		%
		Later we will study this in detail.
		%
		\item
		%
		A few basic embedding operators\footnote{Editor's note: injection function?}:
		%
		\begin{align*}
			\ibot                                            &
			\in \brackets{\braces{\bot} \toc \Omega}         &
			\ibot\prths{\bot} =\;                            &
			\psi\prths{\bot_\Omega}                                                               \\
			\iterm                                           &
			\in \brackets{\Sigma \toc \Omega}                &
			\iterm\prths{\sigma} =\;                         &
			\psi\prths{\chevrons{0,\chevrons{0, \sigma}}} = \psi\prths{\sigma}\footnotemark       \\
			\iabort                                          &
			\in \brackets{\Sigma \toc \Omega}                &
			\iabort\prths{\sigma} =\;                        &
			\psi\prths{\chevrons{0,\chevrons{1, \sigma}}} = \psi\prths{\chevrons{\abort, \sigma}} \\
			\iout                                            &
			\in \brackets{\Z \times \Omega \toc \Omega}      &
			\iout\prths{n, \omega} =\;                       &
			\psi\prths{\chevrons{1, \chevrons{n, \omega}}} = \psi\prths{\chevrons{n, \omega}}     \\
			\iin                                             &
			\in \brackets{\prths{\Z \to \Omega} \toc \Omega} &
			\iin\prths{g} =\;                                &
			\psi\prths{\chevrons{2, g}} = \psi\prths{g}
		\end{align*}
		\footnoteeqn[0]{We will just write $\sigma$. Similar for other cases.}
		%
		\item
		%
		Consider $f \in \brackets{\Sigma \toc \Omega}$
		%
		and $h \in \brackets{\Sigma \toc \Sigma}$.
		%
		We want to define $f_* \in \brackets{\Omega \toc \Omega}$
		%
		and $h_\dagger \in \brackets{\Omega \toc \Omega}$
		%
		such that $f_*$ applies $f$ only for normally terminating state part of a given
		$\omega \in \Omega$ and $h_\dagger$ applies $h$ applies $h$ to the terminating
		or failing state part of $\omega$.
		%
		For instance,
		%
		\begin{align*}
			f_*\prths{\iout\prths{3, \iout\prths{4, \iterm\prths{\sigma}}}}       & = \iout\prths{3, \iout\prths{4, f\prths{\sigma}}} \\
			f_*\prths{\iout\prths{3, \iabort{\sigma}}}                            & = \iout\prths{3, \iabort\prths{\sigma}}           \\
			h_\dagger\prths{\iout\prths{3, \iout\prths{4, \iterm\prths{\sigma}}}} & = \iout\prths{3, \iout\prths{4, h\prths{\sigma}}} \\
			h_\dagger\prths{\iout\prths{3, \iabort{\sigma}}}                      & = \iout\prths{3, \iabort\prths{h\prths{\sigma}}}
		\end{align*}
		%
		How to define $f_*$ and $h_\dagger$?
		%
		Because of (ii), we can define them by case analysis:
		%
		\begin{align*}
			f_*\prths{\ibot}                         =\; & \ibot\prths{\bot}                                     \\
			f_*\prths{\iterm\prths{\sigma}}          =\; & \iterm\prths{f\prths{\sigma}}                         \\
			f_*\prths{\iabort\prths{\sigma}}         =\; & \iabort\prths{\sigma}                                 \\
			f_*\prths{\iout\prths{n, \omega}}        =\; & \iout\prths{n, f_*\prths{\omega}}                     \\
			f_*\prths{\iin\prths{g}}                 =\; & \iin\prths{\lambda n.\; f_*\prths{g\prths{n}}}        \\[1em]
			h_\dagger\prths{\ibot}                   =\; & \ibot\prths{\bot}                                     \\
			h_\dagger\prths{\iterm\prths{\sigma}}    =\; & \iterm\prths{h\prths{\sigma}}                         \\
			h_\dagger\prths{\iabort\prths{\sigma}}   =\; & \iabort\prths{h\prths{\sigma}}                        \\
			h_\dagger\prths{\iout\prths{n, \omega}}  =\; & \iout\prths{n, h_\dagger\prths{\omega}}               \\
			h_\dagger\prths{\iin\prths{g}}           =\; & \iin\prths{\lambda n.\; h_\dagger\prths{g\prths{n}}}.
		\end{align*}
		%
		\begin{exercise}
			In both cases, we have well-defined $f_*$ and $h_\dagger$ because of the minimality condition in (ii).
			%
			Find an appropriate $\alpha$ and $D$.
			%
		\end{exercise}
		%
		\item
		%
		For $\omega, \omega^\prime \in \Omega$,
		%
		intuitively $\omega \sqsubseteq \omega^\prime$ if we can obtain $\omega^\prime$
		by replacing $\bot$ in $\omega$.
		%
		Intuitively, $\sqsubseteq$ represents the progression of computation.
		%
		\begin{example}
			We will write $\botO$ for $\ibot\prths{\bot}$.
			%
			\begin{align*}
				\iout \prths{3, \iout\prths{4, \botO}} \sqsubseteq\;             &
				\iout\prths{3, \iout\prths{4, \iout\prths{5, \iterm\prths{\sigma}}}} \\
				\iin \prths{\lambda n.\; \iout\prths{n+1,  \botO}} \sqsubseteq\; &
				\iin\prths{\lambda n.\; \iout\prths{n+1, \iin\prths{\lambda m.\; \iout\prths{m+n, \iterm\prths{\sigma}}}}}
			\end{align*}
			%
		\end{example}
		%
	\end{enumrm}
	%
	\item
	%
	Interpretation of commands:
	%
	\begin{align*}
		\bbrackets{-}                                & \in
		\brackets{\chevrons{\textit{comm}} \to \brackets{\Sigma \toc \Omega}}                              \\
		\bbrackets{\cskip} \prths{\sigma}            & =
		\iterm\prths{\sigma}                                                                               \\
		\bbrackets{\cseq{c_1}{c_2}} \prths{\sigma}   & =
		\bbrackets{c_2}_* \prths{\bbrackets{c_1} \prths{\sigma}}                                           \\
		\bbrackets{\cif{b}{c_1}{c_2}} \prths{\sigma} & =
		\cif{\bbrackets{b} \prths{\sigma}}{\bbrackets{c_1} \prths{\sigma}}{\bbrackets{c_2} \prths{\sigma}} \\
		\bbrackets{\cwhile{b}{c}} \prths{\sigma}     & =
		\prths{\subscmath{Y}{(\Sigma \toc \Omega)} F}\footnotemark \prths{\sigma}                          \\
		\textrm{where } F\prths{f}\prths{\sigma}     & =
		\cif{\bbrackets{b} \prths{\sigma}}{f_*\prths{\bbrackets{c} \prths{\sigma}}}{\sigma}                \\
		\bbrackets{\cfail} \prths{\sigma}            & =
		\iabort\prths{\sigma}                                                                              \\
		\bbrackets{\cout{e}} \prths{\sigma}          & =
		\iout\prths{\bbrackets{e}\prths{\sigma}, \iterm{\prths{\sigma}}}                                   \\
		\bbrackets{\cin{v}} \prths{\sigma}           & =
		\iin\prths{\lambda n.\; \iterm\prths{\aug{\sigma}{v:n}}}                                           \\
		\bbrackets{\cnew{v}{e}{c}} \prths{\sigma}    & =
		\prths{\lambda \sigma^\prime .\; \aug{\sigma^\prime}{v:\sigma\prths{v}}}_\dagger
		\prths{\bbrackets{c} \aug{\sigma}{v: \bbrackets{e}{\sigma}}}                                       \\
		\bbrackets{\cassign{v}{e}} \prths{\sigma}    & =
		\iterm\prths{\aug{\sigma}{v: \bbrackets{e}{\sigma}}}
	\end{align*}
	\footnoteeqn[0]{$\bigcup_{n=0}^\infty F^n \prths{\bot}$}
	%
	\begin{exercise}
		%
		Prove the following equations:
		%
		\begin{align*}
			\bbrackets{\cseq{\cassign{x}{3}}{\cout{x}}} & = \bbrackets{\cseq{\cassign{x}{3}}{\cout{3}}} \\
			\bbrackets{\cseq{\cfail}{c}}                & = \bbrackets{\cfail}                          \\
		\end{align*}
	\end{exercise}
	%
	\begin{exercise}
		Does the following equation hold?
		%
		\[
			\bbrackets{\cseq{\cin{x}}{\cassign{y}{3}}} = \bbrackets{\cseq{\cassign{y}{3}}{\cin{x}}}
		\]
		%
	\end{exercise}
\end{enumcirc}
\chapter{Transition Semantics}

\section{Motivation or objective}

\begin{enumcirc}
	%
	\item
	%
	So far we defined the meanings of programs in imperative languages using the
	denotational semantics.
	%
	A good denotational semantics reveals an underlying mathematical structure of a
	programming language and hides the intermediate steps of computation as much as
	possible.
	%
	Also, it is compositional, and lets us reason about a piece of program code
	even when we do not know its surrounding program context.
	%
	\item
	%
	However, when a programming language has advanced or complex language
	constructs, defining a denotational semantics of the language may be difficult.
	%
	Also, sometimes we want to have a mathematical semantics of programs that tells
	us what happens in the middle of computation.
	%
	\item
	%
	The operational semantics is an alternative approach to give mathematical
	meanings to programs.
	%
	It is non-compositional, and does not hide the intermediate steps of
	computation.
	%
	But it is usually very simple and also rigorous or formal enough to enable a
	mathematical study of a programming language and language tools such as
	compiler and program verifier.
	%
	Also, an operational semantics of a programming language often serves as a
	blueprint of an interpreter or a compiler of the language.
	%
	\item
	%
	In this chapter, we will study the so-called \ul{small-step} operational
	semantics, which Reynolds calls transition semantics.
	%
\end{enumcirc}

\section{Main idea of the small-step operational semantics}

\begin{enumcirc}
	%
	\item
	%
	The key idea is to formalize one computation step of a program using a
	relation, called transition relation.
	%
	\item
	%
	Typically, a small-step operational semantics has two main parts.
	%
	\begin{enumrm}
		%
		\item
		%
		$\Gamma$ \dots a set of configurations.

		Usually, $\Gamma = \Gamma_N \cup \Gamma_T$ for some sets $\Gamma_N$, $\Gamma_T$
		with $\Gamma_N \cap \Gamma_T = \emptyset$.
		%
		Each element $\gamma \in \Gamma$ describes the status of a machine that runs a
		program.
		%
		If $\gamma \in \Gamma_N$, it is called \ul{nonterminal configuration} and the
		execution of its program is not finished yet.
		%
		If $\gamma \in \Gamma_T$, it is called \ul{terminal configuration} and the
		execution of its program is finished.
		%
		\item
		%
		$\rightarrow \;\subseteq \Gamma_N \times \Gamma$
		\dots transition relation.

		Intuitively, $\prths{\gamma, \gamma^\prime} \in\; \rightarrow$
		%
		\footnote{typically written as $\gamma \rightarrow \gamma^\prime$}
		%
		means that one computation step changes the status of a machine from $\gamma$
		to $\gamma^\prime$.
		%
		Note that $\gamma$ has to be a nonterminal configuration because of the domain
		of $\rightarrow$.
		%
		This condition is consistent with the intuition behind nonterminal and terminal
		configurations.
		%
	\end{enumrm}
	%
	Defining a small-step operational semantics amounts to defining $\Gamma$,
	$\Gamma_N$, $\Gamma_T$, and $\rightarrow$.
	%
	We will see a few examples of the operational semantics in this lecture.
	%
	Often if we define $\Gamma$, $\Gamma_N$, $\Gamma_T$, then the definition of
	$\rightarrow$ follows almost automatically.
	%
	This is a bit similar to the situation in the denotational semantics that if
	the form of the interpretation function for commands $\bbrackets{-}$ is
	determined, the actual definition of the function follows almost automatically.
	%
	\item
	%
	When defining the $\rightarrow$ relation, we usually use the inference rule
	notation $\inferrule{\varphi_1 \;\cdots \;\varphi_n}{\varphi}$ that you saw
	when we discussed Hoare logic.
	%
\end{enumcirc}

\section{Small-step operational semantics of the simple imperative language}

\begin{enumcirc}
	%
	\item
	%
	Let's try to give the operational semantics to the simple imperative language
	that we studied.
	%
	Here is a reminder of this abstract grammar:
	%
	\begin{grammar}
		<comm> ::=
		skip
		\alt <var> := <intexp>
		\alt <comm> ; <comm>
		\alt if <boolexp> then <comm> else <comm>
		\alt while <boolexp> do <comm>
		\alt while <boolexp> do <comm>
	\end{grammar}
	%
	\item
	%
	What should we do?
	%
	First, we have to define
	%
	\ul{the set of nonterminal configurations}
	%
	\footnote{$\Gamma_N$}
	%
	and
	%
	\ul{that of terminal configurations}.
	%
	\footnote{$\Gamma_T$}

	Here are our definitions:
	%
	\[
		\Gamma_N \defeq \chevrons{comm}\footnotemark \times \Sigma\footnotemark
		\qquad
		\Gamma_T \defeq \Sigma\footnotemark
	\]
	\footnoteeqn[-2]{command that records the remaining computation}
	\footnoteeqn{the current state of a machine}
	\footnoteeqn{the $\chevrons{comm}$ part is missing because there is no remaining computation}
	%
	The set of configurations is the union of the above two sets.
	%
	\item
	%
	Second, we should define a binary relation
	%
	\[
		\rightarrow \;\subseteq \Gamma_N \times \Gamma,
	\]
	%
	called transition relation, that describes single-step computation.
	%
	We write $\prths{\gamma, \gamma^\prime}$
	%
	to mean $\prths{\gamma, \gamma^\prime} \in\; \rightarrow$.
	%
	We define the transition relation $\rightarrow$ using the inference rule
	notation.
	%
	\[
		\begin{array}{c}
			\inferrule
			{\ }
			{
				\chevrons{\cskip, \sigma}
				\rightarrow
				\sigma
			}
			\qquad \qquad
			\inferrule
			{\ }
			{
				\chevrons{\cassign{v}{e}, \sigma}
				\rightarrow
				\sigma\prths{v \mapsto \bbrackets{e}\sigma}
			}
			\\[2em]
			\inferrule
			{
				\chevrons{c_1, \sigma}
				\rightarrow
				\sigma^\prime
			}
			{
				\chevrons{\cseq{c_1}{c_2}, \sigma}
				\rightarrow
				\chevrons{c_2, \sigma^\prime}
			}
			\qquad \qquad
			\inferrule
			{
				\chevrons{c_1, \sigma}
				\rightarrow
				\chevrons{c_1^\prime, \sigma^\prime}
			}
			{
				\chevrons{\cseq{c_1}{c_2}, \sigma}
				\rightarrow
				\chevrons{\cseq{c_1^\prime}{c_2}, \sigma^\prime}
			}
			\\[2em]
			\inferrule
			{
				\
			}
			{
				\chevrons{\cif{b}{c_1}{c_2}, \sigma}
				\rightarrow
				\chevrons{c_1, \sigma}
			}
			\prths{\bbrackets{b}\sigma = \ttt}
			\\[2em]
			\inferrule
			{
				\
			}
			{
				\chevrons{\cif{b}{c_1}{c_2}, \sigma}
				\rightarrow
				\chevrons{c_2, \sigma}
			}
			\prths{\bbrackets{b}\sigma = \fff}
			\\[2em]
			\inferrule
			{
				\
			}
			{
				\chevrons{\cwhile{b}{c}, \sigma}
				\rightarrow
				\sigma
			}
			\prths{\bbrackets{b}\sigma = \fff}
			\\[2em]
			\inferrule
			{
				\
			}
			{
				\chevrons{\cwhile{b}{c}, \sigma}
				\rightarrow
				\chevrons{\cseq{c}{\cwhile{b}{c}}, \sigma}
			}
			\prths{\bbrackets{b}\sigma = \ttt}
		\end{array}
	\]
	%
	Note that the right-hand side of $\rightarrow$ may include a command that is
	not a sub-command of the one on the left-hand side.
	%
	Look at
	%
	$ \chevrons{\cseq{c_1}{c_2}, \sigma}
		\rightarrow
		\chevrons{\cseq{c_1^\prime}{c_2}, \sigma^\prime} $
	%
	and
	%
	$ \chevrons{\cwhile{b}{c}, \sigma}
		\rightarrow
		\chevrons{\cseq{c}{\cwhile{b}{c}}, \sigma} $.
	%
	This indicates that the semantics is not compositional.
	%
	All these rules correspond to our intuitive understading of one computation
	step.
	%
	They can form the basis of the implementation of a simple interpreter, which
	just needs to run the $\rightarrow$ step repeatedly.
	%
	\item
	%
	Formal properties of the operational semantics:
	%
	\begin{enumrm}
		%
		\item
		%
		$\gamma \rightarrow \gamma_1
			\textrm{ and }
			\gamma \rightarrow \gamma_2
			\quad\implies\quad
			\gamma_1 = \gamma_2$.

		The semantics is deterministic.
		%
		\item
		%
		$\forall \gamma \in \Gamma_N \; \exists \gamma^\prime
			\textrm{ s.t. }
			\gamma \rightarrow \gamma^\prime$.

		In this semantics, executions never get stuck.
		%
		\item
		%
		From (i) to (ii), it follows that for every
		%
		$\gamma \in \Gamma$,
		%
		there exists a unique maximal sequence.
		%
		\[
			\gamma_0, \gamma_1, \gamma_2, \cdots, \gamma_n\footnotemark
		\]
		\footnoteeqn[0]{may be infinite}
		%
		such that
		%
		\begin{multline*}
			\gamma = \gamma_0 \wedge
			\gamma_0 \rightarrow \gamma_1 \rightarrow \gamma_2 \rightarrow \cdots
			\rightarrow \gamma_n
			\\ \wedge \gamma_n \textrm{ is a terminal configuration or } n \textrm{ is infinite}.
		\end{multline*}
		%
		This maximal finite or infinite sequence represents the full computation
		starting from $\gamma$.
		%
		\item
		%
		We write $\gamma \uparrow$ if the maximal execution sequence from $\gamma$ is
		infinite.
		%
		Then, for all commands $c$ and states $\sigma$,
		%
		\begin{align*}
			\bbrackets{c}\sigma = \bot          &
			\quad \textrm{iff} \quad
			\chevrons{c, \sigma} \uparrow         \\
			\bbrackets{c}\sigma = \sigma^\prime &
			\quad \textrm{iff} \quad
			\chevrons{c, \sigma} \rightarrow^* \sigma^\prime \footnotemark
		\end{align*}
		\footnoteeqn[0]{
			reflexive and transitive closure of $\rightarrow$, i.e.
			$\rightarrow^* \defeq \bigcup_n^\infty \prths{\rightarrow}^n$
		}
	\end{enumrm}
	%
	\begin{exercise}
		%
		Prove (i), (ii), and (iv).
		%
	\end{exercise}
	%
	\begin{exercise}
		%
		Explain why the reasoning in (iii) is true.
		%
	\end{exercise}
	%
\end{enumcirc}

\section{Extension with newvar}

\begin{enumcirc}
	%
	\item
	%
	Extend the language with variable declaration:
	%
	\begin{center}
		\begin{minipage}[c]{0.6\textwidth}
			\begin{grammar}
				<comm> ::=
				\dots\;
				| newvar <var> := <intexp> in <comm>
			\end{grammar}
		\end{minipage}
	\end{center}
	%
	\item
	%
	How should we modify the $\rightarrow$ relation?
	%
	Add a rule for newvar:
	%
	\begin{enumrm}
		%
		\item
		%
		Option 1:
		%
		\[
			\inferrule
			{\ }
			{
				\chevrons{\cnew{v}{e}{c}, \sigma}
				\rightarrow
				\chevrons{\cseq{c}{v:=n}, \aug{\sigma}{v \mapsto \bbrackets{e}\sigma}}
			}
		\]
		%
		\item
		%
		Option 2
		%
		\[
			\inferrule
			{
				\chevrons{c, \aug{\sigma}{v : \bbrackets{e}\sigma}}
				\rightarrow
				\sigma^\prime
			}
			{
				\chevrons{\cnew{v}{e}{c}, \sigma}
				\rightarrow
				\aug{\sigma^\prime}{v : \bbrackets{e}\sigma}
			}
		\]
		%
		\item
		%
		Both options are acceptable.
		%
		But option 2 is better.
		%
		Only option 2 works when we extend the language with primitives for concurrent
		executions.
		%
	\end{enumrm}
	%
	\item
	%
	Note that we did not change $\Gamma_N$ and $\Gamma_T$.
	%
	Thus, adding newvar doesn't change the operational semantics much.
	%
	In a sense, this small change means that newvar doesn't change the language
	much, either.
	%
\end{enumcirc}

\section{Adding fail} \label{sec:5:fail}

\begin{center}
	\begin{minipage}[c]{0.26\textwidth}
		\begin{grammar}
			<comm> ::=
			\dots\;
			| fail
		\end{grammar}
	\end{minipage}
\end{center}

\begin{enumcirc}
	%
	\item
	%
	When we add fail, we have to change the set $\Gamma_T$ of terminal
	configuration, because we now have two types of terminations, normal one and
	abnormal one.
	%
	\[
		\Gamma_T \defeq \Sigma \cup \braces{\abort} \times \Sigma
		\quad
		(\textrm{or } = \Sigma + \Sigma)
	\]
	%
	$\Gamma_N$ remains unchanged.
	%
	\item
	%
	Since $\Gamma_T$ and so $\Gamma$ are changed, we should change the definition
	of $\rightarrow$.
	%
	We will also have to add a rule for fail.
	%
	Here is the new set of rules.
	%
	\[
		\begin{array}{c}
			\inferrule
			{\ }
			{\chevrons{\cfail, \sigma} \rightarrow \chevrons{\abort, \sigma}}
			\footnotemark
			\qquad \quad
			\inferrule
			{\ }
			{\chevrons{\cskip, \sigma} \rightarrow \sigma}
			\qquad \quad
			\inferrule
			{\ }
			{
				\chevrons{\cassign{v}{e}, \sigma}
				\rightarrow
				\sigma\prths{v \mapsto \bbrackets{e}\sigma}
			}
			\\[2em]
			\inferrule
			{
			}
			{
				\chevrons{\cif{b}{c_1}{c_2}, \sigma}
				\rightarrow
				\chevrons{c_1, \sigma}
			}
			\prths{\bbrackets{b}\sigma = \ttt}
			\\[2em]
			\inferrule
			{
			}
			{
				\chevrons{\cif{b}{c_1}{c_2}, \sigma}
				\rightarrow
				\chevrons{c_2, \sigma}
			}
			\prths{\bbrackets{b}\sigma = \fff}
			\\[2em]
			\inferrule
			{
				\chevrons{c_1, \sigma}
				\rightarrow
				\chevrons{c_1^\prime, \sigma^\prime}
			}
			{
				\chevrons{\cseq{c_1}{c_2}, \sigma}
				\rightarrow
				\chevrons{\cseq{c_1^\prime}{c_2}, \sigma^\prime}
			}
			\qquad \quad
			\inferrule
			{
				\chevrons{c_1, \sigma}
				\rightarrow
				\sigma^\prime
			}
			{
				\chevrons{\cseq{c_1}{c_2}, \sigma}
				\rightarrow
				\chevrons{c_2, \sigma^\prime}
			}
			\\[2em]
			\inferrule
			{
				\chevrons{c_1, \sigma}
				\rightarrow
				\chevrons{\abort, \sigma^\prime}
			}
			{
				\chevrons{\cseq{c_1}{c_2}, \sigma}
				\rightarrow
				\chevrons{\abort, \sigma^\prime}
			}
			\footnotemark
			\\[2em]
			\inferrule
			{
			}
			{
				\chevrons{\cwhile{b}{c}, \sigma}
				\rightarrow
				\sigma
			}
			\prths{\bbrackets{b}\sigma = \fff}
			\\[2em]
			\inferrule
			{
				\chevrons{c, \sigma}
				\rightarrow
				\sigma^\prime
			}
			{
				\chevrons{\cwhile{b}{c}, \sigma}
				\rightarrow
				\chevrons{\cseq{c}{\cwhile{b}{c}}, \sigma^\prime}
			}
			\prths{\bbrackets{b}\sigma = \ttt}
			\\[2em]
			\inferrule
			{
				\chevrons{c, \aug{\sigma}{v : \bbrackets{e}\sigma}}
				\rightarrow
				\chevrons{\abort, \sigma^\prime}
			}
			{
				\chevrons{\cnew{v}{e}{c}, \sigma}
				\rightarrow
				\chevrons{\abort, \aug{\sigma^\prime}{v : \sigma\prths{v}}}
			}
			\footnotemark
			\\[2em]
			\inferrule
			{
				\chevrons{c, \aug{\sigma}{v : \bbrackets{e}\sigma}}
				\rightarrow
				\sigma^\prime
			}
			{
				\chevrons{\cnew{v}{e}{c}, \sigma}
				\rightarrow
				\aug{\sigma^\prime}{v : \sigma\prths{v}}
			}
			\\[2em]
			\inferrule
			{
				\chevrons{c, \aug{\sigma}{v : \bbrackets{e}\sigma}}
				\rightarrow
				\chevrons{c^\prime, \sigma^\prime}
			}
			{
				\chevrons{\cnew{v}{e}{c}, \sigma}
				\rightarrow
				\chevrons{
					\cnew{v}{\sigma^\prime\prths{v}}{c^\prime},
					\aug{\sigma^\prime}{v : \sigma\prths{v}}
				}
			}
		\end{array}
	\]
	\footnoteeqn[-2]{new rule: due to fail}
	\footnoteeqn{new rule: due to the change of $\Gamma_T$}
	\footnoteeqn{new rule: due to the change of $\Gamma_T$}
	%
\end{enumcirc}

\section{Handling input and output}

\begin{center}
	\begin{minipage}[c]{0.4\textwidth}
		\begin{grammar}
			<comm> ::=
			\dots\;
			| !<var>
			| ?<var>
		\end{grammar}
	\end{minipage}
\end{center}

\begin{enumcirc}
	%
	\item
	%
	This time we have to change the form or type of $\rightarrow$.
	%
	It is no longer a binary relation, but a ternary relation.
	%
	\[
		\rightarrow \;\subseteq \Gamma_N \times \Lambda \times \Gamma
	\]
	%
	\[
		\lambda \in \Lambda \defeq
		\braces{\varepsilon}\footnotemark \cup
		\set{\cin{n}}{n \in \Z}\footnotemark \cup
		\set{\cout{n}}{n \in \Z}\footnotemark
	\]
	\footnoteeqn[-2]{transition or execution without input or output}
	\footnoteeqn{transition with an input}
	\footnoteeqn{transition with an output}
	%
	We write
	%
	$\chevrons{c, \sigma} \xrightarrow{\lambda} \gamma$
	%
	to mean
	%
	$\chevrons{\chevrons{c, \sigma}, \lambda, \gamma} \in\; \rightarrow$.
	%
	We also often omit $\lambda$ if $\lambda = \varepsilon$.
	%
	\item
	%
	Why do we make this change?
	%
	It is because adding $\cin{n}$ and $\cout{n}$ to the language makes it
	necessary to describe some aspects of intermediate steps of computations
	explicitly.
	%
	\item
	%
	We include all the rules that (except the ones for $\cseq{c_1}{c_2}$ and
	newvar) that we defined in \cref{sec:5:fail}.
	%
	Of course, the occurrences of $\rightarrow$ in those old rules should be
	understood as $\xrightarrow{\varepsilon}$ with $\varepsilon$ omitted for
	simplicity.
	%
	In addition to these rules, we have the following rules:
	%
	\[
		\begin{array}{c}
			\inferrule
			{
				\
			}
			{
				\chevrons{\cin{v}, \sigma}
				\xrightarrow{\cin{n}}
				\aug{\sigma}{v : \sigma\prths{n}}
			}
			\qquad \quad
			\inferrule
			{
				\
			}
			{
				\chevrons{\cout{e}, \sigma}
				\xrightarrow{\cout{\bbrackets{e}\sigma}}
				\sigma
			}
			\\[2em]
			\inferrule
			{
				\chevrons{c_0, \sigma}
				\xrightarrow{\lambda}
				\sigma^\prime
			}
			{
				\chevrons{\cseq{c_0}{c_1}, \sigma}
				\xrightarrow{\lambda}
				\chevrons{c_1, \sigma^\prime}
			}
			\qquad \quad
			\inferrule
			{
				\chevrons{c_0, \sigma}
				\xrightarrow{\lambda}
				\chevrons{c_0^\prime, \sigma^\prime}
			}
			{
				\chevrons{\cseq{c_0}{c_1}, \sigma}
				\xrightarrow{\lambda}
				\chevrons{\cseq{c_0^\prime}{c_1}, \sigma^\prime}
			}
			\\[2em]
			\inferrule
			{
				\chevrons{c_0, \sigma}
				\xrightarrow{\lambda}
				\chevrons{\abort, \sigma^\prime}
			}
			{
				\chevrons{\cseq{c_0}{c_1}, \sigma}
				\xrightarrow{\lambda}
				\chevrons{\abort, \sigma^\prime}
			}
			\\[2em]
			\inferrule
			{
				\chevrons{c, \aug{\sigma}{v : \bbrackets{e}\sigma}}
				\xrightarrow{\lambda}
				\sigma^\prime
			}
			{
				\chevrons{\cnew{v}{e}{c}, \sigma}
				\xrightarrow{\lambda}
				\aug{\sigma^\prime}{v : \sigma\prths{v}}
			}
			\\[2em]
			\inferrule
			{
				\chevrons{c, \aug{\sigma}{v : \bbrackets{e}\sigma}}
				\xrightarrow{\lambda}
				\chevrons{\abort, \sigma^\prime}
			}
			{
				\chevrons{\cnew{v}{e}{c}, \sigma}
				\xrightarrow{\lambda}
				\chevrons{\abort, \aug{\sigma^\prime}{v : \sigma\prths{v}}}
			}
			\\[2em]
			\inferrule
			{
				\chevrons{c, \aug{\sigma}{v : \bbrackets{e}\sigma}}
				\xrightarrow{\lambda}
				\chevrons{c^\prime, \sigma^\prime}
			}
			{
				\chevrons{\cnew{v}{e}{c}, \sigma}
				\xrightarrow{\lambda}
				\chevrons{
					\cnew{v}{\sigma^\prime\prths{v}}{c^\prime},
					\aug{\sigma^\prime}{v : \sigma\prths{v}}
				}
			}
		\end{array}
	\]
	%
	Whenever an old rule contains a premise, we copy the rule and put $\lambda$
	above $\rightarrow$ in the premise and the conclusion, so that the label
	$\lambda$ gets propagated from the execution of a subcommand to that of the
	original command.
	%
	\item
	%
	This operational semantics corresponds to the denotational semantics that we
	studied.
	%
	The correspondence is formalized by the function $F$ in page 134 of the
	textbook:
	%
	\[
		F\prths{\gamma} =
		\left\{
		\begin{array}{ll}
			\bot                                                 &
			\text{when }
			\gamma \uparrow                                        \\
			\iterm\prths{\sigma'}                                &
			\text{when }
			\gamma \rightarrow^* \sigma'                           \\
			\iabort\prths{\sigma'}                               &
			\text{when }
			\gamma \rightarrow^* (\textit{abort}, \sigma')         \\
			\iout \prths{n, F_{\gamma''}}                        &
			\text{when }
			\exists \gamma' \in \Gamma.
			\; \gamma \rightarrow^* \gamma' \text{ and }
			\gamma' \! \xrightarrow{\cout{n}} \gamma''             \\
			\iin \prths {\lambda n \in \mathbb{Z}. F_{\gamma_n}} &
			\text{when }
			\exists \gamma' \in \Gamma.
			\forall n \in \mathbb{Z}.
			\gamma \rightarrow^* \gamma' \text{ and }
			\gamma' \xrightarrow{\cin{n}} \gamma_n.                \\
		\end{array}
		\right.
	\]
	%
	Intuitively, $F$ runs a configuration until it finishes, or outputs a number,
	or waits for an input.
	%
	$F$ then returns what it gets when it completes this execution.

	In a sense, the correspondence says that the denotational semantics comes from
	the operational semantics after unobservable intermediate states are abstracted
	away.
	%
	For detail, look at the textbook.
	%
\end{enumcirc}


\chapter{An Introduction to Category Theory}

Chapter 8 of Tennent's book

\section{Motivation}
\todo

\chapter{Recursively Defined Domains}

Chapter 10 of Tennent's book

\section{Motivation}

\begin{enumcirc}
	%
	\item
	%
	One reason that we studied category theory is to understand a general principle
	behind the construction of recursively defined domains, such as the following
	$\Omega$ that you encountered before:
	%
	\begin{align*}
		\Omega       & \simeq \prths{\hat{\Sigma} + \Z \times \Omega + \prths{\Z \to \Omega}}_\bot \\
		\hat{\Sigma} & \defeq \Sigma \cup \braces{\abort} \times \Sigma \simeq \Sigma + \Sigma     \\
	\end{align*}
	%
	\item
	%
	If we write the RHS of the above isomorphism as $F\prths{\Omega}$, the formula
	says:
	%
	\[
		\Omega \simeq F\prths{\Omega}
	\]
	%
	That is, $\Omega$ is a fixed point of $F$.
	%
	In fact, $\Omega$ is not just a fixed point, but the best fixed point where
	``the best'' means something very similar to ``the least'' in the standard
	fixed point theorem of the domain theory.
	%
	\item
	%
	We will generalize the standard least fixed point theorem of the domain theory
	and obtain a general categorical least fixed point theorem.
	%
	This generalization closely follows the intuition that categories are
	generalized partially ordered sets (and functors are generalized monotone
	functions).
	%
	Then, we will instantiate our generalization with a particular category
	constructed out of domains and a particular kind of continuous functions called
	embeddings.
	%
\end{enumcirc}

\section{$\omega$-chain and co-limit of $\omega$-chain}

\begin{enumcirc}
	%
	\item
	%
	Let's start by remembering ingredients that we needed when expressing the
	standard least fixed point theorem of the domain theory.
	%
	\begin{enumrm}
		%
		\item
		%
		A partially ordered set $D$ has the least element.
		%
		\item
		%
		Every chain in $D$ has the last upper bound
		%
		\item
		%
		A function $f$ on $D$ is continuous (i.e., monotone and
		chain-limit-preserving).
		%
	\end{enumrm}
	%
	Then, the theorem says that $f$ has the least fixed point $x_0$.
	%
	That is, $f\prths{x_0} = x_0$ and for all $y$ s.t. $f\prths{y} = y$, $x_0 \leq
		y$.\footnote{property that is a bit stronger than $x_0$ being the least fixed
		point}
	%
	\item
	%
	In the categorical generalization of the theorem,
	%
	\begin{enumrm}
		%
		\item
		%
		$D$ becomes a category $\Cc$;
		%
		\item
		%
		$f \in \brackets{D \to D}$ becomes a functor $F : \Cc \to \Cc$;
		%
		\item
		%
		$x_0$ becomes an object in $\Cc$;
		%
		\item
		%
		the least element of $D$ becomes the initial object of $\Cc$;
		%
	\end{enumrm}
	%
	Note that \ul{the monotonicity of $f$} \footnote{preservation of the
		$\sqsubseteq$ relation} translates to $F$'s morphism map being type-checked
	with respect to its object map.
	%
	(i.e., for all $g: x \to y$, $F\prths{g}: F\prths{x} \to F\prths{y}$)
	%
	This translated property is a part of the conditions for $F$ being a functor.
	%
	Thus, the monotonicity already holds for $F$ in a sense.
	%
	\item
	%
	Ok, what remain?
	%
	We still need to generalize
	%
	\begin{enumrm}
		%
		\item
		%
		chains
		%
		\item
		%
		least upper bounds (or limiets) of chains
		%
		\item
		%
		limit-preservation
		%
	\end{enumrm}
	%
	\item
	%
	A \ul{$\omega$-chain} in a category $\Cc$ is a countably infinite sequence of
	objects $\prths{x_0, x_1, x_2, \ldots}$ of $\Cc$ and a collection of morphisms
	$\braces{f_i : x_i \to x_{i+1}}_{i \ge 0}$ in $\Cc$.
	%
	The best way to understand this is to imagine the following figure:
	%
	\[
		\begin{tikzcd}[row sep=large]
			x_0 \arrow[r, "f_0"] & x_1 \arrow[r, "f_1"] & x_2 \arrow[r, "f_2"] & \cdots
		\end{tikzcd}
	\]
	%
	We use $\triangle$ to denote a $\omega$-chain.
	%
	\item
	%
	A \ul{co-cone} of an $\omega$-chain $\triangle = \braces{\prths{x_i, f_i}}_{i
			\ge 0}$ is a pair of object $x$ and a collection of morphisms $\braces{g_i :
			x_i \to x}_{i \ge 0}$ such that for all $i \ge 0$, $g_{i+1} \circ f_i = g_i$,
	i.e., in picture,
	%
	\[
		\begin{tikzcd}[row sep=large]
			x_i \arrow[r, "f_i"] \arrow[dr, "g_i"'] & x_{i+1} \arrow[d, "g_{i+1}"] \\
			& x
		\end{tikzcd}
	\]
	%
	commutes.
	%
	\item
	%
	Intuitively, an $\omega$-chain is a generalized chain, and the object $x$ of a
	co-cone $\prths{x, \braces{g_i}_i}$ of $\triangle$ is a generalized upper bound
	of the chain.
	%
	Each $g_i: x_i \to x$ provides a way to view that $x$ is larger than or equal
	to $x_i$.
	%
	Meanwhile, each $f_i: x_i \to x_{i+1}$ provides a way to view that $x_{i+1}$ is
	larger than or equal to $x_i$.
	%
	The commutativity requirement says that these two views should be compatible.
	%
	\item
	%
	A \ul{co-cone} of an $\omega$-chain $\triangle = \braces{\prths{x_i, f_i}}_{i
			\ge 0}$ is \ul{co-limiting} if for every co-cone $\prths{x', \braces{g'_i}_i}$
	of $\triangle$, there exists a unique morphism $h: x \to x'$ such that for all
	$i$,
	%
	\[
		g'_i = h \circ g_i,
	\]
	%
	in a diagram,
	%
	\[
		\begin{tikzcd}[row sep=large]
			x_i \arrow[dr, "g_i"'] \arrow[ddr, swap, "g'_i", bend right = 30] & \\
			& x \arrow[d, dashed, "!h"] \\
			& x' \\
		\end{tikzcd}
	\]
	%
	Intuitively, the very existence of $h$ says that $x$ is smaller than or equal
	to $x'$.
	%
	The commutativity and the uniqueness say that $h$'s explanation about why $x$
	is smaller than or equal to $x'$ follows automatically and canonically from the
	$g_i$ and the $g'_i$.
	%
	\item
	%
	A co-limiting co-cone $\prths{x, \braces{g_i}_i}$ of an $\triangle$ is a
	generalization of the least upper bound of a chain.
	%
	I usually imagine the following visual image whenever I work with an
	$\omega$-chain, a co-cone, or a co-limiting co-cone:
	%
	\[
		\begin{tikzcd}[row sep=large]
			x_0 \arrow[r, "f_0"] \arrow[drrrr, "g_0", bend right = 20] \arrow[ddrrrr, "g'_0", bend right = 100] &
			x_1 \arrow[r, "f_1"] \arrow[drrr, "g_1", bend right = 20]  \arrow[ddrrr, "g'_1", bend right = 100]  &
			x_2 \arrow[r, "f_2"] \arrow[drr, "g_2", bend right = 20]   \arrow[ddrr, "g'_2", bend right = 100]   &
			x_3 \arrow[r, "f_3"] \arrow[dr, "g_3", bend right = 20]    \arrow[ddr, "g'_3", bend right = 100]    & \cdots \\
			& & & & x \arrow[d, dashed, "h"] \\
			& & & & x'
		\end{tikzcd}
	\]
	%
	\begin{exercisetab}
		%
		Construct co-limiting co-cones of $\omega$-chains in the category of Set.
		%
		Do the same thing in the poset (partially ordered set) category $\prths{2^\N,
				\subseteq}$ and in the category of predomains.
		%
	\end{exercisetab}
	%
\end{enumcirc}

\section{$\omega$-continuous functor}

\begin{enumcirc}
	%
	\item
	%
	Let $\Cc$ and $\Dc$ be categories that have {co-limiting co-cones}\footnote{in
		other words, co-limits} for all $\omega$-chains.
	%
	We will call such categories as \ul{chain-complete categories}.
	%
	\item
	%
	A functor $F: \Cc \to \Dc$ is \ul{$\omega$-continuous} if it maps a co-limit of
	an $\omega$-chain to a co-limit of an $\omega$-chain, that is,
	%
	\begin{align*}
		\textrm{for every } \omega\textrm{-chain } & \triangle = \braces{\prths{x_i, f_i}}_{i \ge 0} \textrm{ in } \Cc,                               \\
		\textrm{for every co-limit }               & \prths{x, \braces{g_i}_i} \textrm{ of } \triangle,                                               \\
		                                           & \prths{F\prths{x}, \braces{F\prths{g_i}}_i} \textrm {is a co-limit of }                          \\
		                                           & F\prths{\triangle} = \prths{\braces{F\prths{x_i}}_i, \braces{F\prths{f_i}}_i} \textrm{ in } \Dc.
	\end{align*}
	%
	\item
	%
	Intuitively, this $\omega$-continuity of $F$ means that $F$ preserves the least
	upper bound of an increasing chain.
	%
	\item
	%
	Note that a functor $F$ always maps a co-cone of an $\omega$-chain to a co-cone
	of an $\omega$-chain: visually,
	%
	\[
		\qquad \quad
		\begin{tikzcd}[row sep=large]
			x_0 \arrow[r, "f_0"] \arrow[drr, "g_0", bend right = 30] &
			x_1 \arrow[r, "f_1"] \arrow[dr, "g_1", bend right = 30] &
			x_2 \arrow[r, "f_2"] \arrow[d, "g_2", bend right = 30] & \cdots \\
			& & x
		\end{tikzcd}
		\qquad \qquad
		\textrm{category } \Cc
	\]
	%
	\[
		\begin{tikzcd}
			F\prths{x_0} \arrow[r, "F\prths{f_0}"] \arrow[drr, "F\prths{g_0}", bend right = 30] &
			F\prths{x_1} \arrow[r, "F\prths{f_1}"] \arrow[dr, "F\prths{g_1}", bend right = 30] &
			F\prths{x_2} \arrow[r, "F\prths{f_2}"] \arrow[d, "F\prths{g_2}", bend right = 30] & \cdots \\
			& & F\prths{x}
		\end{tikzcd}
		\qquad
		\textrm{category } \Dc
	\]
	%
	If the diagram in $\Cc$ commutes, the diagram in $\Dc$ also commutes.
	%
	This is because the preservation of $\circ$ and $\id$ by a functor implies that
	the functor maps every commuting diagram to a commuting diagram.

	The situation is similar to the fact that a monotone function $f$ from a
	predomain to a predomain maps an upper bound of a chain to an upper bound of a
	chain.
	%
	\item
	%
	However, the functoriality of $F$ doesn't ensure that if $\prths{x,
			\braces{g_i}_i}$ is a co-limit of $\triangle$, so is $\prths{F\prths{x},
			\braces{F\prths{g'_i}}_i}$.

	When $F$ satisfies this additional property, we say that $F$ is
	$\omega$-continuous.
	%
	\begin{exercisetab}
		%
		Show that the functor from Set to Set
		%
		\begin{align*}
			F\prths{S} & = \Z \times S                                                   \\
			F\prths{f} & = \id_\Z \times f = \lambda \prths{n, s}. \prths{n, f\prths{s}}
		\end{align*}
		%
		is $\omega$-continuous.
		%
	\end{exercisetab}
	%
\end{enumcirc}

\section{Fixed point theorem}

\begin{theorem}
	%
	Let $\Cc$ be a category with an initial object $x_0$.
	%
	Assume that $\Cc$ is chain-complete (i.e., every $\omega$-chain $\triangle$ in
	$\Cc$ has a co-limit).
	%
	Then, for every $\omega$-continuous functor $F: \Cc \to \Cc$, there exists an
	object $\subsctext{x}{fix}$ in $\Cc$ and a morphism $\eta:
		F\prths{\subsctext{x}{fix}} \to \subsctext{x}{fix}$ such that
	%
	\begin{enumrm}
		%
		\item
		%
		$\eta$ is an isomorphism, i.e.,
		%
		$\exists$ a morphism $\psi : \subsctext{x}{fix} \to F\prths{\subsctext{x}{fix}}$ s.t.
		%
		$\eta \circ \psi = \id_{\subsctext{x}{fix}}$ and $\psi \circ \eta = \id_{F\prths{\subsctext{x}{fix}}}$;
		%
		\item
		%
		for every morphism $\eta': F\prths{y} \to y$, there exists a unique morphism
		$\rho: \subsctext{x}{fix} \to y$ such that
		%
		\[
			\begin{tikzcd}
				F\prths{\subsctext{x}{fix}} \arrow[r, "\eta"] \arrow[d, swap, "F\prths{\rho}"] & \subsctext{x}{fix} \arrow[d, dashed, "\rho"] \\
				F\prths{y} \arrow[r, "\eta'"] & y
			\end{tikzcd}
		\]
		%
	\end{enumrm}
	%
\end{theorem}
%
\begin{enumcirc}
	%
	\item
	%
	Intuitively, the theorem says that $subsctext{x}{fix}$ is the least fixed point
	of $F$.
	%
	The condition (i) says that $\subsctext{x}{fix}$ is a fixed point.
	%
	The condition (ii) says that it is the least fixed point.
	%
	\item
	%
	The proof is complex, but note that difficult.
	%
	Very similar to the proof of the standard fixed point theorem of the domain
	theory.
	%
	We will study only some parts of the proof.
	%
	\item
	%
	The key part of the proof is to construct $\subsctext{x}{fix}$.
	%
	Here we use the initial object $x_0$ of $\Cc$, the functoriality of $F$, and
	the chain completeness of $\Cc$
	%
	(They correspond to $\bot$, the monotonicity of a continuous function $f$ and the chain completeness of a predomain $D$
	in the proof of the fixed-point theorem in domain theory).

	We construct a chain:
	%
	\[
		\triangle \defeq
		\begin{tikzcd}[row sep=large]
			x_0 \arrow[r, "f_0"] & F\prths{x_0} \arrow[r, "f_1"] & F^2\prths{x_0} \arrow[r, "f_2"] & F^3\prths{x_0} \arrow[r, "f_3"] & \cdots
		\end{tikzcd}
	\]
	%
	where $f_0$ is a unique morphism from the initial object $x_0$ to
	$F\prths{x_0}$,
	%
	and $f_i = F^i\prths{f_0}$ is a unique morphism from $F^i\prths{x_0}$ to
	$F^{i+1}\prths{x_0}$.

	Since $\Cc$ is chain-complete, there exists a co-limit $\prths{x,
			\braces{g_i}}$ of the chain $\triangle$ that we just built.
	%
	\[
		\begin{tikzcd}[row sep=large]
			x_0 \arrow[r, "f_0"] \arrow[drrr, "g_0", bend right = 20]          &
			F\prths{x_0} \arrow[r, "F\prths{f_0}"] \arrow[drr, "g_1", bend right = 20]  &
			F^2\prths{x_0} \arrow[r, "F^2\prths{f_0}"] \arrow[dr, "g_2", bend right = 20] &
			F^3\prths{x_0} \arrow[r, "F^3\prths{f_0}"] \arrow[d, "g_3", bend right = 20]  & \cdots \\
			& & & \subsctext{x}{fix}  \\
		\end{tikzcd}
	\]
	%
	\item
	%
	Next we build $\eta: F\prths{\subsctext{x}{fix}} \to \subsctext{x}{fix}$.
	%
	Here we use the $\omega$-continuity of $F$.
	%
	Apply $F$ to the diagram.
	%
	This gives us
	%
	\[
		\begin{tikzcd}[row sep=large]
			\triangle' \defeq
			F\prths{x_0} \arrow[r, "F\prths{f_0}"] \arrow[drrr, "F\prths{g_0}", bend right = 20]          &
			F^2\prths{x_0} \arrow[r, "F^2\prths{f_0}"] \arrow[drr, "F\prths{g_1}", bend right = 20]  &
			F^3\prths{x_0} \arrow[r, "F^3\prths{f_0}"] \arrow[dr, "F\prths{g_2}", bend right = 20] &
			F^4\prths{x_0} \arrow[r, "F^4\prths{f_0}"] \arrow[d, "F\prths{g_3}", bend right = 20]  & \cdots \\
			& & & F\prths{\subsctext{x}{fix}}  \\
		\end{tikzcd}
		.
	\]
	%
	Since $F$ is $\omega$-continuous, this is a \ul{co-limiting} co-cone of the
	chain $\triangle'$, which is the $x_0$-truncated version of $\triangle$.
	%
	\item
	%
	Because $x_0$ is the initial object, we can add $x_0$ to the diagram and get a
	\ul{co-limiting} co-cone,
	%
	\[
		\begin{tikzcd}[row sep=large]
			x_0 \arrow[r, "f_0"] \arrow[drrrr, "g_0'", bend right = 30]          &
			F\prths{x_0} \arrow[r, "F\prths{f_0}"] \arrow[drrr, "F\prths{g_0}", bend right = 20]          &
			F^2\prths{x_0} \arrow[r, "F^2\prths{f_0}"] \arrow[drr, "F\prths{g_1}", bend right = 20]  &
			F^3\prths{x_0} \arrow[r, "F^3\prths{f_0}"] \arrow[dr, "F\prths{g_2}", bend right = 20] &
			F^4\prths{x_0} \arrow[r, "F^4\prths{f_0}"] \arrow[d, "F\prths{g_3}", bend right = 20]  & \cdots \\
			& & & & F\prths{\subsctext{x}{fix}}  \\
		\end{tikzcd}
	\]
	%
	where $g_0'$ is a unique morphism from $x_0$ to $F\prths{\subsctext{x}{fix}}$.
	%
	The leftmost triangle commutes because of the initiality of $x_0$.

	Now we have two co-limits, $\subsctext{x}{fix}$ and
	$F\prths{\subsctext{x}{fix}}$, of the same chain $\triangle$.
	%
	One general result (which is easy to show) is that two co-limits of a chain are
	isomorphic, which means in our case that there exist morphisms $\eta:
		F\prths{\subsctext{x}{fix}} \to \subsctext{x}{fix}$ and $\psi:
		\subsctext{x}{fix} \to F\prths{\subsctext{x}{fix}}$ such that
	%
	\[
		\psi \circ \eta = \id_{F\prths{\subsctext{x}{fix}}} \quad \textrm{ and } \quad \eta \circ \psi = \id_{\subsctext{x}{fix}}.
	\]
	%
	We just proved (i) of the theorem.
	%
	We leave the proof of (ii) as an exercise.
	%
	\item
	%
	Why is (ii) in the theorem useful?
	%
	Because it allows us to define a morphism from $\subsctext{x}{fix}$ to some
	$y$.
	%
\end{enumcirc}

\section{Famous example of the fixed point theorem}

\begin{enumcirc}
	%
	\item
	%
	One big motivation for developing domain theory was to find a solution of the
	following equation for space $D$:
	%
	\begin{equation} \label{eq:space}
		D \simeq D \to D.
	\end{equation}
	%
	Such a space is needed (as you will see later in the course) to define a
	mathematical or denotational semantics of the untyped lambda calculus, which
	forms the core of most functional programming languages.
	%
	\item
	%
	But note that if $D$ is a set and $D \to D$ is the set of all functions on $D$,
	the only solution of \cref{eq:space} is the singleton set (of course, in this
	case, $\simeq$ means some bijection between two sets).
	%
	This is because if $D$ contains more than one element, the cardinality of
	$\brackets{D \to D}$ is always strictly larger than that of $D$.
	%
	\item
	%
	Using domains, we can find a solution of \cref{eq:space} (using the fixed point
	theorem).
	%
	But we have to be careful about defining a category on which we apply the
	theorem.
	%
	\item
	%
	Here is the category $\Dom[EP]$ that we use.
	%
	\begin{enumrm}
		%
		\item
		%
		Objects of $\Dom[EP]$ are domains; i.e., partially ordered set where all chains
		have least upper bounds and the least element exists.
		%
		\item
		%
		Morphisms from a domain $D$ to a domain $D'$ are strict (i.e.
		$\bot$-preserving) continuous functions $f$ from $D$ to $D'$ such that there
		exists a continuous function $g: D' \to D$ with
		%
		\[
			g \circ f = \id_D \quad \textrm{ and } \quad f \circ g \sqsubseteq \id_{D'}.
		\]
		%
		\item
		%
		$\circ$ is the usual function composition.
		%
		\item
		%
		$\id_D$ is the identity function on $D$.
		%
	\end{enumrm}
	%
	\item
	%
	Compared with the category $\Dom$ of domains and continuous functions, this
	category $\Dom[EP]$ has a rather unusual notion of morphisms.
	%
	In $\Dom[EP]$, a morphism $f: D \to D'$ should be not just continuous, but also
	strict.
	%
	More importantly, it should have $g: D' \to D$ such that
	%
	\[
		g \circ f = \id_D \quad \textrm{ and } \quad f \circ g \sqsubseteq \id_{D'}.
	\]
	%
	In picture,
	%
	\[
		\begin{tikzcd}[sep=3em]
			D \arrow[r, "f"] \arrow[rd, "\id_{D}"'] & D' \arrow[d, "g"] \\
			& D
		\end{tikzcd}
		\quad \textrm{ and } \quad
		\begin{tikzcd}[column sep= 1.2em, row sep=1.3em]
			D' \arrow[rr, "g"] \arrow[rrdd, "\id_{D'}"'] & & D \arrow[dd, "f"] \\
			& \arrow[ur, phantom, "\sqsupseteq" marking] & \\
			& & D'
		\end{tikzcd}
	\]
	%
	Intuitively, the existence of such $g$ means that $D'$ is built by putting
	additional elements above existing elements of $D$, and $g$ maps all additional
	elements to their best under-approximations in $D$.
	%
	Here is some picture that shows this intuition.
	%
	\begin{center}
		\begin{tikzpicture}
			% Define nodes
			\node[circle, fill=black, inner sep=0pt, minimum size=5pt] (b00) at (0,0) {};
			\node[circle, fill=black, inner sep=0pt, minimum size=5pt] (b10) at (-0.5,1) {};
			\node[circle, fill=black, inner sep=0pt, minimum size=5pt] (b11) at (0.5,1) {};
			\node[circle, fill=black, inner sep=0pt, minimum size=5pt] (b20) at (-1,2) {};
			\node[circle, fill=black, inner sep=0pt, minimum size=5pt] (b21) at (0,2) {};
			\node[circle, fill=black, inner sep=0pt, minimum size=5pt] (b22) at (1,2) {};
			\node[circle, fill=black, inner sep=0pt, minimum size=5pt] (b23) at (2,2) {};
			\node[circle, fill=black, inner sep=0pt, minimum size=5pt] (b30) at (1.5,3) {};

			\node[circle, fill=red, inner sep=0pt, minimum size=5pt] (r30) at (-1.5,3) {};
			\node[circle, fill=red, inner sep=0pt, minimum size=5pt] (r31) at (-0.5,3) {};
			\node[circle, fill=red, inner sep=0pt, minimum size=5pt] (r32) at (0.5,3) {};
			\node[circle, fill=red, inner sep=0pt, minimum size=5pt] (r33) at (2.5,3) {};
			\node[circle, fill=red, inner sep=0pt, minimum size=5pt] (r40) at (-1,4) {};
			\node[circle, fill=red, inner sep=0pt, minimum size=5pt] (r41) at (0,4) {};
			\node[circle, fill=red, inner sep=0pt, minimum size=5pt] (r42) at (1,4) {};
			\node[circle, fill=red, inner sep=0pt, minimum size=5pt] (r43) at (2,4) {};
			\node[circle, fill=red, inner sep=0pt, minimum size=5pt] (r44) at (3,4) {};
			\node[circle, fill=red, inner sep=0pt, minimum size=5pt] (r45) at (4,4) {};

			% Draw black solid edges
			\draw[thick] (b00) -- (b10);
			\draw[thick] (b00) -- (b11);
			\draw[thick] (b10) -- (b20);
			\draw[thick] (b10) -- (b21);
			\draw[thick] (b11) -- (b21);
			\draw[thick] (b11) -- (b22);
			\draw[thick] (b11) -- (b23);
			\draw[thick] (b22) -- (b30);

			% Draw red dashed edges
			\draw[thick, red] (b20) -- (r30);
			\draw[thick, red] (b20) -- (r31);
			\draw[thick, red] (b21) -- (r32);
			\draw[thick, red] (b23) -- (r33);
			\draw[thick, red] (r30) -- (r40);
			\draw[thick, red] (r31) -- (r40);
			\draw[thick, red] (r32) -- (r41);
			\draw[thick, red] (b30) -- (r42);
			\draw[thick, red] (b30) -- (r43);
			\draw[thick, red] (r33) -- (r44);
			\draw[thick, red] (r33) -- (r45);

			\node[color=red] (DD) at (-2.5,3) {$D-D'$};
			\node[] (D) at (-2.5,2) {$D$};

			\draw[->, bend left=40, color=blue, thick] (r33) to node[pos=0.5, right=0.2em] {$g$} (b23);
			\draw[->, bend left=50, color=blue, thick] (r45) to node[pos=0.5, right=0.2em] {$g$} (b23);
			\draw[->, bend right=50, color=blue, thick] (r44) to node[pos=0.5, right=0.2em] {$g$} (b23);
			\draw[->, color=blue, thick] (b23) to [out=220, in=310, looseness=20] node[pos=0.5, below=0.2em] {$g$} (b23);

		\end{tikzpicture}
	\end{center}
	%
	Where $D-D'$ is the set of additional elements in $D'$ and $g$ maps elements of
	$D'$ to their best under-approximations in $D$.
	%
	What this means is that a morphism $f: D \to D'$ is really saying that $D'$ is
	larger than $D$.
	%
	\item
	%
	A morphism $f: D \to D'$ in $\Dom[EP]$ is called \ul{embedding} and a
	corresponding $g: D' \to D$ is called \ul{projection}.

	If you happen to take a course on program analysis, this pair of embedding $f$
	and projection $g$ is closely related to the Galois connection there.

	Note That the definition doesn't say that there exists only one projection $g$
	for a given embedding $f$.
	%
	That is, there may be multiple projections.
	%
	But this doesn't happen.

	\begin{lemma}
		%
		For every embedding $f: D \to D'$, and projections $g_1, g_2: D' \to D$ for
		$f$, we have that
		%
		\[
			g_1 = g_2.
		\]
		%
	\end{lemma}
	%
	\begin{proof}
		%
		We will show that $g_1 \sqsubseteq g_2$.
		%
		A similar argument can show the opposite inequality.
		%
		\[
			g_1 = \id_D \circ g_1 =
			\prths{g_2 \circ f} \circ g_1 =
			g_2 \circ \prths{f \circ g_1} \sqsubseteq
			g_2 \circ \id_{D'} =
			g_2.
		\]
		%
	\end{proof}
	%
	We write $f^\textrm{P}$ to denote the unique projection for an embedding $f$.
	%
	\item
	%
	The category $\Dom[EP]$ has an initial object, which is a singleton domain
	$\prths{\braces{\bot}, \sqsubseteq}$\footnote{$x \sqsubseteq y$ iff $x = y$}.
	%
	It also has a co-limit for any $\omega$-chain.
	%
	We will not prove this.
	%
	But we point out one important property of these co-limits.

	\begin{lemma} \label{lem:co-limit}
		%
		Consider the following co-cone of an $\omega$-chain $\triangle$:
		%
		\[
			\triangle =
			\begin{tikzcd}[row sep=large]
				D_0 \arrow[r, "f_0"] \arrow[drr, "h_0", bend right = 30] &
				D_1 \arrow[r, "f_1"] \arrow[dr, "h_1", bend right = 30] &
				D_2 \arrow[r, "f_2"] \arrow[d, "h_2", bend right = 30] & \cdots \\
				& & D'
			\end{tikzcd}
		\]
		%
		Let $h_i^\textrm{P}$ be the projection for the embedding $h_i$.
		%
		Then, $D'$ is \ul{co-limiting} if and only if
		%
		\[
			\bigsqcup_{i=0}^\infty \prths{h_i \circ h_i^\textrm{P}} = \id_{D'}.
		\]
		%
		Note that $h_i \circ h_i^\textrm{P} \sqsubseteq \id_{D'}$.
		%
		We can easily show that $\braces{h_i \circ h_i^\textrm{P}}_{i \ge 0}$ is an
		increasing chain in $\brackets{D' \toc D'}$.
		%
		The condition says that the least upper bound of the chain is $\id_{D'}$.
		%
	\end{lemma}
	%
	This lemma says that \ul{the order on morphisms} plays an important role in
	deciding whether $D'$ is a co-limit or not.
	%
	One important consequence is the following lemma.
	%
	\begin{lemma} \label{lem:omega-cont}
		%
		A functor $F: \Dom[EP] \to \Dom[EP]$ is $\omega$-continuous if for every
		co-cone of an $\omega$-chain $\triangle$
		%
		\[
			\begin{tikzcd}[row sep=large]
				D_0 \arrow[r, "f_0"] \arrow[drr, "h_0", bend right = 30] &
				D_1 \arrow[r, "f_1"] \arrow[dr, "h_1", bend right = 30] &
				D_2 \arrow[r, "f_2"] \arrow[d, "h_2", bend right = 30] & \cdots \\
				& & D'
			\end{tikzcd}
			,
		\]
		%
		\[
			\bigsqcup_{i=0}^\infty \prths{h_i \circ h_i^\textrm{P}} = \id_{D'}
			\implies
			\bigsqcup_{i=0}^\infty \prths{F\prths{h_i} \circ F\prths{h_i^\textrm{P}}} = \id_{F\prths{D'}}.
		\]
		%
	\end{lemma}
	%
	\begin{proof}
		%
		This is a direct consequence of the previous lemma.
		%
	\end{proof}
	%
	\item
	%
	\cref{lem:omega-cont} is our tool to check the $\omega$-continuity of a functor $F$ on
	$\Dom[EP]$.
	%
	If this check passes, by the fixed point theorem, we know that there exists a
	domain $D$ such that
	%
	\[
		F\prths{D} \simeq D
	\]
	%
	\begin{enumrm}
		%
		\item
		%
		Our functor
		%
		$F\prths{\Omega} = \prths{\Sigma + \Sigma + \Z \times \Omega + \brackets{\Z \to \Omega}}_\bot$
		%
		is an example of such a functor.
		%
		\item
		%
		Another famous example is the following $G$ that defines the function space.

		$G\prths{D} = \brackets{D \toc D}$ \dots the domain of continuous functions in $D$.

		For every $f: D \to D'$,
		%
		\[
			G\prths{f}: G\prths{D} \to G\prths{D'} \textrm{ i.e., } \brackets{D \toc D} \to \brackets{D' \toc D'}
		\]
		%
		\[
			G\prths{f}\prths{h} = f \circ h \circ f^\textrm{P}\footnotemark.
		\]
		\footnoteeqn[0]{
			%
			We are using the projection of $f$ here.
			%
			If $f$ were just a continuous function, we couldn't do it.
			%
		}
		%
	\end{enumrm}
	%
	\begin{exercise}
		%
		Prove that $F$ and $G$ are indeed $\omega$-continuous.
		%
	\end{exercise}
	%
	\begin{exercise}
		%
		Prove \cref{lem:co-limit}.
		%
		(This is not easy)
		%
	\end{exercise}
	%
	If you are familiar with program analysis and abstract interpretation, you
	might have noticed that there we do something similar when defining the
	abstract domain for function space.
	%
	One thing to keep in mind is that in domain theory, $a \sqsubseteq b$ means
	that $b$ is more informative than $a$, while in program analysis, $a
		\sqsubseteq b$ means that $a$ is more informative than $b$.
	%
\end{enumcirc}

\chapter{The Lambda Calculus}

\section{Motivation}

\begin{enumcirc}
	%
	\item
	%
	Most programming languages support a mechanism for declaring functions and
	applying them to arguments.
	%
	In fact, in functional languages, such as Ocaml, Haskell, Clojure and Scala (to
	some extent),
	%
	function declaration and application (not state access and update) is the main
	device of computation.
	%
	\item
	%
	The lambda calculus is a simple formal language that lets us study principles
	behind function declaration and application without being distracted by the
	complexities of usual real-world programming languages.
	%
	If forms the basis of many functional languages.
	%
	Also, it can be used to define a notion of computability.
	%
	\item
	%
	One interesting construct of the lambda calculus is so-called lambda
	abstraction.
	%
	\[
		\lambda x. e
	\]
	%
	which denotes a function with formal argument $x$ and body $e$.
	%
	Nowadays most mainstream languages (c++, Java, Python, etc.) support this
	construct.
	%
	The lambda abstraction is particularly useful when we use higher-order
	functions.
	%
	For instance, to express
	%
	\[
		\int_{0}^{1} dx \int_{0}^{x} dy \prths{x+y}^2
	\]
	%
	in a programming language with the ``\ul{integrate}'' primitive, we can write

	\todo

\end{enumcirc}

\section{Syntax}

\begin{center}
	\begin{minipage}{0.5\textwidth}
		\begin{grammar}
			<exp> ::= <var>
			| <exp> <exp> \footnotemark
			| $\lambda$ <var> . <exp> \footnotemark
		\end{grammar}
	\end{minipage}
\end{center}
\footnoteeqn[-1]{called application}
\footnoteeqn{called abstraction or lambda expression}

\begin{enumcirc}
	%
	\item
	%
	examples
	%
	\todo
	%
	\item
	%
	The set of free variables (in an expression of the lambda calculus), the
	substitution operator, and the $\alpha$-equivalence (or renaming equivalence)
	for the lambda-calculus expressions are defined as you expect.
	%
	Once you get the idea that the lambda operator in $\lambda x. e$ binds the
	variable $x$ in the expression $e$, just as the quantifier $\forall$ in
	$\forall x. p$ binds the variable $x$ in an assertion $p$.
	%
	\begin{enumrm}
		%
		\item
		\begin{align*}
			\fv{v}            & = \braces{v}                  \\
			\fv{e_1 \; e_2}   & = \fv{e_1} \cup \fv{e_2}      \\
			\fv{\lambda v. e} & = \fv{e} \setminus \braces{v} \\
		\end{align*}
		%
		\item
		%
		$\delta$ is a substitution, i.e., a map from $\chevrons{var}$ to $\chevrons{exp}$:
		%
		\begin{align*}
			v / \delta                    & = \delta\prths{v}                                                             \\
			\prths{e_1 \; e_2} / \delta   & = \prths{e_1 / \delta} \prths{e_2 / \delta}                                   \\
			\prths{\lambda v. e} / \delta & = \lambda \subsctext{v}{new}. \prths{e / \aug{\delta}{v: \subsctext{v}{new}}} \\
			where \subsctext{v}{new}      & \notin \bigcup_{w \in \fv{e} \setminus \braces{v}} \fv{\delta\prths{w}}
		\end{align*}
		%
		\item
		%
		The renaming or change of bound variable means the operation of replacing an
		occurrence of $\lambda v. e$ by
		%
		$\lambda \subsctext{v}{new}. \prths{\subst{e}{v}{\subsctext{v}{new}}}$,
		%
		\footnote{
			This means the substitution that maps all variables to themselves except $v$, which
			it maps to $\subsctext{v}{new}$.
		}
		%
		Two expressions $e_1$ and $e_2$ are \ul{$\alpha$-equivalent} or
		\ul{renaming-equivalent} if we can obtain $e_2$ from $e_1$ by applying this
		renaming operation to some subexpressions of $e_1$ zero or multiple times.
		%
		We write $e_1 \equiv e_2$ to denote their $\alpha$-equivalence.
		%
		\begin{example}
			\todo
		\end{example}
		%
	\end{enumrm}
\end{enumcirc}

\section{Reduction}

\begin{enumcirc}
	%
	\item
	%
	When we studied operational semantics, we used transition relation
	$\rightarrow$ to model one-step computation.
	%
	We do something similar for the lambda calculus.
	%
	We define a binary relation
	%
	$\rightarrow \in \chevrons{exp} \times \chevrons{exp}$,
	%
	called \ul{contraction relation}, which models or formalizes single-step
	computation.
	%
	Then, we will call the reflexive transitive closure of $\rightarrow$, i.e.,
	$\rightarrow^*$, the \ul{reduction relation}.
	%
	\item
	%
	Here is the definition of the contraction relation using inference rules.
	%
	\begin{enumrm}
		%
		\item
		%
		$\beta$-reduction
		%
		\[
			\inferrule
			{\;}
			{\prths{\lambda v. e}{e^\prime} \rightarrow \subst{e}{v}{e^\prime}}
		\]
		%
		\item
		%
		Renaming
		%
		\[
			\inferrule
			{e_0 \rightarrow e_1 \\ e_1 \equiv e_1^\prime}
			{e_0 \rightarrow e_1^\prime}
		\]
		%
		\item
		%
		Contextual closure
		%
		\[
			\inferrule
			{e_0 \rightarrow e_1}
			{e_0^\prime \rightarrow e_1^\prime}
		\]
		%
		$e_1^\prime$ is obtained from $e_0^\prime$ by replacing one occurrence of
		$e_0$ in $e_0^\prime$ by $e_1$.
	\end{enumrm}
	%
	\item
	%
	The real change happens by the first rule, $\beta$-reduction.
	%
	The second rule means that the contraction relation is defined on
	$\alpha$-equivalence classes.
	%
	The third rule says that any subexpression of a given $e$ can be contracted by
	the $\beta$-reduction rule.
	%
	Here are a few examples that may help you to see what goes on here.
	%
	\todo
	%
	\item
	%
	Because of the third rule, the contraction relation is not deterministic.
	%
	That is, $e_0 \rightarrow e_1$ and $e_0 \rightarrow e_2$ do not imply that $e_1
		\equiv e_2$.
	%
	For a counterexample, look at the example above.
	%
	However, this nondeterminism comes from any nondeterministic constructs of the
	lambda calculus (which do not exist).
	%
	The following theorem shows one consequence of this, and expresses that the
	contraction relation is essentially deterministic.
	%
	\begin{property}[Church-Rosser Theorem]
		%
		\label{church-rosser}
		\ \\
		%
		If $e \rightarrow^* e_0$\footnote{ reflexive transitive closure of
			$\rightarrow$ } and $e \rightarrow^* e_1$, then there exists $e_2$ s.t. $e_0
			\rightarrow^* e_2$ and $e_1 \rightarrow^* e_2$.
		%
	\end{property}
	%
	\item
	%
	An expression $e$ \ul{is a $\beta$-normal form} if it cannot be contracted.
	%
	Intuitively, such an expression denotes the final result of some computation,
	and the reduction relation $\rightarrow^*$ performs computation by transforming
	a given expression to normal form.
	%
	\cref{church-rosser} implies that every expression can be reduced to at most
	one normal-form expression modulo $\alpha$-equivalence.
	%
	\begin{property}
		%
		\ \\
		If $e_0 \rightarrow^* e_1$, $e_0 \rightarrow e_2$ and $e_1$ and $e_2$ are
		$\beta$-normal forms, then $e_1 \equiv e_2$. (i.e. $e_1$ and $e_2$ are
		$\alpha$-equivalent)
		%
	\end{property}
	%
	\begin{proof}
		%
		By \cref{church-rosser}, $\exists e_3$ s.t.
		%
		$e_1 \rightarrow^* e_3$ and $e_2 \rightarrow^* e_3$.
		%
		Since $e_1$ and $e_2$ are $\beta$-normal forms,
		%
		$e_1 \equiv e_3$ and $e_2 \equiv e_3$.
		%
		$\therefore e_1 \equiv e_2$.
		%
	\end{proof}
	%
	\item
	%
	One natural question is whether we can find a good strategy to use the
	nondeterminism in the third ``Contextual closure'' rule, so that if an
	expression $e$ can be reduced to a normal form, this strategy indeed transforms
	$e$ to such a normal-form expression.

	To see this issue, consider the following two reduction sequences:
	%
	\begin{align*}
		\prths{\lambda u. \lambda v. v}
		\prths{
			\prths{\lambda x. x \; x}
			\prths{\lambda x. x \; x}
		}
		 & \rightarrow \lambda v. v \\
		\prths{\lambda u. \lambda v. v}
		\prths{
			\prths{\lambda x. x \; x}
			\prths{\lambda x. x \; x}
		}
		 & \rightarrow
		\prths{\lambda u. \lambda v. v}
		\prths{
			\prths{\lambda x. x \; x}
			\prths{\lambda x. x \; x}
		}                           \\
		 & \rightarrow \cdots
	\end{align*}
	%
	Only the first gives an expression in the normal form.
	%
	\item
	%
	\ul{The normal-order reduction} is a particular way of using the ``Contextual closue'' rule.
	%
	It picks a redex\footnote{reducible expression} ($\beta$-redex) in an
	expression $e$ that is not included in any other redex.
	%
	Also, if there are multiple such \ul{outermost} redexes, it picks the
	\ul{leftmost} one.
	%
	In our example, the normal-order reduction doesn't pick the redex
	%
	$\prths{\lambda x. x \; x} \prths{\lambda x. x \; x}$
	%
	because it is included in the redex
	%
	$\prths{\lambda u. \lambda v. v} \prths{\prths{\lambda x. x \; x} \prths{\lambda x. x \; x}}$.
	%
	The normal-order reduction is also called \ul{outermost} \ul{leftmost}
	reduction.
	%
	\begin{property}
		%
		If $e \rightarrow^* e^\prime$ for some normal-form $e^\prime$,
		%
		then $e \rightarrow^*_\text{normal} e^\prime$
		%
		where $\rightarrow^*_\text{normal}$ means the contraction relation of the
		normal-order reduction.
		%
	\end{property}
	%
\end{enumcirc}

\section{Normal-order evaluation and eager evaluation}

\begin{enumcirc}
	%
	\item
	%
	In functional languages, we use restricted versions of the reduction relation
	to specify how function calls should be handled.
	%
	We will look at two well-known restrictions used in Haskell and Ocaml, and call
	them normal-order evaluation and eager evaluation.
	%
	Note that we use the world ``evaluation'' instead of ``reduction'' or
	``contraction''.
	%
	\item
	%
	Both normal-order evaluation and eager evaluations are defined for closed
	expressions only, i.e., expressions that do not have any free variables.
	%
	Also, they are formalized as big-step semantics where the evaluation relation
	$\Rightarrow$ transforms an expression to a final result in one go, instead of
	in multiple steps in the reduction relation.
	%
	Finally, these evaluations do not contract any subexpressions inside lambda.
	%
	Thus, their results might not be normal forms.
	%
	They will instead be a \ul{canonical form}.
	%
	\item
	%
	Normal-order evaluation $\Rightarrow$:
	%
	\[
		e\footnotemark \Rightarrow z\footnotemark
	\]
	\footnoteeqn[-1]{closed expression}
	\footnoteeqn{expression in the canonical form (i.e., lambda expression)}
	%
	\ul{A canonical form $z$} is a lambda expression.

	Canonical Forms
	%
	\[
		\inferrule{\;}{\lambda v. e \Rightarrow \lambda v. e}
	\]

	Application ($\beta$-evaluation)
	%
	\[
		\inferrule
		{e \Rightarrow \lambda v . e'' \\ \prths{\subst{e''}{v}{e'}} \Rightarrow z}
		{e \; e' \Rightarrow z}
	\]
	%
	\begin{property}
		%
		For any closed expression $e$ and canonical form $z$,
		%
		$e \rightarrow^* z$ iff $e \Rightarrow z$.
		%
	\end{property}
	%
	\begin{exercise}
		\todo
	\end{exercise}
	%
	\item
	%
	Intuitively, the normal-order evaluation works by postponing the evaluation of
	the arguments of a function.
	%
	The arguments get evaluated when they are needed.
	%
	However, this evaluation strategy may be inefficient because it may repeat the
	evaluation of one argument.
	%
	For instance, look at the following example.
	%
	\[
		\prths{\lambda x. x \; x}
		\prths{
			\prths{\lambda y. y}
			\prths{\lambda z. z}
		}
	\]
	%
	In the normal-order evaluation, the redex
	%
	$\prths{\lambda y. y} \prths{\lambda z. z}$
	%
	gets contracted twice; because of the two occurrences of $x$ in the body of
	%
	$\lambda x. x \; x$.
	%
	\item
	%
	The eager evaluation takes a different approach.
	%
	It evaluates the arguments of a function before applying the function.
	%
	Most programming languages implement this eager evaluation strategy.
	%
	\item
	%
	Eager evaluation (formalized by) $\RightarrowE$:
	%
	\[
		e\footnotemark \RightarrowE z\footnotemark
	\]
	\footnoteeqn[-1]{closed expression}
	\footnoteeqn{canonical form}

	Canonical Forms
	%
	\[
		\inferrule{\;}{\lambda v. e \RightarrowE \lambda v. e}
	\]

	Application ($\beta_\textrm{E}$-evaluation)
	%
	\[
		\inferrule
		{
			e \RightarrowE \lambda v . e'' \\
			e' \RightarrowE z' \\
			\prths{\subst{e''}{v}{z'}} \RightarrowE z}
		{e \; e' \Rightarrow z}
	\]
	%
	\begin{exercise}
		\todo
	\end{exercise}
	%
\end{enumcirc}

\section{Deotational semantics}

\begin{enumcirc}
	%
	\item
	%
	Interpreting the lambda calculus denotationally is not easy.
	%
	It was one of the longstanding open problems in the 1960's, until Scott solved
	it using the techniques from the domain theory (which Scott himself developed
	partly for this purpose).
	%
	\item
	%
	To understand why it was an open problem, let's try to interpret expressions in
	the lambda calculus (denotationally) using sets and functions.
	%
	This trial will fail as we explain shortly and will show the challenge of
	handling the fact that in the lambda calculus, functions can cat on themselves.

	The first thing that we should do (in this trial) is to find an appropriate
	space $S$ (which is this case is a set) for the meanings of the expressions.
	%
	Let's suppose that we somehow managed to find such $S$.
	%
	Then, $S$ should include all functions on $S$ that may be denoted by
	expressions in the lambda calculus.
	%
	Let's make a slightly stronger but nicer assumption that
	%
	\[
		\brackets{S \rightarrow S}\footnotemark \subseteq S
	\]
	\footnoteeqn[0]{set of all functions on $S$}
	%
	we will now show that $S$ is the singleton set.
	%
	This is because the assumption implies that every function $f$ on $S$ has a
	fixed point, which can happen only if $S$ is a singleton set.
	%
	What fixed point does a function $f \in \brackets{S \rightarrow S}$ have?

	Here is an answer for the question.
	%
	Let $p$ a function on $S$ s.t. $p\prths{x} = f\prths{x\prths{x}}$ for all
	%
	$x \in \brackets{S \rightarrow S} \subseteq S$.
	%
	Such $p$ exists.
	%
	Then $p\prths{p} = f\prths{p\prths{p}}$.
	%
	So, $p\prths{p}$ is a fixed point of $f$.

	We have just shown that $\brackets{S \rightarrow S} \notin S$ for any set $S$
	that contains more than one element.
	%
	\item
	%
	What should we do?
	%
	We need to use the domain theory and the categorical tools, in particular
	general categorical fixed-point theorem and the category
	$\textrm{DOM}^\textrm{EP}$, which consists of domains and a particular kind of
	strict conditions functions called embeddings.
	%
	If you are curious about these, look at the notes on
	%
	``5. Famous Example of the Fixed point Theorem'' (6 Nov 2018).
	%
	Using these tools, we can find domains $D_1, D_2, D_3$ and $V_2, V_3$ s.t.
	%
	\begin{enumrm}
		%
		\item
		%
		\[
			D_1\footnotemark \simeq\footnotemark \brackets{D_1 \toc\footnotemark D_1}
		\]
		\footnoteeqn[-2]{also denoted by $D_\infty$ in other textbooks}
		\footnoteeqn{isomorphism between domains}
		\footnoteeqn{
			\begin{minipage}{0.9\textwidth}
				continuous functions. We often omit $c$ but here we wrote it explicitly to
				emphasize the fact that we are considering continuous functions only here.
			\end{minipage}
		}
		%
		\item
		%
		\begin{align*}
			D_2 & \simeq \brackets{D_2 \toc D_2} \times \brackets{D_2 \toc D_2} \\
			D_3 & \defeq \prths{V_3}_\bot                                       \\
		\end{align*}
		%
		\item
		%
		\begin{align*}
			V_2 & \defeq \brackets{D_2 \toc D_2}              \\
			V_3 & \simeq \brackets{V_3 \toc \prths{V_3}_\bot} \\
		\end{align*}
		%
	\end{enumrm}
	%
	\item
	%
	Note that $D_1$, $D_2$ and $V_3$ are solutions of slightly different recursive
	domain equations.
	%
	Why do we consider three such equations, instead of one?
	%
	It is because the contraction relation, the normal-order evaluation and the
	eager evaluation provide three different meanings to the expressions in the
	lambda calculus,
	%
	and these equations capture these differences.
	%
	$D_1$ is for the contraction relation, $D_2$ and $V_2$ are for the normal-order
	evaluation, and $D_3$ and $V_3$ are for the eager evaluation.
	%
	\item
	%
	As before, understanding these domains is the key to understand the
	denotational semantics of the lambda calculus under three different notions of
	computations:
	%
	\begin{enumrm}
		%
		\item
		%
		the contraction relation
		%
		\item
		%
		the normal-order evaluation
		%
		\item
		%
		the eager evaluation
		%
	\end{enumrm}
	%
	The equations for $\prths{D_2, V_2} and \prths{D_3, V_3}$ indicate that under
	(ii) and (iii), we need to differentiate expressions that don't terminate and
	those that do and become lambda expressions.
	%
	$D_2$ and $D_3$ are domains for all expressions, and $V_2$ and $V_3$ are domains
	for the latter kind of expressions.
	%
	Sometimes $V_2$ and $V_3$ are called \ul{domains for values}, and $D_2$ and
	$D_3$ are called \ul{domains for computations}.
	%
	Note that we don't make this kind of distinction for $D_1$.
	%
	For instance, in $D_2$ and $D_3$, $\bot$ is different from the constant
	function that always return $\bot$. (i.e. $\lambda x. \bot$)
	%
	On the other hand, in $D_1$, they are regarded the same.
	%
	(In $D_1$, $\lambda x . \bot$ is the least element up to the isomorphism $D_1
		\simeq \brackets{D_1 \toc D_1}$.)

	Why is $D_1$'s way of defining $\bot$ different from $D_2$ and $D_3$'s?
	%
	Because the normal-order evaluation and the eager evaluation are do not reduce
	subexpressions under the lambda abstraction ($\lambda x. e$), whereas the
	contraction relation do reduce such subexpressions.
	%
	\item
	%
	Now let's try to understand the difference between $\prths{D_2, V_2}$ and
	$\prths{D_3, V_3}$.
	%
	Rewriting isomorphisms and definitions slightly can help us to see this
	difference more easily.
	%
	\begin{align*}
		D_2 & \defeq \prths{V_2}_\bot \qquad V_2 \simeq \prths{\brackets{D_2 \toc D_2}}_\bot \\
		D_3 & \defeq \prths{V_3}_\bot \qquad V_3 \simeq \brackets{V_3 \toc D_3}              \\
	\end{align*}
	%
	The key difference lies in the fact that $V_2$ has a function space
	%
	$\brackets{\underline{D_2} \toc D_2}$,
	%
	whereas $V_3$ has a function space
	%
	$\brackets{\underline{V_3} \toc D_3}$.
	%
	The argument domain for the normal-order evaluation is that for computations,
	while the argument domain for the eager evaluation is that for values.
	%
	This difference comes from the fact that in the eager evaluation, we pass only
	canonical forms (which are lambda expressions) to functions as arguments, while
	in the normal-order evaluation, we pass any expressions as function arguments.
	%
	So, when we use the normal-order evaluation, variables may be bound to
	expressions that may denote any computations.
	%
	But if we use the eager evaluation, variables should be bound to lambda
	expressions or more generally canonical forms that denote values.
	%
	\item
	%
	Here is the denotational semantics for the contraction relation.
	%
	\[
		D_1 \lrsupsubarrow{\phi}{\psi} \brackets{D_1 \toc D_1}
	\]
	%
	\[
		\begin{array}{c}
			\bbrackets{-} \in
			\brackets{ \chevrons{exp} \rightarrow \brackets{ D_1^{\chevrons{var}}\footnotemark \toc D_1 } }
			\qquad \eta \in D_1^{\chevrons{var}} \dots \textrm{called \ul{environment}}                 \\[1em]
			\bbrackets{v}\eta = \eta\prths{v} \qquad
			\bbrackets{e_1 \; e_2}\eta = \phi\prths{\bbrackets{e_1}\eta} \prths{\bbrackets{e_2}\eta}    \\
			\bbrackets{\lambda x. e}\eta = \psi\prths{\lambda a \in D_1 . \bbrackets{e}\aug{\eta}{x:a}} \\
		\end{array}
	\]
	\footnoteeqn[0]{$D_1^{\chevrons{var}}=\brackets{\chevrons{var} \rightarrow D_1}$ ordered pointwise}
	%
	\begin{property}[Textbook 10.8]
		%
		\;\\
		%
		Well-defined. That is,
		%
		$\lambda a \in D_1 . \bbrackets{e}\aug{\eta}{x}{a}$
		%
		is continuous.
		%
	\end{property}
	%
	\begin{property}[Textbook 10.12 and 10.13]
		%
		\[
			\textrm{If } e_0 \rightarrow e_1, \textrm{ then }
			\bbrackets{e_0}\eta = \bbrackets{e_1}\eta \textrm{ for all } \eta \in D_1^{\chevrons{var}}.
		\]
	\end{property}
	%
	The contraction relation preserves the semantics.
	%
	Note that this implies that the reduction relation $\rightarrow^*$ also
	preserves the semantics.
	%
	\item
	%
	Here are the denotational semantics for the normal-order evaluation and that
	for the eager evaluation.

	\ul{normal-order evaluation:}
	%
	\[
		\begin{array}{c}
			\displaystyle
			D_2 = \prths{V_2}_\bot \qquad
			V_2 \lrsupsubarrow{\phi}{\psi} \brackets{D_2 \toc D_2} \times \brackets{V_2 \toc V_2}                                                                      \\[1em]
			\normalsem{-} \in \brackets{\chevrons{exp} \rightarrow \brackets{ D_2^{\chevrons{var}} \toc D_2 }}                                                         \\[1em]
			\normalsem{v}\eta = \eta\prths{v}                                                                                                                          \\
			\normalsem{e_0 \; e_1}\eta = \begin{cases}
				                             \bot                                                       & \quad \textrm{if} \quad \normalsem{e_0}\eta = \bot \\
				                             \phi\prths{\normalsem{e_0}\eta}\prths{\normalsem{e_1}\eta} & \quad \textrm{otherwise}                           \\
			                             \end{cases} \\[2em]
			\normalsem{\lambda x. e}\eta = \psi\prths{\lambda a \in D_2 . \normalsem{e}\aug{\eta}{x:a}}                                                                \\
		\end{array}
	\]

	\ul{eager evaluation:}
	%
	\[
		\begin{array}{c}
			\displaystyle
			D_3 = \prths{V_3}_\bot \qquad
			V_3 \lrsupsubarrow{\phi}{\psi} \brackets{V_3 \toc D_3}                                                                                \\[1em]
			\eagersem{-} \in \brackets{\chevrons{exp} \rightarrow \brackets{ V_3^{\chevrons{var}} \toc D_3 }}                                     \\[1em]
			\eagersem{v}\eta = \eta\prths{v}                                                                                                      \\
			\eagersem{e_0 \; e_1}\eta = \begin{cases}
				                            \bot                                                     &
				                            \quad \textrm{if} \quad \eagersem{e_0}\eta = \bot \textrm{ or } \eagersem{e_1}\eta = \bot \\
				                            \phi\prths{\eagersem{e_0}\eta}\prths{\eagersem{e_1}\eta} &
				                            \quad \textrm{otherwise}                                                                  \\
			                            \end{cases} \\[2em]
			\eagersem{\lambda x. e}\eta = \psi\prths{\lambda a \in V_3 . \eagersem{e}\aug{\eta}{x:a}}                                             \\
		\end{array}
	\]

	Both semantics are well-defined.
	%
	They validate the normal-order evaluation relation and the eager evaluation
	relation.
	%
	That is,
	%
	\begin{align*}
		\prths{e \Rightarrow e'} \quad  & \Rightarrow\quad \normalsem{e}\eta = \normalsem{e'}\eta \textrm{ for all } \eta \in D_2^{\chevrons{var}} \\
		\prths{e \RightarrowE e'} \quad & \Rightarrow\quad \eagersem{e}\eta = \eagersem{e'}\eta \textrm{ for all } \eta \in V_3^{\chevrons{var}}   \\
	\end{align*}
\end{enumcirc}

\chapter{An Eager Functional Language}

\todo

\chapter{Continuations in a Functional Language}

\section{Motivation}

\begin{enumcirc}
	%
	\item
	%
	Intuitively, a continuation means the remaining computation.
	%
	For instance, when evaluating the subexpression $3 + 4$ in
	%
	$\prths{9 \times \prths{8 + \prths{3 + 4}}}$,
	%
	we have the continuation that denotes $9 \times \prths{8 + \brackets{}}$, which
	expresses what we should do after calculating $3 + 4$.
	%
	\item
	%
	Continuations appear in multiple forms in programming languages.
	%
	First, they are used in a particular style of programming, called continuation
	passing style.
	%
	In this style of programming, called CPS, operators like $+$ and $\times$ take
	continuation parameter $\kappa$ additionally.
	%
	For instance,
	%
	\begin{align*}
		\subsctext{+}{CPS} \prths{m, n, \kappa}      & = \kappa \prths{m + n}       \\
		\subsctext{\times}{CPS} \prths{m, n, \kappa} & = \kappa \prths{m \times n}.
	\end{align*}
	%
	Using these new operations, we can write
	%
	$\prths{9 \times \prths{8 + \prths{3 + 4}}}$
	%
	as follows:
	%
	\[
		\subsctext{+}{CPS} \prths{
			3,
			4,
			\lambda r_1. \subsctext{+}{CPS} \prths{
				8,
				r_1,
				\lambda r_2. \subsctext{\times}{CPS} \prths{
					9,
					r_2,
					\lambda r_3. r_3
				}
			}
		}.
	\]
	%
	Second, continuations are first-class values, and they are used to express
	highly generalized \texttt{goto}s in expressive higher-order programming
	languages, such as Scheme.
	%
	Those languages include the construct, \ul{\texttt{callcc}}, and often
	\ul{\texttt{throw}} as well.
	%
	The former is like label in C and C++, and the latter is like \texttt{goto}.
	%
	When used wisely, these constructs lead to really cool programming examples
	that alter the flow of computation in an intricate way.
	%
	They are often used to implement coroutine, backtracking, scheduler, generator,
	etc.

	Third, continuations are also a powerful tool for building a compiler for
	functional languages.
	%
	Some compilers transform programs or expressions to those in continuation
	passing style in the early phase of compilation.
	%
	After this CPS-transformation, expressions no longer depend on whether we use
	eager evaluation or normal-order evaluation.
	%
	Both evaluations give the same result when applied to CPS-transformed
	expressions.

	Fourth, continuations form a powerful tool in the denotational semantics.
	%
	In fact, they frequently feature in mathematics.
	%
	Let $V$ be the predomain for values that we looked at in the previous chapter.
	%
	Then, semantically, continuations are elements in
	%
	\[
		\brackets{V \toc A}
	\]
	%
	for some domain $A$.
	%
	If you studied functional analysis or Banach space or Hilbert space before, you
	might have seen the dual of a vector space $V$ over $\R$,
	%
	\[
		V^* = \brackets{V \subsctext{\to}{linear} \R},
	\]
	%
	which consists of linear maps from $V$ to $\R$.
	%
	$V^*$ can be understood as a space of continuations on $V$.
	%
	\item
	%
	In this chapter, we will primarily study continuations as new language feature
	(second point) and as a tool in the semantics (fourth point).
	%
	But we will say a few words on the CPS transformation (third point).
	%
	\item
	%
	Another big part of this chapter is a semantic version of defunctionalization,
	a technique to replace higher-order functions by records.
	%
	This is one of the key techniques in compilation.
	%
	Also, a large number of PL researchers, especially those working on
	object-oriented languages, use defunctionalized denotational semantics.
	%
\end{enumcirc}

\section{Continuation Semantics}

\begin{enumcirc}
	%
	\item
	%
	One way to understand the idea of continuation is to rewrite the semantics of
	the eager functional programming language in the previous chapter, using
	continuation.
	%
	In this setting, continuations are continuous functions from $V$ to $V_*$.
	%
	\[
		\subsctext{V}{cont} \defeq \brackets{V \toc V_*}
	\]
	%
	They represent the rest of computation.
	%
	If we provide the value $a$ of the current computation step to a continuation
	$\kappa$, the computation (represented by $\kappa\prths{a}$) will perform all
	the remaining computation steps and output the final result, which is the value
	of $\kappa\prths{a}$.
	%
	\item
	%
	Let's define this continuation semantics more formally.
	%
	Recall the semantic domains and predomains that we used in the semantics of the
	eager functional language in the previous chapter.
	%
	\[
		\begin{array}{c}
			V_*                  =
			\prths{V + \braces{\textrm{error}, \textrm{typeerror}}}_\bot                                              \\[1em]
			V                    \lrsupsubarrow{\phi}{\psi}
			\subsctext{V}{int} + \subsctext{V}{bool} + \subsctext{V}{fun} + \subsctext{V}{tuple} + \subsctext{V}{alt} \\[1em]
			\subsctext{V}{int}   =
			\Z \qquad
			\subsctext{V}{bool}  =
			\B \qquad
			\subsctext{V}{fun}   =
			V \toc V_*                                                                                                \\[1em]
			\subsctext{V}{tuple} = \bigcup_{n \ge 0} V^n \qquad \subsctext{V}{alt} = \N \times V                      \\[1em]
			\bbrackets{-}        \in \brackets{\gram{exp} \toc \brackets{V^{\gram{var}} \toc V_*}}
		\end{array}
	\]
	%
	\item
	%
	The key idea of continuation semantics is to add a new input to $\bbrackets{-}$
	that represents continuation, and also to add a new parameter to each function
	that again represents continuation.
	%
	This means the following two changes:
	%
	\begin{align*}
		\subsctext{V}{cont} & = V \toc V_*                                                 \\
		\subsctext{V}{fun}  & = V \toc \brackets{\underline{\subsctext{V}{cont}} \toc V_*} \\
		\bbrackets{-}       & \in \brackets{
			\gram{exp} \toc
			\brackets{
				V^{\gram{var}} \toc
				\brackets{
					\underline{\subsctext{V}{cont}} \toc V_*
				}
			}
		}
	\end{align*}
	%
	The remaining parts of the semantic predomains and domains are defined in the
	same way as before.
	%
	\item
	%
	Altering $\subsctext{V}{fun}$ and the form of $\bbrackets{-}$ has a huge impact
	on the defining clauses of $\bbrackets{-}$.
	%
	When defining $\bbrackets{-}$, we now have to specify how a given continuation
	is used and modified.
	%
	Observed this change in the following definition of $\bbrackets{-}$.
	%
	\begin{align*}
		\bbrackets{v} \eta \; \kappa       & = \kappa \prths{ \eta \prths{v} } \\
		\bbrackets{e \; e'} \eta \; \kappa & = \bbrackets{e} \eta \; \prths{
			\lambda f . \bbrackets{e'} \eta \; \prths{
				\lambda z . f \; z \; \kappa
			}
		}_{\textrm{fun}}                                                       \\
	\end{align*}
	%
	\[ \vdots \]
	%
	Here $\prths{-}_\theta$ is similar to $\prths{-}_{\theta*}$ that we looked at
	before, but it doesn't deal with $\bot$ and errors.
	%
	That is, given $f \in V_\theta \to V_*$,
	%
	\begin{align*}
		f_\theta           & \in V_\theta \to V_* \\
		f_\theta \prths{a} & =
		\begin{cases}
			\chevrons{1, \textrm{typeerror}} & \textrm{if } \neg \prths{\substack{\exists i, b \textrm{ s.t. } b \in V_\theta \\ \wedge \; a = \psi\prths{\chevrons{i, b}}}} \\[1em]
			f \prths{b}                      & \textrm{if }      \prths{\substack{\exists i, b \textrm{ s.t. } b \in V_\theta \\ \wedge \; a = \psi\prths{\chevrons{i, b}}}}
		\end{cases}
	\end{align*}
	%
	\[ \vdots \]
	%
	\begin{align*}
		\bbrackets{\lambda v . e} \eta \; \kappa         & = \kappa \prths{
			\psi \chevrons{
				2, \lambda a . \lambda \kappa' . \bbrackets{e} \aug{\eta}{v:a} \; \kappa'
			}
		}                                                                                                                                               \\
		\bbrackets{n} \eta \; \kappa                     & = \kappa \prths{ \psi \chevrons{0, n} }                                                      \\
		\bbrackets{-e} \eta \; \kappa                    & = \bbrackets{e} \eta \; \prths{ \lambda i. \; \kappa \prths{ \psi \chevrons{0, -i} } }       \\
		\bbrackets{e_1 + e_2} \eta \; \kappa             & = \bbrackets{e_0} \eta \; \prths{
			\lambda i. \bbrackets{e_1} \eta \; \prths{
				\lambda i'. \; \kappa \prths{ \psi \chevrons{0, i + i'} }
			}_{\textrm{int}}
		}_{\textrm{int}}                                                                                                                                \\
		\bbrackets{\cif{e}{e'}{e''}} \eta \; \kappa      & = \bbrackets{e} \eta \; \prths{
			\lambda b . \textrm{ if } b \textrm{ then } \bbrackets{e'} \eta \; \kappa \textrm{ else } \bbrackets{e''} \eta \; \kappa
		}_{\textrm{bool}}                                                                                                                               \\
		\bbrackets{\true} \eta \; \kappa                 & = \kappa \prths{ \psi \chevrons{1, \ttt} }                                                   \\
		\bbrackets{\chevrons{e_0, e_1}} \eta \; \kappa   & = \bbrackets{e_0} \eta \; \prths{
			\lambda a_0 . \bbrackets{e_1} \eta \; \prths{
				\lambda a_1 . \kappa \prths{ \psi \chevrons{3, \chevrons{a_0, a_1}} }
			}
		}                                                                                                                                               \\
		\bbrackets{e.k} \eta \; \kappa                   & = \bbrackets{e} \eta \; \prths{
			\lambda t . \textrm{ if } k \in \textrm{dom}\prths{t}
			\begin{cases}
				\textrm{then } \kappa \prths{ t_k } \\
				\textrm{else } \chevrons{1, \textrm{typeerror} }
			\end{cases}
		}_\textrm{tuple}                                                                                                                                \\
		\bbrackets{\lletrec{v}{u}{e}{e'}} \eta \; \kappa & = \bbrackets{e'} \aug{\eta}{v: Y \; F} \; \kappa                                             \\
		                                                 & F \in \brackets{\subsctext{V}{fun} \toc \subsctext{V}{fun}}                                  \\
		                                                 & F \prths{f_0} \prths{a} \prths{\kappa'} = \bbrackets{e} \augtwo{\eta}{u:a}{v:f_0} \; \kappa' \\
	\end{align*}
	%
	We omit a few definitions.
	%
	You can find them in the page 254-255 of the textbook.
	%
	\item
	%
	Note that whenever we interpret an expression that includes more than one
	immediate subexpression, such as $e_0 + e_1$, we construct a new continuation
	for the subexpressions that will not be evaluated next, such as
	%
	\[
		\prths{
			\lambda i. \bbrackets{e_1} \eta \; \prths{
				\lambda i'. \; \kappa \prths{ \psi \chevrons{0, i + i'} }
			}_{\textrm{int}}
		}_{\textrm{int}}.
	\]
	%
	Intuitively, this means that the semantics is very explicit about evaluation
	order.
	%
	\item
	%
	This semantics can be expressed as syntactic transformation call \ul{CPS
		transformation}.
	%
	Let $\bbrackets{-}_d$ be the direct semantics that we studied in the previous
	chapter.
	%
	Consider an expression $e$ and a fresh variable $\subsctext{v}{cont}$.
	%
	Then, this transformation has the following property:
	%
	\[
		\bbrackets{e} \eta \; \kappa \; ``=" \; \footnotemark
		\bbrackets{\CPS\prths{e, \subsctext{v}{cont}}}_d \aug{\eta}{\subsctext{v}{cont} : \kappa}
	\]
	\footnoteeqn[0]{equal when no errors}
	%
	As mentioned before, this CPS transformation is often used by a compiler as a
	preprocessing step.
	%
	\item
	%
	$
		\CPS \prths{v, \subsctext{v}{cont}} = \subsctext{v}{cont} \prths{v} \\
		\CPS \prths{e \; e', \subsctext{v}{cont}} = \CPS \prths{e,
			\lambda f . \; \CPS \prths{e', \lambda u . \; f \; u \; \subsctext{v}{cont}}
		} \\
		\CPS \prths{\lambda v . e, \subsctext{v}{cont}} = \subsctext{v}{cont} \prths{
			\lambda v.\; \lambda \subsctext{v'}{cont} . \; \CPS \prths{e, \subsctext{v'}{cont}}
		} \\
		\CPS \prths{n, \subsctext{v}{cont}} = \subsctext{v}{cont} \prths{n} \\
		\CPS \prths{-e, \subsctext{v}{cont}} = \CPS \prths{e, \lambda u . \; \subsctext{v}{cont} \prths{-u}} \\
		\CPS \prths{e_0 + e_1, \subsctext{v}{cont}} = \CPS \prths{e_0,
			\lambda u_0 . \; \CPS \prths{e_1, \lambda u_1 . \; \subsctext{v}{cont} \prths{u_0 + u_1}}} \\
		\CPS \prths{\cif{e}{e'}{e''}, \subsctext{v}{cont}} = \CPS \prths{e,
			\lambda b . \textrm{ if } b \; \begin{cases}
				\textrm{then } \CPS \prths{e', \subsctext{v}{cont}} \\
				\textrm{else } \CPS \prths{e'', \subsctext{v}{cont}}
			\end{cases}
		} \\
		\CPS \prths{\true, \subsctext{v}{cont}} = \subsctext{v}{cont} \prths{\true} \\
		\CPS \prths{\chevrons{e_0, e_1}, \subsctext{v}{cont}} = \CPS \prths{e_0,
			\lambda u_0 . \; \CPS \prths{e_1, \lambda u_1 . \; \subsctext{v}{cont} \prths{\chevrons{u_0, u_1}}}} \\
		\CPS \prths{e.k, \subsctext{v}{cont}} = \CPS \prths{e, \lambda u . \; \subsctext{v}{cont} \prths{u.k}} \\
		\CPS \prths{\lletrec{v}{u}{e}{e'}, \subsctext{v}{cont}} \footnotemark \\
		{} \qquad = \lletrec{v}{u}{
			\;\lambda \subsctext{v'}{cont} . \; \CPS \prths{e, \subsctext{v'}{cont}}
		}{\CPS \prths{e', \subsctext{v}{cont}}} \textrm{ for fresh } \subsctext{v}{cont}.
	$
	\footnoteeqn[0]{Sorry, I'm less certain about this case. (editor: seems fine?)}
	%
	\item
	%
	Note that all function calls after the CPS transformation are the applications
	of continuation variables to parameters.
	%
	Since such variables represent the rest of computation, no calls leave anything
	to be done after they are complete.
	%
	Thus, such calls can be implemented as \texttt{jump}, not as procedure call, by
	a compiler.
	%
	Also, as mentioned before, the CPS-transformed programs produce the same result
	regardless of whether we are using eager evaluation or normal-order evaluation.
	%
	These observations indicate that CPS-transformed programs or expressions are
	simpler than the original ones.
	%
	\item
	%
	The transformation in \circled{7} can be obtained systematically from the
	continuation semantics by removing $\eta$ and all the embeddings and converting
	$\kappa$ to the variable $\subsctext{v}{cont}$.
	%
	This is because they are closely related.
	%
\end{enumcirc}

\section{Callcc and throw}

\begin{enumcirc}
	%
	\item
	%
	Some programming languages allow continuations to be denotable values, and
	provide language constructs for manipulating continuation values.
	%
	\item
	%
	Semantically, this means that we change $V$ as follows:
	%
	\[
		V \lrsupsubarrow{\phi}{\psi}
		\subsctext{V}{int} + \subsctext{V}{bool} + \subsctext{V}{fun} + \subsctext{V}{tuple} + \subsctext{V}{alt} + \underline{\subsctext{V}{cont}}
	\]
	%
	Syntactically, it often involves adding the following two constructs:
	%
	\begin{center}
		\begin{minipage}{0.5\textwidth}
			\begin{grammar}
				<exp> ::= callcc <exp> | throw <exp> <exp>
			\end{grammar}
		\end{minipage}
	\end{center}
	%
	callcc expects a function as its argument.
	%
	$\prths{\lcallcc\prths{\lambda f.e}}$ reifies the current continuation, binds $f$ to it,
	and executes $e$.
	%
	The bound continuation $f$ can be invoked by throw, as in
	$\prths{\lthrow{f}{3}}$.
	%
	This calls the continuation $f$ with the value $3$.
	%
	\item
	%
	Here are the semantic clauses for callcc and throw:
	%
	\begin{align*}
		\bbrackets{\lcallcc e} \eta \; \kappa                   & =
		\bbrackets{e} \eta \prths{\lambda f. \; f \prths{\psi \chevrons{5, \kappa \footnotemark}} \kappa }_{\textrm{fun}} \\
		\bbrackets{\lthrow{e}{e'}} \eta \; \kappa \footnotemark & =
		\bbrackets{e} \eta \prths{\lambda \kappa'. \bbrackets{e'} \eta \; \kappa' \footnotemark}_{\textrm{cont}}
	\end{align*}
	\footnoteeqn[-2]{current continuation copied}
	\footnoteeqn{ignored}
	\footnoteeqn{continuation obtained from $e$ is used instead}
	%
	Intuitively, in
	%
	$\prths{\lcallcc \lambda \kappa . \cdots \lthrow{\kappa}{3} \cdots}$,
	%
	$\lcallcc \lambda \kappa$ can be viewed as putting a label denoted by $\kappa$,
	and $\lthrow{\kappa}{3}$ can be understood as a \texttt{goto} to this label.
	%
	\item
	%
	What are the results of the following expressions?
	%
	\begin{enumrm}
		%
		\item
		%
		$\lcallcc \prths{\lambda \kappa . \; 2 + \lthrow{\kappa}{\prths{3 \times 4}}}$
		%
		\item
		%
		$\prths{\lcallcc \prths{\lambda \kappa .\; \lambda x .\; \lthrow{\kappa}{\prths{\lambda y .\; x + y}}}} 6$
		%
	\end{enumrm}
	%
	The first example can be understood as skipping some part of computation.
	%
	The second shows how we can repeat the computation of some part of an
	expression using continuation.
	%
	\item
	%
	The next example is likely very hard to understand because it uses features not
	explained so far, and it is also quite tricky.
	%
	The example is from the page 290 of the textbook.
	%
	Imagine that we would like to implement a routine \ul{backtrack} that takes a
	function and tries the function with a parameter \ul{amb}\footnote{editor:
		ambiguous} representing a nondeterministic choice between true and false.
	%
	It collects the results of all those choices and returns a list of all those
	results.
	%
	For instance,
	%
	\[
		\textrm{backtrack} \prths{
			\lambda \textrm{amb} . \;
			\begin{array}{l}
				\textrm{if } \textrm{amb}\chevrons{}\footnotemark \textrm{ then } \prths{
					\textrm{if } \textrm{amb}\chevrons{} \textrm{ then } 0 \textrm{ else } 1
				} \\
				\textrm{else } \prths{
					\textrm{if } \textrm{amb}\chevrons{} \textrm{ then } 2 \textrm{ else } 3
				}
			\end{array}
		}
	\]
	\footnoteeqn[0]{empty tuple}
	%
	should return
	%
	\[
		@ \; 1 \chevrons{
			3, \;
			@ \; 1 \chevrons{
				2, \;
				@ \; 1 \chevrons{
					1, \;
					@ \; 1 \chevrons{
						0, \;
						@ \; 0 \chevrons{}
					}
				}
			}
		},
	\]
	%
	which is often written as
	%
	\[
		3 :: 2 :: 1 :: 0 :: \nil
	\]
	%
	representing the list of 3, 2, 1, and 0.
	%
	Note that these are all the possible outcomes of the parameter function to
	backtrack.
	%
	To implement backtrack with callcc and throw, we need a few more features in
	our language.
	%
	\begin{center}
		\begin{minipage}{0.8\textwidth}
			\grammarindent=1.5cm
			\begin{grammar}
				<exp> ::= mkref <exp> \footnotemark
				\alt val <exp> \footnotemark
				\alt <exp> := <exp> \footnotemark
				\alt <exp> =\textsubscript{ref} <exp> \footnotemark
			\end{grammar}
		\end{minipage}
	\end{center}
	\footnoteeqn[-3]{allocates a memory cell, initialized it with $\gram{exp}$ and returns the reference to the cell.}
	\footnoteeqn{dereferences a reference}
	\footnoteeqn{updates a reference}
	\footnoteeqn{reference equality check}

	\textbf{Syntactic sugar}.
	%
	\begin{align*}
		\nil \;                                             & \defeq @ \; 0 \chevrons{}                                                     \\
		e :: e'                                             & \defeq @ \; 1 \chevrons{e, e'}                                                \\
		\textrm{listcase } e \textrm{ of } \prths{e_1, e_2} & \defeq
		\textrm{sumcase } e \textrm{ of } \prths{\lambda v. e_1, \lambda v. \prths{\prths{e_2 v.0} v.1}}                                    \\
		\textrm{let } v \equiv e \in e'                     & \defeq \prths{\lambda v . e'} e                                               \\
		e ;\; e'                                            & \defeq \textrm{let } v \equiv e \textrm{ in } e' \qquad \textrm{for fresh } v \\
	\end{align*}
	%
	\begin{align*}
		\textrm{backtrack} \defeq \lambda f. \; & \textrm{let }\textit{rl} \equiv \textrm{mkref} \prths{\nil} \textrm{ in}                                                                                                                    \\
		                                        & \textrm{let }\textit{cl} \equiv \textrm{mkref} \prths{\nil} \textrm{ in}                                                                                                                    \\
		                                        & \textit{rl} := f\prths{\lambda u. \lcallcc \prths{\lambda k. \; \prths{\textit{cl} := k :: \textrm{val }\textit{cl}} ; \; \true}} :: \textrm{val }\textit{rl } ;                            \\
		                                        & \textrm{listcase } \prths{\textrm{val }\textit{cl }} \textrm{ of } \prths{\textrm{val }\textit{rl}, \lambda c. \; \lambda r. \; \prths{\textit{cl} := r \;;\; \textrm{throw } c \; \false}} \\
	\end{align*}
	%
	\textbf{Editor's note on the backtrack function}

	The backtrack function is a bit tricky to understand, so the editor will try to
	explain it.

	\begin{enumrm}
		%
		\item
		%
		$\textit{rl}$ is a reference to a list of results, and $\textit{cl}$ is a reference to a list of continuations.
		%
		\item
		%
		$f$ is applied to a function that uses callcc (call with current continuation) to capture the current continuation $k$.
		%
		This continuation $k$ is added to the list of continuations $\textit{cl}$ along
		with the value $\true$.
		%
		The continuation represents a choice in the computation.
		%
		If the function $f$ decides to backtrack, it can invoke one of these
		continuations to return to the state represented by that continuation.
		%
		\item
		%
		After $f$ has been applied, the listcase operation examines the list of results
		$\textit{rl}$.
		%
		If $\textit{cl}$ is empty, the listcase operation returns the list of results
		$\textit{rl}$.
		%
		If there are any continuations in $\textit{cl}$, one is removed and invoked
		(throw $c$ false) and its associated computation is resumed.
		%
		This effectively backtracks to the point where callcc captured that
		continuation, and the computation tries a different path by returning false
		instead of true.
		%
	\end{enumrm}

\end{enumcirc}

\section{Deriving a First-order Semantics}

\ul{(Semantic version of defunctionalization)}

\begin{enumcirc}
	%
	\item
	%
	The continuation semantics and the direct semantics both use functions so
	heavily, sometimes even higher-order functions, i.e., functions that take
	functions as parameters.
	%
	Can we define a semantics that avoids the use of such functions, or at least
	minimizes the use of higher-order functions?

	More concretely, recall the definitions of predomains and domains involved in
	the continuation semantics:
	%
	\[
		\begin{array}{c}
			V_*                  =
			\prths{V + \braces{\textrm{error}, \textrm{typeerror}}}_\bot                                                                    \\[1em]
			V                    \lrsupsubarrow{\phi}{\psi}
			\subsctext{V}{int} + \subsctext{V}{bool} + \subsctext{V}{fun} + \subsctext{V}{tuple} + \subsctext{V}{alt} + \subsctext{V}{cont} \\[1em]
			\subsctext{V}{int}   =
			\Z \qquad
			\subsctext{V}{bool}  = \B \qquad
			\subsctext{V}{fun}   = \brackets{V \toc \brackets{\subsctext{V}{cont} \toc V_*}}                                                \\[1em]
			\subsctext{V}{tuple} = \bigcup_{n \ge 0} V^n \qquad \subsctext{V}{alt} = \N \times V \qquad \subsctext{V}{cont} = \brackets{V \toc V_*}
		\end{array}
	\]
	%
	If we substitute the definition of $\subsctext{V}{cont}$ in the definition of
	$\subsctext{V}{fun}$, we get
	%
	\[
		\subsctext{V}{fun} = \brackets{V \toc \brackets{V \toc V_*} \toc V_*}.
	\]
	%
	So, elements in $\subsctext{V}{fun}$ are higher-order function.
	%
	We would like to have a semantics that avoids using such higher-order
	functions.
	%
	Such a semantics is called \ul{first-order}.
	%
	\item
	%
	Before answering the question raised in \circled{1}, let me say a few words
	about why we are interested in such a first-order semantics.
	%
	The first reason is a bit theoretical.
	%
	It is that defining such a first-order semantics involves solving much simpler
	and easier recursive domain equations.
	%
	In our original continuation semantics, we assumed that $V$ is a solution of
	the following recursive (pre)domain equation:
	%
	\[
		V \simeq \prths{
			\begin{array}{l}
				\Z + \B                                  \\[0.5em]
				+ \quad \brackets{
					\begin{array}{r}
						\dunderline{V} \toc \brackets{\underline{V} \toc \prths{\dunderline{V} + \textrm{error} + \textrm{typeerror}}_\bot} \\
						\toc \prths{\underline{V} + \braces{\textrm{error}, \textrm{typeerror}}}_\bot
					\end{array}
				}                                        \\[2em]
				+ \quad \bigcup_n^\infty \underline{V}^n \\[0.5em]
				+ \quad \N \times \underline{V}          \\[0.5em]
				+ \quad \brackets{\dunderline{V} \toc \prths{\underline{V} + \textrm{error} + \textrm{typeerror}}_\bot}
			\end{array}
		}
	\]
	%
	Note that $V$ appears on the both sides of $\to$. The occurrences of $V$
	underlined with two lines $\dunderline{V}$ make this recursive predomain
	equation very difficult to solve.
	%
	We should use the categorical fixed point theorem and the category of domains
	with embeddings (which we covered before) to solve this equation.
	%
	On the other hand, in the first-order semantics, we have a recursive predomain
	equation that is much easier to solve.
	%
	It doesn't have those tricky recursive occurrences of $\hat{V}$ (a predomain
	being defined) that appear on the left argument side of $\to$.

	The second reason is that this first-order continuation semantics becomes a
	theoretical basis or guide for a compiler for eager functional languages.
	%
	The situation is analogous to the CPS transformation that we looked at.
	%
	The transformation is derived from the continuation semantics.
	%
	Similarly, from the first-order semantics, we are able to derive a program (or
	expression) transformation sometimes called defunctionalization, which gets rid
	of all higher-order functions.
	%
	By the say, this kind of connection between (denotational) semantics and
	compilation should not be too surprising.
	%
	In a sense, a denotational semantics is a compiler of programs into phrases in
	mathematics.
	%
	If the compiler uses only very restricted phrases, the compiled phrases can be
	understood as instruction sequences in a computer.
	%
	\item
	%
	Let's define the first-order semantics.
	%
	It is based on the observation that when we interpret an expression $e$ in the
	continuation semantics, we do not use all functions, but specific kinds of
	functions.
	%
	In a sense, the first-order semantics replaces $\subsctext{V}{fun}$,
	$\subsctext{V}{cont}$, and $E = \supsctext{V}{var}$ by three sets
	$\subsctext{\hat{V}}{fun}$, $\subsctext{\hat{V}}{cont}$, and $\hat{E}$, that
	consist of mathematical instructions.
	%
	Then, it defines how to interpret those instructions.

	We consider an eager functional language with integers and continuation values.
	%
	Here are predomains and domains used in the first-order semantics.
	%
	\[
		\begin{array}{c}
			\hat{V}_* = \prths{\hat{V} + \braces{\textrm{error}, \textrm{typeerror}}}_\bot \\[1em]
			\hat{V} \lrsupsubarrow{\phi}{\psi} \subsctext{V}{int} + \subsctext{\hat{V}}{fun} + \subsctext{\hat{V}}{cont}
			\qquad
			\subsctext{V}{int} = \Z \qquad
		\end{array}
	\]

	\[
		\subsctext{\hat{V}}{fun} = \braces{\abstr} \times \gram{var} \times \gram{exp} \times \hat{E}
	\]
	%
	$\subsctext{\hat{V}}{fun}$: typical element \dots $\chevrons{\abstr, v, e, \eta}$.
	%
	$\abstr$ indicates this tuple represents a lambda expression $\lambda v . e$
	and an environment $\eta$ for the free variables in the expression.

	\begin{align*}
		\hat{E} = & \braces{\initenv}                                                                          \\
		          & \cup \braces{\extend} \times \gram{var} \times \hat{V} \times \hat{E}                      \\
		          & \cup \braces{\recenv} \times \hat{E} \times \gram{var} \times \gram{var} \times \gram{exp}
	\end{align*}
	%
	$\initenv$ is the initial empty environment.
	%
	$\chevrons{\extend, v, z, \eta}$ is the environment obtained by extending
	$\eta$ with the binding $v \mapsto z$.
	%
	$\chevrons{\recenv, \eta, u, v, e}$ is the environment obtained by extending
	$\eta$ with the recursively defined $u$ (i.e., $u = \lambda v. e$).

	\begin{align*}
		\subsctext{\hat{V}}{cont} = & \braces{\negate} \times \subsctext{\hat{V}}{cont}                                                            \\
		                            & \cup \braces{\addrm_1, \divrm_1, \mulrm_1} \times \gram{exp} \times \hat{E} \times \subsctext{\hat{V}}{cont} \\
		                            & \cup \braces{\addrm_2, \divrm_2, \mulrm_2} \times \subsctext{V}{int} \times \subsctext{\hat{V}}{cont}        \\
		                            & \cup \braces{\apprm_1} \times \gram{exp} \times \hat{E} \times \subsctext{\hat{V}}{cont}                     \\
		                            & \cup \braces{\apprm_2} \times \subsctext{\hat{V}}{fun} \times \subsctext{\hat{V}}{cont}                      \\
		                            & \cup \braces{\ccc} \times \subsctext{\hat{V}}{cont}                                                          \\
		                            & \cup \braces{\thw} \times \gram{exp} \times \hat{E}                                                          \\
		                            & \cup \braces{\initcont}
	\end{align*}
	%
	$\negate$ negates its input and call the continuation.
	%
	The second and third cases are continuations for addition, division, and
	multiplication.
	%
	The fourth and fifth cases are continuations for function application.
	%
	Others are continuations for callcc, throw, and the initial continuation.

	Note that elements of $\subsctext{\hat{V}}{fun}, \subsctext{\hat{V}}{cont}$,
	and $\hat{E}$ are not functions.
	%
	Rather they are like instructions that denote certain functions.
	%
	They are almost like programs.

	The semantic function $\bbrackets{-}$ has a slightly more complex definition.
	%
	It is because the definition should now spell out how we can view elements of
	$\subsctext{\hat{V}}{fun}, \subsctext{\hat{V}}{cont}$, and $\hat{E}$ as
	appropriate functions.
	%
	We define three more functions:
	%
	\begin{align*}
		\bbrackets{-} & \in \brackets{\gram{exp} \to \hat{E} \to \subsctext{\hat{V}}{cont} \to \hat{V}_*}                          \\
		\contf        & \in \brackets{\subsctext{\hat{V}}{cont} \to \brackets{\hat{V} \to \hat{V}_*}}                              \\
		\applyf       & \in \brackets{\subsctext{\hat{V}}{fun} \to \brackets{\hat{V} \to \subsctext{\hat{V}}{cont} \to \hat{V}_*}} \\
		\getf         & \in \brackets{\hat{E} \to \brackets{\gram{var} \to \hat{V}}}
	\end{align*}
	%
	Here cont, apply and get functions provide the meanings of elements (or records
	or instructions) in $\subsctext{\hat{V}}{cont}$, $\subsctext{\hat{V}}{fun}$,
	and $\hat{E}$.
	%
	Whenever we need to use those elements by, say, look-up and function
	application, we use these three functions.
	%
	These three functions and $\bbrackets{-}$ are defined mutually recursively as
	follows:
	%
	\[
		\applyf \chevrons{\abstr, v, e, \eta} a \; \kappa =
		\bbrackets{e} \chevrons{\extend, v, a, \eta} \kappa
	\]
	%
	\[
		\begin{array}{l}
			\getf \chevrons{\initenv} \; v                = \psi \chevrons{0, 0} \quad \substack{(\textrm{initial value 0 assigned to } v)} \\
			\getf \chevrons{\extend, v, a, \eta} \; w     = \textrm{if } v = w
			\begin{cases}
				\textrm{then } a \\
				\textrm{else } \getf \; \eta \; w
			\end{cases}                                                                                                \\
			\getf \chevrons{\recenv, \eta, v, u, e} \; w                                                                                    \\
			\;\qquad \qquad = \textrm{if } v = w
			\begin{cases}
				\textrm{then } \psi \chevrons{1, \chevrons{\abstr, u, e, \chevrons{\recenv, \eta, v, u, e}}} \\
				\textrm{else } \getf \; \eta \; w
			\end{cases}
		\end{array}
	\]
	%
	\[
		\begin{array}{l}
			\contf \chevrons{\negate, \kappa} \; a            = \prths{\lambda i. \; \contf \; \kappa \prths{ \psi \chevrons{0, -i} }}_{\textrm{int}} a         \\[1em]
			\contf \chevrons{\addrm_1, e, \eta, \kappa} \; a  = \prths{\lambda i. \; \bbrackets{e} \eta \chevrons{ \addrm_2, i, \kappa } }_{\textrm{int}} a     \\[1em]
			\contf \chevrons{\addrm_2, i, \kappa} \; a        = \prths{\lambda i'. \; \contf \; \kappa \prths{ \psi \chevrons{0, i + i'} } }_{\textrm{int}} a   \\[1em]
			\contf \chevrons{\divrm_2, i, \kappa} \; a        = \prths{\lambda i'. \; \textrm{if } i' = 0
				\begin{cases}
					\textrm{then } \kappa \prths{ \psi \chevrons{1, \textrm{error}} } \\
					\textrm{else } \contf \; \kappa \prths{ \psi \chevrons{0, i \div i'} }
				\end{cases}
			\!\!}_{\textrm{int}} a                                                                                                                              \\[1em]
			\contf \chevrons{\apprm_1, e, \eta, \kappa} \; a  = \prths{\lambda f. \; \bbrackets{e} \eta \chevrons{ \apprm_2, f, \kappa } }_{\textrm{fun}} a     \\[1em]
			\contf \chevrons{\apprm_2, f, \kappa} \; a        = \applyf \; f \; a \; \kappa                                                                     \\[1em]
			\contf \chevrons{\ccc, \kappa} \; a               = \prths{\lambda f. \; \applyf \; f \prths{\psi \chevrons{2, \kappa}} \; \kappa}_{\textrm{fun}} a \\[1em]
			\contf \chevrons{\thw, e, \eta} \; a               = \prths{\lambda \kappa'. \; \bbrackets{e} \eta \; \kappa'}_{\textrm{cont}} a                    \\[1em]
			\contf \chevrons{\initcont} \; a                  = \psi \chevrons{0, a} \quad \substack{(\textrm{0th component of } \prths{V + \braces{\textrm{error}, \textrm{typeerror}}}_\bot)}
		\end{array}
	\]
	%
	$\chevrons{\mulrm_1, e, \eta, \kappa}$,
	%
	$\chevrons{\mulrm_2, i, \kappa}$ and
	%
	$\chevrons{\divrm_1, e, \eta, \kappa}$
	%
	are all interpreted similarly to \\
	%
	$\chevrons{\addrm_1, e, \eta, \kappa}$ and
	%
	$\chevrons{\addrm_2, i, \kappa}$.

	\[
		\begin{array}{l}
			\bbrackets{n} \eta \; \kappa = \contf \; \kappa \prths{ \psi \chevrons{0, n} }                                           \\[1em]
			\bbrackets{-e} \eta \; \kappa = \bbrackets{e} \eta \; \chevrons{\negate, \kappa}                                         \\[1em]
			\bbrackets{e_0 + e_1} \eta \; \kappa = \bbrackets{e_0} \eta \; \chevrons{\addrm_1, e_1, \eta, \kappa}                    \\[1em]
			\bbrackets{e_0 \div e_1} \eta \; \kappa = \bbrackets{e_0} \eta \; \chevrons{\divrm_1, e_1, \eta, \kappa}                 \\[1em]
			\bbrackets{e_0 \times e_1} \eta \; \kappa = \bbrackets{e_0} \eta \; \chevrons{\mulrm_1, e_1, \eta, \kappa}               \\[1em]
			\bbrackets{v} \eta \; \kappa = \contf \; \kappa \prths{ \getf \; \eta \; v }                                             \\[1em]
			\bbrackets{e_0 \; e_1} \eta \; \kappa = \bbrackets{e_0} \eta \; \chevrons{\apprm_1, e_1, \eta, \kappa}                   \\[1em]
			\bbrackets{\lambda v . e} \eta \; \kappa = \contf \; \kappa \prths{ \psi \chevrons{1, \chevrons{\abstr, v, e, \eta}} }   \\[1em]
			\bbrackets{\lcallcc e} \eta \; \kappa = \bbrackets{e} \eta \; \chevrons{\ccc, \kappa}                                    \\[1em]
			\bbrackets{\lthrow{e}{e'}} \eta \; \kappa = \bbrackets{e} \eta \; \chevrons{\thw, e', \eta}                              \\[1em]
			\bbrackets{\textrm{error}} \eta \; \kappa = \chevrons{1, \textrm{error}}                                                 \\[1em]
			\bbrackets{\textrm{typeerror}} \eta \; \kappa = \chevrons{1, \textrm{typeerror}}                                         \\[1em]
			\bbrackets{\lletrec{v_0}{u_0}{e_0}{e_1}} \eta \; \kappa = \bbrackets{e_1} \chevrons{\recenv, \eta, v_0, u_0, e_0} \kappa \\[1em]
		\end{array}
	\]

	Note that this recursive definition is well-defined because of the following
	two reasons.
	%
	\begin{enumrm}
		%
		\item
		%
		$\getf$ is defined inductively\footnote{
			all recursive calls in the definition of get are over sub-environments.
		} and doesn't depend on $\bbrackets{-}$, $\applyf$ and $\contf$.
		%
		\item
		%
		Since $\hat{V}_*$ is a domain and the function space $\brackets{P \toc D}$ from
		a predomain $P$ to a domain $D$ is a domain, all of
		%
		$D_1 = \brackets{\gram{exp} \to \hat{E} \to \subsctext{\hat{V}}{cont} \to \hat{V}_*}$,
		%
		$D_2 = \brackets{\subsctext{\hat{V}}{cont} \to \brackets{\hat{V} \to \hat{V}_*}}$ and
		%
		$D_3 = \brackets{\subsctext{\hat{V}}{fun} \to \brackets{\hat{V} \to \subsctext{\hat{V}}{cont} \to \hat{V}_*}}$
		%
		are domains.
		%
		$\bbrackets{-}$, $\contf$ and $\applyf$ can be understood as a fixed point (in fact, the least fixed point) of some continuous function
		%
		$F : D_1 \times D_2 \times D_3 \to D_1 \times D_2 \times D_3$.
		%
		This function $F$ is what the semantic definitions of $\bbrackets{-}$, $\contf$
		and $\applyf$ determine.
		%
	\end{enumrm}
	%
	\item
	%
	Let me mention two further points.
	%
	First, the definition in the previous two pages doesn't use lambda in the
	mathematical meta language in a sense.
	%
	Yes, you can see $\lambda$ there.
	%
	But those $\lambda$'s are mainly for enabling the use of $\prths{-}_\theta$
	notation, which does runtime type checking.
	%
	We could have used the unpacked definition of $\prths{-}_\theta$ instead and
	avoided $\lambda$ completely.

	This lack of $\lambda$ confirms that the semantics is first-order.
	%
	Second, the predomain equation for $V$ can be solved in the category of sets,
	i.e., without using domain theory.
	%
	That is, we can define a set $V$ s.t.
	%
	\[
		V =\footnotemark \Z + \subsctext{V}{fun} + \subsctext{V}{cont}
	\]
	\footnoteeqn[0]{equality}
	%
	where $\subsctext{V}{fun}$ and $\subsctext{V}{cont}$ are defined as before.
	%
	\item
	%
	I tried to derive the program transformation from this first-order semantics.
	%
	But I couldn't find a simple way to do so.
	%
	Sorry guys.

	Let me instead show you how one can derive a small-step evaluation relation (or
	more commonly called small-step operational semantics) from the first-order
	denotational semantics.
	%
	The idea is to replace $=$ by a single evaluation step $\to$.

	\begin{align*}
		\chevrons{n, \eta, \kappa}                                   & \to
		\chevrons{\contf, \kappa, n}                                                                                                                 \\
		\chevrons{-e, \eta, \kappa}                                  & \to
		\chevrons{e, \eta, \chevrons{\negate, \kappa}}                                                                                               \\
		\chevrons{e_0 \,\substack{+                                                                                                                  \\ \div \\ \times}\,e_1, \eta, \kappa} & \to
		\chevrons{e_0, \eta, \chevrons{\substack{\addrm_1                                                                                            \\ \divrm_1 \\ \mulrm_1}, e_1, \eta, \kappa}} \\
		\chevrons{v, \eta, \kappa}                                   & \to
		\chevrons{\contf, \kappa, \prths{\getf \; \eta \; v}}                                                                                        \\
		\chevrons{e_0 \; e_1, \eta, \kappa}                          & \to
		\chevrons{e_0, \eta, \chevrons{\apprm_1, e_1, \eta, \kappa}}                                                                                 \\
		\chevrons{\lambda v . e, \eta, \kappa}                       & \to
		\chevrons{\contf, \kappa, \chevrons{\abstr, v, e, \eta}}                                                                                     \\
		\chevrons{\lcallcc e, \eta, \kappa}                          & \to
		\chevrons{e, \eta, \chevrons{\ccc, \kappa}}                                                                                                  \\
		\chevrons{\lthrow{e}{e'}, \eta, \kappa}                      & \to
		\chevrons{e, \eta, \chevrons{\thw, e', \eta}}                                                                                                \\
		\chevrons{\lletrec{v_0}{u_0}{e_0}{e}, \eta, \kappa}          & \to
		\chevrons{e, \chevrons{\recenv, \eta, v_0, u_0, e_0}, \kappa}                                                                                \\
		\chevrons{\contf, \chevrons{\negate, \kappa}, a}             & \to
		\chevrons{\kappa, \psi \chevrons{0, -a}} \quad (\textrm{if } a \in \Z)                                                                       \\
		\chevrons{\contf, \chevrons{\addrm_1, e, \eta, \kappa}, a}   & \to
		\chevrons{e, \eta, \chevrons{\addrm_2, a, \kappa}}                                                                                           \\
		\chevrons{\contf, \chevrons{\addrm_2, a, \kappa}, b}         & \to
		\chevrons{\contf, \kappa, a + b} \quad (\textrm{if } a, b \in \Z)                                                                            \\
		                                                             & \substack{\mulrm_1,\; \mulrm_2 \textrm{ and } \divrm_1 \textrm{ are similar}} \\
		\chevrons{\contf, \chevrons{\divrm_2, a, \kappa}, b}         & \to
		\chevrons{\contf, \kappa, a \div b} \quad (\textrm{if } a, b \in \Z \textrm{ and } b \neq 0)                                                 \\
		\chevrons{\contf, \chevrons{\apprm_1, e, \eta, \kappa}, a}   & \to
		\chevrons{e, \eta, \chevrons{\apprm_2, a, \kappa}}                                                                                           \\
		\chevrons{\contf, \chevrons{\apprm_2, a, \kappa}, b}         & \to
		\chevrons{\applyf, a, b, \kappa} \quad (\textrm{if } a \in \subsctext{\hat{V}}{fun})                                                         \\
		\chevrons{\contf, \chevrons{\ccc, \kappa}, a}                & \to
		\chevrons{\applyf, a, \kappa, \kappa} \quad (\textrm{if } a \in \subsctext{\hat{V}}{fun})                                                    \\
		\chevrons{\contf, \chevrons{\thw, e, \eta}, a}               & \to
		\chevrons{e, \eta, a} \quad (\textrm{if } a \in \subsctext{\hat{V}}{cont})                                                                   \\
		\chevrons{\contf, \chevrons{\initcont}, a}                   & \to
		a                                                                                                                                            \\
		\chevrons{\applyf, \chevrons{\abstr, v, e, \eta}, a, \kappa} & \to
		\chevrons{e, \chevrons{\extend, v, a, \eta}, \kappa}                                                                                         \\
	\end{align*}
	%
	We use definition of get for environments in $\hat{E}$.

\end{enumcirc}



\end{document}

