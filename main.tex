% \documentclass[12pt]{book}
\documentclass[ebook,9pt,oneside,openany]{memoir}

\usepackage[paperwidth=6in, paperheight=9in, margin=0.5in]{geometry}
% \usepackage[paperwidth=148.5mm, paperheight=210mm, margin=10mm]{geometry}
\usepackage[english]{babel}
\usepackage{graphicx}
\usepackage{framed}
\usepackage[normalem]{ulem}
\usepackage{amsmath}
\usepackage{amsthm}
\usepackage{amssymb}
\usepackage{amsfonts}
\usepackage{enumerate}
\usepackage{kotex}
\usepackage{setspace}
\usepackage{listings}
\usepackage{xcolor}
\usepackage{bbm}
\usepackage{lipsum}
\usepackage{tikz}
\usepackage{enumitem}
\usepackage{syntax}
\usepackage{changepage}
\usepackage{mathrsfs}
\usepackage{stmaryrd}

\newtheoremstyle{tabstyle}
  {\topsep} % Space above
  {\topsep} % Space below
  {\normalfont} % Body font
  {} % Indent amount
  {\bfseries} % Theorem head font
  {.} % Punctuation after theorem head
  {.5em} % Space after theorem head
  {} % Theorem head spec (can be left empty, meaning ‘normal’)
\newtheorem{theorem}{Theorem}
\newtheorem{corollary}{Corollary}
\newtheorem{property}{Property}
\newtheorem*{lemma}{Lemma}
\newtheorem*{definition}{Definition}
\newtheorem{note}{Note}

\theoremstyle{tabstyle}
\newtheorem*{exercise}{Exercise}
\newtheorem*{example}{Example}
\newenvironment{exampletab}
  {\begin{example}\;\begin{adjustwidth}{4em}{0em}}
  {\end{adjustwidth}\end{example}}
\newenvironment{exercisetab}
  {\begin{exercise}\;\begin{adjustwidth}{4em}{0em}}
  {\end{adjustwidth}\end{exercise}}


\newcommand*\circled[1]{\tikz[baseline=(char.base)]{
            \node[shape=circle,draw,inner sep=2pt] (char) {#1};}}

\newcommand{\<}{\langle}
\renewcommand{\>}{\rangle}
\newcommand{\abs}[1]{\left\vert #1 \right\vert}
\newcommand{\prths}[1]{\left( #1 \right)}
\newcommand{\braces}[1]{\left\{ #1 \right\}}
\newcommand{\brackets}[1]{\left[ #1 \right]}
\newcommand{\bbrackets}[1]{\left\llbracket #1 \right\rrbracket}
\newcommand{\chevrons}[1]{\left\< #1 \right\>}
\newcommand{\set}[2]{\left\{ #1 \; \middle\vert \; #2 \right\}}

\newcommand{\z}{\mathbb{Z}}
\newcommand{\fv}[1]{\textrm{FV}\prths{ #1 }}
\newcommand{\where}{\quad \textrm{ where } \;}

\renewcommand{\thefootnote}{\fnsymbol{footnote}}

\title{Theory of Programming Languages}
\author{Your Name Here}
\date{\today}

\OnehalfSpacing
\counterwithin*{footnote}{page}

\begin{document}

\maketitle

\tableofcontents

% !TEX root = main.tex

\chapter{Predicate Logic}

\section{Motivation or objective}

\begin{enumerate}[label=\protect\circled{\arabic*}]
  \item
    Learn four key tools in PL that will be used throughout this course.
    \begin{enumerate}[label=(\roman*)]
      \item Abstract Syntax
      \item Denotational Semantics
      \item Inference Rule
      \item Binding
    \end{enumerate}
  \item Learn the basics of predicate logic (or first-order logic)
  \item We plan to go ghrough some of (i) - (iv) twice, first using integer
    expressions and then using predicate logic.
\end{enumerate}

\section{Integer expressions}

\begin{enumerate}[label=\protect\circled{\arabic*}]
  \item
    How to analyze integer expresion found in logic and programming languages
    mathematically? We will first have to define the syntax and teh semantics
    for them.
  \item
    Examples: $x + 3 \times y$ , $x \div 2 + x \times x$ ...
  \item
    We also want to develop mathematical tools to reason about or manipulate
    integer expressions.
\end{enumerate}

\section{Abstract syntax and initial algebra}

\begin{enumerate}[label=\protect\circled{\arabic*}]
  \item Abstract Syntax: \\
    Specification of \emph{abstract phrases}
    \footnote{
      vague words, but will be made rigorous when we define initial algebra.
    }
    in a formal language, such as the language of integer expressions and
    predicate logic.
  \item
    Typically, we use \emph{abstract grammar}
    \footnote{
      \begin{minipage}[t]{\textwidth}
      Here is the explanation of the word with an enumerated list:
      \begin{enumerate}[label=(\roman*)]
        \item grammar without any concern on parsing ofr surface syntax.
        \item In this case, parse trees in the grammar are abstract phrases.
      \end{enumerate}
      \end{minipage}
    }
    to describe abstract syntax.
  \item Abstract grammar for integer expressions:
    \setlength{\grammarindent}{6em} % increase separation between LHS/RHS
    \begin{center}
    \begin{minipage}{0.4\textwidth}
    \begin{grammar}
    <intexp> ::= 0 | 1 | 2 ...
    \alt <var>
    \alt - <intexp>
    \alt <intexp> $+$ <intexp>
    \alt <intexp> $-$ <intexp>
    \alt <intexp> $\times$ <intexp>
    \alt <intexp> $\div$ <intexp>
    \end{grammar}
    \end{minipage}
    \end{center}
    (abstract) integer expressions are finite deriviation trees in this grammar.
    For instance,

    \begin{center}
    \begin{tikzpicture}[
      level 1/.style={sibling distance=10mm, level distance=10mm},
      level 2/.style={sibling distance=10mm},
      level 3/.style={sibling distance=5mm}]

      \node {+}
        child {node {$\times$} }
        child {node {$\times$}
          child {node {3}}
          child {node {y}}
        };
    \end{tikzpicture}
    $\qquad$
    \begin{tikzpicture}[
      level 1/.style={sibling distance=20mm, level distance=10mm},
      level 2/.style={sibling distance=10mm},
      level 3/.style={sibling distance=5mm}]

      \node {+}
        child {node {$\div$}
          child {node {x}}
          child {node {2}}
        }
        child {node {$-$}
          child {node {x}}
          child {node {x}}
        };
    \end{tikzpicture}
    \end{center}

    Note that infinite trees are not included.

  \item
    A more accurate view is to view abstract syntax as an inital algebra. This
    view will help us to see shy we can define various operations on abstract
    phrases or integer expressions using syntax-directed definition.

  \item
    \emph{Algebra} $A$ $\cdots$ Set with operations and constraints. \\
    \emph{Signature} $S$ $\cdots$ Type of an algebra.

    \begin{exampletab}
      \begin{enumerate}[label=(\roman*)]
        \item TODO
        \item TODO
      \end{enumerate}
    \end{exampletab}

  \item
    \emph{Algebra homomorphism} $\cdots$ map between algebras that
    preserves constants and operations.
    \begin{align*}
      S &= \prths{
      t,\; c_1 : t,\; \cdots ,\; c_n : t ,\;
      \textrm{op}_1 : t \times \cdots \times t \rightarrow t ,\; \cdots
      \textrm{op}_m : t \times \cdots \times t \rightarrow t
      }
      \\
      A_0 &= \prths{
      \mathcal{U}_0\footnotemark,\;
      c_1^0 \in \mathcal{U} ,\;
      \cdots ,\;
      c_n^0 \in \mathcal{U} ,\;
      \textrm{op}_1^0 :
      \mathcal{U} \times \cdots \times \mathcal{U} \rightarrow \mathcal{U},\; \cdots
      \textrm{op}_m^0 :
      \mathcal{U} \times \cdots \times \mathcal{U} \rightarrow \mathcal{U}
      }
      \\
      A_1 &= \prths{
      \mathcal{U}_1,\;
      c_1^1 \in \mathcal{U} ,\;
      \cdots ,\;
      c_n^1 \in \mathcal{U} ,\;
      \textrm{op}_1^1 :
      \mathcal{U} \times \cdots \times \mathcal{U} \rightarrow \mathcal{U},\; \cdots
      \textrm{op}_m^1 :
      \mathcal{U} \times \cdots \times \mathcal{U} \rightarrow \mathcal{U}
      }
    \end{align*}
    \footnotetext{notation: $\abs{A_0}$}

    $f \in \mathcal{U}_0 \rightarrow \mathcal{U}_1$ is a \emph{homomorphism} if
    \begin{enumerate}[label=(\roman*)]
      \item $f \prths{c_i^0} = c_i^1$ for all $i$.
      \item $f \prths{\textrm{op}_i^0 \prths{x_1,\; \cdots ,\; x_k}} =
        \textrm{op}_i^1 \prths{x_1,\; \cdots ,\; x_k}$ for all $i$.
    \end{enumerate}

  \item
    \emph{Initial algebra of a signature} $S$
    \begin{enumerate}[label=(\roman*)]
      \item
        An algebra $A$ of the signature $S$ s.t. for all algebras $A^\prime$ of
        the same signature, there is a \emph{unique} homomorphism $f$ from $A$
        to $A^\prime$.
      \item
        $A_{\textrm{grammar}}$ is initial.
      \item
        Formally, an abstract syntax fixes a signature and it denotes an initial
        algebra of the signature. An abstract phrase is an element of that
        algebra.
    \end{enumerate}
    \begin{exercise}
      Prove that $A_{\textrm{grammar}}$ is indeed an initial algebra.
    \end{exercise}
    \begin{exercise}
      Let $A_0$ and $A_0$ be initial algebras of the same signature $S$. Then,
      there are homomorphisms $f \in \abs{A_0} \rightarrow \abs{A_1}$ and
      $g \in \abs{A_1} \rightarrow \abs{A_0}$ s.t. $f \circ g = \textrm{id}$ and
      $g \circ f = \textrm{id}$.\\
      This means that all initial algebrras of $S$ are essentially the same,
      i.e. isomorphic. Prove this fact.
    \end{exercise}

\end{enumerate}

\section{Syntax-directed definition and denotational semantics}
\begin{enumerate}[label=\protect\circled{\arabic*}]
  \item
    Definition of a map on integer expressions using a form of induction and
    case analysis.
  \item
    $
    \textrm{FV (Free Variables)}: \chevrons{intexp} \rightarrow 2^{\chevrons{Var}}
    $
    \begin{align*}
      \fv{e\footnotemark} &= V \footnotemark \\
      \fv{c\footnotemark} &= \phi \\
      \fv{x\footnotemark} &= \braces{x} \\
      \fv{-e} &= \fv{e} \\
      \fv{e_1 \; \substack{+\\-\\\times\\\div} \; e_2} &= \fv{e_1} \cup \fv{e_2}
    \end{align*}
    \addtocounter{footnote}{-3} \footnotetext{integer expression}
    \stepcounter{footnote} \footnotetext{Set of free variables in $e$}
    \stepcounter{footnote} \footnotetext{constant}
    \stepcounter{footnote} \footnotetext{variable} 

  \item
    Two features: case analysis, recursive calls on subphrases.

  \item
    $
      \bbrackets{-} \in \chevrons{intexp} \; \rightarrow \;
      \Sigma \rightarrow \z
      \where \Sigma = \chevrons{var} \rightarrow \z , \textrm{ a set of states }
      \sigma.
    $
    \begin{align*}
      \bbrackets{c}\sigma &= c \\
      \bbrackets{x}\sigma &= \sigma\prths{x} \\
      \bbrackets{-e}\sigma &= -\prths{\bbrackets{e}\sigma} \\
      \bbrackets{
        e_1 \;\substack{+\\-\\\times\\\div}\; e_2
      }\sigma &= \prths{\bbrackets{e_1}\sigma}
      \;\substack{+\\-\\\times\\\div}\;
      \prths{\bbrackets{e_1}\sigma} \footnotemark
    \end{align*}
    \footnotetext{some special treatment of the divide-by-zero case}
    Intuitively, $\bbrackets{-}$ maps tress to mathematical functions in a
    synatax-directed (also called compositional) way. Such a compositional
    mapping from syntactic entities to mathematical entities is called
    \emph{denotational semantics}.

\end{enumerate}


\chapter{The Simple Imperative Language}

\section{Motivation or goal}

\begin{enumcirc}
	%
	\item
	%
	Most real-worlds PLs support computation by state update and that by function
	application.
	%
	The former is often referred to as imperative computation, and the latter as
	functional or applicative computation
	%
	Our goal is to study core PL concepts and ideas for imperative computation.
	%
	\item
	%
	Actually, it is more appropriate to say that our aim is a formal and
	mathematical analysis of the core PL concepts for imperative computation.
	%
	We will study or learn mathematical tools for this.
	%
	Also, we will show how to express and analyze big design decisions of such an
	imperative PLs.
	%
	\item
	%
	We will look at (i) some basic concepts and results of domain theory, (ii)
	variable declaration and binding, (iii) syntactic sugar error handling and (iv)
	the notions of soundness and full abstraction.%
	%
	(Well, I just listed all the key items in the chapter 2 of the book.)
	%
	\item
	%
	A good way to learn the material of this chapter is to ask yourself:
	%
	What should you do in order to design an imperative programming language and
	build a foundation of the designed language?
	%
	Think about this a little, and compare your answer with what I'll explain.
	%
\end{enumcirc}

\section{Syntax}

\begin{enumcirc}
	%
	\item
	%
	Variables, and read and update of them \dots key concepts or operations for
	imperative computation.
	%
	\item
	%
	Syntax that supports these concepts and operations:
	%
	\begin{center}
		\begin{minipage}{0.9\textwidth}
			\begin{grammar}
				<intexp> ::= 0 | 1 | 2 | \dots | <var> | ... \footnotemark

				<boolexp> ::= true | false | ... \footnotemark

				<comm> ::= <var> := <intexp> \footnotemark | skip | <comm> ; <comm> \footnotemark
				\alt if <boolexp> then <comm> else <comm>
				\alt while <boolexp> do <comm>
			\end{grammar}
		\end{minipage}
	\end{center}
	\footnoteeqn[-3]{same as before}
	\footnoteeqn{
		\begin{minipage}{0.9\textwidth}
			%
			almost the same as that of $\chevrons{\textit{assert}}$.
			%
			The exception is that $\chevrons{\textit{boolexp}}$ doesn't include
			quantifiers. (Why?)
			%
		\end{minipage}
	}
	\footnoteeqn{update of a variable}
	\footnoteeqn{order matters}

	As in the case of predicate logic, you can understand $\gram{comm}$ as the set
	of all finite derivation trees, or as a multi-sorted initial algebra for the
	signature determined by the grammar.
	%
	\item
	%
	It is a language for expressing a sequence of variable reads and variable
	updates.
	%
\end{enumcirc}

\section{Baby domain theory}

\begin{enumcirc}
	%
	\item
	%
	Giving a denotational semantics to our simple imperative language is not as
	straightforward as doing so with predicate logic, because of loop.
	%
	\item
	%
	$\bbrackets{-}$ : $\gram{comm} \to \dots$\;\footnote{don't worry about this target set now.}

	We want the following equation for loop unrolling to hold:
	%
	\begin{align*}
		\bbrackets{\cwhile{b}{c}} & = \bbrackets{\cif{b}{c ; \cwhile{b}{c}}{\cskip}}             \\
		                          & = \dots \bbrackets{\cwhile{b}{c}} \dots                      \\
		                          & = F \prths{\bbrackets{\cwhile{b}{c}}} \textrm{ for some } F.
	\end{align*}
	%
	But a function $F$ on a set may or may not have such a fixed point.
	%
	\item
	%
	Then, why should $F$ in the above have a fixed point?
	%
	Because it is something that can be implemented by a program.
	%
	\[
		F \; ``=" \footnotemark \bbrackets{\cif{b}{c ; \Box}{\cskip}}.
	\]
	\footnoteeqn[0]{informally}
	%
	One objective of domain theory is to formalize good properties enjoyed by such
	program implementable functions without going into the low-level details of
	computability theory.
	%
	\item
	%
	High-level meta heuristic behind domain theory:
	%
	\begin{enumrm}
		%
		\item
		%
		Consider a set together with some structure.
		%
		\item
		%
		Use functions between such sets that respect the structures.
		%
		\item
		%
		Why on (i) and (ii)?
		%
		Because if done well, functions in (ii) will always have fixed points.
		%
	\end{enumrm}
	%
	\item
	%
	Key definitions:
	%
	\begin{definition}[partial order]
		%
		A binary relation $\sqsubseteq$ on a set $S$ is a \ul{partial order} if
		%
		\begin{enumrm}
			%
			\item
			%
			$x \sqsubseteq x$ for all $x \in S$ (reflexivity);
			%
			\item
			%
			for all $x, y, z \in S$, if $x \sqsubseteq y$ and $y \sqsubseteq z$,
			%
			then $x \sqsubseteq z$ (transitivity); and
			%
			\item
			%
			for all $x, y \in S$, if $x \sqsubseteq y$ and $y \sqsubseteq x$, then $x = y$
			(anti-symmetry).
			%
		\end{enumrm}
		%
		A set $S$ with a partial order $\sqsubseteq$ is called a
		%
		\ul{partially ordered set} or \ul{poset}.
		%
	\end{definition}
	%
	\begin{definition}
		%
		A \ul{chain} in a poset $\prths{S, \sqsubseteq}$ is a (countably) infinite
		sequence
		%
		\[
			x_0, x_1, x_2, \dots , x_n, \dots
		\]
		%
		of elements in $S$ s.t. $x_n \sqsubseteq x_{n+1}$ for all $n \geq 0$.
	\end{definition}
	%
	\begin{definition}
		A \ul{pre-domain} is a poset $\prths{S, \sqsubseteq}$ such that \ul{all
			chains have least upper bounds}.
		%
		That is, for every chain $\braces{x_n}_{n \geq 0}$ in $S$, there exists $y$
		%
		\footnote{notation for $y$: $\displaystyle\bigsqcup_{n \geq 0} x_n$}
		%
		in $S$ s.t.
		%
		\begin{enumrm}
			%
			\item
			%
			\footnote{$y$ is an upper bound}
			%
			$x_n \sqsubseteq y$ for all $n \geq 0$
			%
			\item
			%
			\footnote{$y$ is the least such}
			%
			for any $z$ in $S$, if $x_n \sqsubseteq z$ for every $n \geq 0$, then $y
				\sqsubseteq z$.
			%
		\end{enumrm}
		%
	\end{definition}
	%
	\begin{definition}
		%
		A \ul{domain} is a pre-domain $\prths{S, \sqsubseteq}$ that has
		%
		\ul{the least element}, often denoted $\bot$.
		%
		(meaning: for all $x \in S$, $\bot \sqsubseteq x$)
		%
	\end{definition}
	%
	\begin{definition}
		%
		Let $\prths{S_1, \sqsubseteq_1}$ and $\prths{S_2, \sqsubseteq_2}$ be
		pre-domains.
		%
		A function $f : S_1 \to S_2$ is \ul{continuous} if for every chain
		%
		$\braces{x_n}_{n \geq 0}$ in $S_1$,
		%
		$f \prths{\bigsqcup_{n \geq 0} x_n}$ is the least upper bound of
		%
		$\braces{f \prths{x_n}}_{n \geq 0}$ in $S_2$.
		%
	\end{definition}
	%
	\begin{definition}
		%
		A function $f : S_1 \to S_2$ is \ul{monotone} if for all $x, y \in S_1$,
		%
		\[
			x \sqsubseteq_1 y \implies f \prths{x} \sqsubseteq_2 f \prths{y}.
		\]
		%
		When $S_1$ and $S_2$ are domains with least elements $\bot_1$ and $\bot_2$, we
		say that a function
		%
		$f : S_1 \to S_2$ is \ul{strict} if
		%
		$f \prths{\bot_1} = \bot_2$.
		%
	\end{definition}
	%
	\begin{exercise} \label{ex:cont-mon}
		%
		Show that if $f$ is continuous, it is monotone.
		%
	\end{exercise}
	\item
	%
	What's going on here?
	%
	What are the intuitions behind these definitions?
	%
	\begin{enumrm}
		%
		\item
		%
		$x \sqsubseteq y$ \dots intuitively means that $y$ has more information than $x$ or
		$x$ and $y$ have the same amount of information.
		%
		\newpage
		%
		\begin{exampletab}
			%
			\begin{enumalpha}
				%
				\item
				%
				$\Z^* \cup \Z^\omega$ \dots finite or infinite sequence \footnote{
					results produced so far by sequence-producing computation.
				} of integers.
				%
				(without the infinite part, not a pre-domain)
				%
				$x \sqsubseteq y$ iff $x$ is an initial subsequence (or prefix) of $y$.
				%
				\begin{align*}
					\chevrons{3, 1, 4} & \sqsubseteq \chevrons{3, 1, 4, 1, 5, 9} \\
					\chevrons{3, 1, 4} & \not\sqsubseteq \chevrons{3, 1, 5, 9}   \\
				\end{align*}
				%
				\item
				%
				$\Z_{\bot} \defeq \Z \cup \braces{\bot}$ \dots lifted $\Z$.
				%
				\footnote{integer output of an integer-returning computation}
				%
				\[
					\forall x, y \in \Z_{\bot}, \quad x \sqsubseteq y \iff x = \bot \textrm{ or } x = y.
				\]
				%
				\item
				%
				$\prths{2^\Z, \subseteq}$ \dots power set of $\Z$.
				%
				\item
				%
				vertical domain of natural numbers \dots
				%
				$
					\left.
					\begin{array}{c}
						\infty =: \top \\
						\vdots         \\
						2              \\
						1              \\
						0 =: \bot
					\end{array}
					\right)
					=: \N^\top
				$
				%
			\end{enumalpha}
		\end{exampletab}
		%
		\item
		%
		The monotonicity means the preservation of the approximation
		relation.\footnote{ editor: if $a$ is an approximation of $b$, then $f(a)$ is
			an approximation of $f(b)$ }
		%
		One intuition behind continuity of $f$ is that in order to produce finite
		amount of information in its output, $f$ consumes only finite amount
		information in its input.
		%
		\begin{exampletab}
			%
			\begin{enumalpha}
				%
				\item
				%
				\begin{align*}
					\textrm{set}                                    & \in \brackets{Z^{*,\omega} \to 2^\Z} \\
					\textrm{set} \prths{\chevrons{x_1, \dots, x_n}} & = \braces{x_1, \dots, x_n}           \\
					\textrm{set} \prths{\chevrons{x_1, x_2, \dots}} & = \braces{x_1, x_2, \dots}           \\
				\end{align*}
				%
				If $A \subseteq \textrm{set}\prths{s}$ and $A$ is finite, then there is a
				finite prefix $s_0$ of $s$ (i.e., $s_0 \sqsubseteq s$) s.t. $A \subseteq
					\textrm{set}\prths{s_0}$.
				%
				\begin{exercisetab}
					%
					Show that if $f \in \brackets{Z^{*,\omega} \to 2^\Z}$ is continuous, it
					satisfies the above property.
					%
					Also, show that if $f \in \brackets{Z^{*,\omega} \to 2^\Z}$ is monotone and
					satisfies the property, then $f$ is continuous.
					%
				\end{exercisetab}
				%
				\item
				%
				$f \in \brackets{2^\Z \to \N^\top}$
				%
				\[
					f \prths{A} = \begin{cases}
						\abs{A} \textrm{ if } $A$ \textrm{ is finite } \\
						\top \textrm{ if } $A$ \textrm{ is infinite }
					\end{cases}
				\]
				%
				\begin{exercisetab}
					%
					A function from $2^\Z$ to a predomain $P$ is \ul{finitely generated} if for all
					$A \in 2^\Z$, $f \prths{A}$ is the least upper bound of
					%
					$\set{f\prths{A_0}}{A_0 \subseteq A \textrm{ and } A_0 \textrm{ is finite}}$.
					%
					Show that $f$ is continuous iff it is finitely generated.
					%
				\end{exercisetab}
				%
			\end{enumalpha}
			%
		\end{exampletab}
		%
		\item
		%
		John Reynolds' phrase in page 108: \\
		%
		\emph{
			``\dots Instead, it is based on the
			physical \footnote{ Domain theory attempts to capture this aspect of computation
			} limitations of communication: one cannot predict the future of input, nor
			receive an infinite amount of information in a finite amount of time, nor
			produce output except at finite times \dots''
		}
		%
	\end{enumrm}
	%
	\item
	%
	One important reason of doing domain theory is to have the following ``least
	fixed-point theorem'':
	%
	\begin{property}[Least Fixed-Point Theorem]
		%
		If $D$ is a domain and $f$ is a continuous function from $D$ to $D$,
		%
		\[
			x = \bigsqcup_{n = 0}^{\infty} f^n \prths{\bot \footnotemark}
		\]
		\footnoteeqn[0]{least element of $D$}
		%
		is the least fixed-point of $f$.
		%
		(That is, $f \prths{x} = x$ and for all $y \in D$ s.t. $f \prths{y} = y$, $x \sqsubseteq y$.)
		%
	\end{property}
	%
	\begin{proof}
		%
		By \cref{ex:cont-mon}, $f$ is monotone.
		%
		Using induction and $\bot$'s being the least element, we can show that
		%
		$\braces{f^n \prths{\bot}}_{n \geq 0}$ is a chain in $D$.
		%
		Since $D$ is a domain, the least upper bound
		%
		$\bigsqcup_{n = 0}^{\infty} f^n \prths{\bot}$ exists.
		%
		Furthermore, by the continuity of $f$,
		%
		\[
			f \prths{x} =
			f \prths{\bigsqcup_{n = 0}^{\infty} f^n \prths{\bot}} =
			\bigsqcup_{n = 0}^{\infty} f^{n + 1} \prths{\bot} =
			\bigsqcup_{n = 0}^{\infty} f^n \prths{\bot} =
			x.
		\]
		%
		$\therefore x$ is a fixed point of $f$.

		To show that x is a least such, consider a fixed point $y$ of $f$.
		%
		Then, by induction, we can show that $y$ is an upper bound of the chain
		%
		$\braces{f^n \prths{\bot}}_{n \geq 0}$.
		%
		$\therefore x \sqsubseteq y$.
		%
	\end{proof}
	%
	\item
	%
	When $P, P'$ are predomains, we write $\brackets{P \toc P'}$ for the set of
	continuous functions.
	%
	When $\brackets{P \toc P'}$\footnote{
		\begin{minipage}{0.9\textwidth}
			%
			Although we rush here, this function space construction is very important.
			%
			It lets domain theory be applicable to functional languages.
			%
		\end{minipage}
	} is given pointwise order
	$\sqsubseteq$,
	%
	\[
		f \sqsubseteq g \iff f \prths{x} \sqsubseteq_{P'} g \prths{x}
		\textrm{ for all } x \in P, \quad f, g \in \brackets{P \toc P'},
	\]
	it becomes a predomain where the limit of a chain $\braces{f_n}_{n \geq 0}$ is
	defined pointwise $x \mapsto \bigsqcup_{n = 0}^\infty f_n \prths{x}$.
	%
	Furthermore, if $P'$ is a domain with the least element $\bot'$, then
	$\brackets{P \toc P'}$ is also a domain with $x \mapsto \bot'$ as its least
	element.
	%
	\item
	%
	$D$ is a domain with the least element $\bot$.
	%
	Define $Y_D$ to be the following function from $\brackets{D \toc D}$ to $D$:
	%
	\[
		Y_D \prths{f} = \bigsqcup_{n = 0}^{\infty} f^n \prths{\bot}.
	\]
	%
	\begin{lemma}
		%
		$Y_D$ is continuous\footnote{%
			This means that the very act of computing a fixed point of a given function is continuous.%
		}.
		%
	\end{lemma}
	\begin{proof}
		%
		Exercise.
		%
	\end{proof}
	%
	\item
	%
	There are a lot of interesting results in domain theory, some of which we will
	cover later in the course.
	%
	Before finishing this mini review of domain theory, I want to explain the
	lifting construction.
	%
	\begin{enumrm}
		%
		\item
		%
		\begin{itemize}
			%
			\item
			      %
			      $P_\bot := P \cup \braces{\bot}$ for a predomain $P$.
			      %
			\item
			      %
			      $x \sqsubseteq_{P_\bot} y$ iff $x = \bot$ or $x, y \in P \textrm{ and } x \sqsubseteq_P y$
			      for $x, y \in P_\bot$.
			      %
			\item
			      %
			      Intuitively, we are adding the least element to $P$ and converting $P$ to a
			      domain.
			      %
		\end{itemize}
		%
		\item
		%
		$i_\uparrow \in \brackets{P \toc P_\bot}$, sometimes called \ul{unit}.
		%
		\[
			i_\uparrow \prths{x} = x \textrm{ for all } x \in P.
		\]
		%
		\item
		%
		For each $f \in \brackets{P \toc P'_\bot}$,
		%
		\begin{align*}
			f_\doublebot              & \in \brackets{P_\doublebot \toc P'_\doublebot} \\
			f_\doublebot \prths{\bot} & = \bot                                         \\
			f_\doublebot \prths{x}    & = f \prths{x} \textrm{ for all } x \in P.
		\end{align*}
		%
		sometimes called \ul{Kleisli extension}.
		%
		\item
		%
		Why should we care about (ii) and (iii)?
		%
		Because they allow us to compose continuous functions from $P$ to $P'_\bot$
		%
		\[
			f \in \brackets{P \toc P'_\bot} \; \textrm{ and } \;
			g \in \brackets{P' \toc P''_\bot} \implies
			g_\doublebot \circ f \in \brackets{P \toc P''_\bot}.
		\]
		%
		We can view $(-)_\doublebot \circ (-)$ as a new composition operator $\circ'$.
		%
		Then, $\circ'$ is associative and $i_\uparrow \circ' f = f = f \circ'
			i_\uparrow$.
		%
		This means that $\prths{-_\bot, i_\uparrow, -_\doublebot}$ gives rise to a
		monad on predomains.
		%
	\end{enumrm}
	%
\end{enumcirc}

\section{Denotational semantics of the simple imperative language}

\begin{enumcirc}
	%
	\item
	%
	Recall that $\Sigma = \gram{var} \to \Z$.
	%
	$\Sigma$ is a predomain when given the discrete order $\sqsubseteq$.
	%
	(That is, $x \sqsubseteq y$ iff $x = y$ for all $x, y \in \Sigma$.)
	%
	\begin{align*}
		\intexpsem{-}  & \in \gram{intexp} \to \brackets{\Sigma \to \Z} (same as before)  \\
		\boolexpsem{-} & \in \gram{boolexp} \to \brackets{\Sigma \to \B} (same as before) \\
		\commexpsem{-} & \in \gram{comm} \to \brackets{\Sigma \toc \Sigma_\bot}
	\end{align*}
	%
	\begin{align*}
		\bbrackets{\cassign{x}{e}} \sigma     & = \aug{\sigma}{x: \bbrackets{e} \sigma}                                                        \\
		\bbrackets{\cskip} \sigma             & = \sigma                                                                                       \\
		\bbrackets{\cseq{c_1}{c_2}} \sigma    & = \bbrackets{c_2}_\doublebot \prths{\bbrackets{c_1} \sigma}                                    \\
		\bbrackets{\cif{b}{c_1}{c_2}} \sigma  & = \cif{\bbrackets{b} \sigma}{\bbrackets{c_1} \sigma}{\bbrackets{c_2} \sigma}                   \\
		\bbrackets{\cwhile{b}{c}} \sigma      & = Y_{\Sigma_\bot} \prths{F}                                                                    \\
		\textrm{ where } F                    & \in \brackets{\Sigma \toc \Sigma_\bot} \toc \brackets{\Sigma \toc \Sigma_\bot}                 \\
		\textrm{and }F\prths{f}\prths{\sigma} & := \cif{\bbrackets{b} \sigma}{\prths{f_\doublebot \circ \bbrackets{c}} \prths{\sigma}}{\sigma}
	\end{align*}
	%
	Note: $Y_{\Sigma_\bot}$ and $\toc$ are where we get something by using domain
	theory.
	%
	\item
	%
	Why \ul{least} fixed point?
	%
	Because the least fixed point maps an input state to $\bot$
	%
	(denoting non-termination, hence, the absence of any information)
	%
	whenever the corresponding output state is not uniquely determined by the
	equation $F(f) = f$.
	%
	For example,
	%
	\begin{enumrm}
		%
		\item
		%
		least fixed point \dots
		%
		$\bbrackets{\cwhile{\textrm{true}}{\cskip}} \sigma = \bot$.
		%
		\item
		%
		non-least fixed point
		%
		\dots $\bbrackets{\cwhile{\textrm{true}}{\cskip}} \sigma = \sigma$.
		%
		\begin{align*}
			F\prths{f}\prths{\sigma} & = \cif{\bbrackets{\textrm{true}} \sigma}{\prths{f_\doublebot \circ \bbrackets{\cskip}} \prths{\sigma}}{\sigma} \\
			                         & = \prths{f_\doublebot \circ \bbrackets{\cskip}} \prths{\sigma}                                                 \\
			                         & = f_\doublebot \prths{\bbrackets{\cskip} \prths{\sigma}}                                                       \\
			                         & = f_\doublebot \prths{\sigma}                                                                                  \\
			                         & = f\prths{\sigma}                                                                                              \\
		\end{align*}
		%
		That is, $F(f) = f$.
		%
	\end{enumrm}
	%
	Later when we consider the correspondence between denotational semantics and
	operational semantics, we will answer this question more rigorously.
	%
	\item
	%
	Design decision of our language seen in the semantics:
	%
	\[
		\intexpsem{-} : \gram{intexp} \to \Sigma \to \Z
	\]
	%
	all integer expressions terminate and do not raise exceptions.
	%
	A similar remark applies to boolean expressions as well.
	%
	\item
	%
	Choosing the type of the semantics function such as
	%
	$\gram{intexp} \to \Sigma \to \Z$
	%
	is the most important step in defining the semantics.
	%
	It also clarifies certain major design decisions of the target programming
	language.
	%
\end{enumcirc}

\section{Variable declaration and substitution}

\begin{enumcirc}
	%
	\item
	%
	\begin{grammar}
		<comm> ::= ... | newvar <var> := <intexp> in <comm>
	\end{grammar}
	%
	This is a construct that doesn't increase the expressivity of the language but
	enables the programmers to combat the complexity of software by introducing the
	ideas of scope and local variables.
	%
	\item
	%
	\[
		\bbrackets{\cnew{v}{e}{c}} \sigma =
		\prths{
			\prths{
				\lambda \sigma' \in \Sigma \;.\; \aug{\sigma'}{v : \sigma v}
			}_\doublebot \circ \bbrackets{c}
		} \prths{
			\aug{\sigma}{v : \bbrackets{e} \sigma}
		}
	\]
	%
	\dots
	%
	$\lambda \sigma' \in \Sigma \;.\; \aug{\sigma'}{v : \sigma v}$
	%
	means the function $\sigma' \mapsto \aug{\sigma'}{v: \sigma v}$, restoring the
	old value.
	%
	We didn't have to do something like this when we interpreted quantifications in
	predicate logic.
	%
	This is because there we didn't return a state, but a boolean value.
	%
	\item
	%
	How do we know that this is a sensible definition?
	%
	By checking expected properties like \cref{prop:coincidence} and
	\cref{prop:renaming}.

	$\fv{c}$ \dots free variables appearing in $c$. (textbook page 40)

	$\fa{c}$ \dots free assigned variables appearing in $c$. (textbook page 41)

	\begin{property}[Coincidence]\label{prop:coincidence}
		%
		\;\\
		%
		\vspace{-1.5em}
		%
		\begin{enumalpha}
			%
			\item
			%
			$\sigma w = \sigma' w \textrm{ for all } w \in \fv{c}$
			%
			\begin{align*}
				\implies & \prths{\bbrackets{c}\sigma = \bbrackets{c}\sigma' = \bot} \textrm{ or } \\
				         & \prths{
					\bbrackets{c}\sigma, \bbrackets{c}\sigma' \in \Sigma \textrm{ and }
					\prths{\bbrackets{c}\sigma} w = \prths{\bbrackets{c}\sigma'} w \textrm{ for all } w \in \fv{c}
				}
			\end{align*}
			%
			\item
			%
			$
				\bbrackets{c}\sigma \neq \bot \implies
				\prths{\bbrackets{c}\sigma} w = \sigma w \textrm{ for all } w \notin \fa{c}.
			$
		\end{enumalpha}
		%
	\end{property}
	%
	\begin{property}[Renaming]\label{prop:renaming}
		%
		\begin{multline*}
			\subsctext{v}{new} \notin \fv{c'} - \braces{v} \\
			\implies \bbrackets{\cnew{v}{e}{c'}} \sigma =
			\bbrackets{\cnew{\subsctext{v}{new}}{e}{\subst{c'}{v}{\subsctext{v}{new}}}} \sigma
		\end{multline*}
		%
	\end{property}
	%
\end{enumcirc}

\section{Syntactic Sugar}

\begin{enumcirc}
	%
	\item
	%
	Introduction of a construct by defining its meaning in terms of existing
	constructs in the language.
	%
	\item
	%
	Three definitions of for loop:
	%
	\begin{enumrm}
		%
		\item
		%
		$
			\prths{
				\cfor{v}{e_0}{e_1}{c}} :=
			\prths{
				\cseq{
					\cassign{v}{e_0}
				}{
					\cwhile{v \le e_1}{
						\prths{
							\cseq{c}{\cassign{v}{v + 1}}
						}
					}
				}
			}
		$
		%
		\item
		%
		$
			\prths{
				\cfor{v}{e_0}{e_1}{c}} :=
			\prths{
				\cnew{v}{e_0}{
					\cwhile{v \le e_1}{
						\prths{
							\cseq{c}{\cassign{v}{v + 1}}
						}
					}
				}
			}
		$
		%
		\item
		%
		$
			\prths{
				\cfor{v}{e_0}{e_1}{c}} := \\
			\textrm{\quad}
			\prths{
				\cnew{w}{e_1}{
					\cnew{v}{e_0}{
						\cwhile{v \le w}{
							\prths{
								\cseq{c}{\cassign{v}{v + 1}}
							}
						}
					}
				}
			}
		$,

		where $w \ne v$ and $w \notin \fv{e_0} \cup \fv{c}$.
		%
		\item
		      %
		      (iii) with the condition $v \notin \fv{c}$.
		%
	\end{enumrm}
	%
	\item
	%
	The for loop should be something easier to understand than while.
	%
	In this regard, (i) $<$ (ii) $<$ (iii) $<$ (iv).
	%
\end{enumcirc}

\section{Arithmetic errors}

\begin{enumcirc}
	%
	\item
	%
	How should we deal with $x \div 0$, underflow and overflow?
	%
	\item
	%
	Two approaches:
	%
	\begin{enumrm}
		%
		\item
		%
		early stop with error
		%
		\item
		%
		some default choice and computation continued:
		%
		ad hoc but it can become less ad hoc if we ensure that the default choices
		satisfy \ul{certain properties} such as
		%
		\begin{align*}
			\bbrackets{\prths{x + y} \times 0} \sigma                       & = 0                     \\
			\bbrackets{x \div 0 = x \div 0} \sigma                          & = \ttt                  \\
			\bbrackets{\cseq{\cassign{y}{x \div 0}}{\cassign{y}{e}}} \sigma & =
			\bbrackets{\cassign{y}{e}} \sigma \textrm{ when } y \notin \fv{e}                         \\
			\bbrackets{\cif{x + y = z}{c}{c}} \sigma                        & = \bbrackets{c} \sigma. \\
		\end{align*}
		%
	\end{enumrm}
	%
\end{enumcirc}

\section{Soundness and full abstraction}

\begin{enumcirc}
	%
	\item
	%
	The semantics defined so far looks ok, but is there any formal way to confirm
	this?
	%
	\item
	%
	One approach is to show that the semantics assigns the same meaning to two
	commands $c_1$ and $c_2$ only when $c_1$ and $c_2$ should be equal intuitively.
	%
	That is,
	%
	\[
		\bbrackets{c_1} = \bbrackets{c_2} \implies c_1 ``=" \footnotemark c_2.
	\]
	\footnoteeqn[0]{our intuitive notion of equality defined separately}
	%
	This property is called \ul{soundness}.
	%
	Its converse
	%
	\[
		\bbrackets{c_1} = \bbrackets{c_2} \impliedby c_1 ``=" c_2
	\]
	%
	is called \ul{full abstraction}.
	%
	\item
	%
	Now how to define $``="$?
	%
	We use a set of observable phrases with a hole or \ul{contexts}, $\Cc$, and a
	set of \ul{observations}, $\Oc$, which are functions from observable phrases to
	outcomes.
	%
	\[
		c_1 ``=" c_2 \iff \forall c \in \Cc \footnotemark ,\; \forall o \in \Oc \footnotemark,\;
		o \prths{c \brackets{c_1}\footnotemark} = o \prths{c \brackets{c_2}}
	\]
	\footnoteeqn[-2]{intuitively means all use cases}
	\footnoteeqn{intuitively means user's observations}
	\footnoteeqn{filling the hole of $c$ with $c_1$}
	%
	\begin{enumrm}
		%
		\item
		%
		Intuitively, this condition says that under all use cases, the user cannot
		observe the difference between $c_1$ and $c_2$.
		%
		\item
		%
		This is sometimes called \ul{observataional equivalence}.
		%
		\item
		%
		Note that this is not a syntax-directed (or compositional) definition.
		%
	\end{enumrm}
	%
	\begin{example}
		%
		Assume that $v_0, \dots , v_{n-1}$ are all the free variables in $c_1$ and
		$c_2$.
		%
		\[
			\Cc = \set{
				\begin{array}{c}
					\begin{aligned}
						\cnew{v_0     & }{k_0}{        \\
						\cnew{v_1     & }{k_1}{        \\
						              & \vdots         \\
						\cnew{v_{n-1} & }{k_{n-1}}{}}}
					\end{aligned}
					\\
					\prths{
						\begin{aligned}
							\brackets{-}; \textrm{ if } v_i = k & \textrm{ then } \cskip                         \\
							                                    & \textrm{ else } \cwhile{\textrm{true}}{\cskip}
						\end{aligned}
					}
				\end{array}
			}{
				\begin{array}{c}
					k, k_0, \dots , k_{n-1} \in \Z \\
					i \in \braces{0, \dots , n-1}
				\end{array}
			}
		\]
		%
		\vspace{1em}
		%
		\[
			\Oc = \braces{\lambda c \;.\; \cif{\bbrackets{c} \sigma_0 = \bot}{0}{1}}
		\]
		%
		where $\sigma_0 \prths{x} = 0$ for all $x$.
	\end{example}

\end{enumcirc}
\chapter{Program Specifications and Their Proofs}

\section{Motivation}

\begin{enumcirc}
	%
	\item
	%
	Are there any methods that let us specify desired properties or intended
	behaviors of a program and prove that the specified properties indeed hold?
	%
	For instance, consider the program:
	%
	\[
		\subsctext{c}{div3} =
		\prths{
			\begin{array}{l}
				a := 0;                                        \\
				b := x;                                        \\
				\textrm{while } \prths{b \geq 3} \textrm{ do } \\
				\qquad b := b - 3;                             \\
				\qquad a := a + 1                              \\
			\end{array}
		}
	\]
	%
	We want to express formally that the program divides $x$ by $3$ and stores the
	quotient in $a$ and the remainder in $b$.
	%
	We also want to prove this formal specification.
	%
	\item
	%
	We will study Hoare logic and its variant for total correctness.
	%
	They provide the kind of methods that we are looking for.
	%
	\item
	%
	Hoare logic and its total-correctness variant are the basis of modern automatic
	software verifiers, such as Facebook's infer.
	%
	\item
	%
	Another reason for studying Hoare logic is that it shows why we need or where
	we use denotational semantics that we studied.
	%
\end{enumcirc}

\section{Syntax and semantics of specifications}

\begin{enumcirc}
	%
	\item
	%
	Specifications are a new type of phrases that formally express properties of
	programs.
	%
	\item
	%
	Syntax in terms of abstract grammar:
	%
	\begin{center}
		\begin{minipage}{0.5\textwidth}
			\grammarindent5em
			\begin{grammar}
				<spec> ::= \{<assert>\} <comm> \{<assert>\}
				\alt [<assert>] <comm> [<assert>]
			\end{grammar}
		\end{minipage}
	\end{center}
	%
	\begin{exampletab}
		%
		\[
			\subsctext{c}{fib} = \prths{
				\begin{array}{c}
					k := 1;\; y := 0;\; x:=1;\; \\
					\cwhile{k \ne n}{           \\ \prths{t := y;\; y := x;\; x := x + t;\; k := k + 1}}
				\end{array}
			}
		\]
		%
		\[
			\begin{array}{c}
				\braces{n \ge 0}
				\subsctext{c}{fib}
				\braces{x = \textrm{fib}\prths{n}}
				\qquad \qquad
				\braces{\true}
				\subsctext{c}{fib}
				\braces{x = \textrm{fib}\prths{n}}
				\\[0.3em]
				\brackets{n \ge 0}
				\subsctext{c}{fib}
				\brackets{x = \textrm{fib}\prths{n}}
				\qquad \qquad
				\brackets{\true}
				\subsctext{c}{fib}
				\brackets{x = \textrm{fib}\prths{n}}
				\\[0.3em]
				\braces{x \ge 0}
				\subsctext{c}{div3}
				\braces{x = 3a + b \wedge 0 \le b < 3 \wedge a \ge 0}
				\\[0.3em]
				\brackets{x \ge 0}
				\subsctext{c}{div3}
				\brackets{x = 3a + b \wedge 0 \le b < 3 \wedge a \ge 0}
				\\[0.3em]
				\brackets{\true}
				\subsctext{c}{div3}
				\brackets{x = 3a + b \wedge 0 \le b < 3 \wedge a \ge 0}
				\\[0.3em]
				\brackets{\true}
				\subsctext{c}{div3}
				\brackets{x = 3a + b \wedge 0 \le b < 3 \wedge a \ge 0}
			\end{array}
		\]
		%
	\end{exampletab}
	%
	\item
	%
	Intuitive reading:
	%
	\begin{enumrm}
		%
		\item
		%
		$\braces{p} c \braces{q}$ holds iff when $c$ is run in a state satisfying $p$,
		\ul{and it terminates normally}, \footnote{condition} then the final state satisfies $q$.
		%
		\item
		%
		$\brackets{p} c \brackets{q}$ holds iff when $c$ is run in a state satisfying $p$,
		then \ul{it terminates normally}, \footnote{conclusion} and the final state satisfies $q$.
		%
	\end{enumrm}
	%
	Note that $\brackets{p} c \brackets{q}$ expresses a stronger property than
	$\braces{p} c \braces{q}$.
	%
	The former is called \ul{total correctness specification}, \footnote{also total
		correctness triple} and the latter \ul{partial correctness specification}.
	\footnote{also partial correctness triple, Hoare triple or triple}

	$p$ \dots precondition or precedent.

	$q$ \dots postcondition or consequent.

	\begin{exercise}
		%
		Among all the partial correctness and total correctness specification from
		above, pick those that hold.
		%
	\end{exercise}
	%
	\item
	%
	Formal semantics:
	%
	\begin{align*}
		\bbrackets{-}                           & \in \brackets{\gram{spec} \to \B} \\
		\bbrackets{\braces{p} c \braces{q}}     & = \ttt \quad \textrm{ iff } \quad
		\bbrackets{p} \sigma = \ttt \wedge \bbrackets{c} \sigma \neq \bot \implies
		\bbrackets{q} \prths{\bbrackets{c} \sigma} = \ttt                           \\
		\bbrackets{\brackets{p} c \brackets{q}} & = \ttt \quad \textrm{ iff } \quad
		\bbrackets{p} \sigma = \ttt \implies \bbrackets{c} \sigma \neq \bot \wedge
		\bbrackets{q} \prths{\bbrackets{c} \sigma} = \ttt
	\end{align*}
	%
\end{enumcirc}

\section{Inference rules}

\begin{enumcirc}
	%
	\item
	%
	Methods or rules for proving or deriving partial or total correctness triples.
	%
	\item
	%
	\;\vspace{-1.5em}
	%
	\[
		\inferrule{\textrm{premises}}{\textrm{conclusion}}
		\qquad \qquad
		\inferrule{\varphi_1 \quad \varphi_2 \quad \dots \quad \varphi_n}{\psi}
	\]
	%
	(if $\varphi_1, \varphi_2, \dots, \varphi_n$ are true, then $\psi$ is true)
	%
	\item
	%
	Rules associated with program constructs:
	%
	\[
		\begin{array}{c}
			\inferrule
			{ }
			{\braces{p} \cskip \braces{p}}
			\qquad \qquad
			\inferrule
			{ }
			{\brackets{p} \cskip \brackets{p}}
			\\[2em]
			\inferrule
			{\braces{p} c_1 \braces{r}                    \\ \braces{r} c_2 \braces{q}}
			{\braces{p} c_1 ; c_2 \braces{q}}
			\qquad \qquad
			\inferrule
			{\brackets{p} c_1 \brackets{r}                \\ \brackets{r} c_2 \brackets{q}}
			{\brackets{p} c_1 ; c_2 \brackets{q}}
			\\[2em]
			\inferrule
			{\braces{p \wedge b} c_1 \braces{q}           \\ \braces{p \wedge \neg b} c_2 \braces{q}}
			{\braces{p} \cif{b}{c_1}{c_2} \braces{q}}
			\qquad \qquad
			\inferrule
			{\brackets{p \wedge b} c_1 \brackets{q}       \\ \brackets{p \wedge \neg b} c_2 \brackets{q}}
			{\brackets{p} \cif{b}{c_1}{c_2} \brackets{q}}
			\\[2em]
			\inferrule
			{ }
			{\braces{\subst{q}{x}{e}} x := e \braces{q}}
			\qquad \qquad
			\inferrule
			{ }
			{\brackets{\subst{q}{x}{e}} x := e \brackets{q}}
			\\[2em]
			\inferrule
			{\braces{i \wedge b} c \braces{i}}
			{\braces{i} \cwhile{b}{c} \braces{i \wedge \neg b}}
			\qquad \quad
			\inferrule
			{i \wedge b \Rightarrow e \ge 0 \footnotemark \\ \brackets{i \wedge b \wedge e = v_0 \footnotemark} c \brackets{i \wedge e < v_0}}
			{\brackets{i} \cwhile{b}{c} \brackets{i \wedge \neg b}}
		\end{array}
	\]
	\footnoteeqn[-1]{$i \wedge b \Rightarrow e \ge 0$ should be valid. That is, $\bbrackets{i \wedge b \Rightarrow e \ge 0} \sigma = \ttt$ for all $\sigma$.}
	\footnoteeqn{when $v_0$ does not occur free in $i$, $b$, $c$ or $e$}
	%
	\begin{enumrm}
		%
		\item
		%
		Note that rules for total correctness and the corresponding ones for partial
		correctness are identical except for the case of loop.
		%
		This is expected because these two notions differ only in their treatment of
		non-termination.
		%
		\item
		%
		The rules for while require that $i$ should be preserved by the body of the
		loop.
		%
		The one for total correctness additionally requires that the value of $e$
		should decrease whenever we run the loop body $c$ once, but it cannot be
		negative.
		%
		All these requirements together give the conclusions of the rules.
		%
		\item
		%
		The rules for assignment also deserve some thoughts.
		%
		They in a sense say that running an assignment backward symbolically is the
		same as doing substitution.
		%
		It holds because of the substitution theorem (Prop 1.3 and Prop 1.4 in the
		textbook).

		Reminder of the theorem specialized to our case here:
		%
		\[
			\begin{tikzcd}[cramped, sep=7em]
				\Sigma
				\arrow[r, "\lambda \sigma . \aug{\sigma}{x:\bbrackets{e}\sigma}"]
				\arrow[d, swap, "\bbrackets{\subst{q}{x}{e}}"] &
				\Sigma
				\arrow[d, "\bbrackets{q}"]\\
				\B \arrow[r,-,double, "="] &
				\B
			\end{tikzcd}
		\]
		%
	\end{enumrm}
	%
	Rules not associated with any specific program constructs
	%
	(sometimes called structural rules or adaptation rules):
	%
	\[
		\inferrule
		{p \Rightarrow p' \\ \braces{p'} c \braces{q'} \\ q' \Rightarrow q}
		{\braces{p} c \braces{q}}
		\qquad \qquad
		\inferrule
		{p \Rightarrow p' \\ \brackets{p'} c \brackets{q'} \\ q' \Rightarrow q}
		{\brackets{p} c \brackets{q}}
	\]
	%
	They are called \ul{the rule of consequence}.
	%
	They often enable us to use the other rules, in particular, those for loop and
	if.
	%
	\item
	%
	I omit many structural rules and the rule for newvar.
	%
	Look at the textbook if you are interested.
	%
\end{enumcirc}

\section{Example proof}

\[
	\subsctext{c}{div3} =
	\prths{
		\begin{array}{l}
			a := 0;                                        \\
			b := x;                                        \\
			\textrm{while } \prths{b \geq 3} \textrm{ do } \\
			\qquad b := b - 3;                             \\
			\qquad a := a + 1                              \\
		\end{array}
	}
\]

Goal: prove that
%
\[
	\braces{x \ge 0}
	\subsctext{c}{div3}
	\braces{x = 3a + b \wedge 0 \le b < 3}
\]

Proof:

{
\fontsize{3pt}{3pt}\selectfont
\[
	\scriptscriptstyle
	\inferrule{
		x \ge 0 \Rightarrow x = 3 \times 0 + x \wedge x \ge 0
		\\
		\inferrule{
			\inferrule{ }{
				\braces{x = 3 \times 0 + x \wedge x \ge 0}
				a := 0
				\braces{x = 3a + x \wedge x \ge 0}
			}
		}{
			\braces{x \ge 0}
			a := 0
			\braces{x = 3a + x \wedge x \ge 0}
		}
		\\
		\inferrule{ }{
			\braces{x = 3a + x \wedge x \ge 0}
			b := x
			\braces{x = 3a + b \wedge b \ge 0}
		}
	}{
		\braces{x \ge 0}
		a := 0;\; b := x
		\braces{x = 3a + b \wedge b \ge 0}
	}
\]

\[
	\inferrule{
		\inferrule{
			\inferrule{
				\substack{
					x = 3a + b \wedge b \ge 0 \wedge b \ge 3\\
					\Downarrow\\
					x = 3(a + 1) + (b - 3) \wedge b - 3 \ge 0\\
				}
				\\
				\inferrule{ }{
					\braces{x = 3(a + 1) + (b - 3) \wedge b - 3 \ge 0}
					b := b - 3
					\braces{x = 3(a + 1) + b \wedge b \ge 0}
				}
			}{
				\braces{x = 3a + b \wedge b \ge 0 \wedge b \ge 3}
				b := b - 3
				\braces{x = 3(a + 1) + b \wedge b \ge 0}
			}
			\\
			\inferrule{ }{
				\braces{x = 3(a + 1) + b \wedge b \ge 0}
				a := a + 1
				\braces{x = 3a + b \wedge b \ge 0}
			}
		}{
			\braces{x = 3a + b \wedge b \ge 0 \wedge b \ge 3}
			b := b - 3;\; a := a + 1
			\braces{x = 3a + b \wedge b \ge 0}
		}
	}{
		\braces{x = 3a + b \wedge b \ge 0}
		\cwhile{b \ge 3}{b := b - 3;\; a := a + 1}
		\braces{x = 3a + b \wedge 0 \le b < 3}
	}
\]

\[
	\inferrule
	{
		\braces{x \ge 0}
		a := 0;\; b := x
		\braces{x = 3a + b \wedge b \ge 0}
		\\
		\braces{x = 3a + b \wedge b \ge 0}
		\cwhile{b \ge 3}{b := b - 3;\; a := a + 1}
		\braces{x = 3a + b \wedge 0 \le b < 3}
	}{
		\braces{x \ge 0}
		\textrm{c}_{div3}
		\braces{x = 3a + b \wedge 0 \le b < 3}
	}
\]
}

\begin{enumcirc}
	%
	\item
	%
	In practice, people use the rule of consequence, without mentioning it
	explicitly.
	%
	Also, they use many derived rules.
	%
	\item
	%
	This proof has the flavor of running a program backward symbolically because of
	its heavy use of the assignment rule and the fact that the rule of consequence
	is used only when it is necessary.

	\begin{exercisetab}
		%
		Prove:
		%
		\begin{enumalpha}
			%
			\item
			%
			\[
				\braces{n \ge 1}
				\subsctext{c}{fib}
				\braces{x = \textrm{fib}\prths{n}}
			\]
			%
			\item
			%
			\[
				\subsctext{c}{Euclid} =
				\prths{
					\begin{array}{l}
						while \prths{a \ne b} \textrm{ do }                  \\
						\qquad \textrm{if } a > b \textrm{ then } a := a - b \\
						\qquad \textrm{else } b := b - a
					\end{array}
				}
			\]
			%
			\[
				\braces{a \ge 1 \wedge b \ge 1 \wedge a = a_0 \wedge b = b_0}
				\subsctext{c}{Euclid}
				\braces{a = \textrm{gcd}\prths{a_0, b_0}}
			\]
			%
		\end{enumalpha}
		%
	\end{exercisetab}

	\begin{exercisetab}
		%
		Find a forward rule for assignment.
		%
		That is, for all $p$ and $x, e$, find $q$ s.t.
		%
		\[
			\inferrule{ }{\brackets{p} x := e \brackets{q}}
		\]
		%
	\end{exercisetab}
	%
\end{enumcirc}

\section{Soundness}

\begin{theorem}[Soundness theorem]
	%
	If $\brackets{p} c \brackets{q}$ is derivable using
	%
	\ul{the rules that we studied},
	%
	\footnote{called rules in Hoare logic}
	%
	then $\bbrackets{\braces{p} c \braces{q}} = \ttt$, i.e.,
	%
	the triple $\braces{p} c \braces{q}$ holds.
	%
	If $\brackets{p} c \brackets{q}$ is derivable, then
	%
	$\bbrackets{\brackets{p} c \brackets{q}} = \ttt$.
	%
\end{theorem}
%
\begin{proof}
	%
	Intuitively, the theorem says that all rules are correct.
	%
	In fact, typical proofs of the theorem show the correctness of the rules in the
	following sense:

	If $\inferrule{\varphi_1 \quad \varphi_2 \quad \dots \quad \varphi_n}{\psi}$
	%
	then
	%
	\[
		\bbrackets{\varphi_1} = \ttt \wedge
		\bbrackets{\varphi_2} = \ttt \wedge \dots \wedge
		\bbrackets{\varphi_n} = \ttt \implies
		\bbrackets{\psi} = \ttt.
	\]
	%
	The rules for loop (or while) are the most important cases.
	%
	We will consider only the one for partial correctness.
	%
	\[
		\inferrule
		{\braces{i \wedge b} c \braces{i}}
		{\braces{i} \cwhile{b}{c} \braces{i \wedge \neg b}}
	\]
	%
	We first do a bit of rewriting for the semantics of specifications.
	%
	\begin{multline*}
		\bbrackets{\brackets{p} c \brackets{q}}
		\quad \textrm{ iff } \\
		\forall \sigma \in \Sigma ,\;
		\bbrackets{p} \sigma = \ttt \implies
		\bbrackets{q}_\bot \prths{\bbrackets{c} \sigma} \sqsubseteq \ttt \\
		\textrm{ where }
		\bbrackets{q}_\bot \in \brackets{\Sigma_\bot \to \B_\bot} \\
		\textrm{ s.t. }
		\bbrackets{q}_\bot \sigma = \bbrackets{q} \sigma \textrm{ for all } \sigma \in \Sigma
		\textrm{ and }
		\bbrackets{q}_\bot \prths{\bot} = \bot
	\end{multline*}
	%
	We need to prove that, if
	%
	\begin{equation}{} \label{eq:while-partial-correctness} \tag{$\star$}
		\forall \sigma .\;
		\bbrackets{i \wedge b} \sigma = \ttt \implies
		\bbrackets{i}_\bot \prths{\bbrackets{c} \sigma} \sqsubseteq \ttt
	\end{equation}
	%
	then
	%
	\[
		\forall \sigma .\;
		\bbrackets{i} \sigma = \ttt \implies
		\bbrackets{i \wedge \neg b}_\bot \prths{\bbrackets{\cwhile{b}{c}} \sigma} \sqsubseteq \ttt
	\]

	Assume that \cref{eq:while-partial-correctness} holds.
	%
	Let
	%
	\begin{align*}
		F                          & \in \brackets{\prths{\Sigma \to \Sigma_\bot} \toc \prths{\Sigma \to \Sigma_\bot}}                      \\
		F \prths{f} \prths{\sigma} & = \cif{\bbrackets{b} \sigma = \ttt}{\prths{ f_\doublebot \circ \bbrackets{c}} \prths{\sigma}}{\sigma}.
	\end{align*}
	%
	Define $f_n := F^n \prths{\bot}$ for all $n \ge 0$.
	%
	Then, $\displaystyle \bbrackets{\cwhile{b}{c}} = \bigcup_{n = 0}^\infty f_n$.
	%
	We will show that for all $n \ge 0$,
	%
	\begin{equation} \label{eq:while-partial-correctness-2} \tag{$\star\star$}
		\forall \sigma .\;
		\bbrackets{i} \sigma = \ttt \implies
		\bbrackets{i \wedge \neg b}_\bot \prths{f_n \prths{\sigma}} \sqsubseteq \ttt.
	\end{equation}
	%
	This is sufficient because for all $\sigma \in \Sigma$ s.t. $\bbrackets{i}
		\sigma = \ttt$,
	%
	\begin{align*}
		\bbrackets{i \wedge \neg b}_\bot \prths{\bbrackets{\cwhile{b}{c}} \sigma}
		 & = \bbrackets{i \wedge \neg b}_\bot \prths{\prths{\bigcup_{n = 0}^\infty f_n} \prths{\sigma}}       \\
		 & = \bbrackets{i \wedge \neg b}_\bot \prths{\bigcup_{n = 0}^\infty f_n \prths{\sigma}}               \\
		 & = \footnotemark \bigcup_{n = 0}^\infty \bbrackets{i \wedge \neg b}_\bot \prths{f_n \prths{\sigma}} \\
		 & \sqsubseteq \footnotemark \bigcup_{n = 0}^\infty \ttt = \ttt.
	\end{align*}
	\footnoteeqn[-1]{because $\bbrackets{i \wedge \neg b}_\bot$ is continuous}
	\footnoteeqn{because of \cref{eq:while-partial-correctness-2}}
	%
	Our proof of \cref{eq:while-partial-correctness-2} uses induction on $n$.

	\begin{itemize}
		%
		\item
		      %
		      Base case $n = 0$: $f_0 = \bot$.
		      %
		      \[
			      \therefore
			      \bbrackets{i \wedge \neg b}_\bot \prths{f_0 \prths{\sigma}} = \bbrackets{i \wedge \neg b}_\bot \prths{\bot} = \bot \sqsubseteq \ttt.
		      \]
		      %
		\item
		      %
		      Inductive case $n = m + 1$:
		      %
		      Pick $\sigma$ s.t. $\bbrackets{i} \sigma = \ttt$.
		      %
		      \begin{align*}
			       & \quad\, \bbrackets{i \wedge \neg b}_\bot \prths{f_{m + 1} \prths{\sigma}}                                    \\
			       & = \bbrackets{i \wedge \neg b}_\bot \prths{F\prths{f_m} \prths{\sigma}}                                       \\
			       & = \bbrackets{i \wedge \neg b}_\bot
			      \prths{\cif{\bbrackets{b} \sigma = \ttt}{\prths{ {f_m}_\doublebot \circ \bbrackets{c}} \prths{\sigma}}{\sigma}} \\
			       & = \cif
			      {\bbrackets{b} \sigma = \ttt}
			      {\prths{\bbrackets{i \wedge \neg b}_\bot \circ {f_m}_\doublebot} \prths{\bbrackets{c} \prths{\sigma}}}
			      {\bbrackets{i \wedge \neg b}_\bot \prths{\sigma}}                                                               \\
		      \end{align*}
		      %
		      \vspace{-3em}\\
		      %
		      Since $\bbrackets{i} \sigma = \ttt$, if $\bbrackets{b} \sigma \ne \ttt$, then
		      %
		      \[
			      \bbrackets{i \wedge \neg b}_\bot \prths{\sigma} = \bbrackets{i}_\bot \prths{\sigma} = \ttt \sqsubseteq \ttt.
		      \]
		      %
		      If $\bbrackets{b} \sigma = \ttt$ and $\bbrackets{c} \sigma = \bot$, then
		      %
		      \[
			      \prths{\bbrackets{i \wedge \neg b}_\bot \circ {f_m}_\doublebot} \prths{\bbrackets{c} \prths{\sigma}} = \bot \sqsubseteq \ttt.
		      \]
		      %
		      If $\bbrackets{b} \sigma = \ttt$ and $\bbrackets{c} \sigma \ne \bot$, then
		      %
		      $\bbrackets{i}_\bot \prths{\bbrackets{c}\sigma} = \ttt$
		      %
		      by \cref{eq:while-partial-correctness}.
		      %
		      Thus,
		      %
		      \[
			      \prths{\bbrackets{i \wedge \neg b}_\bot \circ {f_m}_\doublebot} \prths{\bbrackets{c} \prths{\sigma}} =
			      \bbrackets{i \wedge \neg b}_\bot \prths{ f_m \prths{\bbrackets{c} \prths{\sigma}} } \sqsubseteq \ttt
		      \]
		      %
		      by induction hypothesis.
		      %
	\end{itemize}

\end{proof}



\end{document}

