\chapter{Transition Semantics} \label{ch:transition}

\section{Motivation or objective}

\begin{enumcirc}
	%
	\item
	%
	So far we defined the meanings of programs in imperative languages using the
	denotational semantics.
	%
	A good denotational semantics reveals an underlying mathematical structure of a
	programming language and hides the intermediate steps of computation as much as
	possible.
	%
	Also, it is compositional, and lets us reason about a piece of program code
	even when we do not know its surrounding program context.
	%
	\item
	%
	However, when a programming language has advanced or complex language
	constructs, defining a denotational semantics of the language may be difficult.
	%
	Also, sometimes we want to have a mathematical semantics of programs that tells
	us what happens in the middle of computation.
	%
	\item
	%
	The operational semantics is an alternative approach to give mathematical
	meanings to programs.
	%
	It is non-compositional, and does not hide the intermediate steps of
	computation.
	%
	But it is usually very simple and also rigorous or formal enough to enable a
	mathematical study of a programming language and language tools such as
	compiler and program verifier.
	%
	Also, an operational semantics of a programming language often serves as a
	blueprint of an interpreter or a compiler of the language.
	%
	\item
	%
	In this chapter, we will study the so-called \ul{small-step} operational
	semantics, which Reynolds calls transition semantics.
	%
\end{enumcirc}

\section{Main idea of the small-step operational semantics}

\begin{enumcirc}
	%
	\item
	%
	The key idea is to formalize one computation step of a program using a
	relation, called transition relation.
	%
	\item
	%
	Typically, a small-step operational semantics has two main parts.
	%
	\begin{enumrm}
		%
		\item
		%
		$\Gamma$ \dots a set of configurations.

		Usually, $\Gamma = \Gamma_N \cup \Gamma_T$ for some sets $\Gamma_N$, $\Gamma_T$
		with $\Gamma_N \cap \Gamma_T = \emptyset$.
		%
		Each element $\gamma \in \Gamma$ describes the status of a machine that runs a
		program.
		%
		If $\gamma \in \Gamma_N$, it is called \ul{nonterminal configuration} and the
		execution of its program is not finished yet.
		%
		If $\gamma \in \Gamma_T$, it is called \ul{terminal configuration} and the
		execution of its program is finished.
		%
		\item
		%
		$\rightarrow \;\subseteq \Gamma_N \times \Gamma$
		\dots transition relation.

		Intuitively, $\prths{\gamma, \gamma^\prime} \in\; \rightarrow$
		%
		\footnote{typically written as $\gamma \rightarrow \gamma^\prime$}
		%
		means that one computation step changes the status of a machine from $\gamma$
		to $\gamma^\prime$.
		%
		Note that $\gamma$ has to be a nonterminal configuration because of the domain
		of $\rightarrow$.
		%
		This condition is consistent with the intuition behind nonterminal and terminal
		configurations.
		%
	\end{enumrm}
	%
	Defining a small-step operational semantics amounts to defining $\Gamma$,
	$\Gamma_N$, $\Gamma_T$, and $\rightarrow$.
	%
	We will see a few examples of the operational semantics in this lecture.
	%
	Often if we define $\Gamma$, $\Gamma_N$, $\Gamma_T$, then the definition of
	$\rightarrow$ follows almost automatically.
	%
	This is a bit similar to the situation in the denotational semantics that if
	the form of the interpretation function for commands $\bbrackets{-}$ is
	determined, the actual definition of the function follows almost automatically.
	%
	\item
	%
	When defining the $\rightarrow$ relation, we usually use the inference rule
	notation $\inferrule{\varphi_1 \;\cdots \;\varphi_n}{\varphi}$ that you saw
	when we discussed Hoare logic.
	%
\end{enumcirc}

\section{Small-step operational semantics of the simple imperative language}

\begin{enumcirc}
	%
	\item
	%
	Let's try to give the operational semantics to the simple imperative language
	that we studied.
	%
	Here is a reminder of this abstract grammar:
	%
	\begin{grammar}
		<comm> ::=
		skip
		\alt <var> := <intexp>
		\alt <comm> ; <comm>
		\alt if <boolexp> then <comm> else <comm>
		\alt while <boolexp> do <comm>
		\alt while <boolexp> do <comm>
	\end{grammar}
	%
	\item
	%
	What should we do?
	%
	First, we have to define
	%
	\ul{the set of nonterminal configurations}
	%
	\footnote{$\Gamma_N$}
	%
	and
	%
	\ul{that of terminal configurations}.
	%
	\footnote{$\Gamma_T$}

	Here are our definitions:
	%
	\[
		\Gamma_N \defeq \chevrons{comm}\footnotemark \times \Sigma\footnotemark
		\qquad
		\Gamma_T \defeq \Sigma\footnotemark
	\]
	\footnoteeqn[-2]{command that records the remaining computation}
	\footnoteeqn{the current state of a machine}
	\footnoteeqn{the $\chevrons{comm}$ part is missing because there is no remaining computation}
	%
	The set of configurations is the union of the above two sets.
	%
	\item
	%
	Second, we should define a binary relation
	%
	\[
		\rightarrow \;\subseteq \Gamma_N \times \Gamma,
	\]
	%
	called transition relation, that describes single-step computation.
	%
	We write $\prths{\gamma, \gamma^\prime}$
	%
	to mean $\prths{\gamma, \gamma^\prime} \in\; \rightarrow$.
	%
	We define the transition relation $\rightarrow$ using the inference rule
	notation.
	%
	\[
		\begin{array}{c}
			\inferrule
			{\ }
			{
				\chevrons{\cskip, \sigma}
				\rightarrow
				\sigma
			}
			\qquad \qquad
			\inferrule
			{\ }
			{
				\chevrons{\cassign{v}{e}, \sigma}
				\rightarrow
				\sigma\prths{v \mapsto \bbrackets{e}\sigma}
			}
			\\[2em]
			\inferrule
			{
				\chevrons{c_1, \sigma}
				\rightarrow
				\sigma^\prime
			}
			{
				\chevrons{\cseq{c_1}{c_2}, \sigma}
				\rightarrow
				\chevrons{c_2, \sigma^\prime}
			}
			\qquad \qquad
			\inferrule
			{
				\chevrons{c_1, \sigma}
				\rightarrow
				\chevrons{c_1^\prime, \sigma^\prime}
			}
			{
				\chevrons{\cseq{c_1}{c_2}, \sigma}
				\rightarrow
				\chevrons{\cseq{c_1^\prime}{c_2}, \sigma^\prime}
			}
			\\[2em]
			\inferrule
			{
				\
			}
			{
				\chevrons{\cif{b}{c_1}{c_2}, \sigma}
				\rightarrow
				\chevrons{c_1, \sigma}
			}
			\prths{\bbrackets{b}\sigma = \ttt}
			\\[2em]
			\inferrule
			{
				\
			}
			{
				\chevrons{\cif{b}{c_1}{c_2}, \sigma}
				\rightarrow
				\chevrons{c_2, \sigma}
			}
			\prths{\bbrackets{b}\sigma = \fff}
			\\[2em]
			\inferrule
			{
				\
			}
			{
				\chevrons{\cwhile{b}{c}, \sigma}
				\rightarrow
				\sigma
			}
			\prths{\bbrackets{b}\sigma = \fff}
			\\[2em]
			\inferrule
			{
				\
			}
			{
				\chevrons{\cwhile{b}{c}, \sigma}
				\rightarrow
				\chevrons{\cseq{c}{\cwhile{b}{c}}, \sigma}
			}
			\prths{\bbrackets{b}\sigma = \ttt}
		\end{array}
	\]
	%
	Note that the right-hand side of $\rightarrow$ may include a command that is
	not a sub-command of the one on the left-hand side.
	%
	Look at
	%
	$ \chevrons{\cseq{c_1}{c_2}, \sigma}
		\rightarrow
		\chevrons{\cseq{c_1^\prime}{c_2}, \sigma^\prime} $
	%
	and
	%
	$ \chevrons{\cwhile{b}{c}, \sigma}
		\rightarrow
		\chevrons{\cseq{c}{\cwhile{b}{c}}, \sigma} $.
	%
	This indicates that the semantics is not compositional.
	%
	All these rules correspond to our intuitive understading of one computation
	step.
	%
	They can form the basis of the implementation of a simple interpreter, which
	just needs to run the $\rightarrow$ step repeatedly.
	%
	\item
	%
	Formal properties of the operational semantics:
	%
	\begin{enumrm}
		%
		\item
		%
		$\gamma \rightarrow \gamma_1
			\textrm{ and }
			\gamma \rightarrow \gamma_2
			\quad\implies\quad
			\gamma_1 = \gamma_2$.

		The semantics is deterministic.
		%
		\item
		%
		$\forall \gamma \in \Gamma_N \; \exists \gamma^\prime
			\textrm{ s.t. }
			\gamma \rightarrow \gamma^\prime$.

		In this semantics, executions never get stuck.
		%
		\item
		%
		From (i) to (ii), it follows that for every
		%
		$\gamma \in \Gamma$,
		%
		there exists a unique maximal sequence (may be infinite)
		%
		\[
			\gamma_0, \gamma_1, \gamma_2, \cdots, \gamma_n
		\]
		% \footnoteeqn[0]{may be infinite}
		%
		such that
		%
		\begin{multline*}
			\gamma = \gamma_0 \wedge
			\gamma_0 \rightarrow \gamma_1 \rightarrow \gamma_2 \rightarrow \cdots
			\rightarrow \gamma_n
			\\ \wedge \gamma_n \textrm{ is a terminal configuration or } n \textrm{ is infinite}.
		\end{multline*}
		%
		This maximal finite or infinite sequence represents the full computation
		starting from $\gamma$.
		%
		\item
		%
		We write $\gamma \uparrow$ if the maximal execution sequence from $\gamma$ is
		infinite.
		%
		Then, for all commands $c$ and states $\sigma$,
		%
		\begin{align*}
			\bbrackets{c}\sigma = \bot          &
			\quad \textrm{iff} \quad
			\chevrons{c, \sigma} \uparrow         \\
			\bbrackets{c}\sigma = \sigma^\prime &
			\quad \textrm{iff} \quad
			\chevrons{c, \sigma} \rightarrow^*\footnotemark \; \sigma^\prime
		\end{align*}
		\footnoteeqn[0]{
			reflexive and transitive closure of $\rightarrow$, i.e.
			$\rightarrow^* \defeq \bigcup_n^\infty \prths{\rightarrow}^n$
		}
	\end{enumrm}
	%
	\begin{exercise}
		%
		Prove (i), (ii), and (iv).
		%
	\end{exercise}
	%
	\begin{exercise}
		%
		Explain why the reasoning in (iii) is true.
		%
	\end{exercise}
	%
\end{enumcirc}

\section{Extension with newvar}

\begin{enumcirc}
	%
	\item
	%
	Extend the language with variable declaration:
	%
	\begin{center}
		\begin{minipage}[c]{0.6\textwidth}
			\begin{grammar}
				<comm> ::=
				\dots\;
				| newvar <var> := <intexp> in <comm>
			\end{grammar}
		\end{minipage}
	\end{center}
	%
	\item
	%
	How should we modify the $\rightarrow$ relation?
	%
	Add a rule for newvar:
	%
	\begin{enumrm}
		%
		\item
		%
		Option 1:
		%
		\[
			\inferrule
			{\ }
			{
				\chevrons{\cnew{v}{e}{c}, \sigma}
				\rightarrow
				\chevrons{\cseq{c}{v:=n}, \aug{\sigma}{v \mapsto \bbrackets{e}\sigma}}
			}
		\]
		%
		\item
		%
		Option 2
		%
		\[
			\inferrule
			{
				\chevrons{c, \aug{\sigma}{v : \bbrackets{e}\sigma}}
				\rightarrow
				\sigma^\prime
			}
			{
				\chevrons{\cnew{v}{e}{c}, \sigma}
				\rightarrow
				\aug{\sigma^\prime}{v : \bbrackets{e}\sigma}
			}
		\]
		%
		\item
		%
		Both options are acceptable.
		%
		But option 2 is better.
		%
		Only option 2 works when we extend the language with primitives for concurrent
		executions.
		%
	\end{enumrm}
	%
	\item
	%
	Note that we did not change $\Gamma_N$ and $\Gamma_T$.
	%
	Thus, adding newvar doesn't change the operational semantics much.
	%
	In a sense, this small change means that newvar doesn't change the language
	much, either.
	%
\end{enumcirc}

\section{Adding fail} \label{sec:5:fail}

\begin{center}
	\begin{minipage}[c]{0.26\textwidth}
		\begin{grammar}
			<comm> ::=
			\dots\;
			| fail
		\end{grammar}
	\end{minipage}
\end{center}

\begin{enumcirc}
	%
	\item
	%
	When we add fail, we have to change the set $\Gamma_T$ of terminal
	configuration, because we now have two types of terminations, normal one and
	abnormal one.
	%
	\[
		\Gamma_T \defeq \Sigma \cup \braces{\abort} \times \Sigma
		\quad
		(\textrm{or } = \Sigma + \Sigma)
	\]
	%
	$\Gamma_N$ remains unchanged.
	%
	\item
	%
	Since $\Gamma_T$ and so $\Gamma$ are changed, we should change the definition
	of $\rightarrow$.
	%
	We will also have to add a rule for fail.
	%
	Here is the new set of rules.
	%
	\[
		\begin{array}{c}
			\inferrule
			{\ }
			{\chevrons{\cfail, \sigma} \rightarrow \chevrons{\abort, \sigma}}
			\footnotemark
			\qquad \quad
			\inferrule
			{\ }
			{\chevrons{\cskip, \sigma} \rightarrow \sigma}
			\qquad \quad
			\inferrule
			{\ }
			{
				\chevrons{\cassign{v}{e}, \sigma}
				\rightarrow
				\sigma\prths{v \mapsto \bbrackets{e}\sigma}
			}
			\\[2em]
			\inferrule
			{
			}
			{
				\chevrons{\cif{b}{c_1}{c_2}, \sigma}
				\rightarrow
				\chevrons{c_1, \sigma}
			}
			\prths{\bbrackets{b}\sigma = \ttt}
			\\[2em]
			\inferrule
			{
			}
			{
				\chevrons{\cif{b}{c_1}{c_2}, \sigma}
				\rightarrow
				\chevrons{c_2, \sigma}
			}
			\prths{\bbrackets{b}\sigma = \fff}
			\\[2em]
			\inferrule
			{
				\chevrons{c_1, \sigma}
				\rightarrow
				\chevrons{c_1^\prime, \sigma^\prime}
			}
			{
				\chevrons{\cseq{c_1}{c_2}, \sigma}
				\rightarrow
				\chevrons{\cseq{c_1^\prime}{c_2}, \sigma^\prime}
			}
			\qquad \quad
			\inferrule
			{
				\chevrons{c_1, \sigma}
				\rightarrow
				\sigma^\prime
			}
			{
				\chevrons{\cseq{c_1}{c_2}, \sigma}
				\rightarrow
				\chevrons{c_2, \sigma^\prime}
			}
			\\[2em]
			\inferrule
			{
				\chevrons{c_1, \sigma}
				\rightarrow
				\chevrons{\abort, \sigma^\prime}
			}
			{
				\chevrons{\cseq{c_1}{c_2}, \sigma}
				\rightarrow
				\chevrons{\abort, \sigma^\prime}
			}
			\footnotemark
			\\[2em]
			\inferrule
			{
			}
			{
				\chevrons{\cwhile{b}{c}, \sigma}
				\rightarrow
				\sigma
			}
			\prths{\bbrackets{b}\sigma = \fff}
			\\[2em]
			\inferrule
			{
				\chevrons{c, \sigma}
				\rightarrow
				\sigma^\prime
			}
			{
				\chevrons{\cwhile{b}{c}, \sigma}
				\rightarrow
				\chevrons{\cseq{c}{\cwhile{b}{c}}, \sigma^\prime}
			}
			\prths{\bbrackets{b}\sigma = \ttt}
			\\[2em]
			\inferrule
			{
				\chevrons{c, \aug{\sigma}{v : \bbrackets{e}\sigma}}
				\rightarrow
				\chevrons{\abort, \sigma^\prime}
			}
			{
				\chevrons{\cnew{v}{e}{c}, \sigma}
				\rightarrow
				\chevrons{\abort, \aug{\sigma^\prime}{v : \sigma\prths{v}}}
			}
			\footnotemark
			\\[2em]
			\inferrule
			{
				\chevrons{c, \aug{\sigma}{v : \bbrackets{e}\sigma}}
				\rightarrow
				\sigma^\prime
			}
			{
				\chevrons{\cnew{v}{e}{c}, \sigma}
				\rightarrow
				\aug{\sigma^\prime}{v : \sigma\prths{v}}
			}
			\\[2em]
			\inferrule
			{
				\chevrons{c, \aug{\sigma}{v : \bbrackets{e}\sigma}}
				\rightarrow
				\chevrons{c^\prime, \sigma^\prime}
			}
			{
				\chevrons{\cnew{v}{e}{c}, \sigma}
				\rightarrow
				\chevrons{
					\cnew{v}{\sigma^\prime\prths{v}}{c^\prime},
					\aug{\sigma^\prime}{v : \sigma\prths{v}}
				}
			}
		\end{array}
	\]
	\footnoteeqn[-2]{new rule: due to fail}
	\footnoteeqn{new rule: due to the change of $\Gamma_T$}
	\footnoteeqn{new rule: due to the change of $\Gamma_T$}
	%
\end{enumcirc}

\newpage

\section{Handling input and output}

\begin{center}
	\begin{minipage}[c]{0.4\textwidth}
		\begin{grammar}
			<comm> ::=
			\dots\;
			| !<var>
			| ?<var>
		\end{grammar}
	\end{minipage}
\end{center}

\begin{enumcirc}
	%
	\item
	%
	This time we have to change the form or type of $\rightarrow$.
	%
	It is no longer a binary relation, but a ternary relation.
	%
	\[
		\rightarrow \;\subseteq \Gamma_N \times \Lambda \times \Gamma
	\]
	%
	\[
		\lambda \in \Lambda \defeq
		\braces{\varepsilon}\footnotemark \cup
		\set{\cin{n}}{n \in \Z}\footnotemark \cup
		\set{\cout{n}}{n \in \Z}\footnotemark
	\]
	\footnoteeqn[-2]{transition or execution without input or output}
	\footnoteeqn{transition with an input}
	\footnoteeqn{transition with an output}
	%
	We write
	%
	$\chevrons{c, \sigma} \xrightarrow{\lambda} \gamma$
	%
	to mean
	%
	$\chevrons{\chevrons{c, \sigma}, \lambda, \gamma} \in\; \rightarrow$.
	%
	We also often omit $\lambda$ if $\lambda = \varepsilon$.
	%
	\item
	%
	Why do we make this change?
	%
	It is because adding $\cin{n}$ and $\cout{n}$ to the language makes it
	necessary to describe some aspects of intermediate steps of computations
	explicitly.
	%
	\item
	%
	We include all the rules that (except the ones for $\cseq{c_1}{c_2}$ and
	newvar) that we defined in \cref{sec:5:fail}.
	%
	Of course, the occurrences of $\rightarrow$ in those old rules should be
	understood as $\xrightarrow{\varepsilon}$ with $\varepsilon$ omitted for
	simplicity.
	%
	In addition to these rules, we have the following rules:
	%
	\[
		\begin{array}{c}
			\inferrule
			{
				\
			}
			{
				\chevrons{\cin{v}, \sigma}
				\xrightarrow{\cin{n}}
				\aug{\sigma}{v : \sigma\prths{n}}
			}
			\qquad \quad
			\inferrule
			{
				\
			}
			{
				\chevrons{\cout{e}, \sigma}
				\xrightarrow{\cout{\bbrackets{e}\sigma}}
				\sigma
			}
			\\[2em]
			\inferrule
			{
				\chevrons{c_0, \sigma}
				\xrightarrow{\lambda}
				\sigma^\prime
			}
			{
				\chevrons{\cseq{c_0}{c_1}, \sigma}
				\xrightarrow{\lambda}
				\chevrons{c_1, \sigma^\prime}
			}
			\qquad \quad
			\inferrule
			{
				\chevrons{c_0, \sigma}
				\xrightarrow{\lambda}
				\chevrons{c_0^\prime, \sigma^\prime}
			}
			{
				\chevrons{\cseq{c_0}{c_1}, \sigma}
				\xrightarrow{\lambda}
				\chevrons{\cseq{c_0^\prime}{c_1}, \sigma^\prime}
			}
			\\[2em]
			\inferrule
			{
				\chevrons{c_0, \sigma}
				\xrightarrow{\lambda}
				\chevrons{\abort, \sigma^\prime}
			}
			{
				\chevrons{\cseq{c_0}{c_1}, \sigma}
				\xrightarrow{\lambda}
				\chevrons{\abort, \sigma^\prime}
			}
			\\[2em]
			\inferrule
			{
				\chevrons{c, \aug{\sigma}{v : \bbrackets{e}\sigma}}
				\xrightarrow{\lambda}
				\sigma^\prime
			}
			{
				\chevrons{\cnew{v}{e}{c}, \sigma}
				\xrightarrow{\lambda}
				\aug{\sigma^\prime}{v : \sigma\prths{v}}
			}
			\\[2em]
			\inferrule
			{
				\chevrons{c, \aug{\sigma}{v : \bbrackets{e}\sigma}}
				\xrightarrow{\lambda}
				\chevrons{\abort, \sigma^\prime}
			}
			{
				\chevrons{\cnew{v}{e}{c}, \sigma}
				\xrightarrow{\lambda}
				\chevrons{\abort, \aug{\sigma^\prime}{v : \sigma\prths{v}}}
			}
			\\[2em]
			\inferrule
			{
				\chevrons{c, \aug{\sigma}{v : \bbrackets{e}\sigma}}
				\xrightarrow{\lambda}
				\chevrons{c^\prime, \sigma^\prime}
			}
			{
				\chevrons{\cnew{v}{e}{c}, \sigma}
				\xrightarrow{\lambda}
				\chevrons{
					\cnew{v}{\sigma^\prime\prths{v}}{c^\prime},
					\aug{\sigma^\prime}{v : \sigma\prths{v}}
				}
			}
		\end{array}
	\]
	%
	Whenever an old rule contains a premise, we copy the rule and put $\lambda$
	above $\rightarrow$ in the premise and the conclusion, so that the label
	$\lambda$ gets propagated from the execution of a subcommand to that of the
	original command.
	%
	\item
	%
	This operational semantics corresponds to the denotational semantics that we
	studied.
	%
	The correspondence is formalized by the function $F$ in page 134 of the
	textbook:
	%
	\[
		F\prths{\gamma} =
		\left\{
		\begin{array}{ll}
			\bot                                                 &
			\text{when }
			\gamma \uparrow                                        \\
			\iterm\prths{\sigma'}                                &
			\text{when }
			\gamma \rightarrow^* \sigma'                           \\
			\iabort\prths{\sigma'}                               &
			\text{when }
			\gamma \rightarrow^* (\textit{abort}, \sigma')         \\
			\iout \prths{n, F_{\gamma''}}                        &
			\text{when }
			\exists \gamma' \in \Gamma.
			\; \gamma \rightarrow^* \gamma' \text{ and }
			\gamma' \! \xrightarrow{\cout{n}} \gamma''             \\
			\iin \prths {\lambda n \in \mathbb{Z}. F_{\gamma_n}} &
			\text{when }
			\exists \gamma' \in \Gamma.
			\forall n \in \mathbb{Z}.
			\gamma \rightarrow^* \gamma' \text{ and }
			\gamma' \xrightarrow{\cin{n}} \gamma_n.                \\
		\end{array}
		\right.
	\]
	%
	Intuitively, $F$ runs a configuration until it finishes, or outputs a number,
	or waits for an input.
	%
	$F$ then returns what it gets when it completes this execution.

	In a sense, the correspondence says that the denotational semantics comes from
	the operational semantics after unobservable intermediate states are abstracted
	away.
	%
	For detail, look at the textbook.
	%
\end{enumcirc}

