% !TEX root = main.tex

\chapter{Predicate Logic}

\section{Motivation or objective}

\begin{enumcirc}
	%
	\item
	Learn four key tools in PL that will be used throughout this course.
	%
	\begin{enumrm}
		%
		\item Abstract Syntax
		%
		\item Denotational Semantics
		%
		\item Inference Rule
		%
		\item Binding
		%
	\end{enumrm}
	%
	\item
	Learn the basics of predicate logic (or first-order logic)
	%
	\item
	We plan to go through some of (i) - (iv) twice, first using integer expressions
	and then using predicate logic.
	%
\end{enumcirc}

\section{Integer expressions}

\begin{enumcirc}
	%
	\item How to analyze integer expression found in logic and programming languages
	mathematically?
	%
	We will first have to define the syntax and the semantics for them.
	%
	\item Examples: $x + 3 \times y$, $x \div 2 + x \times x$ ...
	%
	\item We also want to develop mathematical tools to reason about or manipulate
	integer expressions.
	%
\end{enumcirc}

\section{Abstract syntax and initial algebra}

\begin{enumcirc}
	%
	\item
	Abstract Syntax: \\
	%
	Specification of \emph{abstract phrases}\footnote{ vague words, but will be
		made rigorous when we define initial algebra. }
	%
	in a formal language,
	%
	such as the language of integer expressions and predicate logic.
	%
	\item
	Typically, we use \emph{abstract grammar}\footnote{
		\begin{minipage}[t]{\textwidth}
			Here is the explanation of the word with an enumerated list:
			\begin{enumrm}
				\item
				grammar without any concern on parsing for surface syntax.
				%
				\item
				In this case, parse trees in the grammar are abstract phrases.
			\end{enumrm}
		\end{minipage}
	}
	%
	to describe abstract syntax.
	%
	\item Abstract grammar for integer expressions: \setlength{\grammarindent}{6em} % increase separation between LHS/RHS
	%
	\begin{center}
		\begin{minipage}{0.4\textwidth}
			\begin{grammar}
				<intexp> ::= 0 | 1 | 2 ...
				\alt <var>
				\alt - <intexp>
				\alt <intexp> $+$ <intexp>
				\alt <intexp> $-$ <intexp>
				\alt <intexp> $\times$ <intexp>
				\alt <intexp> $\div$ <intexp>
			\end{grammar}
		\end{minipage}
	\end{center}
	%
	(abstract) integer expressions are finite derivation trees in this grammar.
	%
	For instance,
	%
	\begin{center}
		\begin{tikzpicture}[
				level 1/.style={sibling distance=10mm, level distance=10mm},
				level 2/.style={sibling distance=10mm},
				level 3/.style={sibling distance=5mm}]
			\node {+}
			child {node {$\times$} }
			child {node {$\times$}
					child {node {3}}
					child {node {y}}
				};
		\end{tikzpicture}
		%
		$\qquad$
		%
		\begin{tikzpicture}[
				level 1/.style={sibling distance=20mm, level distance=10mm},
				level 2/.style={sibling distance=10mm},
				level 3/.style={sibling distance=5mm}]

			\node {+}
			child {node {$\div$}
					child {node {x}}
					child {node {2}}
				}
			child {node {$-$}
					child {node {x}}
					child {node {x}}
				};
		\end{tikzpicture}
	\end{center}
	%
	Note that infinite trees are not included.
	%
	\item
	A more accurate view is to view abstract syntax as an initial algebra.
	%
	This view will help us to see shy we can define various operations on abstract
	phrases or integer expressions using syntax-directed definition.

	\item
	%
	\emph{Algebra} $A$ $\cdots$ Set with operations and constraints. \\
	%
	\emph{Signature} $S$ $\cdots$ Type of an algebra.
	%
	\begin{exampletab}
		%
		$\subsctext{S}{group} = \prths{t,\; \textrm{id}: t ,\; \textrm{o} : t \times t \rightarrow t,\; \prths{-}^{-1} : t \rightarrow t}$
		%
		\begin{align*}
			A_0 & = \prths{\Z,\; 0,\; +,\; -}                         \\
			A_1 & = \prths{\R_{>0} ,\; 1,\; \times,\; \prths{-}^{-1}}
		\end{align*}
		%
		$A_0: \subsctext{S}{group}$ is a group with integers and addition.
		%
		$A_1: \subsctext{S}{group}$ is a group with positive reals and multiplication.
		%
	\end{exampletab}
	%
	\item
	\emph{Algebra homomorphism} $\cdots$ map between algebras that preserves
	constants and operations.
	%
	\begin{align*}
		S   & = \prths{
			t,\; c_1 : t,\; \cdots ,\; c_n : t ,\;
			\textrm{op}_1 : t \times \cdots \times t \rightarrow t ,\; \cdots
			\textrm{op}_m : t \times \cdots \times t \rightarrow t
		}
		\\
		A_0 & = \prths{
			\mathcal{U}_0\footnotemark,\;
			c_1^0 \in \mathcal{U} ,\;
			\cdots ,\;
			c_n^0 \in \mathcal{U} ,\;
			\textrm{op}_1^0 :
			\mathcal{U} \times \cdots \times \mathcal{U} \rightarrow \mathcal{U},\; \cdots
			\textrm{op}_m^0 :
			\mathcal{U} \times \cdots \times \mathcal{U} \rightarrow \mathcal{U}
		}
		\\
		A_1 & = \prths{
			\mathcal{U}_1,\;
			c_1^1 \in \mathcal{U} ,\;
			\cdots ,\;
			c_n^1 \in \mathcal{U} ,\;
			\textrm{op}_1^1 :
			\mathcal{U} \times \cdots \times \mathcal{U} \rightarrow \mathcal{U},\; \cdots
			\textrm{op}_m^1 :
			\mathcal{U} \times \cdots \times \mathcal{U} \rightarrow \mathcal{U}
		}
	\end{align*}
	\leavevmode
	\footnotetext{notation: $\abs{A_0}$}
	%
	$f \in \mathcal{U}_0 \rightarrow \mathcal{U}_1$ is a \emph{homomorphism} if
	%
	\begin{enumrm}
		\item $f \prths{c_i^0} = c_i^1$ for all $i$.
		\item $f \prths{\textrm{op}_i^0 \prths{x_1,\; \cdots ,\; x_k}} =
			\textrm{op}_i^1 \prths{x_1,\; \cdots ,\; x_k}$ for all $i$.
	\end{enumrm}

	\item
	\emph{Initial algebra of a signature} $S$
	%
	\begin{enumrm}
		%
		\item An algebra $A$ of the signature $S$ s.t. for all algebras $A^\prime$ of the
		same signature, there is a \emph{unique} homomorphism $f$ from $A$ to
		$A^\prime$.
		%
		\item
		$A_{\textrm{grammar}}$ is initial.
		%
		\item
		Formally, an abstract syntax fixes a signature and it denotes an initial
		algebra of the signature.
		%
		An abstract phrase is an element of that algebra.
		%
	\end{enumrm}

	\begin{exercise}
		Prove that $A_{\textrm{grammar}}$ is indeed an initial algebra.
	\end{exercise}

	\begin{exercisetab}
		Let $A_0$ and $A_0$ be initial algebras of the same signature $S$.
		%
		Then, there are homomorphisms
		%
		$f \in \abs{A_0} \rightarrow \abs{A_1}$ and
		%
		$g \in \abs{A_1} \rightarrow \abs{A_0}$ s.t. $f \circ g = \textrm{id}$ and
		%
		$g \circ f = \textrm{id}$.
		%
		This means that all initial algebras of $S$ are essentially the same, i.e.
		isomorphic. Prove this fact.
	\end{exercisetab}

\end{enumcirc}

\section{Syntax-directed definition and denotational semantics}

\begin{enumcirc}
	%
	\item
	Definition of a map on integer expressions using a form of induction and
	case analysis.
	%
	\item
	%
	$ \textrm{FV (Free Variables)}: \chevrons{intexp} \rightarrow 2^{\chevrons{Var}} $
	%
	\begin{align*}
		\fv{e\footnotemark} & = V \footnotemark \\
		\fv{c\footnotemark} & = \phi            \\
		\fv{x\footnotemark} & = \braces{x}      \\
		\fv{-e}             & = \fv{e}          \\
		\fv{e_1 \; \substack{+                  \\-\\\times\\\div} \; e_2} &= \fv{e_1} \cup \fv{e_2}
	\end{align*}
	\footnoteeqn[-3]{integer expression}
	\footnoteeqn{Set of free variables in $e$}
	\footnoteeqn{constant}
	\footnoteeqn{variable}
	%
	\item
	Two features: case analysis, recursive calls on subphrases.
	%
	\item
	%
	$ \bbrackets{-} \in \chevrons{intexp} \; \rightarrow \;
		\Sigma \rightarrow \Z
		\where \Sigma = \chevrons{var} \rightarrow \Z , \textrm{ a set of states }
		\sigma. $
	%
	\begin{align*}
		\bbrackets{c}\sigma  & = c                             \\
		\bbrackets{x}\sigma  & = \sigma\prths{x}               \\
		\bbrackets{-e}\sigma & = -\prths{\bbrackets{e}\sigma}  \\
		\bbrackets{
		e_1 \;\substack{+                                      \\-\\\times\\\div}\; e_2
		}\sigma              & = \prths{\bbrackets{e_1}\sigma}
		\;\substack{+                                          \\-\\\times\\\div}\;
		\prths{\bbrackets{e_1}\sigma} \footnotemark
	\end{align*}
	\footnotetext{some special treatment of the divide-by-zero case}
	%
	Intuitively, $\bbrackets{-}$ maps tress to mathematical functions in a
	syntax-directed (also called compositional) way.
	%
	Such a compositional mapping from syntactic entities to mathematical entities
	is called \emph{denotational semantics}.
	%
	\item
	In both cases, we are using the initiality of $\chevrons{intexp}$.
	%
	What are the target algebras in those cases?
	%
	\begin{enumrm}
		%
		\item
		%
		\begin{align*}
			\abs{A_1}                  & = 2^{\chevrons{var}} \\
			\underline{c^1}            & = \phi               \\
			\underline{x^1}            & = \braces{x}         \\
			\underline{-^1}\prths{X}   & = X                  \\
			\underline{+^1}\prths{X,Y} & = X \cup Y
		\end{align*}
		%
		$\underline{\times^1}$, $\underline{\div^1}$ and
		$\underline{\textrm{rem}^1}$ are defined similarly.
		%
		\item
		%
		\begin{align*}
			\abs{A_2}                                & = \Sigma \rightarrow \Z             \\
			\underline{c^2}\prths{\sigma}            & = c                                 \\
			\underline{x^2}\prths{\sigma}            & = \sigma\prths{x}                   \\
			\underline{-^2}\prths{f}\prths{\sigma}   & = -f\prths{\sigma}                  \\
			\underline{+^2}\prths{f,g}\prths{\sigma} & = f\prths{\sigma} + g\prths{\sigma}
		\end{align*}
		%
		$\underline{\times^2}$, $\underline{\div^2}$ and
		$\underline{\textrm{rem}^2}$ are defined similarly.

	\end{enumrm}
	%
	\item
	\begin{exercisetab}
		\[
			\delta \in \chevrons{var} \rightarrow \chevrons{intexp} \quad \textrm{(substitution)}
		\]
		Define $e/\delta$, the application of substitution $\delta$ to $e$.
	\end{exercisetab}
\end{enumcirc}

\section{Structural induction}
\begin{enumcirc}
	%
	\item
	We want to show that some property $\Psi$ holds for all integer expressions.
	%
	What should we do?
	%
	\item
	Use induction on the structure of expressions.
	%
	That is, for each expression $e$, prove the property assuming that the property
	holds for the subexpressions of $e$.
	%
	\begin{lemma}[Coincidence]
		For every expression $e$ and states $\sigma$ and $\sigma^\prime$,
		%
		if $\sigma\prths{x} = \sigma^\prime\prths{x}$ for all $x \in \fv{e}$,
		%
		then $\bbrackets{e}\sigma = \bbrackets{e}\sigma^\prime$.
	\end{lemma}
	%
	\begin{proof}
		%
		By structural induction.
		%
		\begin{itemize}
			%
			\item $e \equiv c$\;:\quad $\bbrackets{c}\sigma = c = \bbrackets{c}\sigma^\prime$.
			      %
			\item $e \equiv x$\;:\quad $\bbrackets{x}\sigma = \sigma\prths{x} =
				      \sigma^\prime\prths{x} = \bbrackets{x}\sigma^\prime$ $\because x \in \fv{x}$.
			      %
			\item $e \equiv -e^\prime$\;:\quad $\bbrackets{-e^\prime}\sigma =
				      -\bbrackets{e^\prime}\sigma = -\bbrackets{e^\prime}\sigma^\prime =
				      \bbrackets{-e^\prime}\sigma^\prime$ $\because$ induction hypothesis.
			      %
			\item $e \equiv e_1 + e_2$\;:\quad $\bbrackets{e_1 + e_2}\sigma =
				      \bbrackets{e_1}\sigma + \bbrackets{e_2}\sigma =
				      \bbrackets{e_1}\sigma^\prime + \bbrackets{e_2}\sigma^\prime =
				      \bbrackets{e_1 + e_2}\sigma^\prime$.
			      %
			\item $e_1 \substack{-\\\times\\\div} e_2$\;:\quad similar.
			      %
		\end{itemize}
	\end{proof}

	\begin{lemma}[Substitution]
		\[
			\sigma\prths{x} = \bbrackets{\delta\prths{x}}\sigma^\prime \quad \textrm{for all } x
			\quad \implies \quad \bbrackets{e/\delta}\sigma^\prime = \bbrackets{e}\sigma
		\]
	\end{lemma}
	\begin{proof}
		By structural induction.
	\end{proof}

	Notation:
	%
	\begin{enumrm}
		%
		\item
		$x_1 \rightarrow e_1, x_2 \rightarrow e_2, \cdots , x_n \rightarrow e_n$
		means the substitution that maps $x_i$ to $e_i$ and $y \ne x_i$ to $y$.
		%
		\item
		$
			\aug{\sigma}{x:v}\prths{y} =
			\begin{cases}
				\sigma\prths{y} & \textrm{if } y \ne x \\
				v               & \textrm{if } y = x
			\end{cases}
		$
	\end{enumrm}

	\begin{corollary}
		%
		\[
			\bbrackets{e / x_1 \rightarrow e_1, \cdots , x_n \rightarrow e_n}\sigma =
			\bbrackets{e} \aug{\sigma}{x_1 : \bbrackets{e_1} \sigma \middle\vert \cdots \middle\vert x_n : \bbrackets{e_n}\sigma}
		\]
		%
	\end{corollary}
	%
	This intuitively says the correspondence between syntactic and semantic
	substitutions.
	%
	\item
	Structural induction holds because of the initiality of $\chevrons{intexp}$.
	%
	Can you explain why it is the case?
	%
\end{enumcirc}

\section{Predicate logic (or first-order logic) informally}

\begin{enumcirc}
	\item
	%
	Language for expressing (boolean) properties (also called assertions).
	%
	\item
	%
	Extension of boolean expressions in programming languages with universal and
	existential quantification.
	%
	\item
	%
	Examples:\\
	%
	$ \forall x .\; \forall y .\; \exists m .\; \exists n .\; x \times m + y \times n = 1 $ \\
	%
	$ \forall x .\; \exists y .\; y > x$
	%
	\item
	%
	Quantifiers are over integers, reals, and other first-order entities (i.e.
	\underline{not} over sets of integers, and functions etc.).
	%
	The ``first-order'' in first-order logic refers to this restriction.
	%
	We will consider a version of predicate logic or first-order logic where
	quantifiers range over integers and all variables are integer variables.

\end{enumcirc}

\section{Abstract syntax of predicate logic}

\begin{enumcirc}
	%
	\item
	%
	Described in terms of the following abstract grammar:
	%
	\begin{center}
		\begin{minipage}{0.8\textwidth}
			\begin{grammar}
				<intexp> ::= 0 | 1 | 2 | ...
				| <var>\footnotemark | - <intexp> | <intexp> $\substack{+ \\ \times }$ <intexp>

				<assert> ::= true | false | <intexp> $\substack{= \\ > }$ <intexp>
				\alt $\neg$ <assert> | <assert> $\wedge$ <assert> | $\forall$ <var> . <assert>

			\end{grammar}
		\end{minipage}
	\end{center}
	\footnotetext{some countably-infinite set disjoint from 0, 1, 2, ...}
	%
	To simplify presentations, we will consider only $+$, $-$, $\times$, $=$, $>$,
	$\neg$, $\wedge$, and $\forall$.
	%
	\item
	%
	What do we mean by abstract grammar and abstract syntax here?
	%
	If you got confused about initial-algebra stuff, just think that our abstract
	syntax is the set of all finite derivation or parse trees.
	%
	If not you can view the abstract syntax as the initial algebra of the signature
	induced by the grammar signature $S_{\textrm{PL}}$:
	%
	\[
		\textrm{Signature} \quad S_{\textrm{PL}} =
		\begin{pmatrix}
			t\footnotemark,\; u\footnotemark,\; 0:t,\; 1:t\;, 2:t,\; \cdots,                                         \\
			x: t,\; y: t,\; \cdots ,                                                                                 \\
			- : t \rightarrow t,\; + : t \times t \rightarrow t,\; \times : t \times t \rightarrow t,\;              \\
			\textrm{true}: u,\; \textrm{false}: u,\; = : t \times t \rightarrow u,\; > : t \times t \rightarrow u,\; \\
			\neg: u \rightarrow u,\; \wedge: u \times u \rightarrow u,\; \forall: \chevrons{var} \times u \rightarrow u
		\end{pmatrix}
	\]
	\footnoteeqn[-1]{integer expression}
	\footnoteeqn{assertions}
	%
	\[
		\textrm{Algebra} \quad A_0 =
		\begin{pmatrix}
			\mathcal{U}^0,\; \mathcal{V}^0,\; 0^0 \in \mathcal{U}^0,\; 1^0 \in \mathcal{U}^0,\; 2^0 \in \mathcal{U}^0,\; \cdots, \\
			x^0 \in \mathcal{U}^0,\; y^0 \in \mathcal{U}^0,\; \cdots,                                                            \\
			- \in \brackets{\mathcal{U}^0 \rightarrow \mathcal{U}^0},\;
			\plustimes \in \brackets{\mathcal{U}^0 \times \mathcal{U}^0 \rightarrow \mathcal{U}^0},\;                            \\
			\textrm{true}^0 \in \mathcal{V}^0,\;
			\textrm{false}^0 \in \mathcal{V}^0,\;
			\twostack{=^0}{<^0} \in \brackets{\mathcal{U}^0 \times \mathcal{U}^0 \rightarrow \mathcal{V}^0},\;                   \\
			\neg^0 \in \brackets{\mathcal{V}^0 \rightarrow \mathcal{V}^0},\;
			\wedge^0 \in \brackets{\mathcal{V}^0 \times \mathcal{V}^0 \rightarrow \mathcal{V}^0},\;                              \\
			\forall^0 \in \brackets{\chevrons{var} \times \mathcal{V}^0 \rightarrow \mathcal{V}^0}
		\end{pmatrix}
	\]
	%
	\[
		\textrm{Algebra} \quad A_1 =
		\begin{pmatrix}
			\mathcal{U}^1,\; \mathcal{V}^1,\; 0^1 \in \mathcal{U}^1,\; 1^1 \in \mathcal{U}^1,\; 2^1 \in \mathcal{U}^1,\; \\
			\cdots                                                                                                       \\
			\forall^1 \in \brackets{\chevrons{var} \times \mathcal{V}^1 \rightarrow \mathcal{V}^1}
		\end{pmatrix}
	\]
	%
	An algebra homomorphism from $A_0$ to $A_1$ is a pair
	%
	$h: \mathcal{U}^0 \rightarrow \mathcal{U}^1$ and
	%
	$k: \mathcal{V}^0 \rightarrow \mathcal{V}^1$
	%
	that preserves all constraints and operations.
	%
	For instance,
	%
	\begin{enumrm}
		%
		\item
		%
		$k \prths{=^0 \prths{a, b}} \quad = \quad =^1 \prths{h\prths{a}, h\prths{b}}$
		%
		\item
		%
		for any
		%
		$\substack{
				x \in \chevrons{var}\\
				a \in \mathcal{U}^0
			}$,\ \
		%
		$k \prths{ \forall^0 \prths{x, =^0 \prths{a, x}} } \quad = \quad
			\forall^1 \prths{x, =^1 \prths{h\prths{a}, h\prths{x}} }$
		\leavevmode
		\footnote{same $x$}
		%
	\end{enumrm}
	%
	As before, the abstract syntax is
	%
	\underline{ the initial algebra of this signature $S_{PL}$}.\footnote{
		isomorphic to the algebra built with derivation trees
	}
	%
	Because of initiality, we can define a function in a syntax-directed way.
	%
	Also, we can use structural induction to prove properties of assertions.
	%
\end{enumcirc}

\section{Denotational semantics}

\begin{enumcirc}
	%
	\item
	%
	We define two functions:
	%
	\begin{align*}
		\intexpsem{-} & \in \brackets{\chevrons{intexp} \rightarrow \Sigma\footnotemark \rightarrow \Z} \\
		\assertsem{-} & \in \brackets{\chevrons{assert} \rightarrow \Sigma \rightarrow \B\footnotemark}
	\end{align*}
	%
	\footnoteeqn[-1]{$\chevrons{var} \rightarrow \Z$}
	\footnoteeqn{$\braces{\ttt, \fff}$}
	%
	\item
	%
	the definition of $\intexpsem{-}$ is the same as before.
	%
	Recall that it is syntax-directed.
	%
	\item
	%
	The definition of $\assertsem{-}$ is as follows:
	%
	\begin{align*}
		\assertsem{\true} \sigma                       & = \ttt                                                                    \\
		\assertsem{\false} \sigma                      & = \fff                                                                    \\
		\assertsem{e_1 \;\twostack{=}{>}\; e_2} \sigma & = \prths{\bbrackets{e_1}\sigma\; \twostack{=}{>}\; \bbrackets{e_2}\sigma} \\
		\assertsem{\neg p} \sigma                      & = \prths{\neg \assertsem{p}\sigma}                                        \\
		\assertsem{p_1 \wedge p_2} \sigma              & = \prths{\assertsem{p_1}\sigma \wedge \assertsem{p_2}\sigma}              \\
		\assertsem{\forall x .\; p} \sigma             & = \prths{\forall n \in \Z .\; \assertsem{p}\prths{\aug{\sigma}{x:n}}}
	\end{align*}
	%
	\item
	%
	Don't forget that what appears inside $\bbrackets{}$ is a tree, while $\forall$
	and $\wedge$ on the RHS of $=$ are the usual mathematical notations.
	%
	As we discussed already, here we are really defining an algebra $A$ of $S_{PL}$
	s.t.
	%
	\[
		A = \prths{\Sigma \rightarrow \Z, \Sigma \rightarrow \B, \cdots}
	\]
	%
	Then, we are using the initiality of the abstract syntax to get a map from it
	to $A$.

	\section{Inference rules}

	\begin{enumcirc}
		%
		\item
		%
		Rules for deriving always-true (in other words, valid\footnote{ $p$ is valid if
			$\assertsem{p}=\ttt$ for all $\sigma \in \Sigma$ }) assertions.
		%
		\item
		%
		Expressed using the inference-rule notation.
		%
		\[
			\begin{array}{c}
				\inferrule{ p_0 \\ p_0 \rightarrow p_1 }{p_1}
				\qquad \qquad
				\inferrule{ p }{\forall x .\; p}
				\qquad \qquad
				\inferrule{\ }{e_1 = e_2 \rightarrow e_1 = e_2}
				\\
			\end{array}
		\]
		%
		general form: $ \inferrule{p_0 \\ \cdots \\ p_n}{p}$
		%
		\begin{enumrm}
			%
			\item
			%
			expresses if all of $p_0, \cdots , p_n$ are valid, then $p$ is valid.
			%
			\item
			%
			\underline{doesn't} say that for all $\sigma \in \Sigma$, if all of
			$\assertsem{p_0}\sigma, \cdots , \assertsem{p_n}\sigma$ are $\ttt$, then
			$\assertsem{p}\sigma$ is $\ttt$\footnote{
				$\sigma$ satisfies $p$ or $p$ holds for $\sigma$.
			}.
			%
		\end{enumrm}
		%
		\begin{exercise}
			Prove why the above three rules are \underline{correct}\footnote{also called sound.}.
		\end{exercise}
		%
		\item
		%
		A big part of research on or study about predicate logic is to study these
		rules.
		%
		In this course, however, we will not do this.
		%
	\end{enumcirc}
	%
\end{enumcirc}

\section{Binding and substitution}

\begin{enumcirc}
	%
	\item
	%
	$\forall v$, $\exists v$:
	%
	\begin{enumrm}
		%
		\item
		%
		example of binders
		%
		\item
		%
		They have the scope of binding
		%
		\item
		%
		Informally, they
		%
		\ul{introduce a new variable},
		%
		\ul{give a nave $v$ to it}, and
		%
		\ul{make it available within its scope.}
		%
		Morally, renaming $v$ to some other variable $w$ should not change the meaning
		of the assertion.
		%
	\end{enumrm}
	%
	\item
	%
	An occurrence of a variable $x$ is \ul{bound} in $p$ if $x$ is within the scope
	of a binder for $x$ in $p$.
	%
	\item
	%
	An \underline{occurrence} of $x$ is \underline{free} in $p$ if it is not bound.
	%
	\item
	%
	A \underline{variable} is \underline{free} in $p$ if it has a free occurrence
	in $p$.
	%
	\item
	%
	We can define functions $\textrm{FV}_{\textrm{\tiny assert}}$ and
	$\textrm{FV}_{\textrm{\tiny intexp}}$ \footnote{ subscripts to be omitted }
	that compute the sets of free variables of assertions and integer expressions,
	in a syntax-directed way:
	%
	\begin{align*}
		\fvintexp{e}                           & \cdots                     \textrm{defined as above} \\
		\fvassert{\true}                       & = \phi                                               \\
		\fvassert{\false}                      & = \phi                                               \\
		\fvassert{e_1 \;\twostack{=}{>}\; e_2} & = \textrm{FV}\prths{e_1} \cup \textrm{FV}\prths{e_2} \\
		\fvassert{\neg p}                      & = \textrm{FV}\prths{p}                               \\
		\fvassert{p_1 \wedge p_2}              & = \textrm{FV}\prths{p_1} \cup \textrm{FV}\prths{p_2} \\
		\fvassert{\forall x .\; p}             & = \textrm{FV}\prths{p} \setminus \braces{x}
	\end{align*}
	%
	\begin{exercisetab}
		Define an algebra for $S_{\textrm{PL}}$ such that the algebra homomorphism from the abstract syntax to
		this algebra is $\prths{\textrm{FV}_{\textrm{\tiny intexp}}, \textrm{FV}_{\textrm{\tiny assert}}}$.
	\end{exercisetab}
	%
	\item
	%
	Now, how should we deal with binders during substitution?
	%
	It is not entirely obvious.
	%
	Several textbooks had wrong definitions in old days.\footnote{ mistake:
		$\forall y .\; y > x / x \rightarrow y = \forall y .\; y > y + 1$ }\\
	%
	Correct definition:\\
	%
	\[
		\begin{array}{c}
			\true / \delta = \true \qquad \qquad \false / \delta = \false                                                    \\[1em]
			\prths{e_1 \;\twostack{=}{>}\; e_2 \; / \delta} = \prths{e_1 / \delta \;\twostack{=}{>}\; e_2 / \delta} \qquad \qquad
			\neg p / \delta = \neg \prths{p / \delta}                                                                        \\[1em]
			\prths{p_1 \wedge p_2} / \delta = \prths{p_1 / \delta \wedge p_2 / \delta} \qquad \qquad
			\prths{\forall x .\; p} / \delta = \forall v_{\textrm{\tiny new}} .\; p / \aug{\delta}{v:v_{\textrm{\tiny new}}} \\[1em]
			\displaystyle
			\textrm{where} \quad v_{\textrm{\tiny new}} \notin
			\bigcup_{w \in \textrm{FV}\prths{p} \backslash \braces{v}}
			\textrm{FV}\prths{\delta\prths{w}}\footnotemark
		\end{array}
	\]
	\footnoteeqn{
		\begin{minipage}[t]{0.8\textwidth}
			device that lets us avoid variable capturing; if $v$ is not in the set then $v_{\textrm{\tiny new}} = v$.
			Otherwise, $v_{\textrm{\tiny new}}$ is the first variable not in the set.
		\end{minipage}
	}
	%
	\begin{property}[Coincidence]
		%
		\[
			\bbrackets{p}\sigma = \bbrackets{p}\sigma^\prime
		\]
		%
		$p \cdots$ assertion or integer expression \\
		%
		$\sigma, \sigma^\prime \cdots$ states s.t.
		%
		$\sigma\prths{w} = \sigma^\prime\prths{w}$ for all $w \in \textrm{FV}\prths{p}$.
		%
	\end{property}
	%
	\begin{proof}
		By structural induction.\footnote{
			we can use it because the abstract syntax for predicate logic is an initial algebra.
		}
		%
		\begin{itemize}
			%
			\item
			      %
			      $p \equiv \true$ or $\false$:\quad trivial.
			      %
			\item
			      %
			      $p \equiv e_1 \;\twostack{=}{>}\; e_2$:
			      %
			      \[
				      \fv{e_i} \subseteq \fv{p} \quad \textrm{for } i = 1, 2 \quad
				      \therefore \forall w \in \fv{e_i}, \; \sigma\prths{w} = \sigma^\prime\prths{w}
			      \]
			      %
			      We can use induction and get
			      %
			      $ \bbrackets{e_i}\sigma = \bbrackets{e_i}\sigma^\prime $.
			      %
			      \begin{align*}
				      \bbrackets{p}\sigma & = \bbrackets{e_1}\sigma \;\twostack{=}{>}\; \bbrackets{e_2}\sigma               \\
				                          & = \bbrackets{e_1}\sigma^\prime \;\twostack{=}{>}\; \bbrackets{e_2}\sigma^\prime \\
				                          & = \bbrackets{p}\sigma^\prime
			      \end{align*}
			      %
			\item
			      %
			      $p \equiv \neg p^\prime$:\quad similar.
			      %
			\item
			      %
			      $p \equiv p_1 \wedge p_2$:\quad similar.
			      %
			\item
			      %
			      $p \equiv \forall x .\; p^\prime$:\quad
			      %
			      For all $n \in \Z$,
			      %
			      \[
				      \sigma_1 := \aug{\sigma}{x:n}\prths{w} \qquad \sigma_1^\prime := \aug{\sigma^\prime}{x:n}\prths{w}
			      \]
			      %
			      Then, $\forall w \in \fv{p^\prime}$,
			      %
			      $\sigma_1\prths{w} = \sigma_1^\prime\prths{w}$.
			      %
			      Therefore, by induction hypothesis,
			      %
			      \[
				      \bbrackets{p^\prime}\sigma_1 = \bbrackets{p^\prime}\sigma_1^\prime
			      \]
			      %
			      \begin{align*}
				      \therefore \bbrackets{p}\sigma & = \forall n \in \Z .\; \bbrackets{p^\prime}\sigma_1        \\
				                                     & = \forall n \in \Z .\; \bbrackets{p^\prime}\sigma_1^\prime \\
				                                     & = \bbrackets{p}\sigma^\prime
			      \end{align*}
		\end{itemize}
	\end{proof}

	\begin{property}[Substitution]
		%
		\[
			\sigma\prths{w} = \bbrackets{\delta\prths{w}}\sigma^\prime
			\quad \implies \quad \bbrackets{p/\delta}\sigma^\prime = \bbrackets{p}\sigma
		\]
		%
	\end{property}
	%
	\begin{proof}
		By structural induction.
	\end{proof}

	\begin{property}[Finite substitution theorem\footnote{correspondence between syntactic and semantic substitutions.}]
		%
		\[
			\bbrackets{p/v_0 \rightarrow e_0, \cdots , v_n \rightarrow e_n}\sigma =
			\bbrackets{p}\aug{\sigma}{v_0 : \bbrackets{e_0}\sigma \middle\vert \cdots \middle\vert v_n : \bbrackets{e_n}\sigma}
		\]
		%
	\end{property}
	%
	\begin{proof}
		By structural induction.
	\end{proof}

	\begin{property}[Renaming\footnote{renaming doesn't change the meaning of an assertion}]
		%
		\[
			w \notin \fvassert{q} \backslash \braces{v} \quad \implies \quad
			\bbrackets{\forall w .\; q / v \rightarrow w}\sigma = \bbrackets{\forall v .\; q}\sigma
		\]
	\end{property}

\end{enumcirc}