\chapter{The Simple Imperative Language}

\section{Motivation or goal}

\begin{enumcirc}
	%
	\item
	%
	Most real-worlds PLs support computation by state update and that by function
	application.
	%
	The former is often referred to as imperative computation, and the latter as
	functional or applicative computation
	%
	Our goal is to study core PL concepts and ideas for imperative computation.
	%
	\item
	%
	Actually, it is more appropriate to say that our aim is a formal and
	mathematical analysis of the core PL concepts for imperative computation.
	%
	We will study or learn mathematical tools for this.
	%
	Also, we will show how to express and analyze big design decisions of such an
	imperative PLs.
	%
	\item
	%
	We will look at (i) some basic concepts and results of domain theory, (ii)
	variable declaration and binding, (iii) syntactic sugar error handling and (iv)
	the notions of soundness and full abstraction.%
	%
	(Well, I just listed all the key items in the chapter 2 of the book.)
	%
	\item
	%
	A good way to learn the material of this chapter is to ask yourself:
	%
	What should you do in order to design an imperative programming language and
	build a foundation of the designed language?
	%
	Think about this a little, and compare your answer with what I'll explain.
	%
\end{enumcirc}

\section{Syntax}

\begin{enumcirc}
	%
	\item
	%
	Variables, and read and update of them \dots key concepts or operations for
	imperative computation.
	%
	\item
	%
	Syntax that supports these concepts and operations:
	%
	\begin{center}
		\begin{minipage}{0.9\textwidth}
			\begin{grammar}
				<intexp> ::= 0 | 1 | 2 | \dots | <var> | ... \footnotemark

				<boolexp> ::= true | false | ... \footnotemark

				<comm> ::= <var> := <intexp> \footnotemark | skip | <comm> ; <comm> \footnotemark
				\alt if <boolexp> then <comm> else <comm>
				\alt while <boolexp> do <comm>
			\end{grammar}
		\end{minipage}
	\end{center}
	\footnoteeqn[-3]{same as before}
	\footnoteeqn{
		\begin{minipage}{0.9\textwidth}
			%
			almost the same as that of $\chevrons{\textit{assert}}$.
			%
			The exception is that $\chevrons{\textit{boolexp}}$ doesn't include
			quantifiers. (Why?)
			%
		\end{minipage}
	}
	\footnoteeqn{update of a variable}
	\footnoteeqn{order matters}

	As in the case of predicate logic, you can understand $\gram{comm}$ as the set
	of all finite derivation trees, or as a multi-sorted initial algebra for the
	signature determined by the grammar.
	%
	\item
	%
	It is a language for expressing a sequence of variable reads and variable
	updates.
	%
\end{enumcirc}

\section{Baby domain theory}

\begin{enumcirc}
	%
	\item
	%
	Giving a denotational semantics to our simple imperative language is not as
	straightforward as doing so with predicate logic, because of loop.
	%
	\item
	%
	$\bbrackets{-}$ : $\gram{comm} \to \dots$\;\footnote{don't worry about this target set now.}

	We want the following equation for loop unrolling to hold:
	%
	\begin{align*}
		\bbrackets{\cwhile{b}{c}} & = \bbrackets{\cif{b}{c ; \cwhile{b}{c}}{\cskip}}             \\
		                          & = \dots \bbrackets{\cwhile{b}{c}} \dots                      \\
		                          & = F \prths{\bbrackets{\cwhile{b}{c}}} \textrm{ for some } F.
	\end{align*}
	%
	But a function $F$ on a set may or may not have such a fixed point.
	%
	\item
	%
	Then, why should $F$ in the above have a fixed point?
	%
	Because it is something that can be implemented by a program.
	%
	\[
		F \; ``=" \footnotemark \bbrackets{\cif{b}{c ; \Box}{\cskip}}.
	\]
	\footnoteeqn[0]{informally}
	%
	One objective of domain theory is to formalize good properties enjoyed by such
	program implementable functions without going into the low-level details of
	computability theory.
	%
	\item
	%
	High-level meta heuristic behind domain theory:
	%
	\begin{enumrm}
		%
		\item
		%
		Consider a set together with some structure.
		%
		\item
		%
		Use functions between such sets that respect the structures.
		%
		\item
		%
		Why on (i) and (ii)?
		%
		Because if done well, functions in (ii) will always have fixed points.
		%
	\end{enumrm}
	%
	\item
	%
	Key definitions:
	%
	\begin{definition}[partial order]
		%
		A binary relation $\sqsubseteq$ on a set $S$ is a \ul{partial order} if
		%
		\begin{enumrm}
			%
			\item
			%
			$x \sqsubseteq x$ for all $x \in S$ (reflexivity);
			%
			\item
			%
			for all $x, y, z \in S$, if $x \sqsubseteq y$ and $y \sqsubseteq z$,
			%
			then $x \sqsubseteq z$ (transitivity); and
			%
			\item
			%
			for all $x, y \in S$, if $x \sqsubseteq y$ and $y \sqsubseteq x$, then $x = y$
			(anti-symmetry).
			%
		\end{enumrm}
		%
		A set $S$ with a partial order $\sqsubseteq$ is called a
		%
		\ul{partially ordered set} or \ul{poset}.
		%
	\end{definition}
	%
	\begin{definition}
		%
		A \ul{chain} in a poset $\prths{S, \sqsubseteq}$ is a (countably) infinite
		sequence
		%
		\[
			x_0, x_1, x_2, \dots , x_n, \dots
		\]
		%
		of elements in $S$ s.t. $x_n \sqsubseteq x_{n+1}$ for all $n \geq 0$.
	\end{definition}
	%
	\begin{definition}
		%
		A \ul{domain} is a pre-domain $\prths{S, \sqsubseteq}$ that has
		%
		\ul{the least element}, often denoted $\bot$.
		%
		(meaning: for all $x \in S$, $\bot \sqsubseteq x$)
		%
	\end{definition}
	%
	\begin{definition}
		%
		Let $\prths{S_1, \sqsubseteq_1}$ and $\prths{S_2, \sqsubseteq_2}$ be
		pre-domains.
		%
		A function $f : S_1 \to S_2$ is \ul{continuous} if for every chain
		%
		$\braces{x_n}_{n \geq 0}$ in $S_1$,
		%
		$f \prths{\bigsqcup_{n \geq 0} x_n}$ is the least upper bound of
		%
		$\braces{f \prths{x_n}}_{n \geq 0}$ in $S_2$.
		%
	\end{definition}
	%
	\begin{definition}
		%
		A function $f : S_1 \to S_2$ is \ul{monotone} if for all $x, y \in S_1$,
		%
		\[
			x \sqsubseteq_1 y \implies f \prths{x} \sqsubseteq_2 f \prths{y}.
		\]
		%
		When $S_1$ and $S_2$ are domains with least elements $\bot_1$ and $\bot_2$, we
		say that a function
		%
		$f : S_1 \to S_2$ is \ul{strict} if
		%
		$f \prths{\bot_1} = \bot_2$.
		%
	\end{definition}
	%
	\begin{exercise} \label{ex:cont-mon}
		%
		Show that if $f$ is continuous, it is monotone.
		%
	\end{exercise}
	\item
	%
	What's going on here?
	%
	What are the intuitions behind these definitions?
	%
	\begin{enumrm}
		%
		\item
		%
		$x \subseteq y$ \dots intuitively means that $y$ has more information than $x$ or
		$x$ and $y$ have the same amount of information.
		%
		\begin{exampletab}
			%
			\begin{enumalpha}
				%
				\item
				%
				\todo
				%
				\item
				%
				\todo
				%
				\item
				%
				\todo
				%
				\item
				%
				\todo
				%
			\end{enumalpha}
		\end{exampletab}
		%
		\item
		%
		The monotonicity means the preservation of the approximation
		relation.\footnote{ editor: if $a$ is an approximation of $b$, then $f(a)$ is
			an approximation of $f(b)$ }
		%
		One intuition behind continuity of $f$ is that in order to produce finite
		amount of information in its output, $f$ consumes only finite amount
		information in its input.
		%
		\begin{exampletab}
			%
			\begin{enumalpha}
				%
				\item
				%
				\todo
				%
				\begin{exercisetab}
					%
					\todo
					%
				\end{exercisetab}
				%
				\item
				%
				\todo
				%
				\begin{exercisetab}
					%
					\todo
					%
				\end{exercisetab}
				%
			\end{enumalpha}
			%
		\end{exampletab}
		%
		\item
		%
		John Reynolds' phrase in page 108: \\
		%
		\emph{
			``\dots Instead, it is based on the
			physical \footnote{ Domain theory attempts to capture this aspect of computation
			} limitations of communication: one cannot predict the future of input, nor
			receive an infinite amount of information in a finite amount of time, nor
			produce output except at finite times \dots''
		}
		%
	\end{enumrm}
	%
	\item
	%
	One important reason of doing domain theory is to have the following ``least
	fixed-point theorem'':
	%
	\begin{property}[Least Fixed-Point Theorem]
		%
		If $D$ is a domain and $f$ is a continuous function from $D$ to $D$,
		%
		\[
			x = \bigsqcup_{n = 0}^{\infty} f^n \prths{\bot \footnotemark}
		\]
		\footnoteeqn[0]{least element of $D$}
		%
		is the least fixed-point of $f$.
		%
		(That is, $f \prths{x} = x$ and for all $y \in D$ s.t. $f \prths{y} = y$, $x \sqsubseteq y$.)
		%
	\end{property}
	%
	\begin{proof}
		%
		By \cref{ex:cont-mon}, $f$ is monotone.
		%
		Using induction and $\bot$'s being the least element, we can show that
		%
		$\braces{f^n \prths{\bot}}_{n \geq 0}$ is a chain in $D$.
		%
		Since $D$ is a domain, the least upper bound
		%
		$\bigsqcup_{n = 0}^{\infty} f^n \prths{\bot}$ exists.
		%
		Furthermore, by the continuity of $f$,
		%
		\[
			f \prths{x} =
			f \prths{\bigsqcup_{n = 0}^{\infty} f^n \prths{\bot}} =
			\bigsqcup_{n = 0}^{\infty} f^{n + 1} \prths{\bot} =
			\bigsqcup_{n = 0}^{\infty} f^n \prths{\bot} =
			x.
		\]
		%
		$\therefore x$ is a fixed point of $f$.

		To show that x is a least such, consider a fixed point $y$ of $f$.
		%
		Then, by induction, we can show that $y$ is an upper bound of the chain
		%
		$\braces{f^n \prths{\bot}}_{n \geq 0}$.
		%
		$\therefore x \sqsubseteq y$.
		%
	\end{proof}
	%
	\item
	%
	When $P, P'$ are predomains, we write $\brackets{P \toc P'}$ for the set of
	continuous functions.
	%
	When $\brackets{P \toc P'}$\footnote{
		\begin{minipage}{0.9\textwidth}
			%
			Although we rush here, this function space construction is very important.
			%
			It lets domain theory be applicable to functional languages.
			%
		\end{minipage}
	} is given pointwise order
	$\sqsubseteq$,
	%
	\[
		f \sqsubseteq g \iff f \prths{x} \sqsubseteq_{P'} g \prths{x}
		\textrm{ for all } x \in P, \quad f, g \in \brackets{P \toc P'},
	\]
	it becomes a predomain where the limit of a chain $\braces{f_n}_{n \geq 0}$ is
	defined pointwise $x \mapsto \bigsqcup_{n = 0}^\infty f_n \prths{x}$.
	%
	Furthermore, if $P'$ is a domain with the least element $\bot'$, then
	$\brackets{P \toc P'}$ is also a domain with $x \mapsto \bot'$ as its least
	element.
	%
	\item
	%
	$D$ is a domain with the least element $\bot$.
	%
	Define $Y_D$ to be the following function from $\brackets{D \toc D}$ to $D$:
	%
	\[
		Y_D \prths{f} = \bigsqcup_{n = 0}^{\infty} f^n \prths{\bot}.
	\]
	%
	\begin{lemma}
		%
		$Y_D$ is continuous\footnote{%
			This means that the very act of computing a fixed point of a given function is continuous.%
		}.
		%
	\end{lemma}
	\begin{proof}
		%
		Exercise.
		%
	\end{proof}
	%
	\item
	%
	There are a lot of interesting results in domain theory, some of which we will
	cover later in the course.
	%
	Before finishing this mini review of domain theory, I want to explain the
	lifting construction.
	%
	\begin{enumrm}
		%
		\item
		%
		\begin{itemize}
			%
			\item
			      %
			      $P_\bot := P \cup \braces{\bot}$ for a predomain $P$.
			      %
			\item
			      %
			      $x \sqsubseteq_{P_\bot} y$ iff $x = \bot$ or $x, y \in P \textrm{ and } x \sqsubseteq_P y$
			      for $x, y \in P_\bot$.
			      %
			\item
			      %
			      Intuitively, we are adding the least element to $P$ and converting $P$ to a
			      domain.
			      %
		\end{itemize}
		%
		\item
		%
		$i_\uparrow \in \brackets{P \toc P_\bot}$, sometimes called \ul{unit}.
		%
		\[
			i_\uparrow \prths{x} = x \textrm{ for all } x \in P.
		\]
		%
		\item
		%
		For each $f \in \brackets{P \toc P'_\bot}$,
		%
		\begin{align*}
			f_\doublebot              & \in \brackets{P_\doublebot \toc P'_\doublebot} \\
			f_\doublebot \prths{\bot} & = \bot                                         \\
			f_\doublebot \prths{x}    & = f \prths{x} \textrm{ for all } x \in P.
		\end{align*}
		%
		sometimes called \ul{Kleisli extension}.
		%
		\item
		%
		Why should we care about (ii) and (iii)?
		%
		Because they allow us to compose continuous functions from $P$ to $P'_\bot$
		%
		\[
			f \in \brackets{P \toc P'_\bot} \; \textrm{ and } \;
			g \in \brackets{P' \toc P''_\bot} \implies
			g_\doublebot \circ f \in \brackets{P \toc P''_\bot}.
		\]
		%
		We can view $(-)_\doublebot \circ (-)$ as a new composition operator $\circ'$.
		%
		Then, $\circ'$ is associative and $i_\uparrow \circ' f = f = f \circ'
			i_\uparrow$.
		%
		This means that $\prths{-_\bot, i_\uparrow, -_\doublebot}$ gives rise to a
		monad on predomains.
		%
	\end{enumrm}
	%
\end{enumcirc}

\section{Denotational semantics of the simple imperative language}

\begin{enumcirc}
	%
	\item
	%
	Recall that $\Sigma = \gram{var} \to \Z$.
	%
	$\Sigma$ is a predomain when given the discrete order $\sqsubseteq$.
	%
	(That is, $x \sqsubseteq y$ iff $x = y$ for all $x, y \in \Sigma$.)
	%
	\begin{align*}
		\intexpsem{-}  & \in \gram{intexp} \to \brackets{\Sigma \to \Z} (same as before)  \\
		\boolexpsem{-} & \in \gram{boolexp} \to \brackets{\Sigma \to \B} (same as before) \\
		\commexpsem{-} & \in \gram{comm} \to \brackets{\Sigma \toc \Sigma_\bot}
	\end{align*}
	%
	\begin{align*}
		\bbrackets{\cassign{x}{e}} \sigma     & = \aug{\sigma}{x: \bbrackets{e} \sigma}                                                        \\
		\bbrackets{\cskip} \sigma             & = \sigma                                                                                       \\
		\bbrackets{\cseq{c_1}{c_2}} \sigma    & = \bbrackets{c_2}_\doublebot \prths{\bbrackets{c_1} \sigma}                                    \\
		\bbrackets{\cif{b}{c_1}{c_2}} \sigma  & = \cif{\bbrackets{b} \sigma}{\bbrackets{c_1} \sigma}{\bbrackets{c_2} \sigma}                   \\
		\bbrackets{\cwhile{b}{c}} \sigma      & = Y_{\Sigma_\bot} \prths{F}                                                                    \\
		\textrm{ where } F                    & \in \brackets{\Sigma \toc \Sigma_\bot} \toc \brackets{\Sigma \toc \Sigma_\bot}                 \\
		\textrm{and }F\prths{f}\prths{\sigma} & := \cif{\bbrackets{b} \sigma}{\prths{f_\doublebot \circ \bbrackets{c}} \prths{\sigma}}{\sigma}
	\end{align*}
	%
	Note: $Y_{\Sigma_\bot}$ and $\toc$ are where we get something by using domain
	theory.
	%
	\item
	%
	Why \ul{least} fixed point?
	%
	Because the least fixed point maps an input state to $\bot$
	%
	(denoting non-termination, hence, the absence of any information)
	%
	whenever the corresponding output state is not uniquely determined by the
	equation $F(f) = f$.
	%
	For example,
	%
	\begin{enumrm}
		%
		\item
		%
		least fixed point \dots
		%
		$\bbrackets{\cwhile{\textrm{true}}{\cskip}} \sigma = \bot$.
		%
		\item
		%
		non-least fixed point
		%
		\dots $\bbrackets{\cwhile{\textrm{true}}{\cskip}} \sigma = \sigma$.
		%
		\begin{align*}
			F\prths{f}\prths{\sigma} & = \cif{\bbrackets{\textrm{true}} \sigma}{\prths{f_\doublebot \circ \bbrackets{\cskip}} \prths{\sigma}}{\sigma} \\
			                         & = \prths{f_\doublebot \circ \bbrackets{\cskip}} \prths{\sigma}                                                 \\
			                         & = f_\doublebot \prths{\bbrackets{\cskip} \prths{\sigma}}                                                       \\
			                         & = f_\doublebot \prths{\sigma}                                                                                  \\
			                         & = f\prths{\sigma}                                                                                              \\
		\end{align*}
		%
		That is, $F(f) = f$.
		%
	\end{enumrm}
	%
	Later when we consider the correspondence between denotational semantics and
	operational semantics, we will answer this question more rigorously.
	%
	\item
	%
	Design decision of our language seen in the semantics:
	%
	\[
		\intexpsem{-} : \gram{intexp} \to \Sigma \to \Z
	\]
	%
	all integer expressions terminate and do not raise exceptions.
	%
	A similar remark applies to boolean expressions as well.
	%
	\item
	%
	Choosing the type of the semantics function such as
	%
	$\gram{intexp} \to \Sigma \to \Z$
	%
	is the most important step in defining the semantics.
	%
	It also clarifies certain major design decisions of the target programming
	language.
	%
\end{enumcirc}

\section{Variable declaration and substitution}

\begin{enumcirc}
	%
	\item
	%
	\begin{grammar}
		<comm> ::= ... | newvar <var> := <intexp> in <comm>
	\end{grammar}
	%
	This is a construct that doesn't increase the expressivity of the language but
	enables the programmers to combat the complexity of software by introducing the
	ideas of scope and local variables.
	%
	\item
	%
	\[
		\bbrackets{\cnew{v}{e}{c}} \sigma =
		\prths{
			\prths{
				\lambda \sigma' \in \Sigma \;.\; \aug{\sigma'}{v : \sigma v}
			}_\doublebot \circ \bbrackets{c}
		} \prths{
			\aug{\sigma}{v : \bbrackets{e} \sigma}
		}
	\]
	%
	\dots
	%
	$\lambda \sigma' \in \Sigma \;.\; \aug{\sigma'}{v : \sigma v}$
	%
	means the function $\sigma' \mapsto \aug{\sigma'}{v: \sigma v}$, restoring the
	old value.
	%
	We didn't have to do something like this when we interpreted quantifications in
	predicate logic.
	%
	This is because there we didn't return a state, but a boolean value.
	%
	\item
	%
	How do we know that this is a sensible definition?
	%
	By checking expected properties like \cref{prop:coincidence} and
	\cref{prop:renaming}.

	$\fv{c}$ \dots free variables appearing in $c$. (textbook page 40)

	$\fa{c}$ \dots free assigned variables appearing in $c$. (textbook page 41)

	\begin{property}[Coincidence]\label{prop:coincidence}
		%
		\;\\
		%
		\vspace{-1.5em}
		%
		\begin{enumalpha}
			%
			\item
			%
			$\sigma w = \sigma' w \textrm{ for all } w \in \fv{c}$
			%
			\begin{align*}
				\implies & \prths{\bbrackets{c}\sigma = \bbrackets{c}\sigma' = \bot} \textrm{ or } \\
				         & \prths{
					\bbrackets{c}\sigma, \bbrackets{c}\sigma' \in \Sigma \textrm{ and }
					\prths{\bbrackets{c}\sigma} w = \prths{\bbrackets{c}\sigma'} w \textrm{ for all } w \in \fv{c}
				}
			\end{align*}
			%
			\item
			%
			$
				\bbrackets{c}\sigma \neq \bot \implies
				\prths{\bbrackets{c}\sigma} w = \sigma w \textrm{ for all } w \notin \fa{c}.
			$
		\end{enumalpha}
		%
	\end{property}
	%
	\begin{property}[Renaming]\label{prop:renaming}
		%
		\begin{multline*}
			\subsctext{v}{new} \notin \fv{c'} - \braces{v} \\
			\implies \bbrackets{\cnew{v}{e}{c'}} \sigma =
			\bbrackets{\cnew{\subsctext{v}{new}}{e}{\subst{c'}{v}{\subsctext{v}{new}}}} \sigma
		\end{multline*}
		%
	\end{property}
	%
\end{enumcirc}

\section{Syntactic Sugar}

\begin{enumcirc}
	%
	\item
	%
	Introduction of a construct by defining its meaning in terms of existing
	constructs in the language.
	%
	\item
	%
	Three definitions of for loop:
	%
	\begin{enumrm}
		%
		\item
		%
		$
			\prths{
				\cfor{v}{e_0}{e_1}{c}} :=
			\prths{
				\cseq{
					\cassign{v}{e_0}
				}{
					\cwhile{v \le e_1}{
						\prths{
							\cseq{c}{\cassign{v}{v + 1}}
						}
					}
				}
			}
		$
		%
		\item
		%
		$
			\prths{
				\cfor{v}{e_0}{e_1}{c}} :=
			\prths{
				\cnew{v}{e_0}{
					\cwhile{v \le e_1}{
						\prths{
							\cseq{c}{\cassign{v}{v + 1}}
						}
					}
				}
			}
		$
		%
		\item
		%
		$
			\prths{
				\cfor{v}{e_0}{e_1}{c}} := \\
			\textrm{\quad}
			\prths{
				\cnew{w}{e_1}{
					\cnew{v}{e_0}{
						\cwhile{v \le w}{
							\prths{
								\cseq{c}{\cassign{v}{v + 1}}
							}
						}
					}
				}
			}
		$,

		where $w \ne v$ and $w \notin \fv{e_0} \cup \fv{c}$.
		%
		\item
		      %
		      (iii) with the condition $v \notin \fv{c}$.
		%
	\end{enumrm}
	%
	\item
	%
	The for loop should be something easier to understand than while.
	%
	In this regard, (i) $<$ (ii) $<$ (iii) $<$ (iv).
	%
\end{enumcirc}

\section{Arithmetic errors}

\begin{enumcirc}
	%
	\item
	%
	How should we deal with $x \div 0$, underflow and overflow?
	%
	\item
	%
	Two approaches:
	%
	\begin{enumrm}
		%
		\item
		%
		early stop with error
		%
		\item
		%
		some default choice and computation continued:
		%
		ad hoc but it can become less ad hoc if we ensure that the default choices
		satisfy \ul{certain properties} such as
		%
		\begin{align*}
			\bbrackets{\prths{x + y} \times 0} \sigma                       & = 0                     \\
			\bbrackets{x \div 0 = x \div 0} \sigma                          & = \ttt                  \\
			\bbrackets{\cseq{\cassign{y}{x \div 0}}{\cassign{y}{e}}} \sigma & =
			\bbrackets{\cassign{y}{e}} \sigma \textrm{ when } y \notin \fv{e}                         \\
			\bbrackets{\cif{x + y = z}{c}{c}} \sigma                        & = \bbrackets{c} \sigma. \\
		\end{align*}
		%
	\end{enumrm}
	%
\end{enumcirc}

\section{Soundness and full abstraction}

\begin{enumcirc}
	%
	\item
	%
	The semantics defined so far looks ok, but is there any formal way to confirm
	this?
	%
	\item
	%
	One approach is to show that the semantics assigns the same meaning to two
	commands $c_1$ and $c_2$ only when $c_1$ and $c_2$ should be equal intuitively.
	%
	That is,
	%
	\[
		\bbrackets{c_1} = \bbrackets{c_2} \implies c_1 ``=" \footnotemark c_2.
	\]
	\footnoteeqn[0]{our intuitive notion of equality defined separately}
	%
	This property is called \ul{soundness}.
	%
	Its converse
	%
	\[
		\bbrackets{c_1} = \bbrackets{c_2} \impliedby c_1 ``=" \footnotemark c_2
	\]
	%
	is called \ul{full abstraction}.
	%
	\item
	%
	Now how to define $``="$?
	%
	We use a set of observable phrases with a hole or \ul{contexts}, $\Cc$, and a
	set of \ul{observations}, $\Oc$, which are functions from observable phrases to
	outcomes.
	%
	\[
		c_1 ``=" c_2 \iff \forall c \in \Cc \footnotemark ,\; \forall o \in \Oc \footnotemark,\;
		o \prths{c \brackets{c_1}\footnotemark} = o \prths{c \brackets{c_2}}
	\]
	\footnoteeqn[-2]{intuitively means all use cases}
	\footnoteeqn{intuitively means user's observations}
	\footnoteeqn{filling the hole of $c$ with $c_1$}
	%
	\begin{enumrm}
		%
		\item
		%
		Intuitively, this condition says that under all use cases, the user cannot
		observe the difference between $c_1$ and $c_2$.
		%
		\item
		%
		This is sometimes called \ul{observataional equivalence}.
		%
		\item
		%
		Note that this is not a syntax-directed (or compositional) definition.
		%
	\end{enumrm}
	%
	\begin{example}
		%
		Assume that $v_0, \dots , v_{n-1}$ are all the free variables in $c_1$ and
		$c_2$.
		%
		\[
			\Cc = \set{
				\begin{array}{c}
					\begin{aligned}
						\cnew{v_0     & }{k_0}{        \\
						\cnew{v_1     & }{k_1}{        \\
						              & \vdots         \\
						\cnew{v_{n-1} & }{k_{n-1}}{}}}
					\end{aligned}
					\\
					\prths{
						\begin{aligned}
							\brackets{-}; \textrm{ if } v_i = k & \textrm{ then } \cskip                         \\
							                                    & \textrm{ else } \cwhile{\textrm{true}}{\cskip}
						\end{aligned}
					}
				\end{array}
			}{
				\begin{array}{c}
					k, k_0, \dots , k_{n-1} \in \Z \\
					i \in \braces{0, \dots , n-1}
				\end{array}
			}
		\]
		%
		\vspace{1em}
		%
		\[
			\Oc = \braces{\lambda c \;.\; \cif{\bbrackets{c} \sigma_0 = \bot}{0}{1}}
		\]
		%
		where $\sigma_0 \prths{x} = 0$ for all $x$.
	\end{example}

\end{enumcirc}