\chapter{An Introduction to Category Theory}

Chapter 8 of Tennent's book

\section{Motivation}

\begin{enumcirc}
	%
	\item
	%
	The category theory is a branch of mathematics that studies essentially same or
	common notions and principles in different areas of mathematics.
	%
	It is very abstract, but provides good guidelines about finding right
	definitions and right or meaningful questions to ask in a new mathematical
	theory.
	%
	\item
	%
	The category theory had huge influence on the research on programming
	languages, such as Scala, Haskell, and Rust.
	%
	\item
	%
	In this course, we will focus on the influence of the category theory on the
	semantics research.
	%
	We will study abstract categorical concepts that give rise to popular notions
	appearing in the semantics of programming languages.
	%
	Another big objective is to understand a result in the category theory (or a
	categorical formulation of domain theory) that ensures the existence of certain
	recursively defined domains, such as the following $\Omega$ that you saw in the
	chapter 5 of Reynolds' book:
	%
	\begin{align*}
		\Omega       & \simeq \prths{\hat{\Sigma} + \Z \times \Omega + \brackets{\Z \to \Omega}}_\bot \\
		\hat{\Sigma} & \defeq \Sigma + \Sigma
	\end{align*}
	%
	\item
	%
	We will study only a tiny part of the category theory.
	%
	If you are excited about it and are willing to read a math book, I recommend
	Mac Lane's ``Categories for the Working Mathematician''.
	%
\end{enumcirc}

\section{Definition of Category}

\begin{definition}
	%
	A \ul{category} $\Cc$ is a tuple $\prths{\Obj, \Hom, \circ, \id}$ where
	%
	\begin{enumarab}
		%
		\item
		%
		$\Obj$ is a collection of elements called \ul{objects};
		%
		\item
		%
		for all objects $x, y \in \Obj$, $\Homset{x}{y}$ is a collection of elements
		called \ul{morphisms} from $x$ to $y$;
		%
		\item
		%
		for all objects $x, y, z \in \Obj$, $\circ_{x,y,z}$ (or simply $\circ$) is a
		map from $\Homset{y}{z} \times \Homset{x}{y}$ to $\Homset{x}{z}$, and is called
		composition;
		%
		\item
		%
		for every object $x \in \Obj$, $\id_x$ is an element in $\Homset{x}{x}$, and is
		called \ul{identity morphism};
		%
		\item
		%
		these data should satisfy associativity and identity axioms:
		%
		\begin{enumalpha}
			%
			\begin{minipage}[t]{0.4\textwidth}
				\item
				%
				Associativity:
				%
				\begin{align*}
					 & \forall                   w, x, y, z \in \Obj,         \\
					 & \forall                   f \in \Homset{w}{x},         \\
					 & \forall                   g \in \Homset{x}{y},         \\
					 & \forall                   h \in \Homset{y}{z},         \\
					 & h \circ \prths{g \circ f}  = \prths{h \circ g} \circ f \\
				\end{align*}
				%
			\end{minipage}
			%
			\begin{minipage}[t]{0.4\textwidth}
				\item
				%
				Identity:
				%
				\begin{align*}
					 & \forall x, y \in \Obj,                \\
					 & \forall f \in \Homset{x}{y},          \\
					 & f \circ \id_x = f = \id_y \circ f = f
				\end{align*}
				%
			\end{minipage}
			%
		\end{enumalpha}
		%
	\end{enumarab}
	%
\end{definition}

\begin{enumcirc}
	%
	\item
	%
	Although not perfect, a reasonably good intuition is that a category $\Cc$ is a
	collection of spaces.
	%
	\footnote{spaces that you encounter in mathematics, such
		as metric spaces, vector spacess, topological spaces, etc.}
	%
	The $\Obj$ part of $\Cc$ consists of spaces, and the $\Homset{x}{y}$ part of
	$\Cc$ consists of maps between spaces that preserve the structures of the
	spaces.
	%
	$\circ$ is then the usual function composition and $\id_x$ is the identity function on $x$.
	%
	\item
	%
	Here are some examples that match the intuition that I just explained:
	%
	\begin{enumrm}
		%
		\item
		%
		Set \dots category of all sets
		%
		\begin{itemize}
			%
			\item
			      %
			      $\Obj$ is the collection of all sets
			      %
			\item
			      %
			      $\Homset{x}{y}$ is the set of all functions from $x$ to $y$.
			      %
			      Note that since $x, y \in \Obj$, they are sets.
			      %
			\item
			      %
			      $\circ$ is usual function composition
			      %
			\item
			      %
			      $\id_x$ is the identity function on $x$
			      %
		\end{itemize}
		%
		\item
		%
		Predom \dots category of all predomains and continuous functions
		%
		\begin{itemize}
			%
			\item
			      %
			      $\Obj$ is the collection of all predomains
			      %
			\item
			      %
			      $\Homset{x}{y}$ is the set of all continuous functions from $x$ to $y$.
			      %
			\item
			      %
			      $\circ$ is the function composition
			      %
			\item
			      %
			      $\id_x$ is the identity function on $x$
			      %
		\end{itemize}
		%
		\item
		%
		Dom \dots Category of domains and continuous functions
		%
		\begin{itemize}
			%
			\item
			      %
			      $\Obj$ is the collection of all domains
			      %
			\item
			      %
			      $\Homset{x}{y}$ is the set of all continuous functions from $x$ to $y$.
			      %
			\item
			      %
			      $\circ$ is the function composition
			      %
			\item
			      %
			      $\id_x$ is the identity function on $x$
			      %
		\end{itemize}
		%
	\end{enumrm}
	%
	Note that Dom is in a sense included in Predom.
	%
	(Technically, it is a full subcategory of Predom.)
	%
	\item
	%
	As indicated by my use of the phrase ``not perfect'', there are categories that
	do not match the intuition well.
	%
	Here is a very well-known example:

	Let $\prths{X, \sqsubseteq}$ be a partially ordered set.
	%
	It can be understood as a category:
	%
	\begin{itemize}
		%
		\item
		      %
		      $\Obj = X$
		      %
		\item
		      %
		      $\Homset{x}{y} = \begin{cases}
				      \emptyset                    & \text{if } x \not\sqsubseteq y \\
				      \braces{\star} \footnotemark & \text{if } x \sqsubseteq y     \\
			      \end{cases}
		      $
		      \footnoteeqn[0]{a set with one element. It doesn't matter what that element is.}
		      %
		\item
		      %
		      $\id_X = \star$
		      %
		\item
		      %
		      $\circ_{x,y,z} \in \brackets{\Homset{y}{z} \times \Homset{x}{y} \mapsto \Homset{x}{z}}$

		      If $x \sqsubseteq y$ and $y \sqsubseteq z$ (i.e., $\Homset{x}{y} \neq
			      \emptyset$ and $\Homset{y}{z} \neq \emptyset$),
		      %
		      then $\circ_{x,y,z}$ is the constant function to the unique element in
		      $\Homset{x}{z}$.
		      %
		      ($\Homset{x}{z} \ne \emptyset$ in this case because of the transitivity of $\sqsubseteq$.)

		      Otherwise (i.e., $\Homset{x}{y} = \emptyset$ or $\Homset{y}{z} = \emptyset$),
		      %
		      $\circ_{x,y,z}$ is the empty function
		      %
		      (the function whose graph is the empty set).
		      %
	\end{itemize}
	%
	\item
	%
	In the category theory, we often use commutative diagrams to express the
	equality of two morphisms.
	%
	For instance,
	%
	\[
		\begin{tikzcd}[cramped, sep=3em]
			x
			\arrow[r, "f"]
			\arrow[d, swap, "k"] &
			y
			\arrow[d, "g"]\\
			a
			\arrow[r,"h"] &
			z
		\end{tikzcd}
	\]
	%
	expresses that $x, y, a, z$ are objects in a category and $f, g, h, k$ are
	morphisms with domains and codomains indicated by the arrows (for instance, $f
		\in \Homset{x}{y}$), and $g \circ f = h \circ k$.
	%
	\footnote{the most important bit}
	%
	\item
	%
	Another intuition about categories is that objects in a category are types in a
	programming language and morphisms from $x$ to $y$ are functions in the
	language from the input of type $x$ to the output of type $y$.
	%
\end{enumcirc}

\section{Initial and terminal objects. Product and co-product}

\begin{enumcirc}
	%
	\item
	%
	In a category $\Cc$, we can build a new object out of existing ones.
	%
	Often this new object satisfies one of the well-known conditions, and has a
	well-known name.
	%
	We will study a few such names.
	%
	\item
	%
	Consider a category $\Cc$ and its objects $x, y \in \Obj$.
	%
	\begin{enumrm}
		%
		\item
		%
		An object $z$ is a \ul{product} of $x$ and $y$, often written as
		%
		$z = x \times y$,
		%
		if there are morphisms
		%
		$\pi_0 \in \Homset{z}{x}$\footnote{often written as $\pi_0 : x \times y \to x$}
		%
		and $\pi_1 \in \Homset{z}{y}$\footnote{often written as $\pi_1 : x \times y \to
				y$}
		%
		s.t.

		for all objects $w$ and morphisms
		%
		$f: w \to x$\footnote{same as $f \in \Homset{w}{x}$}
		%
		and $g: w \to y$, there exists a unique morphism
		%
		$h: w \to x \times y$
		%
		\footnote{we often write $h$ as $\chevrons{f, g}$. Reynolds writes it as
			$f\otimes g$.}
		%
		with $f = \pi_0 \circ h$ and $g = \pi_1 \circ h$.
		%

		In practice,
		%
		\footnote{! means uniqueness and $\dasharrow$ means existence}
		%
		\[
			\begin{tikzcd}[cramped, sep=3em]
				x&
				&
				y\\
				&
				x \times y \arrow[ul, "\pi_0"] \arrow[ur, swap, "\pi_1"]
				&
				\\
				&
				z \arrow[u, dashed, "!h"] \arrow[uul, "f", bend left = 30] \arrow[uur, swap, "g", bend right = 30]
				&
				\\
			\end{tikzcd}
		\]
		%
		\item
		%
		An object $z$ is a \ul{co-product} (or \ul{sum}) of $x$ and $y$, often written
		as $z = x + y$,
		%
		if there are morphisms $\inj_0: x \to x + y$ and $\inj_1: y \to x + y$ s.t.

		for all objects $w$ and morphisms $f: x \to w$ and $g: y \to w$, there exists a
		unique morphism $h: x + y \to w$
		%
		\footnote{often written as $\brackets{f, g}$. Reynolds writes it as $f \oplus g$}
		%
		with $f = h \circ \inj_0$ and $g = h \circ \inj_1$.

		In practice,
		%
		\[
			\begin{tikzcd}[cramped, sep=3em]
				&
				w
				&
				\\
				&
				x + y \arrow[u, dashed, "!h"]
				&
				\\
				x \arrow[ur, swap, "\inj_0"] \arrow[uur, "f", bend left = 30]&
				&
				y \arrow[ul, "\inj_1"] \arrow[uul, swap, "g", bend right = 30]\\
			\end{tikzcd}
		\]
		%
		\item
		%
		An object $x$ is \ul{initial} if for every object $w$, there exists the unique
		morphism $h: x \to w$.
		%
		\item
		%
		An object $x$ is \ul{final} if for every object $w$, there exists the unique
		morphism $h: w \to x$.
		%
		\footnote{$h$ is often written as $!_w$ or $!$}
		%
	\end{enumrm}
	%
	\item
	%
	Note that all of these notions are defined purely in terms of morphisms,
	without referring to elements of objects.
	%
	This is a bit like specifying a property of an abstract data type in terms of
	its operations, not in terms of its implementations.
	%
	\item
	%
	In the category Predom, the product of two predomains $P_0$ and $P_1$ that we
	studied in \cref{ch:failure} is indeed a categorical product in (i).
	%
	Also, the sum of $P_0$ and $P_1$ in \cref{ch:failure} is indeed a categorical
	sum or co-product.
	%
	The predomain of the singleton set $\prths{\braces{\star}, \sqsubseteq}$ is a
	terminal object.
	%
	The predomain of the empty set $\prths{\emptyset, \sqsubseteq}$ is an initial
	object.
	%
	\item
	%
	Consider the category corresponding to a pratially ordered set $\prths{X,
			\sqsubseteq}$.
	%
	Then, an object $x \in X$ is initial if and only if it is the least element in
	$X$.
	%
	It is final if and only if it is the greatest element.
	%
	For objects $x, y$ in $X$, $x + y$ is the least upper bound of $x$ and $y$, and
	$x \times y$ is the greatest lower bound of $x$ and $y$.
	%
\end{enumcirc}

\begin{exercise}
	%
	Prove \circled{4} and \circled{5}.
	%
\end{exercise}

\section{Functor and Natural Transformation}

\begin{enumcirc}
	%
	\item
	%
	Intuitively, a functor is a structure-preserving map from one category to
	another.
	%
	A natural transformation is a uniform map from one functor to another.
	%
	In PL terms, a functor is a type constructor.
	%
	(Recall that in this analogy, an object in a category is a type.)
	%
	And a natural transformation is a polymorphic function.
	%
	\item
	%
	Let $\Cc$ and $\Dc$ be categories.
	%
	\begin{definition}
		%
		A \ul{functor} $F$ from $\Cc$ to $\Dc$ is a pair of two maps
		$\subsctext{F}{obj}$ and $\subsctext{F}{mor}$ s.t.
		%
		\begin{enumrm}
			%
			\item
			%
			$\subsctext{F}{obj}$ maps objects in $\Cc$ to objects in $\Dc$.
			%
			\item
			%
			for objects $x, y \in \Obj\prths{\Cc}$\footnote{objects of $\Cc$},
			%
			\[
				\prths{\subsctext{F}{mor}}_{x,y} \in
				\brackets{
					\Homset[\Cc]{x}{y} \to
					\Homset[\Dc]{\subsctext{F}{obj}\prths{x}}{\subsctext{F}{obj}\prths{y}}}
			\]
			%
			\item
			%
			$\subsctext{F}{mor}$ preserves $\circ$ and $\id$:
			%
			\begin{enumalpha}
				%
				\item
				%
				for all objects $x$ of $\Cc$,
				%
				\[
					\prths{\subsctext{F}{mor}}_{x,x}\prths{\id_x} = \id_{\subsctext{F}{obj}\prths{x}}
				\]
				%
				\item
				%
				for all morphisms $f \in \Homset[\Cc]{x}{y}$ and $g \in \Homset[\Cc]{y}{z}$,
				%
				\[
					\prths{\subsctext{F}{mor}}_{x,z}\prths{g \circ f} =
					\prths{\subsctext{F}{mor}}_{y,z}\prths{g} \circ
					\prths{\subsctext{F}{mor}}_{x,y}\prths{f}
				\]
				%
			\end{enumalpha}
			%
		\end{enumrm}
		%
	\end{definition}
	%
	We use $F$ to mean $\subsctext{F}{obj}$ and $\prths{\subsctext{F}{mor}}_{x,y}$.
	%
	\item
	%
	To see common examples, we need to understand to product of two categories.
	%
	When $\Cc$ and $\Dc$ are categories, \ul{the product category} $\Cc \times \Dc$
	is defined as follows:
	%
	\begin{itemize}
		%
		\item
		      %
		      $\Obj\prths{\Cc \times \Dc} = \set{\prths{x, y}}{x \in \Obj\prths{\Cc} \wedge y \in \Obj\prths{\Dc}}$
		      %
		\item
		      %
		      $\Homset[\Cc \times \Dc]{\prths{u, v}}{\prths{x, y}} =
			      \set{\prths{f, g}}{f \in \Homset[\Cc]{u}{x} \wedge g \in \Homset[\Dc]{v}{y}}$
		      %
		\item
		      %
		      $\prths{f, g} \circ \prths{f', g'} = \prths{f \circ f', g \circ g'}$
		      %
		\item
		      %
		      $\id_{\prths{x, y}} = \prths{\id_x, \id_y}$
		      %
	\end{itemize}
	%
	\item
	%
	The $\times, +, \bot$ operators on predomains are in fact functors.
	%
	\begin{itemize}
		%
		\item
		      %
		      $\prths{-}_\bot$ : Predom $\to$ Predom

		      $\prths{-}_\bot \prths{P} = P_\bot$

		      $\prths{-}_\bot \prths{f} = \lambda x.
			      \begin{cases}
				      f\prths{x} & \text{if } f\prths{x} \neq \bot \\
				      \bot       & \text{otherwise}
			      \end{cases}
		      $
		      %
		\item
		      %
		      $\times$\footnote{product functor} : Predom $\times$\footnote{product of categories} Predom $\to$ Predom.

		      $\times \prths{P_0, P_1} = P_0 \times\footnotemark P_1$
		      \footnoteeqn[0]{product of predomains}

		      $\times \prths{f, g} = \lambda \prths{x, y}. \prths{f\prths{x}, g\prths{y}} = f \times g$
		      \footnote{Reynolds' notation (and standard notation) that we covered in \cref{ch:failure}}
		      %
		\item
		      %
		      $+$\footnote{sum functor} : Predom $\times$ Predom $\to$ Predom.

		      $+ \prths{P_0, P_1} = P_0 +\footnotemark P_1$
		      \footnoteeqn[0]{sum of predomains}

		      $+ \prths{f, g} = \lambda \prths{x, y}. \begin{cases}
				      \inj_0 \prths{f\prths{u}} & \text{if } x = \inj_0 \prths{u} \\
				      \inj_1 \prths{g\prths{v}} & \text{if } x = \inj_1 \prths{v}
			      \end{cases}
			      = f + g$
		      \footnote{notation that we used in \cref{ch:failure}}
		      %
	\end{itemize}
	%
	\item
	%
	One other important functor is the identity functor:

	$\Id$: Predom$\to$ Predom

	$\Id\prths{X} = X$

	$\Id\prths{f} = f$
	%
	\item
	%
	Let $F, G$ be functors from $\Cc$ to $\Dc$.
	%
	\begin{definition}
		%
		A \ul{natural transformation} $\eta$ from $F$ to $G$,
		%
		denoted $\eta: F \xrightarrow{\bullet} G: \Cc \to \Dc$ or
		%
		\begin{tikzcd}[cramped, sep=1em]
			\Cc
			\arrow[rr, "F", bend left = 40]{}
			\arrow[rr, swap, "G", bend right = 40]{}
			&
			\;\;\, \Downarrow \eta
			&
			\Dc
		\end{tikzcd}%
		, is a collection of morphisms in $\Dc$ indexed by objects in $\Cc$:
		%
		\[
			\prths{
				\eta_x : F\prths{x} \to G\prths{x}
			}_{x \in \Obj\prths{\Cc}}
		\]
		%
		such that
		%
		\begin{enumrm}
			%
			\item
			%
			for each object $x$ in $\Cc$, $\eta_x$ is a $\Dc$-morphism from $F\prths{x}$ to
			$G\prths{x}$.
			%
			\item
			%
			for every morphism $f: x \to y$ in $\Cc$ (i.e., $f \in \Homset[\Cc]{x}{y}$),
			%
			\[
				\begin{tikzcd}[cramped, sep=3em]
					F\prths{x}
					\arrow[r, "\eta_x"]
					\arrow[d, swap, "F\prths{f}"] &
					G\prths{x}
					\arrow[d, "G\prths{f}"] \\
					F\prths{y}
					\arrow[r, swap, "\eta_y"] &
					G\prths{y}
				\end{tikzcd}
			\]
			%
			This is called naturality condition.
			%
		\end{enumrm}
		%
	\end{definition}
	%
	\item
	%
	One intuition is that $F$ and $G$ are type constructors and $\eta$ is a
	polymorphic function from $F$ to $G$.
	%
	Whenever we are given a type $x$, we have an instantiation of $\eta$ at $x$,
	that is a function from the type $F \prths{x}$ to the type $G \prths{x}$.

	The condition (i) says that $\eta$ should type check.
	%
	The condition (ii) says that what $\eta$ does at $x$ should be identical in a
	sense to what it does at $y$.
	%
	In other words, $\eta$ should bot depend much on its type parameter $x$.
	%
	This is called uniformity.
	%
	\item
	%
	Here are a few examples.

	\begin{enumrm}
		%
		\item
		%
		$\textrm{unit}$: $\Id \xrightarrow{\bullet} \prths{-}_\bot: \text{Predom} \to \text{Predom}$

		\begin{center}
			$\textrm{unit}_P \in \brackets{P \to P_\bot}$

			$\textrm{unit}_P\prths{x} = x$
		\end{center}
		%
		\item
		%
		Let $\textrm{fst}$ be a functor from $\Cc \times \Dc$ to $\Cc$, defined by

		\begin{center}
			$\textrm{fst}\prths{x, y} = x$ (for objects)

			$\textrm{fst}\prths{f, g} = f$ (for morphisms)
		\end{center}

		Then,

		$\pi_0 : \prths{-} \times \prths{-} \xrightarrow{\bullet} \textrm{fst}: \text{Predom} \times \text{Predom} \to \text{Predom}$

		\begin{center}
			$\prths{\pi_0}_{P, P'} \in \brackets{P \times P' \to P}$

			$\prths{\pi_0}_{P, P'}\prths{a, b} = a$
		\end{center}
		%
	\end{enumrm}
	%
	\begin{exercisetab}
		%
		Show that $\textrm{unit}$ and $\pi_0$ are indeed natural transformations.
		%
	\end{exercisetab}
	%
	\begin{exercisetab}
		%
		$\pi_1$, $\inj_0$, and $\inj_1$ are also natural transformations if we pick
		appropriate functors.
		%
		Find such functors.
		%
	\end{exercisetab}
	%
	\item
	%
	Notice that all of $\textrm{unit}$, $\pi_0$, $\pi_1$, $\inj_0$, and $\inj_1$ do
	something very straightforward intuitively.
	%
	They don't do any clever tricks.
	%
	Naturality condition says in a sense that a natural transformation doesn't do
	anything clever.
	%
	In more positive terms, it says that a natural transformation only performs
	canonical operations.
	%
\end{enumcirc}
